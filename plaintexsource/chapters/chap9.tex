\documentstyle{book}
\input extra.tex
\input le2.sty
\input epsf.tex
\input chap8.lnk
\maketoctrue
\indextrue
\mardiagtrue
\openseg{chap9}
\Monograph
\topmatter
\nextchap{Series Solutions of Differential Equations}
\title
\chapter{\chapno}Series Solutions of Differential Equations
\endtitle
\endtopmatter
\document
\nextsec{Power Series Solutions at Ordinary Points}
\head \sn.  Power Series Solutions at Ordinary Points
\endhead

We want to solve equations of the form
\nexteqn
\xdef\DE{\eqn}
$$
y'' + p(t)y' + q(t)y = 0\tag\eqn
$$
where $p(t)$ and $q(t)$ are analytic functions on some interval
in $\R$.   We saw several examples of such equations in the
previous chapter, and we shall see others.  They arise commonly
in solving physical problems.

The functions $p(t)$ and $q(t)$ are supposed to be analytic on
their domains, but they often will have singularities at other
points.  We assume that these singularities occur at {\it isolated\/}
points of $\R$.   Points where the functions are analytic are
called {\it ordinary\/} points and other points are called
\outind{ordinary point}%
\outind{singular point}%
{\it singular\/} points.

\nextex
\xdef\ExOne{\en}
\example{Example \en} The coefficients of   Legendre's equation
$$
y'' - \frac{2t}{1 - t^2}y' + \frac{\alpha(\alpha + 1)}{1 - t^2}y = 0
$$
are analytic on any interval not containing the points $t = \pm 1$.
$\pm 1$ are singular points. 
\endexample
\outind{Legendre's equations}%

Let $t_0$ be an ordinary point of (\eqn).  It makes sense to look
for a solution $y = y(t)$ which is analytic in an interval containing
$t_0$.   Such a solution will have a power series centered at
$t_0$
\nexteqn
$$ 
y = \sum_{n=0}^\infty a_n(t - t_0)^n\tag\eqn
$$
with a positive radius of convergence $R$.   Hence, one way to
try to solve the equation is to substitute the series (\eqn)
in the equation (\DE) and see if we can determine the coefficients
$a_n$.   
Generally, these coefficients may be expressed in terms of
the first two,
$a_0 = y(t_0)$ and $a_1 = y'(t_0)$.  (See Chapter VIII, Section 1,)
%%% Reference
We illustrate the method by an example.

\example{Example \en, {\rm continued}}  Consider
$$
y'' - \frac{2t}{1 - t^2}y' + \frac{\alpha(\alpha + 1)}{1 - t^2}y = 0
$$
in the neighborhood of the point $t_0 = 0$.   Before trying a
series solution, it is better in this case to multiply
through by the factor $1 - t^2$ so as to avoid fractions.
(This is not absolutely necessary, but in most cases it simplifies
the calculations.)   This yields
\nexteqn
\xdef\DEE{\eqn}
$$
(1 - t^2)y'' - 2ty' + \alpha(\alpha +1)y = 0.\tag\eqn
$$
We try a power series of the form $y = \sum_{n=0}^\infty a_nt^n$
and calculate
$$
\align
y &= \sum_{n=0}^\infty a_nt^n \\
y' &= \sum_{n=1}^\infty na_nt^{n-1} \\
y'' &= \sum_{n=2}^\infty n(n-1)a_nt^{n-2} .
\endalign $$
We now write down the terms in the differential equation and
{\it renumber the indices\/} so that we may collect coefficients of 
common powers of $t$.
$$\alignat 2
y'' &= \sum_{n=2}^\infty n(n-1)a_nt^{n-2} 
&&=\sum_{n+2=2}^\infty (n+2)(n+2 -1)
a_{n+2}t^{n+2 -2}\\
&&&\text{replacing $n$ by $n+2$} \\
&&&=\sum_{n=0}^\infty (n+2)(n+1)a_{n+2}t^n\\
-t^2y'' &=  \sum_{n=2}^\infty(-1) n(n-1)a_nt^{2+n-2} &&= 
\sum_{n=0}^\infty(-1) n(n-1)a_nt^{n}\\
&&&\text{terms for $n= 0,1$ are 0}\\
-2ty' &= \sum_{n=1}^\infty(-2) na_nt^{1+n-1} &&= \sum_{n=0}^\infty(-2) na_nt^n\\
&&&\text{term for $n=0$ is 0}\\
\alpha(\alpha +1)y &=&&\ \sum_{n=0}^\infty\alpha(\alpha +1) a_nt^n.
\endalignat $$
(Make sure you understand each step!)   Now add everything up.
On the left side, we get zero, and on the right side we collect
terms involving the same power of $t$.
$$
0 = \sum_{n=0}^\infty
[(n+2)(n+1)a_{n+2} - n(n-1)a_n - 2na_n + \alpha(\alpha +1) a_n]t^n .
$$ 
A power series in $t$ is zero if and only if the coefficient of each power
of $t$ is zero, so we obtain
$$
\gather
(n+2)(n+1)a_{n+2} - n(n-1)a_n - 2na_n + \alpha(\alpha +1) a_n = 0\qquad
\text{for } n \ge 0 \\
(n+2)(n+1)a_{n+2} =  [n(n-1) + 2n - \alpha(\alpha +1)] a_n \\
(n+2)(n+1)a_{n+2} =  [n^2 -n + 2n - \alpha(\alpha +1)] a_n 
 = [n^2 +n - \alpha(\alpha +1)] a_n\\
(n+2)(n+1)a_{n+2} = (n +\alpha +1)(n - \alpha)a_n\\
a_{n+2} = \frac{(n +\alpha +1)(n - \alpha)}{(n+2)(n+1)}a_n\qquad\text{for }
n \ge 0.
\endgather $$
The last equation is an example of what is called a
{\it recurrence relation}.  Once $a_0$ and $a_1$ are known,
\outind{recurrence relation}%
it is possible iteratively to determine any $a_n$ with $n \ge 2$.
%Exer  Write a computer program to determine a_n from a
%Exer recurrence relation.
It is better, of course, to find a general {\it formula\/} for
$a_n$, but this is not always possible.   In the present example,
it is possible to find such a formula, but it is very complicated. 
(See {\it Differential Equations\/} by G. F. Simmons,
Section 27.)   We derive the formula
for two particular values of $\alpha$.  
First take $\alpha = 1$.  The recurrence relation is
$$
a_{n+2} = \frac{(n+2)(n -1)}{(n+2)(n+1)}a_n
 = \frac{n-1}{n+1}a_n \qquad\text{for } n\ge 0.
$$
Thus, for even $n$,
$$
\alignat 2
n&=0 &\qquad\qquad a_2 &= -a_0 \\
n&=2 &\qquad\qquad a_4 &= \frac 13a_2 = - \frac 13 a_0 \\
n&=4 &\qquad\qquad a_6 &= \frac 35 a_4 =\frac 35 \left(-\frac 13 \right)a_0
 = -\frac 15 a_0 \\
n&=6 &\qquad\qquad a_8 &= \frac 57 a_6 = \frac 57 \left(-\frac 15 \right)a_0
 = -\frac 17 a_0.
\endalignat
$$
The general rule is now clear
$$
a_n = -\frac 1{n-1} a_0\qquad\text{for } n = 2, 4, 6, \dots \,.
$$
This could also be written
$$
a_{2k} = -\frac 1{2k -1} a_0\qquad\text{for } k = 1, 2, 3 \dots \,.
$$

For $n$ odd, we get for $n = 1$,
$$
a_3 = \frac{0}{2}a_1 = 0,
$$
so $a_n = 0$ for every odd $n \ge 3$.

We may now write out the general solution 
$$\align
y &= \sum_{n=0}^\infty a_nt^n\\
&= \undersetbrace{\text{even n}}\to{
a_0 + \sum_{k=1}^\infty \left(-\frac 1{2k -1}\right)a_0t^{2k}}
\quad+\quad a_1t \\
&= a_0\left(1 -\sum_{k=1}^\infty \frac {t^{2k}}{2k-1}\right) + a_1t.
\endalign
$$
Define $y_1 = 1 -\sum_{k=1}^\infty \dfrac {t^{2k}}{2k-1}$
and $y_2 = t$.   Then, we have shown that any solution may
be written
\nexteqn
$$
y = a_0y_1 + a_1y_2.\tag\eqn
$$
Moreover it is clear that $y_1$ and $y_2$ form a linearly
independent pair.  (Even if it weren't obvious by inspection,
we could always verify it by calculating the Wronskian at
$t = 0$.)
%Exer Calculate Wronskian at $t= 0$.
Hence, (\eqn) gives a general solution of Legendre's  differential
equation for $\alpha = 1$.   

(Since $y_2 = t$ is a solution, you could use the method of
reduction of order to find a second independent solution.
This yields
$$
 y = 1 - \frac t2 \ln\left(\frac{1+t}{1-t}\right)
 = 1 - \frac t2(\ln(1 + t) - \ln(1 -t)).
$$
See if you can expand this out and get $y_1$.)
%Exer   Expand to get $y_1$

Let's see what happens for $\alpha = -\dfrac 12$.   The
recurrence relation may be rewritten
$$
\align
a_{n+2} &= \frac{(n + 1/2)^2}{(n+2)(n+1)}a_n\\
        &= \frac {(2n + 1)^2}{4(n+2)(n+1)}a_n
\qquad\text{for }
n \ge 0.
 \endalign $$
(Both numerator and denominator were multiplied by 4.)
Thus, for $n$ even, we have
$$
\alignat 2
n&=0 &\qquad\qquad a_2 &= \frac {1^2}{4\cdot 2} a_0 \\
n&=2 &\qquad\qquad a_4 &= \frac {5^2}{4\cdot 4\cdot 3}a_2 
=  \frac {(5\cdot 1)^2}{4^2\cdot 4!} a_0 \\
n&=4 &\qquad\qquad a_6 &= \frac {9^2}{4\cdot 6\cdot 5} a_4 
=\frac {(9\cdot 5 \cdot 1)^2}{4^3\cdot 6!}a_0\\
n&=6 &\qquad\qquad a_8 &= \frac {13^2}{4\cdot 8\cdot 7} a_6 
= \frac {(13\cdot 9\cdot 5\cdot 1)^2}
{4^4\cdot 8!}a_0 .
\endalignat $$
Note that $4^36! = 2^66!$ and $4^48! = 2^88!$.
We begin see the following rule for $n$ even,
$$
a_n = \frac{[(2n -3)(2n - 7)\dots 5\cdot 1)]^2}
{2^nn!} a_0\qquad\text{for } n = 2, 4, 6,\dots\,.
$$
In the numerator, we start with $2n - 3$, reduce successively
by 4 until we get down to 1, multiply all those numbers together,
and square the whole thing.

For $n$ odd, the same analysis yields a similar result.
$$
a_n = \frac{[(2n-3)(2n -7)\dots 3]^2}{2^{n-1}n!}a_1
\qquad\text{for } n = 3, 5, 7, \dots\, .
$$
As above, we may define
$$\align
y_1 &= 1 + \sum\Sb n\ \text{even }\\ n > 0\endSb
 \frac{[(2n -3)(2n - 7) \dots 5 \cdot 1]^2 }{2^nn!}t^n \\
y_2 &= t + \sum\Sb n\ \text{odd }\\ n > 1\endSb
\frac{[(2n-3)(2n -7)\dots 3]^2}{2^{n-1}n!}t^n
\endalign$$
and
$$
y = \sum_{n=0}^\infty a_nt^n = a_0y_1(t) + a_1 y_2(t).
$$
Neither $y_1$ nor $y_2$ is a polynomial.   
 For any power series  in
$t$, the constant term is the value of the sum at $t= 0$ and the 
coefficient of $t$ is the value of its derivative at $t = 0$.
Hence,
$$
\alignat 2
   y_1(0) &= 1 &\qquad y_1'(0) &= 0 \\
   y_2(0) & = 0&\qquad y_2'(0) &= 1.
\endalignat $$
(Why?)
Hence, the Wronskian  $W(0) = 1\cdot 1 - 0\cdot 0 = 1 \not= 0$,
and it follows that the two solutions form a linearly independent
pair.   (Note also, that it is fairly clear that neither is
a constant multiple of the other since $y_1$ starts with $1$ 
and $y_2$ starts with $t$.) 
\endexample

\nextex
\example{Example \en}  Consider the equation
$y'' -2t y' + 2y = 0$.
The coefficients $p(t) = -2t$ and $q(t) = 2$ have no singularities,
so every point is an ordinary point.  To make the problem a trifle
more interesting, we find a series solution centered at $t_0 = 1$,
i.e., we try a series of the form $y = \sum_{n=0}^\infty a_n(t -1)^n$.
To simplify the algebra, we introduce a new variable $s = t - 1$.
Then, since $t = s + 1$, the equation may be rewritten
$$
y'' - 2(s + 1)y' + 2y = 0.
$$
Subtle point: 
 $y' = \dfrac{dy}{dt}= \dfrac{dy}{ds}$ and
similarly for $y''$. (Why?)   Hence, we are not begging the question
in the above equation by treating $s$ as the independent variable!

We proceed exactly as before, but we shall skip some steps.

$$
\alignat 2
y'' &= \sum_{n=2}^\infty n(n-1) a_n s^{n-2} &&=\sum_{n=0}^\infty
(n+2)(n+1)a_{n+2}s^n \\
-2sy' &= \sum_{n=1}^\infty (-2n a_n)s^{1+n-1} &&=
\sum_{n=0}^\infty(-2n a_n)s^n \\
-2y' &= \sum_{n=1}(-2na_n)s^{n-1} &&= \sum_{n=0}^\infty (-2(n+1)a_{n+1})s^n \\
2y &= &&\quad\sum_{n=0}^\infty (2a_n)s^n.
\endalignat $$
Adding up and comparing corresponding coefficients of $s^n$ yields
$$\gather
0 = (n+2)(n+1)a_{n+2} - 2na_n - 2(n+1)a_{n+1} + 2 a_n \\
(n+2)(n+1)a_{n+2} = 2(n+1)a_{n+1} + (2n -2)a_n \\
a_{n+2} = 2\frac{(n+1)a_{n+1} + (n-1)a_n}{(n+2)(n+1)}\qquad n \ge 0.
\endgather $$
In this case, we cannot separate the terms into even and odd
terms, and the general term is not so easy to determine.
Here are some of the terms
$$
\alignat 2
n &= 0&\qquad\qquad a_2 &= 2\frac 12 (a_1 - a_0) = a_1 - a_0 \\
n &=1 &\qquad\qquad a_3 &= 2\frac{2a_2 +0\cdot a_1}{3\cdot 2}
         = \frac 23(a_1 - a_0) \\
n &= 2 &\qquad\qquad a_4 &=
2\frac{3a_3 + 1\cdot a_2}{4\cdot 3} = \frac 16(2(a_1 -a_0) + (a_1 - a_0))
\\
&&&= \frac 12(a_1 - a_0) \\
n &= 3 &\qquad\qquad
a_5 &= 2\frac{4a_4 + 2a_3}{5\cdot 4} =\dots = \frac 13(a_1 - a_0)\\
&\vdots
\endalignat $$
I gave up trying to find the general terms.   The general solution
is
$$
\align
y &= \sum_{n=0}^\infty a_ns^n \\
  &= a_0 + a_1s + (a_1 - a_0)s^2 + \frac 23 (a_1 - a_0)s^3
   + \frac 12(a_1 - a_0)s^4 + \frac 13(a_1 - a_0) s^5 + \dots \\
  &=a_0
\undersetbrace{y_1}\to{(1 - s^2 -\frac 23s^3 - \frac 12s^4 - \frac 13
 s^5 - \dots)} \\
  &\qquad +a_1\undersetbrace{y_1}\to
{(s + s^2 + \frac 23 s^3 + \frac 12 s^4 + \frac 13s^5 + \dots)}.
\endalign $$
Hence, if we put back  $s = t- 1$, we may write
$$
y = a_0y_1(t) + a_1y_2(t)
$$
where
$$\align
y_1(t) &= 1 - (t-1)^2 -\frac 23(t-1)^3 - \frac 12(t-1)^4 -
 \frac 13(t-1)^5 - \dots \\
y_2(t) &= (t-1) + (t-1)^2 + \frac 23 (t-1)^3 + 
\frac 12 (t-1)^4 + \frac 13(t-1)^5 + \dots\,.
\endalign $$
\endexample

\subhead The radius of convergence of a series solution \endsubhead
An extension of the {\it basic existence and 
uniqueness theorem\/}---which we won't try to prove in this course---tells
us that a solution of
$y'' + p(t)y' + q(t)y = 0$  is analytic {\it at least\/} where
the  coefficients  $p(t)$ and $q(t)$ are analytic.   (The solution
could be
analytic on an even larger domain.)  We may use this fact to
\outind{radius of convergence of a solution}%
study the radius of convergence of a series
solution of a linear differential equation.  
Namely, in Chapter VIII, Section 8,
%Back Reference
we described a rule for determining the radius of convergence
of the Taylor series of a function:  calculate the
distance to the nearest singularity.  Unfortunately, there
was an important caveat to keep in mind.  You have to look also at
singularities in the complex plane.

\example{Example \ExOne, {\rm revisited}}  Legendre's
equation
$$
y'' - \frac{2t}{1 - t^2} y' + \frac{\alpha(\alpha +1)}{1 - t^2}y = 0
$$
has singularities when $1 - t^2 = 0$, i.e., at $t = \pm 1$.
The distance from $t_0 = 0$ to the nearest singularity is 1.
Hence, the radius
of convergence of a power series solution centered at $t_0 = 0$ is
at least 1, but it may be larger.   Consider in particular
the case $\alpha = 1$.   One of the solutions $y_2(t) = t$
is a polynomial and so it converges for all $t$.  Its radius of
converges is infinite.   The other solution is
$$
y_1(t) = 1 - \sum_{k=1}^\infty \frac{t^{2k}}{2k -1}
$$
and it is easy to check by the ratio test that its radius
of convergence is precisely 1.
\endexample

\nextex
\example{Example \en}  Consider
$$
y'' + \frac{3t}{1 + t^2}y' + \frac 1{1+ t^2} y = 0.
$$
The coefficients have singularities where $1 + t^2 = 0$,
i.e., at $t = \pm i$.  The distance from $t_0$ to
$\pm i$ in the complex plane is 1.   Hence, series solutions
of this equation centered at $t_0 = 0$ will have radius of
convergence at least 1.

(See Braun, Section 2.8, Example 2 for solutions of this equation.)
\endexample

\subhead  More complicated coefficients \endsubhead
In all the above examples, the coefficients were quite
simple.  They were what we call {\it rational functions\/},
i.e.,  quotients of polynomials.   For such functions, it is
possible to multiply through by a common denominator and thereby
deal only with polynomial coefficients.   Of course, in general
that may not be possible.  For example, for the equation
$$
y'' + (\sin t) y' + (\cos t) y = 0
$$
the simplification employed previously won't work.  However, the method
still works.  Namely, we may put the expansions
$$\align
\sin t &= \sum_{n=0}^\infty (-1)^n \frac{t^{2n+1}}{(2n+1)!} \\
\cos t &=  \sum_{n=0}^\infty (-1)^n \frac{t^{2n}}{(2n)!} 
\endalign $$
along with the expansions 
$$\align
y &= \sum_{n=0}^\infty a_n t^n \\
y &= \sum_{n=1}^\infty na_n t^{n-1} \\
y &= \sum_{n=2}^\infty n(n-1)a_n t^{n-2} 
\endalign $$
in the differential equation, multiply everything out, and
collect coefficients as before.  Although this is theoretically
feasible, you wouldn't want to try it unless it were absolutely
necessary.   Fortunately, that is seldom the case.

\bigskip
\input chap9.ex1
\bigskip
\nextsec{Partial Differential Equations}
\head \sn. Partial Differential Equations.  A Preview \endhead

You probably have learned by now that certain partial differential
equations such as Laplace's Equation or the Wave Equation govern
the behavior of important physical systems.   The solution of such
equations leads directly to the consideration of second order
linear differential equations, and it is this fact that lends
such equations much of their importance.   In this section,
we show how an interesting physical problem leads to
Bessel's equation
$$
t^2y'' + ty' + (t^2 - m^2)y = 0\qquad\text{where } m = 0, 1, 2, \dots .
$$
In the sections that follow we shall discuss methods for solving
Bessel's equation and related equations by use of infinite series.

\mar{s9-1.ps}
The physical problem we shall consider is that of determining the
possible vibrations of a circular drum head.   We model such
\outind{drum head, vibrations of}%
a drum head as a disk of radius $a$ in the $x,y$-plane  centered
at the origin.   We suppose that the circumference of the disk is
fixed, but that other points may be displaced upward or downward
in the $z$-direction.   The displacement $z$ is a function of
both
position $(x,y)$ and time $t$.  If we assume that the displacement is
small, then to a high degree of approximation, this function
satisfies the wave equation
$$
\frac 1{v^2}\frac{\d^2 z}{\d t^2} = \nabla^2 z
$$
where $v$ is a constant and $\nabla^2$ is the Laplace operator
$$
\nabla^2 = \frac{\d^2}{\d x^2} + \frac{\d^2}{\d y^2}.
$$
(You will study this equation shortly in physics if you have not
done so already.)

Because the problem exhibits circular symmetry, it is appropriate
to switch to polar coordinates, and then the Laplace operator takes
the form
$$
\nabla^2 = \frac{\d^2}{\d r^2} + \frac 1r\frac{\d }{\d r} + \frac 1{r^2}
\frac{\d^2 }{\d\theta^2}.
$$
(See the exercises for Chapter V, Section 13.)
Thus the wave equation may be rewritten
\nexteqn
\xdef\Eqn{\eqn}
$$
\frac 1{v^2}\frac{\d^2 z}{\d t^2} 
 = \frac{\d^2z}{\d r^2} + \frac 1r\frac{\d z}{\d r} + \frac 1{r^2}
\frac{\d^2 z}{\d\theta^2}.\tag\eqn
$$
Since the circumference of the disk is fixed, we must add the
{\it boundary\/} condition  $z(a, \theta, t) = 0$ for all $\theta$ and
all $t$.

A complete study of such equations will be undertaken in your course
next year
in Fourier series and boundary value problems.   For the moment
we shall consider only solutions which
can be expressed
\nexteqn
\xdef\Sep{\eqn}
$$
z = T(t) R(r) \Theta(\theta)\tag\eqn
$$
where the variables have been {\it separated\/} out in three functions,
each of which
depends only on one of the variables.   (The method employed here is
called {\it separation of variables\/}.   It is  similar in spirit to
\outind{separation of variables for partial differential equation}%
the method employed previously for ordinary differential equations,
but of course the context is entirely different, so the two methods should
not be confused with one another.)  It should be emphasized that
the general solution of the wave equation
cannot be so expressed, but as you shall see next year, it can
be expressed as a sum (usually infinite) of such functions.
The boundary condition for a separated solution in this case is
simply  $R(a) = 0$.

If we substitute (\Sep) in equation (\Eqn), the partial derivatives
become ordinary derivatives of the relevant functions and we obtain
$$
\frac 1{v^2} T''R\Theta = TR''\Theta + \frac 1r TR'\Theta +
\frac 1{r^2}TR\Theta''.
$$
Divide through by $z = TR\Theta$ to obtain
$$
\frac 1{v^2} \frac {T''}T = \frac{R''}R + \frac 1r\frac{R'}R +
\frac 1{r^2}\frac{\Theta''}\Theta.
$$
In this equation, the left hand side 
$\dfrac 1{v^2}\dfrac{T''}T$ depends only on $t$, and the right hand
side does not depend on $t$.  Hence, both equal the same constant
$\gamma$, i.e.,
$$
\gather
\frac 1{v^2} \frac {T''}T = \gamma \\
 \frac{R''}R + \frac 1r\frac{R'}R +
\frac 1{r^2}\frac{\Theta''}\Theta = \gamma.
\endgather
$$
The second of these equations may be rewritten
$$
\frac{\Theta''}\Theta = r^2(\text{an expression depending only on $r$}),
$$
so by similar reasoning, it must be equal to  a constant $\mu$.
Thus,
$$
\gather
\frac{\Theta''}{\Theta} = \mu \\
\Theta'' -\mu \Theta = 0.
\endgather
$$
This is a simple second order equation with known solutions.  If
$\mu$ is positive, the general solution has the form $C_1e^{\sqrt\mu\, \theta}
+C_2e^{-\sqrt\mu\, \theta}$.   However, the function $\Theta$ must satisfy
the {\it periodicity\/} condition
$$
\Theta(\theta + 2\pi) = \Theta(\theta)\qquad\text{for every }\theta
$$
since adding $2\pi$ to the polar angle $\theta$ does not change
the point represented.   The solution listed above does not satisfy
this condition so we conclude that $\mu \le 0$.   In that case,
the general solution is $\Theta = C_1\cos \sqrt{|\mu|}\theta
+ C_2 \sin \sqrt{|\mu|}\theta$.    Moreover, the periodicity condition
tells us that $\sqrt{|\mu|}$ {\it must be an integer}.   Thus we
may take $\mu = -m^2$ where $m$ is an integer.   We may even assume that
that $m \ge 0$ since changing the sign of $m$ makes no essential difference
in the form of the general solution
$$
\Theta = C_1\cos m\theta + C_2\sin m\theta\qquad m = 0, 1, 2, \dots\,.
$$

If we now put $\Theta''/\Theta = -m^2$ back into the separated equation,
we obtain
$$
\frac 1{v^2} \frac {T''}T = \frac{R''}R + \frac 1r\frac{R'}R +
\frac 1{r^2}(-m^2) = \gamma
$$
where $\gamma$ is a constant.   It turns out that $\gamma < 0$,
but the argument is somewhat more complicated than that given above
for $\mu$.   One way to approach this would be as follows.  The
above equation gives the following equation for $T$
$$
T'' - \gamma v^2 T = 0.
$$
We know by
observation (and common sense) that the motion of the drum head
is oscillatory.
If $\gamma > 0$, this equation has non-periodic exponential solutions
(as in the previous argument for $\Theta$). 
 Similarly, $\gamma = 0$ implies that 
$T = c_1t + c_2$, which would make sense only if $c_1 = c_2 = 0$.
That corresponds to the solution in which the drum does not vibrate
at all, and it is not very interesting.
 Hence, the only remaining possibility is
  $\gamma < 0$,  in which case we
get periodic oscillations: 
$$
T(t) = C_1\cos (\sqrt{|\gamma|}\,vt) + C_2 \sin (\sqrt{|\gamma|}\,vt).
$$

  This argument is a bit
unsatisfactory for the following reason.   We should be able to
derive {\it as a conclusion\/} the fact that the solution is
periodic in time.  After all, the purpose of a physical theory is
to predict as much as we can with as few assumptions as possible.
It is in fact possible to show that $\gamma < 0$ by another
argument (discussed in the Exercises) which only uses the fact
that the drum head is fixed at its circumference.   

Write
 $\gamma = -\lambda$ where $\lambda = |\gamma|$,
and
consider the equation for $R$
$$
\gather
\frac{R''}R + \frac 1r\frac{R'}R +
\frac 1{r^2}(-m^2) = -\lambda \\
r^2 R''  +r R' + (\lambda r^2 - m^2)R = 0
\endgather
$$
where $m = 0, 1, 2, \dots,\, \lambda > 0$, and $R$ satisfies the
boundary condition   $R(a) = 0$.   It is usual to make one last
transformation to simplify this equation, namely introduce a
new variable $s = \sqrt\lambda\, r$.    Then  if  $S(s) = R(r)$,
we have
$$
\gather
\frac {dR}{dr} =   \frac {dS}{ds}\frac{ds}{dr} = \sqrt\lambda\, \frac{dS}{ds}\\
\frac{d^2R}{dr^2} = \frac{d}{dr}\frac{dR}{dr} = \sqrt\lambda\,
  \frac{d}{dr}\frac{dS}{ds} = \sqrt\lambda\,\sqrt\lambda\,\frac{d}{ds}
\frac{dS}{ds} = \lambda \frac{d^2S}{ds^2}.
\endgather
$$
Thus
$$
\gather
\lambda r^2 S'' + \sqrt\lambda\, r S' + (\lambda r^2 - m^2)S = 0
\\
\text{or}\qquad s^2 S'' + sS' + (s^2 - m^2) S = 0.
\endgather
$$
The last equation is just {\it Bessel's Equation\/}, and we
shall see how to solve it and similar equations in the next
sections in this chapter.   Note that in terms of the function
$S(s)$, the boundary condition becomes
$$
S(\sqrt\lambda a) = 0.
$$
On the other hand, as mentioned above, $\sqrt\lambda = \sqrt{|\gamma|}$
is a factor in determining the frequency of oscillation of the
drum.   Hence, finding the roots of the equation $S(x) = 0$ is
a matter of some interest.

If we switch back to calling the independent variable $t$
and the dependent variable $y$, Bessel's  Equation takes the form
used earlier
$$
t^2 y'' + ty' + (t^2 - m^2)y = 0
$$
or
$$
y'' + \frac 1t y' + \frac{t^2 - m^2}{t^2} y = 0.
$$
Of course, $t$ here need not bear any relation to `time'.
In fact, in the above analysis, $t$ came from the polar
coordinate $r$.   That means we have the following quandary.
The value $t = 0$  (the origin in the above discussion) may
be a specially interesting point.  However, it is also a
singular point for the differential equation.   Thus,
we may have to use series solutions 
{\it centered at a singular point\/}, and that
is rather different from what we did previously for ordinary
points. 

\bigskip
\input chap9.ex2
\bigskip

\nextsec{Regular Singular Points and the Method of Frobenius}
\head \sn.  Regular Singular Points and the Method of Frobenius \endhead

The general problem we want to consider now is how to solve
an equation of the form $y'' + p(t)y' + q(t)y = 0$ by expanding
in a series {\it centered at a singular point}.   To understand
the process, we start by reviewing the simplest case which is {\it Euler's
Equation}
\outind{Euler's equation}%
$$ 
y'' + \frac \alpha t y' + \frac \beta{t^2} y = 0
$$
where $\alpha$  and $\beta$ are constants.  To solve it in the
vicinity of the singular point $t = 0$, we try a solution
of the form $y = t^r$.   Then, $y' = rt^{r-1}$ and $y'' = r(r-1)t^{r-2}$,
so the equation becomes
$$\gather
r(r-1)t^{r-2} + \alpha r \frac {t^{r-1}}t + \beta \frac{t^r}{t^2} = 0 \\
(r(r-1) + \alpha r + \beta)t^{r -2} = 0 \\
r(r-1) + \alpha r + \beta = 0 \\
r^2 + (\alpha -1)r + \beta = 0.
\endgather $$
This is a quadratic equation with two roots $r_1, r_2$, so we
get two solutions $y_1 = t^{r_1}$ and $y_2 = t^{r_2}$.
If the roots are different, these form a linearly independent pair,
and the general solution is
$$
y = C_1t^{r_1} + C_2t^{r_2}.
$$
If the roots are equal, i.e., $r_1 = r_2 = r$, then $y_1 = t^r$
is one solution, and we may
use reduction of order to find another solution.  It turns
out to be  $y_2 = t^r\ln t$.  (See the Exercises.)
     The general solution is
$$
y = C_1t^r + C_2 t^r\ln t.
$$

There are a couple of things to notice about the above process.
First, the method depended critically on the fact that $t$ occurred
precisely to the right powers in the two denominators.  Otherwise,
we might not have ended up with the common factor $t^{r-2}$. 
Secondly, the solutions of the quadratic equation need not
be positive integers; they could be negative integers, fractions,
or worse.  In such cases $t^r$ may exhibit some singular behavior
at $t = 0$.   This will certainly be the case if $r < 0$, since
in that case $t^r = 1/t^{|r|}$ blows up as $t \to 0$.  If $r >0$
but $r$ is not an integer, then $t^r$ is continuous at $t = 0$, but
it may have discontinuous derivatives of some order.  For example,
if $y = t^{5/3}$, then $y' = (5/3)t^{2/3}$ and $y'' = (10/9)t^{-1/3}$,
which is not continuous at $t = 0$.  Hence, the singularity at
$t = 0$ in the differential equation tends to show up in some way
in the solution.  (In general, the roots $r$ could even be complex
numbers, which complicates the matter even more.  In this course,
we shall ignore that possibility.)

Consider now the general equation
$$
y'' + p(t) y' + q(t) y = 0,
$$
and suppose $p(t)$ or $q(t)$ is singular at $t = 0$.  We say
that $t = 0$ is a {\it regular singular point\/} if we
\outind{regular singular point}%
can write
$$
    p(t) = \frac {\overline p(t)}t\qquad q(t) = 
\frac{\overline q(t)}{t^2}
$$
where $\overline p(t)$ and $\overline q(t)$ are {\it analytic\/}
in the vicinity of $t = 0$.   This means that the differential equation
has the form
$$\gather
y'' + \frac {\overline p(t)}t y' + \frac{\overline q(t)}{t^2} y = 0 \\
\text{or}\qquad t^2 y'' + t\,\overline p(t)y' + \overline q(t) y = 0.
\endgather $$
(This is what we get if we replace the constants in Euler's Equation
by analytic functions.)

	More generally, we say that $t = t_0$ is a regular singular
point of the differential equation if it may be rewritten
$$\gather
y'' + \frac {\overline p(t)}{t-t_0} y' + \frac{\overline q(t)}{(t-t_0)^2} y = 0 \\
\text{or}\qquad (t- t_0)^2 y'' + (t- t_0)\,\overline p(t)y'
 + \overline q(t) y = 0
\endgather $$
where $\overline p(t)$ and $\overline q(t)$ are analytic functions
in the vicinity of $t = t_0$.

\nextex
\example{Example \en}  Bessel's Equation
$$
y'' + \frac 1t y' + \frac{t^2 - m^2}{t^2} y = 0
$$
has a regular singular point at $t = 0$.
\outind{Bessel's equation}%
\endexample

\nextex
\example{Example \en}  Legendre's Equation
$$
y'' - \frac {2t}{1 - t^2} y' + \frac{\alpha(\alpha +1)}{1 - t^2} y = 0
$$
has regular singular points both at $t = 1$ and $t = -1$.
\outind{Legendre's equation}%

For, at $t = 1$, we may write
$$\align
p(t) &= \frac{-2t}{1-t^2} = \frac{2t}{(t -1)(t +1)}
   = \frac{2t/(t + 1)}{t - 1} \\
q(t) &= \frac{\alpha(\alpha +1)}{1 -t^2}
      = \frac{-\alpha(\alpha +1)}{(t -1)(t+1)}
=  \frac{-\alpha(\alpha +1)(t -1)/(t +1)}{(t -1)^2}
\endalign $$
and
$$\align
\overline p(t) &= \frac {2t}{t + 1} \\
\overline q(t) &= \frac{-\alpha(\alpha +1)(t-1)}{t + 1}
\endalign $$
are analytic functions near $t = 1$.  (They are of course singular
at $t = -1$, but that is far enough way not to matter.)

 A similar argument which reverses the roles of $t -1$ and $t +1$
shows that $t = -1$ is also a regular singular point.
\endexample

\nextex
\example{Example \en}  The equation
$$
y'' - \frac 2{t^2}y' + 5 y = 0
$$
has an {\it irregular\/} singular point at $t = 0$.  In this case,
the best we can do with $p(t)$ is
$$
  \frac{-2}{t^2} = \frac {-2/t}t
$$
and $-2/t$ is certainly not analytic at $t = 0$.

To solve an equation with a regular singular point at $t = t_0$, we allow
for the possibility that the solution is singular at $t = t_0$,
but we hope that the singularity won't be worse than a negative or
fractional power of $t - t_0$, as in the case of Euler's Equation.
 That is, we try for a solution of the
form
$$
  y = (t - t_0)^r\sum_{n=0}^\infty a_n(t -t_0)^n.
$$
Since the power series is analytic, this amounts to trying
for a solution of the form  $y = (t - t_0)^r g(t)$ where $g(t)$
is analytic near $t_0$.   This method is called the
{\it method of Frobenius}.
\outind{method of Frobenius}%
\outind{Frobenius, method of}%

There is one technical problem with the method of Frobenius.
If $r$ is not an integer, then {\it by definition\/}
$(t - t_0)^r = e^{r\ln(t - t_0)}$.  Unfortunately,
$\ln(t - t_0)$ is undefined for $t - t_0 < 0$.  Fortunately,
since $t = t_0$ is a singular point of the differential equation,
one is usually interested either in the case $t > t_0$ or
$t < t_0$, but one does not usually have to worry about going
from one to the other.   We shall concentrate in this course
on the case $t > t_0$.   For the case $t < t_0$, similar
methods work except that you should use $|t - t_0|^r$ instead
of $(t - t_0)^r$.

\subhead The Method of Frobenius for Bessel's Equation \endsubhead

We want to solve
$$
t^2 y'' + t y' + (t^2 - m^2) y = 0
$$
\outind{Bessel's equation}%
near the regular singular point $t_0 = 0$.  We look for a 
solution defined for $t > 0$.  Clearly, we may assume
$m$ is non-negative since its square is what appears in the equation.
In interesting applications
$m$ is a non-negative integer, but for the moment we make no
assumptions about $m$ except that $m \ge 0$.  
 The method of Frobenius suggests
trying
$$
y = t^r\sum_{n=0}^\infty a_nt^n = \sum_{n=0}a_nt^{n+r}.
$$
Note that we may assume here that $a_0 \not = 0$ since if
it were zero that would mean the series would start with
a positive power of $t$ which could be factored out from each
term and absorbed in $t^r$ by increasing $r$.

As before, we calculate 
$$y' = \sum_{n=0}^\infty (n+r)a_n t^{n+r -1}
\quad\text{and  }y'' = \sum_{n=0}^\infty (n+r)(n+r -1)a_n t^{n+r -2}.$$
(Note however that we can't play any games with the lower index
as we did for ordinary points.)   Thus,
$$\align
t^2y'' &= \sum_{n=0}^\infty (n+r)(n+r -1)a_nt^{n+r} \\
ty' &= \sum_{n=0}^\infty (n+r)a_n t^{n+r}\\
t^2y &= \sum_{n=0}^\infty a_nt^{n+r+2} = \sum_{n-2 = 0}^\infty
  a_{n-2}t^{n+r} \\
   &= \sum_{n=2}^\infty a_{n-2}t^{n+r} \\
-m^2y &= \sum_{n=0}^\infty (-m^2a_n)t^{n+r}.
\endalign$$
Note that after the adjustments, one of the sums starts at $n = 2$.
That means that when we add up the terms for each $n$, we have to consider
those for $n = 0$ and $n = 1$ separately since they don't involve
terms from the aforementioned sum.   Thus, for $n = 0$, we get
\nexteqn
$$
0 = r(r-1)a_0 + ra_0 - m^2a_0
$$
while for $n = 1$, we get
$$
0 = (r+1)r a_1 + (r +1)a_1 - m^2 a_1.
$$
The general rule starts with $n = 2$
$$
0 = (n+r)(n+r -1)a_n + (n+r)a_n + a_{n-2} - m^2a_n\qquad\text{for } n\ge 2.
$$
These relations may be simplified.  For, $n = 0$, we get
$$
(r^2 - m^2)a_0 = 0.
$$
However, since by assumption $a_0 \not=0$, we get
$$
r^2 - m^2 = 0.
$$
This quadratic equation in $r$ is called the {\it indicial equation}.
\outind{indicial equation}%
In this particular case, it is easy to solve
$$
r = \pm m.
$$
(Note that if $m = 0$, this is a double root!)
For $n = 1$, (using $r^2 = m^2$) we get
\nexteqn
\xdef\NOne{\eqn}
$$
((r + 1)^2 - m^2)a_1 = (2r +1)a_1 = 0\tag\eqn
$$ 
Except in the case $r = -1/2$, this implies that $a_1 = 0$.
Finally, for $n \ge 2$, the coefficient of $a_n$ is
$(n+r)(n+r-1) + (n+r) - m^2 = (n+r)^2 - m^2$, so we get 
\nexteqn
\xdef\NHigher{\eqn}
$$
[(n + r)^2 - m^2]a_n + a_{n-2} = 0.\tag\eqn
$$

We shall now consider separately what happens for each root
$r = \pm m$  of the indicial equation.  (Some people prefer
to see how far they can get
without specifying $r$, and then they put $r$ equal
to each of the roots when they can't proceed further.)   
\medskip
\subhead Solution of Bessel's Equation for the Positive Root of
the Indicial Equation \endsubhead
Take $r = m \ge 0$.
Then, $r \not= -1/2$, so $a_1 = 0$.   For $n \ge 2$, the
coefficient of $a_n$ is $(n+m)^2 - m^2 = n^2 + 2nm = n(n + 2m)$, and 
we may solve
$$
a_n = -\frac{a_{n-2}}{n(n + 2m)}.
$$
This recurrence relation allows us to determine $a_n$ for all
$n > 0$.   First of all, since $a_1 = 0$, it follows that
$a_3 = a_5 = \dots = 0$, i.e., all odd numbered coefficients are
zero.   For $n$ even, we have
$$
\alignat 2
n &= 2&\qquad a_2 &= -\frac{a_0}{2(2 + 2m)} = -\frac{a_0}{2^2(m + 1)} \\
n &= 4 &\qquad a_4 &= -\frac{a_2}{4(4 + 2m)} = \frac{a_0}{2^5(m+2)(m+1)}\\
n &= 6 &\qquad a_6 &= -\frac{a_4}{6(6 + 2m)} = 
-\frac{a_0}{3\cdot 2^7(m+3)(m+2)(m+1)}\\
n &= 8 &\qquad a_8 &= -\frac{a_4}{6(6 + 2m)} = 
\frac{a_0}{4\cdot3\cdot 2^9(m+4)(m+3)(m+2)(m+1)}\\
&&&=\frac{a_0}{4! 2^8(m+4)(m+3)(m+2)(m+1)}\\
&\vdots
\endalignat
$$
The general rule may be written
$$
a_{2k} = (-1)^k\frac{a_0}{k!2^{2k}(m+k)(m+k-1)\dots (m+2)(m+1)}
\qquad k = 0, 1, 2, \dots
$$
Note the quantity $(m +k)(m +k-1)\dots (m+2)(m+1)$  is similar to a
factorial in which each term has an extra addend $m$.  If we interpret
this product to be 1 if $k=0$, and recall that $0! = 1$, then the
formula is valid, as indicated, for $k = 0$.

The corresponding series solution is
$$
y = t^m\sum_{n=0}^\infty a_n t^n = a_0t^m\sum_{k=0}^\infty
\frac{(-1)^k}{k!2^{2k}(m+k)(m+k-1)\dots (m+2)(m+1)}t^{2k}.
$$
Note that since $m \ge 0$, this solution is actually continuous
at $t = 0$.  (It is the product of the continuous function $t^m$
with the sum of a power series, i.e., an analytic function.)   If
$m$ is a non-negative integer, the solution is even analytic
at $t = 0$.  If $m$ is positive but not an integer, the solution
is definitely not analytic since a sufficiently high derivative
of $t^m$ will involve a negative power and so fail to exist at
$t = 0$.   (A function which is analytic at $t = 0$ has derivatives
of every order at $t = 0$. They are just the coefficients (except
for factorials) of
its Taylor series centered at $t = 0$.)

The constant $a_0$ is arbitrary and determined by initial conditions.
However, we may pick out one specific solution and any other such
solution will be a constant multiple of it.  It is often useful to adjust
the constant $a_0$ for the distinguished solution so that the formulas
work out nicely.   If $m$ is a non-negative integer
the most common choice is  $a_0 = 1/(2^mm!)$.   This yields
$$
\align
y &= \frac 1{2^mm!}t^m\sum_{k=0}^\infty
\frac{(-1)^k}{2^{2k}k!(m+k)(m+k-1)\dots (m+2)(m+1)}t^{2k} \\
&= \sum_{k=0}^\infty \frac{(-1)^k}{2^{2k+m}k!(m+k)!}t^{2k + m} \\
&= \sum_{k=0}^\infty \frac{(-1)^k}{k!(m+k)!}\left(\frac t2\right)^{2k + m}.
\endalign
$$
For $m$ a non-negative integer,
this solution is   called a {\it Bessel Function of the first kind\/},
\outind{Bessel function}%
and it is denoted
$$
J_m(t) = 
\sum_{k=0}^\infty \frac{(-1)^k}{k!(m+k)!}\left(\frac t2\right)^{2k + m}.
 $$
\outind{$J_m(t)$}%
The ratio test shows that the series converges for all $t$, so 
its sum is an analytic function for all $t$.
Two interesting cases are
$$\align
J_0(t) &= \sum_{k=0}^\infty \frac{(-1)^k}{(k!)^2}\left(\frac t2\right)^{2k}\\
J_1(t) &= \sum_{k=0}^\infty \frac{(-1)^k}{(k+1)!k!}\left(\frac t2\right)^{2k+1}.
\endalign$$

If $m > 0$ but $m$ is not an integer, everything above works except
that the resulting function is not analytic, but as mentioned above,
it is bounded and continuous as $t \to 0$.   The choice of the constant
$a_0$ for a distinguished solution is trickier.  The term $2^m$
poses no problems, but we need an
analogue of $m!$ for $m$ a positive real number which is not
an integer.   It turns out that there
is a real valued function $\Gamma(x)$ of the real variable $x$, which
is even analytic for non-negative values of $x$, and which satisfies
the rules
\nexteqn
$$
  \Gamma(x+1) = x\Gamma(x)\qquad \Gamma(1) = 1.\tag\eqn
$$
It is called appropriately enough the {\it Gamma function}.
(See the Exercises for its definition.)
\outind{Gamma function}%
\outind{$\Gamma(x)$}%
It follows from the rules  (\eqn) that if $m$ is a positive integer,
$$
\split
\Gamma(m) = (m-1)\Gamma(m-1) = (m-1)(m-2)\Gamma(m-2) = \dots \\
 =  (m-1)\dots 2\,1\,\Gamma(1) = (m-1)!.
\endsplit
$$
This is usually rewritten with $m$ replaced by $m+1$
$$
\Gamma(m + 1) = m!\qquad\text{for } m = 0, 1, 2, \dots.
$$
You will study the Gamma function
in more detail in
 your complex variables course.

 Using the Gamma function, we may take  $a_0 = \dfrac 1{2^m\Gamma(m+1)}$.
Then,  $2^m2^{2k} = 2^{2k + m}$, and
$$
\Gamma(m+1)(m+1)(m +2)\dots (m+k) = \Gamma(m+ k + 1).
$$
So, combining $a_0$ with 
$a_{2k}$, we obtain the solution
$$\align
J_m(t) &=
\sum_{k=0}^\infty \frac{(-1)^k}{k!\Gamma(m+k+1)}
\left(\frac t2\right)^{2k + m}\\
&=\left(\frac t2\right)^m
\sum_{k=0}^\infty \frac{(-1)^k}{k!\Gamma(m+k+1)}\left(\frac t2\right)^{2k}
\endalign$$
where the fractional power has been put in front in the second equation to
emphasize the non-analytic part of the solution.

One interesting case is $m = 1/2$.
$$\align
J_{1/2}(t)
&=\frac {t^{1/2}}{2^{1/2}\Gamma(3/2)}
\sum_{k=0}^\infty 
\frac{(-1)^k}{2^{2k}k!(1/2 +k)(1/2 + k -1)\dots (1/2 +1)}t^{2k}\\
&= \frac 1{t^{1/2}2^{1/2}\Gamma(3/2)}
\sum_{k=0}^\infty (-1)^k\frac{t^{2k+1}}{2^kk!(2k +1)(2k -1)\dots 3}\\
&= \frac 1{t^{1/2}2^{1/2}\Gamma(3/2)}
\sum_{k=0}^\infty (-1)^k
\frac{t^{2k+1}}{(2k)(2k-2)\dots 4\cdot 2\,(2k +1)(2k -1)\dots 3}\\
&= \frac 1{t^{1/2}2^{1/2}\Gamma(3/2)}
\sum_{k=0}^\infty (-1)^k\frac{t^{2k+1}}{(2k+1)!}\\
&= \frac 1{t^{1/2}2^{1/2}\Gamma(3/2)}\sin t.
\endalign $$
However,  $\Gamma(3/2) = (1/2)\Gamma(1/2)$, and it
may be shown that $\Gamma(1/2) = \sqrt \pi$.  Thus, after
some algebra,
we get
$$
J_{1/2}(t) = \sqrt{\frac 2\pi}\,\frac{\sin t}{\sqrt t}.
$$

\bigskip
\subhead  Solution of Bessel's Equation for the Negative Root
of the Indicial Equation \endsubhead
Suppose $m > 0$.
By considering the positive root $r = m$  of the indicial equation
$r^2 - m^2 = 0$,
we found one solution of Bessel's Equation.
We now attempt to find a second linearly independent solution
by considering the 
negative root $r = -m$. 
\outind{Bessel's equation}%
For this root, 
 equation (\NOne) for $n=1$ becomes
$$
(2r +1)a_1 = (1 - 2m)a_1 = 0,
$$
which, as earlier, implies that $a_1 = 0$ except
in the case $m = 1/2, r = -1/2$.  However, even in that case   
we need only find one additional solution, so in any event
  we shall concentrate
on the {\it even numbered\/} coefficients.

For $n \ge 2$,  equation (\NHigher) becomes
$$
(n^2 + 2nr)a_n + a_{n-2} = (n^2 - 2nm)a_n + a_{n-2} = 0
$$
which may be solved to obtain
$$
 a_n = \frac{-a_{n-2}}{n(n - 2m)}\qquad\text{for } n \ge 2, 
$$
{\it provided\/} $n - 2m \not= 0$.   Thus, we may use the
recurrence relation to generate coefficients 
(for $n$ even)
 for a second
solution as long as $m \not= n/2$, i.e., $m$ {\it is not an integer}.
     Assume
that is the case.  Then we get
$$
\alignat 2
n&=2 &\qquad a_2 &= -\frac{a_0}{2(2 - 2m)} = -\frac{a_0}{2^2(1 - m)} \\
n&=4 &\qquad a_4 &= -\frac{a_2}{4(4 - 2m)} = \frac{a_0}{2^42(1-m)(2-m)} \\
n&=6 &\qquad a_6 &= -\frac{a_4}{6(6 - 2m)}
  = -\frac{a_0}{2^63\cdot 2 (1 - m)(2-m)(3 -m)}\\
&&\vdots
\endalignat$$
Putting $n = 2k$, we get the general rule
$$
a_{2k} = \frac{(-1)^ka_0}{2^{2k}k!(1-m)(2-m)\dots (k-m)}\qquad k \ge 0.
$$
The corresponding solution is
$$
y = a_0t^{-m}\sum_{k=0}^\infty \frac{(-1)^ka_0}{k!(1-m)(2-m)\dots (k-m)}
\left(\frac t2\right)^{2k}.
$$
If $m > 0$ is not a positive integer, we may set
$a_0 = \dfrac 1{2^{-m}\Gamma(-m + 1)}$.  That is simiilar
to what we did above, except that $m$ is replaced by $-m$.
The resulting solution is 
$$
J_{-m}(t) = \sum_{k=0}^\infty \frac{(-1)^k}{k!\Gamma(-m + k + 1)}
\left (\frac t2\right )^{2k -m}.
$$
 Note that this solution has a singularity for
$t=0$ because of the common factor $t^{-m}$, so it is certainly not
a constant multiple of $J_m(t)$.   Hence, we have a linearly
independent pair of solutions and we conclude that
{\it if $m$ is not an integer\/},  the general solution of
Bessel's equation is
$$
y = C_1J_m(t) + C_2J_{-m}(t).
$$

The case $m = 1/2$ is interesting.  
The series can be rewritten
$$\align
J_{-1/2}(t)&= \frac 1{2^{-1/2}\Gamma(1/2)}\sum_{k=0}^\infty
(-1)^k\frac 1{k!(1 -1/2)(2 - 1/2)\dots (k - 1/2)2^{2k}}t^{2k- 1/2} \\
J_{-1/2}(t)&= \sqrt{\frac 2\pi}\frac 1{\sqrt t}\sum_{k=0}^\infty
(-1)^k\frac 1{k!(1 -1/2)(2 - 1/2)\dots (k - 1/2)2^{2k}}t^{2k} \\
&= \sqrt{\frac 2\pi}\frac 1{\sqrt t}\sum_{k=0}^\infty
(-1)^k\frac {t^{2k}}{2^kk!(2 -1)(4 - 1)\dots (2k - 1)}t^{2k}\\
&= \sqrt{\frac 2\pi}\frac 1{\sqrt t}\sum_{k=0}^\infty(-1)^k\frac {t^{2k}}{(2k)!} \\
&= \sqrt{\frac 2\pi} \frac{\cos t}{\sqrt t}.
\endalign
$$
\medskip
Note that we still have to deal with the case that 
$m$ is an integer since the above method breaks down.
\bigskip
\input chap9.ex3
\bigskip
\nextsec{The Method of Frobenius.  General Theory}
\head \sn.  The Method of Frobenius.  General Theory \endhead

In the previous section, we applied the method of Frobenius to
Bessel's Equation.   For some cases ($m$ not an integer), the
method gave a complete solution, but for other cases ($m$ a 
non-negative integer),
it gave only one solution.   In the `bad' cases, we need another
method to find a linearly independent pair of solutions from which
we can form a general solution of the differential equation.
\outind{method of Frobenius}%

Before attempting to deal with the `bad' cases, we should discuss
how the method of Frobenius works for other differential equations.

\nextex
\example{Example \en}  We shall try to solve
$$
t^2y'' + 2ty' - (2 +t)y = 0\qquad\qquad t > 0
$$
by a series of the form $y = t^r\sum_{n=0}^\infty a_nt^n
= \sum_{n=0}^\infty a_nt^{n+r}$.
Calculating as previously, we have
$$\align
t^2y'' &= \sum_{n=0}^\infty (n+r)(n+r-1)a_nt^{n+r}\\
2ty' &= \sum_{n=0}^\infty 2(n+r)a_n t^{n+r} \\
-2y  &= \sum_{n=0}^\infty (-2a_nt^{n+r} \\
-ty &= \sum_{n=0}^\infty (-a_n)t^{n+r+1} = \sum_{n-1=0}^\infty
(-a_{n-1})t^{n-1+r+1} \\
    &= \sum_{n=1}^\infty (-a_{n-1})t^{n+r}.
\endalign$$
Adding up corresponding powers of $t$ yields for $n=0$
\nexteqn
\xdef\ZeroEq{\eqn}
$$
[r(r-1) + 2r - 2]a_0 = 0\tag\eqn
$$
and for $n\ge 1$
\nexteqn
\xdef\HigherEq{\eqn}
$$
[(n+r)(n+r-1) + 2(n+r) - 2]a_n - a_{n-1} = 0.\tag\eqn
$$
Since $a_0\not=0$, (\ZeroEq) yields the {\it indicial equation}
\nexteqn
\xdef\IndEq{\eqn}
$$
f(r) = r(r-1) + 2r - 2 = r^2 + r - 2 = (r -1)(r +2) = 0.
$$
Note also that for $n \ge 1$, (\HigherEq) has the form
$$
f(n+r)a_n - a_2 = 0
$$
where
$$
f(n+r) = (n+r)(n+r-1) + 2(n+r) - 2 = (n+r)^2 + (n+r) - 2 = (n+r-1)(n+r+2).
$$
(You should look back at the previous section at this point to see
what happened for Bessel's equation.  The indicial equation was
$f(r) = r^2 - m^2 = 0$,
while the coefficient of $a_n$ in each of the equations for $n \ge 1$ was 
$f(r) = (n+r)^2 - m^2$.)

The roots of the indicial equation are  $r = 1$ and $r = -2$.   Consider
first $r = 1$.  For $n \ge 1$,  $f(n+1) = n(n+3)$
so (\HigherEq) becomes
$$
 n(n+3)a_n  - a_{n-1} = 0
$$
which may be solved to obtain the recurrence relation
\nexteqn
$$
a_n = \frac{a_{n-1}}{n(n+3)}\qquad\text{for } n \ge 1.\tag\eqn
$$
  Note that there is no problem with this relation since
the denominator never vanishes.   This is not an accident.  We shall
see below why it happened.

Clearly, this recurrence relation may be used to determine all the
coefficients $a_n$ in terms of $a_0$.  (We leave it to the student
to actually do that.)  Thus, we obtain one solution of the differential
equation which is in fact uniquely determined except for the
{\it non-zero\/}
 multiplicative
constant $a_0$.

Consider next the root $r = -2$ of the indicial equation.   For $n\ge 1$,
$f(n-2) = (n-3)n$, so (\HigherEq) becomes
$$
(n-3)na_n = a_{n-1}.
$$
This can be solved for $n =1$ and $n = 2$ to obtain
\nexteqn
$$\align
a_1 &= \frac{a_0}{(-2)1} = -\frac{a_0}2 \\  
a_2 &= \frac{a_1}{(-1)2} = \frac{a_0}4. \tag\eqn
\endalign$$
However,  for $n = 3$ we encounter
a problem.   The equation becomes
$$
0\cdot a_3 = a_2
$$
which is consistent only if $a_2 = 0$, and by (\eqn) that contradicts
the assumption
that $a_0 \not= 0$.    
Hence, for the root $r = -2$ of the indicial equation, the process
breaks down.
\endexample

Let's see what was going on both in our discussion of Bessel's equation
and in the last example.   In each case, we had a quadratic indicial equation
$$
f(r) = 0.
$$
In addition, we had for $n \ge 1$ equations of the form
$$
f(n+r)a_n = \text{an expression involving $a_j$ with $j < n$}.
$$
(The expression on the right might be zero
as for Bessel's equation with $n=1$.) 
 Suppose
$r_1 \ge r_2$ are the two roots of the indicial equation.   If
$n \ge 1$, it can never be the case that $f(n+r_1) = 0$ since the
only other root of the equation $f(r) = 0$  is $r_2$ which is not
larger than $r_1$.   Hence, we may always solve the above equation
to obtain recurrence relations
$$
a_n = 
\frac{\text{expression involving $a_j$ with $j < n$}}{f(n +r_1)},\qquad n\ge 1.
$$

For the other root, $r_2$ the situation is more complicated.   If
$r_2 + n$ is never a root of the quadratic equation
$f(r) = 0$, the above reasoning applies and we obtain recurrence
relations
$$
a_n =
\frac{\text{expression involving $a_j$ with $j < n$}}{f(n +r_2)},\qquad n\ge 1.
$$
Thus, we obtain a second solution, and it is not hard to see
that the two solutions form a linearly independent pair.  (See
the appendix to this section.)

There is the possibility, however, that for some integer $n = k$,
$r_2 + k = r_1$ is the {\it larger\/} root of the 
indicial equation, i.e., $f(r_2 + k) = 0$.
In that case, the $k$th recursion relation becomes
$$
 0 = f(k+r_2)a_k = \text{an expression involving $a_j$ with $j < k$},
$$
so the process will break down 
{\it unless
we are incredibly lucky and the expression on the right happens to
be zero.}   In that case, we can seize on our great
fortune, and set $a_k$ equal to any convenient value.
Usually, we just set $a_k = 0$.  In any case, the process may continue
unimpeded for $n > k$ as soon as we successfully get past the `barrier'
at $k$.

The upshot of the above analysis is 
 that the method of Frobenius
may break down in the `bad case' that
$r_1 - r_2$ is a positive integer.  Of course, if $r_1 = r_2$,
the method only gives one solution in any case.   
Hence, you must be on guard whenever {\it $r_1 - r_2$ is a
non-negative integer}.

If you look back at the previous section, you will see
that Bessel's Equation  exhibits the phenomena we just described.
\outind{Bessel's equation}%
The roots are $\pm m$, so  $m = 0$ is is the case of equal
roots,  and the method
of Frobenius only gives one solution.  If $m > 0$, the larger
of the two roots is $r_1 = m$, and this gives
a solution in any case.
The smaller of the two roots is $r_2 = -m$, and provided
$m -(-m) = 2m$ is not a positive integer, the method of
Frobenius also generates a solution for $r_2 = -m$.
On the other hand, if
 $2m$ is a positive integer, then 
the recurrence relations for $r_2 = -m$
take the form
$$\align
1(1 - 2m)a_1 &= 0 \qquad\text{for } n = 1\\
n(n - 2m)a_n &= -a_{n-2}\qquad\text{for } n \ge 2.
\endalign
$$
The first equation ($n = 1$) implies that $a_1 = 0$ except
in the case $m = 1/2$.
Even in that case, our luck holds out,
and we may take $a_1 = 0$ since the right hand side of
the equation is zero.  Similarly, if  $m = k/2$
for some odd integer $k > 1$, then  the recursion relation will
imply that $a_n = 0$ for every odd $n < k$, and the $k$th recurrence
relation will
read
$$
k(0)a_k = -a_{k-2} = 0,
$$
so we may set $a_k = 0$.  It follows that if $m$ is half an odd integer,
then we may assume all the odd numbered $a_n$ are zero.  For
 even
$n$, the coefficient $n(n -2m) = n(n - k)$
is never zero, so there is no problem determining
the $a_n$ as previously. 

 The only
remaining case is when $m$ is itself a positive integer.  In this
case, 
for $n = 2m$, the coefficient on the left $n(n -2m)$ is zero, but
the quantity $a_{n-1}$ is not,
so there is no way to recover.  Hence, we must find another
method to generate a second solution.


\subhead The General Theory \endsubhead
We shall explain here why the method of Frobenius behaves the way it
does.   This section may be omitted your first time through the material,
but you should come back a look at it after you have worked some
\outind{method of Frobenius}%
more examples.

Consider the differential equation
$$
t^2y'' + t\overline p(t) y' + \overline q (t) y = 0
$$
where $\overline p(t)$ and $\overline q(t)$ are analytic functions
in a neighborhood of $t = 0$.   (That is what it means to say
$t = 0$ is a regular singular point.)   We shall try to find a
solution of the form $y = t^r\sum_{n=0}^\infty a_nt^n$ for
$t > 0$.   (As mentioned earlier, if you want a solution for
$t < 0$, replace $t^r$ by $|t|^r$.)   Since $\overline p(t)$
and $\overline q(t)$ are analytic at $t = 0$, they have
 power series expansions
$$
\align
\overline p(t) &= p_0 + p_1t + p_2t^2 + \dots \\
\overline q(t) &= q_0 + q_1t + q_2t^2 + \dots\,.
\endalign$$
Putting these in the differential equation, one term at a time,
we have
$$\align
t^2y'' &= \sum_{n=0}^\infty a_n(n+r)(n+r-1)t^{n+r} \\
p_0ty' &= \sum_{n=0}^\infty p_0a_n(n+r)t^{n+r} \\
p_1t\,ty' =p_1t^2y' &= \sum_{n=0}^\infty p_1a_n(n+r)t^{n+r+1}
 = \sum_{n=1}p_1a_{n-1}(n+r-1)t^{n+r} \\
p_2t^2\,ty' = p_2t^3y' &= \sum_{n=0}^\infty p_2a_n(n+r)t^{n+r+2}
= \sum_{n=2}^\infty p_2a_{n-2}(n+r-2)t^{n+r} \\
&\vdots\qquad\text{sums starting with }n =3, 4, \dots \\
q_0y = \sum_{n=0}^\infty q_0a_nt^{n+r} \\
q_1ty &= \sum_{n=0}^\infty q_1a_nt^{n+r+1} =
\sum_{n=1}^\infty q_1a_{n-1}t^{n+r} \\
q_2t^2y &= \sum_{n=0}^\infty q_2a_nt^{n+r+2} =
\sum_{n=2}^\infty q_2a_{n-2}t^{n+r} \\
&\vdots\qquad\text{sums starting with } n = 3, 4,\dots \, .
\endalign$$
Adding up coefficients of corresponding powers of $t$ yields for
$n = 0$
$$
[r(r-1) + p_0r + q_0]a_0 = 0,
$$
and since by assumption $a_0 \not= 0$, we get the indicial equation
$$
f(r) = r(r-1) + p_0r + q_0 = 0.
$$
(Notice the similarity to the equation obtained for Euler's equation.)
For $n \ge 1$, we obtain equations of the form
$$
[(n+r)(n+r -1) + p_0(n+r) + q_0]a_n + \text{lesser numbered terms}= 0.
$$
The coefficient of $a_n$ will always be
$$
f(n+r) = (n+r)(n+r -1) + p_0(n+r) + q_0,
$$
and the additional terms will depend in general on the exact nature of the
coefficients $p_1, p_2, \dots, q_1, q_2, \dots$.   (You should work out
the cases $n = 1$ and $n=2$ to make sure you understand the argument!)

The above calculation justifies {\it in general\/}
 the conclusions drawn earlier by
looking at examples.   However, there is still one point that has not
been addressed.  Even if the method unambiguously generates the coefficients
of the series (in terms of $a_0$), it won't be of much use unless that
series has a positive radius of convergence.  Determining the radius
of convergence of the series generated by the method of Frobenius 
in any specific case is
usually not difficult, but showing that it is not
zero in  general is hard.   We shall leave that for you to investigate
by yourself at a future date if you are sufficiently interested.
{\it Introduction to Differential Equations\/} by Simmons has a good
treatment of the question.
\bigskip
\subhead Appendix.  Why the solution pair $\{y_1, y_2\}$  is linearly
independent when $r_1 - r_2$ is not an integer \endsubhead
We have
$$
y_1 = t^{r_1}\,g_1(t)\qquad\text{and}\qquad y_2 = t^{r_2}\,g_2(t)
$$
where $g_1(t)$ and $g_2(t)$ are the sums of the series obtained
in each case.  Since by assumption, the leading coefficient $a_0
\not= 0$, neither of these functions vanishes.  It follows that
the quotient of the solutions has the form
$$
\frac {y_1}{y_2} = \frac{t^{r_1}g_1(t)}{t^{r_2}g_2(t)}
 = t^{r_1 - r_2}g(t)
$$
where $g(t) = g_1(t)/g_2(t)$ is a quotient of two analytic functions,
neither of which vanishes at $t = 0$, so $g(t)$ is also analytic
at $t= 0$ and does not vanish there.  If this were constant,
we could write
$$
t^{r_1 - r_2} = \frac c{g(t)}
$$
and since $g(0)\not= 0$, the right hand side is analytic.
  On the other hand, if $r_1 - r_2$ is not
a positive integer, $t^{r_1 - r_2}$ is definitely not analytic.
(If it is a positive integer, then it vanishes at $t = 0$,
but the right hand side does not.)

\bigskip
\input chap9.ex4
\bigskip

\nextsec{Second Solutions in the Bad Cases}
\head \sn. Second Solutions in the Bad Cases \endhead

We consider next what to do when the method of Frobenius
fails to produce a second solution.  We suppose that we
\outind{method of Frobenius}%
are working at $t = 0$ which is assumed to be a regular
singular point.   It is clear how to modify the formulas and
arguments for an arbitrary regular singular point $t_0$.
\medskip

\subhead  The Case of Equal Roots \endsubhead
Suppose first that the indicial equation
$$
  f(r) = 0
$$
has a double root $r$.   Let $y_1(t)$ denote the solution obtained
\outind{method of Frobenius, equal roots}%
by the method of Frobenius for this root.   Then, we may apply
the method of reduction of order to obtain a second solution.
\outind{reduction of order}%
It turns out that this second solution always has the form
\nexteqn
\xdef\EqRootSol{\eqn}
$$
y_2(t) = y_1(t)\ln t + \sum_{n=1}^\infty c_nt^{n+r}.\tag\eqn
$$
Note that the summation starts with $n=1$.
You should compare (\eqn)  with the second solution for Euler's equation
in the
case of a double root.   In that case, we had $y_1 = t^r$ and
$y_2 = t^r\ln t = y_1\ln t$, so the logarithmic term is not a
complete surprise.

   We shall see later how the method of reduction of order leads
to such a solution, but first let's look at an example.

\nextex
\example{Example \en}  Consider Bessel's equation for $m = 0$
$$
t^2y'' + ty' + (t^2 - 0^2)y = t^2y'' + ty' + t^2y = 0.
$$
$r = 0$ is a double root and the first solution is given by
$$
y_1 = J_0(t) = \sum_{k=0}^\infty (-1)^k\frac{1}{(k!)^2}\left(\frac t2
\right)^{2k}.
$$ 
As suggested above, let's try a second solution of the form
$$
y = y_1(t)\ln t + \sum_{n=1}^\infty c_n t^n.
\outind{Bessel's equation}%
$$
Then
$$\align
y' &= y_1'(t)\ln t + \frac{y_1(t)}t + \sum_{n=1}^\infty nc_n t^{n-1} \\
y'' &= y_1''(t)\ln t + 2\frac{y_1'(t)}{t}
 - \frac{y_1(t)}{t^2} + \sum_{n=2}^\infty n(n-1)c_nt^{n-2}.
\endalign$$
Thus, renumbering as needed, we get
$$
\alignat 4
t^2y''&= t^2 y_1''\ln t\;  &+\quad &2ty_1'\; &-\quad &y_1\; &+\quad &\sum_{n=1}^\infty n(n-1)c_nt^n  \\
  ty' &= ty_1'\ln t      &\quad  &\quad  &+\quad &y_1\; &+\quad &\sum_{n=1}^\infty nc_n t^n \\
t^2y  &= t^2y_1\ln t     &\quad   &\quad &\quad  &{}  &+\quad &\sum_{n=3}^\infty c_{n-2}t^n
\endalignat$$

Add this all up to get zero.   The right side includes the terms
$$
t^2y_1'' \ln t + ty_1' \ln t + t^2 y_1 \ln t
 = (t^2y_1'' + ty_1' + t^2 y_1)\ln t = 0
$$
since $y_1$ is a solution of the differential equation.   The
terms $-y_1$ and $+y_1$ cancel, so we are left only with
$2ty_1'(t)$ and the summation terms.  The coefficient of $t^n$
for $n \ge 3$ is 
$$
n(n-1)c_n + nc_n + c_{n-2} = n^2c_n + c_{n-2},
$$
but we also have two additional terms, those for $n = 1$ and
$n = 2$.   Putting this all together, we get
$$
2ty_1'(t) + c_1t + 4c_2t^2 + \sum_{n=3}^\infty[n^2c_n + c_{n-2}]t^n
 = 0,
$$
or
$$
c_1t + 4c_2t^2 +  \sum_{n=3}^\infty[n^2c_n + c_{n-2}]t^n
 = -2ty_1'(t).
$$
However, we may evaluate $ty_1'(t)$ without too much difficulty.
$$\align
y_1 &= \sum_{k=0}^\infty (-1)^{k}\frac 1{(k!)^2 2^{2k}}t^{2k}\\
\intertext{so}
-2ty_l' &= -2t\sum_{k=1}^\infty (-1)^{k}\frac 1{(k!)^2 2^{2k}}2kt^{2k-1}\\
      &=\sum_{k=1}^\infty (-1)^{k+1}\frac 1{k!(k-1)! 2^{2k-2}}t^{2k}.
\endalign$$
Thus,
$$
c_1t + 4c_2t^2 +  \sum_{n=3}^\infty[n^2c_n + c_{n-2}]t^n
=\sum_{k=1}^\infty (-1)^{k+1}\frac 1{k!(k-1)! 2^{2k-2}}t^{2k}.
$$
Now compare corresponding powers of $t$ on the two sides, and
remember that only even powers appear on the right.
For $n = 1$, there are no terms on the right, so
$$
c_1 =  0.
$$
For $n =2, k = 1$, we have
$$\align
4c_2 &= +\frac 1{1!0!2^0} = 1\\
c_2 &= \frac 14. 
\endalign$$
For $n = 3$, we have
$$\align
9c_3 + c_1 &= 0 \\
c_3 &= 0.
\endalign$$
For $n = 4, k = 2$, we have
$$\align
16c_4 + c_2 &= - \frac 1{2!1! 2^2} = 1\frac 1{8} \\
c_4 &= -\frac 1{16}(\frac 14+ \frac 18) = -\frac 1{64}(1 + \frac 12).
\endalign$$
This process may be continued indefinitely to generate
coefficients.   Clearly, all the odd numbered coefficients
will end up being zero.   The general rule for the even numbered
coefficients is not at all obvious.  It turns out to be
$$
c_{2k} = (-1)^{k+1}\frac 1{(k!)^22^{2k}}\left(1 + \frac 12 + \frac 13
+ \dots +\frac 1k\right)\qquad k \ge 1.
$$
Hence, the second solution of Bessel's equation with $m = 0$
is
$$
y_2 = J_0(t)\ln t + \sum_{k=1}^\infty\frac{(-1)^{k+1}}
{(k!)^2}(1 + 1/2 + \dots + 1/k)\left(\frac t2\right)^{2k}.
$$ 
It is not hard to see that $\{y_1, y_2\}$ is a linearly
independent pair of solutions.

It should be noted that one seldom needs to know the exact form
of the second solution.   The most important thing about it
is that it is not continuous as $t \to 0$ because of the
\outind{logarithmic singularity}%
logarithmic term.   The first solution
$y_1 = J_0(t)$ is continuous.   This has the following important
consequence.   Suppose we know on physical grounds that the
solution is continuous and bounded as $t \to 0$.  Then
it follows that in
$$
y = C_1y_1(t) + C_2y_2(t)
$$
the coefficient of $y_2$ must vanish.  For example, this must
be the case for the vibrating membrane.   The displacement
must be bounded at the origin ($r = 0$), so we know the
solution can involve only the Bessel function of the first
kind.

\bigskip
\subhead   The Case of Roots Differing by a Positive Integer \endsubhead
Suppose the roots $r_1, r_2$ of the indicial equation satisfy
$r_1 - r_2 = k$ where $k$ is a positive integer.   If $y_1(t)$
is a solution obtained by the method of Frobenius for $r = r_1$,
the larger of the two roots, then the method of reduction of order
yields a second solution of the form
\outind{method of Frobenius, roots differing by an integer}%
\nexteqn
\xdef\IntDiff{\eqn}
$$
y_2(t) = ay_1(t)\ln t + t^{r_2}\sum_{n=0}^\infty c_nt^n.\tag\eqn
$$
{\it Note that the summation starts with $n = 0$.}
It is possible that the method of Frobenius works for the
smaller root $r_2$ (because crucial terms vanish),
 and in that case we would have $a= 0$ in 
(\eqn); the series in the second term is what the method of Frobenius
generates.   Otherwise, $a\not=0$, and, since we may always
adjust a solution by a non-zero multiplicative constant,
 we may take $a = 1$. 
 In any case the solution definitely exhibits a
logarithmic singularity.   In applying the method, you should
\outind{logarithmic singularity}%
first see if the method of Frobenius can be made to work for
the lesser root.  If it doesn't work (because the recursion
at some point unavoidably yields a non-zero numerator and a zero denominator),
then try a solution of the form (\IntDiff) with $a = 1$.   
This will yield a
set of recursion rules for the coefficients $c_0, c_1, c_2, \dots$
roughly as in Example \en.   To find these rules, you will have
to evaluate some expression involving the series for the first
solution $y_1(t)$ and its derivative $y'_1(t)$.   See the
Exercises for some examples.

Bessel's equation with $m > 0$  and $2m$ an
integer illustrates the above principles.   If 
$m$ is half an odd integer, then the logarithmic term is missing
and we may take
$y_2 = J_{-m}(t)$.   However, if $m$ is a positive integer,
then the logarithmic term is definitely present and
we must take
\outind{logarithmic singularity}%
\nexteqn
$$
y_2  = J_m(t)\ln t + t^{-m}\sum_{n=0}^\infty c_n t^n,\tag\eqn
$$
where the coefficients $c_n$ are determined by the method outlined
above.
Such solutions are called
{\it Bessel Functions of
the second kind}.  For technical reasons, it is sometimes
\outind{Bessel functions of 2nd kind}%
better to add to (\eqn) an appropriate
multiple of $J_m(t)$ (a solution of the first kind).   There is
one particular family of such solutions called
{\it Neumann functions\/} and denoted
\outind{Neumann functions}%
\outind{$Y_m(t)$}%
$Y_m(t)$.  We won't study these in this course, but we mention
them in case you encounter the notation.   The most important
thing about Neumann functions is that they have logarithmic
singularities.
\outind{logarithmic singularity}%
\medskip
  It is possible to see in general that
$\{y_1, y_2\}$ is a linearly independent pair of solutions,
so the general solution of the differential equation has the
form
$$
y = C_1y_1(t) + C_2y_2(t).
$$
If $r_1 > 0$, then $y_1(t) = t^{r_1}\sum_{n=0}^\infty a_nt^n$
is bounded as $t \to 0$.   Usually, the second solution $y_2$
does not have this property.  If the logarithmic term is present
or if $r_2 < 0$, $y_2$ will not be bounded as $t \to 0$.  In
those cases, if we know the solution is bounded by physical
considerations, we may conclude that $C_2 = 0$.


\bigskip
\subhead Using Reduction of Order for the Second Solution \endsubhead
You might want to skip this section the first time through.

%%Warning to instructor.  Skip this section
There are a variety of ways to show the second solution has the
desired form in each of the `bad' cases.   We shall use the
method of reduction of order.   There is an alternate method based
\outind{reduction of order}%
on solving the recursion relations in terms of $r$ before setting
$r$ equal to either of the roots of the indicial equation.   It
is possible thereby to derive a set of formulas for the coefficients
$c_n$.   (See {\it Braun\/}, Section 2.8.3 for a discussion of this
method.)   However, no method gives a really practical approach for
finding the coefficients, so in most cases it is enough to know the
form of the solution and then try to find the coefficients by
substituting in the equation as we did above.

First, assume $r$ is a double root of the indicial equation
$f(r) = 0$ and $y_1 = t^rg(t)$ (where $g(t) =\sum_{n=0}^\infty a_nt^n$)
is a solution
obtained by the method of Frobenius.   (Note that $g(t)$ is analytic
at $t = 0$ and $g(0) \not= 0$.) 
The indicial equation has the form
$$
f(r) = r(r-1) + p_0r + q_0 = r^2 + (p_0 -1)r + q_0 = 0
$$
where $p(t) = \overline p(t)/t = (p_0 + p_1t + \dots)/t$
and $q(t) = \overline q(t)/t^2 = (q_0 + q_1t + \dots)/t^2$.
Since $r$ is double root, we have
$(p_0 -1)^2 - 4q_0 = 0$ and
$$
r = -\frac{p_0 -1}2.
$$
The method of reduction of order tells us that there is a second
solution of the form
$y_2 = y_1v$ where
$$
v' = \frac 1{y_1{}^2}e^{-\int p(t)dt}.
$$
However, as above, 
$$
\align
p(t) = \frac{\overline p(t)}t &= \frac 1t(p_0 + p_1t + p_2t^2 + \dots)\\
&= \frac{p_0}t + p_1 + p_2t + \dots + p_nt^{n-1} + \dots\,.
\endalign
$$
Hence,
$$
\int p(t)dt =  p_0\ln t + p_1t + \frac{p_2}2t^2 + \dots,
$$
so
$$
e^{-\int p(t)dt} = e^{-p_0\ln t}e^{ -p_1t - \dots} = t^{-p_0}h_1(t)
$$
where $h_1(t) = e^{-p_1t - \dots}$ is an analytic function at $t = 0$
such that $h_1(0)\not=0$.   On the other
hand
$$
\frac 1{y_1{}^2} = \frac 1{t^{2r} g(t)^2} = t^{-2r}h_2(t) 
$$
where,  since 
$g(0) \not=0$,
$h_2(t) = 1/g(t)^2$ is also analytic at $t = 0$ and $h_2(0)\not=0$.
 It follows that
$$
v' = t^{-2r}h_2(t)t^{-p_0}h_1(t) = t^{-(2r + p_0)}h_1(t)h_2(t).
$$
However,  $r = -(p_0 - 1)/2$, so $2r + p_0 = 1$.   Moreover,
$h(t) = h_1(t)h_2(t)$ is analytic and $h(0)\not= 0$, so it may be expanded in
a power series
$$
h(t) = h_0 + h_1t + h_2t^2 + \dots\,
$$
with $h_0\not=0$.  Thus,
$$\align
v' &= t^{-1}(h_0 + h_1t + h_2t^2 + \dots)\\
&= \frac{h_0}t + h_1 + \frac{h_2}t + \dots \\
v &= h_0\ln t + h_1t + \frac{h_2}2t^2 + \dots\,
\endalign$$
It follows that 
$$
y_2 = y_1v = h_0 y_1(t)\ln t + y_1(t)(h_1t + \frac{h_2}2t^2 + \dots).
$$
The solution generated by this process may always be modified by
multiplying by a constant.  Hence, since we know that $h_0 \not=0$,
we may multiply by its reciprocal, so in effect we may assume
$h_0 = 1$.   Moreover, since $y_1 = t^rg(t)$, the
second term has the form
$$
t^rg(t)(h_1t + \frac{h_2}2t^2 + \dots).
$$
Since $g(t)$ and $h_1t + \frac{h_2}2t^2 + \dots$ are analytic,
so is their product which can be expressed as a power series
$\sum_{n=0}c_nt^n$.  Moreover, since the `h series' does not
have a constant term, its sum vanishes at $t= 0$, and the
same can be said for the product.  That implies $c_0 = 0$,
i.e., the summation may be assumed to start with $n = 1$.  Thus
we see that the solution obtained by reduction of order has
the form
$$
y_2 = y_1(t)\ln t + t^r\sum_{n=1}^\infty c_n t^n
$$
as claimed. 

A very similar analysis of the process of reduction of order
works in the case $r_1 - r_2$ is a positive integer.
\bigskip
\input chap9.ex5
\bigskip

\nextsec{More about Bessel Functions}
\head \sn. More about Bessel Functions \endhead

We saw that $J_{1/2}(t) = \sqrt{2/\pi} \sin t/\sqrt t$ and
$J_{-1/2}(t) = \sqrt{2/\pi} \cos t/\sqrt t$.  In general,
it turns out that the Bessel functions $J_m(t)$
all behave like sines and cosines
with an `amplitude' $1/\sqrt t$.
\outind{Bessel function, roots of}%
\medskip
\centerline{\epsfbox{s9-2.ps}}
\medskip
To see why this is the case, put $y = u/\sqrt t$ in the
differential equation
$$
t^2y'' + ty' + (t^2 - m^2)y = 0.
$$
We have
$$\align
y' &= \frac {u'}{\sqrt t} - \frac 12 \frac u{t^{3/2}}\\
y'' &= \frac {u''}{\sqrt t} - \frac{u'}{t^{3/2}} + \frac 34 \frac u{t^{5/2}}.
\endalign $$
Hence,
$$\align
t^2y'' + ty' &+ (t^2 - m^2)y \\
 &=  t^{3/2}u'' -t^{1/2}u' + \frac 34 \frac u{t^{1/2}}
+ t^{1/2}u' - \frac 12 \frac u{t^{1/2}} +
t^{3/2}u - m^2 \frac u{t^{1/2}} \\
&= t^{3/2}u'' + t^{3/2}u + (\frac 14 - m^2)\frac u{t^{1/2}}.
\endalign $$
Divide the last expression by $t^{3/2}$.   We see thereby that if
$y$ is a solution of Bessel's equation, then $u = y\sqrt t$ satisfies
the differential equation
\nexteqn
\xdef\ModBess{\eqn}
$$
u'' + u + \frac{1/4 - m^2}{t^2}u  = 0.\tag\eqn
$$
Let $t \to \infty$.  Then $\dfrac{1/4 -m^2}{t^2} \to 0$, so for large
$t$, the equation is close to
\nexteqn
\xdef\SinEqn{\eqn}
$$
u'' + u = 0.\tag\eqn
$$
The general solution of (\eqn) is
$$
u = C_1\cos t + C_2\sin t = A\cos(t + \delta).
$$
Hence, it is {\it plausible\/} that for large $t$, the solution of
(\ModBess) should look very similar.   That means that for large
$t$,  any solution $y = u/\sqrt t$ of Bessel's equation should look
like
$$
y \approx A\frac {\cos(t + \delta)}{\sqrt t}.
$$
(It is not clear in general that the limiting behavior
of the solution of a differential equation is the same as a solution
of the limit of the differential equation, but in this particular
case, it happens to be true.)

Since $J_m(t) \approx A\dfrac{\cos t}{\sqrt t}$ for large $t$,
we would expect it to oscillate with approximate period $2\pi$.
That means that successive roots of the equation
$$
J_m(t) = 0
$$
should differ roughly by $\pi$ for large $t$.  This is indeed the
case.    As mentioned earlier, the roots of this equation are
important in determining the frequencies of the
basic vibrations of a vibrating drum (and similar physical
systems).   The first root of the equation
$J_0(t) = 0$ is approximately $2.4048$.

\bigskip
\subhead Functions Related to Bessel Functions \endsubhead
The differential equation
\nexteqn
\eqn
$$
t^2z'' + 2tz' + (t^2 - p^2)z = 0\tag\eqn
$$
often arises in solving problems with spherical
symmetry.   (Note that the coefficient of $z'$
is $2t$ rather than $t$.)
    This equation is related to Bessel's equation as
follows.  In
$$
t^2y'' + ty' + (t^2 - m^2)y = 0
$$
put $y = \sqrt t\, z$.   Then
$$\align
y' &= \sqrt t z' + \frac 12 \frac z{t^{1/2}} \\
y'' &= \sqrt t z'' + 2\frac 12 \frac {z'}{t^{1/2}} - \frac 14\frac z{t^{3/2}}.
\endalign$$
We obtain
$$
t^{5/2}z'' +t^{3/2}z' -\frac 14 t^{1/2}z
+ t^{3/2}z' + \frac 12 t^{1/2}z + (t^2 - m^2)t^{1/2}z = 0.
$$
If we divide through by $t^{1/2}$ and rearrange the terms, we obtain
$$
t^2z'' + 2tz' + (t^2 - m^2 + 1/4)z = 0.
$$
If we put $p^2 = m^2 - 1/4$, then we obtain  equation (\eqn).   It
follows that solutions of (\eqn) may be obtained by
solving Bessel's equation with $m = \sqrt{p^2 + 1/4}$.    Thus,
Bessel functions for such $m$ (where $p$ is a non-negative integer)
are often called {\it spherical Bessel functions}.
\outind{spherical Bessel function}%
\outind{Bessel function, spherical}%

\medskip
The solutions of the equation
$$
z'' + tz' -(t^2 + m^2)z = 0
$$
are called {\it modified Bessel functions\/}, and they also arise
\outind{Bessel function, modified}%
\outind{modified Bessel function}%
in important physical applications.   (Note that the coefficient
of $z$ is $-t^2 - m^2$ rather than $+t^2 - m^2$.)
  There are two methods for
obtaining solutions.   First, apply the method of Frobenius.
The process is almost identical to that in the case of the
%Exer  Method of Frobenius for modified Bessel functions
%Exer Use $it$.
ordinary Bessel functions.   The indicial equation is
also $r^2 - m^2 = 0$, and the only real difference is that
the sign $(-1)^k$ does not occur.   In particular, if $m$
is a non-negative integer, a normalized solution obtained
for the root $r = m$ is
$$
I_m(t) = \sum_{k=0}^\infty\frac 1{k!(m+k)!}\left(\frac t2\right)^{2k+m}.
$$
For $m$ a non-negative integer, a second solution is obtained for the root
$r = -m$ by reduction of order and has a logarithmic singularity.

An alternate approach to the modified Bessel functions is to replace
$t$ by $it$ in Bessel's equation where $i =\sqrt{-1}$.   Then,
we see that $I_m(t)$ and $J_m(it)$ just differ by
a multiplicative constant.   (Check that for yourself!)
You may learn more about this in your complex variables course.

\medskip
One could go on almost without end discussing properties of Bessel
functions and of the other {\it special functions\/} of mathematical
physics.   We leave such pleasures for other courses in mathematics,
physics, geophysics, etc.
\bigskip
\input chap9.ex6
\bigskip

\endchapter
\closeseg{chap9}
\enddocument
