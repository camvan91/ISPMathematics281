\documentstyle{book}
\input extra.tex
\input le2.sty
\input chap5.lnk
\input epsf.tex
\maketoctrue
\indextrue
\mardiagtrue
\openseg{chap6}
\Monograph
\topmatter
\nextchap{First Order Differential Equations}
\title\chapter{\chapno}First Order Differential Equations\endtitle
\endtopmatter
\document

\nextsec{Differential Forms}
\head \sn. Differential Forms \endhead

Differential forms are an alternate way to talk about vector fields,
and for some applications they are the preferred way.  They are
used frequently in thermodynamics and also in the study of first
order differential equations.
\outind{differential form}%
\outind{form, differential}%

We shall concentrate on the case of forms in $\R^2$.  There is
a corresponding theory for $\R^n$ with $n > 2$, but it is
much more difficult.   

Let $\F = \lb F_1, F_2 \rb$ be a
plane vector field, and let $\Cal C$ be a path in the plane.
Unless $\F$ is conservative, the line integral $W = \int_{\Cal C} \F\cdot
d\r$  is a path dependent quantity, so we might write 
symbolically $W = W(\Cal C)$.  You may also have encountered path
dependent quantities in your chemistry course when studying
thermodynamics.  Namely, if a gas satisfies an equation of
state of the form  $f(p,v,T) = 0$, it is possible to choose
two of the three variables to be independent variables and
(at least locally) solve for the third in terms of those.  
Suppose, for example, that $v, T$ are the independent variables.
%D   path in v,T space.
In this context, we can think of the path $\Cal C$ as representing
a sequence of changes of state of the gas leading from
the state characterized by $(v_1,T_1)$ to the state characterized
by $(v_2,T_2)$.   Such a sequence of changes of state will yield
a change in the {\it heat\/} $q$ in the system, 
and this change will generally depend on the path $\Cal C$.
For `infinitesimal changes' $dv, dT$, {\it the first law of
thermodynamics\/} asserts that the {\it change in\/} $q$ is given
by
$$
  du - p\,dv
$$
where $u = u(p, v, T)$ is a function of the state of the system
called its {\it internal energy}. 
 $du$ can be expressed
in terms of $dp, dv$ and $dT$, and since $dp$ can
\outind{thermodynamics}%
be expressed in terms of $dv$ and $dT$,  the  above expression
for the change in $q$ can be put ultimately in the form
$$
   M(v,T)dv + N(v,T)dT.
$$
Then the total change in the heat $q$ along the path
$\Cal C$ will be the line integral
$$
\int_{\Cal C} du -p\,dv = \int_{\Cal C} M\, dv + N\, dT. 
$$
As mentioned above, it depends on the path $\Cal C$.

\mar{s6-1.ps}
Leaving the thermodynamics to your chemistry professors, let
us consider the basic mathematical situation.  Given a pair
of functions $M(x,y), N(x,y)$, the expression
$$
    M(x,y)\,dx + N(x,y)\,dy
$$
is called a {\it differential form}. 
We shall always assume that the component functions $M$ and $N$
are as smooth as we need them to be on the domain of the form.
You can think of the form as giving the change of some quantity for
a displacement $d\r = \lb dx, dy \rb$ in $\R^2$.   Associated
to such a differential form is the vector field $\F = \lb M, N \rb$,
and the form is just the expression $\F\cdot d\r$ appearing in
line integrals for $\F$.  Similarly, given a vector field
$\F = \lb M, N \rb$, we may consider the differential
form $M\,dx + N\,dy$. Thus, the two formalisms, vector fields
and differential forms, are really just alternate
notations for the same concept. This may seem to add needless
complication, but  there are situations,
e.g., in thermodynamics, where it is easier to think
about a problem in the language of forms than it is in the
equivalent language of vector fields.

We can translate many of the notions we encountered in studying
plane vector fields to the language of differential forms.

Recall first that a plane vector field $\F = \lb M, N \rb$
is conservative if and only if it is the gradient of a scalar
function  $\F = \nabla f$.  Then  
$$
   M\,dx + N\,dy =  \F\cdot d\r = \nabla f\cdot d\r.
$$
You should recognize the expression on the right as the
{\it differential of the function\/} 
$$
df = \nabla f\cdot d\r = \frac{\d f}{\d x}dx + \frac{\d f}{\d y}dy.
$$
Thus, $\lb M, N\rb$ is conservative if and only if the
associated form $M\,dx + N\,dy$ equals the
 differential $df$ of a function.
Such forms are called {\it exact}.   For exact forms, the
\outind{exact form}%
path independence property looks particularly simple:
$$
\int_{\Cal C} M\,dx + N\,dy = \int_{\Cal C} df =
   f(\text{end of }\Cal C) - f(\text{start of }\Cal C).
$$ 

 Recall  the screening test for the field $\F = \lb M, N \rb$ 
\outind{screening test, two dimensions}%
\nexteqn
$$
\frac{\d N}{\d x} = \frac{\d M}{\d y}\tag\eqn
$$
which in some (but not all) circumstances will tell us if the
field is conservative.  We say that the corresponding differential
form $M\,dx + N\,dy$ is {\it closed\/} if its components
satisfy equation (\eqn).
\outind{closed form}%
\outind{conservative vector field}%

We now translate some of the things we learned about vector
fields to the language of forms.

(a) ``Every conservative field passes the screening test (\eqn)''
becomes ``Every exact form is closed''.

(b)``The field $\dfrac 1r\u_\theta$ satisfies the screening test
but is not conservative'' becomes
``$\dfrac{-y}{x^2 + y^2}dx + \dfrac x{x^2 + y^2} dy$ is closed
but not exact''.

(c) ``If the domain of a field is simply connected, then the field is
conservative if and only if it passes the screening test''
becomes ``If the domain of a form is simply connected, then the field
is exact if and only if it is closed''.

Note that because of (c), a closed form
$M\,dx + N\,dy$ defined on some open set in the plane
 is always {\it
locally\/} exact.  That is, for any point in its domain, we can
always choose a small neighborhood, i.e., a disk or a rectangle,
and a function $f$ defined on that neighborhood such that
$$
  M\,dx + N\,dy = df
$$
{\it on that neighborhood}.   However, we can't necessarily
find a single function which will work everywhere.  If we
can find an $f$ which works everywhere in the domain of the
form,
 the form would be exact, but we might say
`globally exact' to emphasize the distinction. 

\bigskip
\input chap6.ex1
\bigskip

\nextsec{Using Differential Forms}
\head \sn. Using Differential Forms to Solve Differential Equations \endhead

We want to develop general methods for solving first order
differential equations.   In most applications, the independent
variable represents time, so instead of calling the variables
$x$ and $y$, we shall call them $t$ and $y$ with $t$ being the
independent variable and $y$ being the dependent variable.  
Then a first order differential equation can {\it usually\/} be put
in the form
\nexteqn
$$
  \frac{dy}{dt} = f(t,y)\tag\eqn
$$
where $f$ is some specified function.  A solution  of (\eqn)
is a function  $y = y(t)$ defined on some  real $t$-interval
 such that
$$
y'(t) = f(t,y(t))
$$
is valid for each $t$ in the domain of the solution function $y(t)$.

\nextex
\example{Example \en}   Consider $\dfrac{dy}{dt} = -\dfrac{2t+y}{t +2y}$.
We shall 
`solve' this equation by some {\it formal\/} calculations using differential
forms, and later we shall try to see why the calculations should be
expected to work.

First, cross multiply to obtain
$$
  (t + 2y)dy = -(2t + y)dt
$$
or
\nexteqn
$$
  (2t + y)dt + (t + 2y)dy = 0.\tag\eqn
$$
Consider the differential form appearing on the left of this
equation.  It is closed since
$$
\frac{\d}{\d t} (t + 2y) = 1 = \frac{\d}{\d y}(2t + y).
$$
(Don't get confused by the fact that what was previously called
$x$ is now called $t$.)  Since the domain of this form is all
of $\R^2$ (the $t,y$-plane), and that is simply connected, the form
is exact.  Thus, it is the differential of a function,
 $df = (2t + y)dt + (t + 2y)dy$.  We may find that
$f$ by the same method we used to find a function 
with a specified 
 conservative vector field as gradient.  After all, it is just
the same theory in other notation.  Thus, since
$df = \dfrac{\d f}{\d t}dt + \dfrac{\d f}{\d y} dy$,
we want
$$
\frac{\d f}{\d t} = 2t + y\qquad\frac{\d f}{\d y} = t + 2y.
$$
Integrating each of these, we obtain
$$
f(t,y) = t^2 + yt + C(y)\qquad f(t,y) =  ty + y^2 + D(t),
$$
and comparing, we see that we may take $C(y) = y^2, D(t) = t^2$.
Thus,
$$
f(t,y) = t^2 + ty + y^2 
$$
will work.  (Check it by taking its differential!)
Thus, equation (\eqn) may be rewritten
$$
df = d (t^2 + ty + y^2) = 0
$$
from which we conclude
$$
  f(t,y) = t^2 + ty + y^2 = c
$$
for some constant $c$. 
We obtain this way an infinite collection
of level curves of the function $f$ as a ``solution'' to
the original differential equation.  In order to determine a
unique solution, we must specify a point $(t_0, y_0)$ lying
on the level curve or impose some other equivalent condition.
This corresponds to requiring that the solution of the differential
equation satisfy an {\it initial condition}  $y(t_0) = y_0$.
(Refer back to Chapter II where the issue of initial conditions
was first discussed.)   For example, if we know that
$y = 2$ when $t = 1$, we have
$$
   1^2 + (1)(2) + 2^2 = c\qquad\text{or}\qquad c = 7.
$$
Hence, the corresponding ``solution'' is $t^2 + ty + y^2 = 7$.
We can solve this for $y$ in terms of $t$ by applying the quadratic formula
to the equation $y^2 + ty + (t^2 - 7) = 0$.  This yields
$$
  y(t) = \frac{ -t \pm \sqrt{t^2 -4(t^2 - 7)}}2  = 
\frac{ -t \pm \sqrt{28 - 3t^2}}2
$$
but only the plus sign gives a solution satisfying the condition
$y = 2$ when $t = 1$.  Hence the solution we end up with
is  $y(t) = \dfrac 12(-t + \sqrt{28 - 3t^2})$.

\emar{s6-2.ps}{200}
Note that not every value of $c$ yields a solution which makes
sense.  Thus, requiring $y = 0$ when $t = 0$ yields $c = 0$
or $t^2 + yt + y^2 = 0$ and the locus of this equation in
$\R^2$ is the point $(0,0)$.  (Why?)   This is not a total
surprise, since the right hand side of the  original
differential equation $\dfrac{dy}{dt} = -\dfrac{2t+y}{t +2y}$
is undefined at $(0,0)$.

\mar{s6-3.ps}
\subhead Analysis of the Solution Process \endsubhead
We went from an equation of the form
\nexteqn
\xdef\EqA{\eqn}
$$
\frac{dy}{dt} = f(t,y)\tag\eqn
$$
to one of the form
\nexteqn
$$
M\,dt + N\,dy = 0\tag\eqn
$$
where $f(t,y) = -\dfrac{M(t,y)}{N(t,y)}$.  (We could take $M = f$
and $N = 1$, but in many cases, as in Example 1, $f$ is actually
a quotient.)  
The two equations are not quite the same.  A solution of (\EqA)
is a collection of functions $y = y(t)$,
one for each initial condition,  satisfying the differential equation,
while a solution of (\eqn) is a family of level curves
$f(t,y) = c$ in $\R^2$.  The relation between the two is
that the {\it graph\/} of each $y(t)$ is a subset of
some one of the level curves.

In general, I shall say that a smooth curve $\Cal C$
 in the plane is a
solution curve to equation (\eqn) if at each point on the
curve, the equation is true for each displacement vector
$\lb dt, dy \rb$ which is {\it tangent\/} to the curve.  Note that at
 any point in the domain of the differential form, the equation
(\eqn) determines the ratio $dy/dt$, so it
determines the vector  $\lb dt, dy \rb$
 up to multiplication by a scalar.  That means that the
{\it line\/} along which the vector  points is determined.  
(There is some fiddling you
have to do if one or the other of the coefficients $M, N$
vanishes.  However, if they both vanish at a point, the argument
fails and $\lb dt, dy \rb$ is not restricted at all.)    
A general solution of (\eqn), then, will be a
 family of such curves, one passing through each
point of the domain of the differential form---except possibly
where both coefficients vanish. 

\mar{s6-4.ps}
If $y = y(t)$ is a solution of the differential equation (\EqA),
then,  at each point of its graph, we also
have the relation
$$
 dy = y'(t) dt = f(t,y) dt = -\frac{M(t,y)}{N(t,y)} dt
$$
so the graph is at least part of a solution curve to
$$
 M(t,y)dt + N(t,y)dy = 0.\tag\eqn
$$ 
Conversely, the reasoning can be reversed, so that any
section of a solution curve to (\eqn) which happens to be
the graph of a function will be a solution to the original
differential equation.

\mar{s6-5.ps}
Example \en\ illustrates this quite well.  The curve
$t^2 + ty + y^2 = 7$ is in fact an ellipse in the $t,y$-plane,
but we can pick out a segment of that ellipse
by solving  $y(t) = \dfrac 12(-t + \sqrt{28 - 3t^2}$ 
 for $28 - 3t^2 > 0$ (i.e., $-\sqrt{28/3} < t < \sqrt{28/3}$),
and that yields a solution of the original differential
equation satisfying $y = 2$ when $t = 1$.

\emar{s6-6.ps}{-100}
There are differences between curves satisfying
$M\,dt + N\,dy = 0$ and graphs of functions satisfying
$dy/dt = f(t,y)$.  First of all, a solution curve for the
differential form could be
a vertical line, and that certainly can't be the
graph of a function.  Secondly, at a point where $N = 0$
but $M \not=0$, we must have $dt = 0$.  At such a point
the proposed tangent line to the curve is vertical, and
the curve might not look like the graph of a function.
However, this is quite reasonable, since at such a
point the differential equation $dy/dt = -M/N$ has
a singularity on the right.   Finally, at points at
which $M = N = 0$, all bets are off since no unique
tangent line is specified.  Such points are called
{\it critical points\/} of the differential form.
\outind{critical points of a differential form}%

\medskip
\subhead Integrating Factors \endsubhead
The strategy suggested in Example 1 for solving an equation of
the form 
$$
  M\,dt + N\,dy = 0
$$
is to look for a function $f$ so that $df =  M\,dt + N\,dy$,
and then to take the level curves of that function.
Of course, this will only be feasible if the differential
form is exact, so you may wonder what to try if it is
{\it not\/} exact.  The answer is to try to make it exact
by multiplying by an appropriate function.  Such
a function is called an {\it integrating factor}.  Thus,
\outind{integrating factor for a differential form}%
we want to look for a function $\mu(t,y)$ such that
$$
  \mu(M\,dt + N\,dy) = (\mu M)dt + (\mu N)dy
$$
{\it is\/} exact.

\nextex
\example{Example \en}  We shall try to solve
$y\,dt - t\,dy = 0$.   Note that
$$
   \frac{\d (-t)}{\d t} = -1 \not= 1 = \frac{\d y}{\d y}
$$
so the form is not closed and hence it cannot be exact.
We shall try to find $\mu(t,y)$ so that
$$
   (\mu y)dt - (\mu t) dy
$$
is exact.  Since every exact form is closed, we must have
\nexteqn
$$
\align
       \frac{\d(-\mu t)}{\d t} &= \frac{\d(\mu y)}{\d y}\qquad\text{or} \\
   - \frac{\d\mu }{\d t}t - \mu &= \frac{\d\mu}{\d y}y + \mu\qquad\text{or}\\
\frac{\d\mu }{\d t}t  +  \frac{\d\mu}{\d y}y  &= -2\mu.\tag\eqn
\endalign
$$
At this point, it is not clear how to proceed.  Equation (\eqn)
alone does not completely determine $\mu$, and indeed that is
no surprise because in general there may be many possible
integrating factors.  Of course, we need only one.  To find
a $\mu$ we proceed in effect by making educated guesses.    In
particular, we shall look for an integrating factor
  which depends only on $t$,
i.e., we assume $\mu = \mu(t)$.  Of course, this may not work, but we have
nothing to lose by trying.
With this assumption, equation (\eqn) becomes
$$
\align
\frac{d\mu}{dt}t  &= -2\mu \qquad\text{or} \\
\frac{d\mu}{\mu}  &= -2\frac{dt}{t}\qquad\text{which yields} \\
\ln |\mu| &= -2\ln |t| + c.
\endalign
$$
Since we only need {\it one\/} integrating factor, we may take
$c = 0$, so we get  
$$\align
\ln |\mu| &= \ln |t|^{-2}\qquad\text{or} \\
\mu &= \pm\frac 1{t^2}.
\endalign $$
Again, since we only need one integrating factor, we may
as well take  $\mu = 1/t^2$.

Having done all that work, you may think you are done, but of
course, all we now know is that
$$
   \frac 1{t^2}(y\,dt - t\, dy) = \frac y{t^2}dt - \frac 1t dy
$$
is probably exact.  Hence, it now makes sense to look for a
function $f(t,y)$ such that $df = \dfrac 1{t^2}(y\,dt - t\, dy)$.
If you remember the quotient rule, you will see immediately
that $f(t,y) = -y/t$ is just such a function.  However,
let's be really dumb and proceed as if we hadn't noticed
that.  We  use the usual method to find $f$.
Integrate the equations
$$
\frac{\d f}{\d t} = \dfrac y{t^2}\qquad
\frac{\d f}{\d y} = -\dfrac 1 t
$$
 to obtain
$$
 f = -\frac yt + C(y)\qquad  f =  -\frac yt + D(t). 
$$
Clearly, we can take $C = D = 0$, so we do obtain $f(t,y) =
-y/t$ as expected.  Hence, the general solution curves are
just the level curves of this function
$$
  f(t,y) = -y/t = c\qquad \text{or}\qquad y = -c t.
$$
Clearly, it doesn't matter what we call the constant, so we
can drop the sign and write the general solution as
$$
   y = ct.
$$
This is the family of all straight lines through the origin with
the exception of the $y$-axis.


\mar{s6-7.ps}
There is one problem with the above analysis.  What we actually
obtained were solution curves for the
equation
$$
 \mu(M\,dt + N\,dy) = 0.
$$
rather than the original equation $M\,dt + N\,dy = 0$.
If for example, $\mu(t,y) = 0$ has some solutions, that
locus would be added as an additional {\it extraneous\/}
solution which is not a solution of the original equation.
In this example, $\mu = 1/t^2$ never vanishes, so we
don't have any extraneous solutions.  However, it has
a singularity at $t = 0$, so its domain is smaller than
that of the original form $y\,dt - t\,dy$ (which is
defined everywhere in $\R^2$.)  Thus, it is possible
that the curve $t = 0$, i.e., the $y$-axis is a solution
curve for the original problem, which got lost when
the integrating factor was introduced.  In
fact that is the case.  If $t = 0$, so is $dt = 0$,
so 
$y\,dt - t\,dy = 0$.  
 
Hence, the general solution is the set of all straight lines
through the origin.  Note also that the origin itself
is a critical point of $y\,dt - t\,dy = 0$.

In general, there are many possible integrating factors
for a given differential form.  Thus,  in this example, 
$\mu = \dfrac 1{yt}$  is also an integrating factor and results
in the equation
$$\gather
\frac 1{yt}(y\,dt - t\,dy) = \frac{dt}t - \frac{dy}y = 0\\
\text{or}\qquad\qquad \frac{dt}t = \frac{dy}y
\endgather
$$
which is what we would have obtained by trying separation of
variables.  That would be the most appropriate method for
this particular problem, but we did it the other way in
order to illustrate the method in a simple case.

\endexample

If you thoroughly understand the above example, you will
be able to work many problems, but it is worthwhile also
describing the {\it general case}.   To see if the
form
$$
  (\mu M)dt + (\mu N)dy 
$$
is closed, apply the screening test
$$
   \frac{\d(\mu N)}{\d t} =  \frac{\d(\mu M)}{\d y}
$$
which 
yields
\nexteqn
$$
\align
\frac{\d\mu}{\d t}N + \mu\frac{\d N}{\d t} &=
\frac{\d\mu}{\d y}M + \mu\frac{\d M}{\d y}\qquad\text{or} \\
\frac 1\mu\left[\frac{\d\mu}{\d t}N - \frac{\d\mu}{\d y}M\right]
&= \frac{\d M}{\d y}- \frac{\d N}{\d t}. \tag\eqn
\endalign $$

Unfortunately, {\it it is not generally possible to solve
this equation!}  This is quite disappointing because it also means
that there is no general method to
solve explicitly the equation $M\,dt + N\,dy = 0$ or the related
equation $dy/dt = f(t,y)$.  There are however many special cases
in which the method does work.
These are discussed in great detail in books devoted to
differential equations.
 %%%  Put in some references to Braun and Simmons.
Typically one tries integrating factors  depending
only on one or the other of the variables or perhaps on
their sum.  This is basically a matter of trial and error.

\nextex
\example{Example \en}  Consider
$$
   (t + e^y) dt + (\frac 12 t^2 + 2t e^y) dy = 0.
$$
You should apply the screening test to check that the differential
form on the left is not closed.   We look for an integrating
factor by considering equation (\eqn) which in this case
becomes
$$
\align
\frac 1\mu\left[\frac{\d\mu}{\d t}(\frac 12 t^2 + 2t e^y)
 - \frac{\d\mu}{\d y} (t + e^y)\right]
&= \frac{\d  (t + e^y)}{\d y}- \frac{\d (\frac 12 t^2 + 2t e^y)}{\d t} \\
\intertext{or}
\frac 1\mu\left[\frac{\d\mu}{\d t}(\frac 12 t^2 + 2t e^y)
 - \frac{\d\mu}{\d y} (t + e^y)\right]
&=  e^y - (t + 2e^y) = -(t + e^y).
\endalign
$$
(This was derived by substituting in equation (\eqn), but
you would have been just as well off deriving it from scratch
by applying the screening test directly to the form 
$\mu(t + e^y) dt + \mu(\frac 12 t^2 + 2t e^y) dy$.  There is
no need to memorize equation (\eqn).) 
It is apparent that assuming that
$\partial\mu/\partial y = 0$
 (i.e., $\mu$ depends only on $t$) leads nowhere.    Try instead
assuming $\partial\mu/\partial t = 0$. In that case, the
equation becomes
$$\align
-\frac 1\mu \frac{d\mu}{dy}(t + e^y) &= -(t + e^y) \qquad\text{or}\\
	\frac 1\mu \frac{d\mu}{dy} &= 1.
\endalign $$
(Note that the partial derivative becomes an ordinary derivative
since we have assumed $\mu$ is a function only of $y$.)  This
can be solved easily to obtain as a possible integrating factor
$$
\mu = e^y.
$$
I leave the details to you.

To continue, we expect that the form
$$
e^y(t + e^y) dt + e^y(\frac 12 t^2 + 2t e^y) dy
$$
is exact.  Hence, we look for $f(t,y)$ such that
$$
\frac{\d f}{\d t} = te^y + e^{2y}\qquad \frac{\d f}{\d y} =
\frac 12 t^2e^y + 2t e^{2y}.
$$
Integrating the first equation with respect to $t$ and the
second with respect to $y$ yields
$$
f(t,y) = \frac 12 t^2 e^y + te^{2y} + C(y)
\qquad
f(t,y) = \frac 12 t^2 e^y + te^{2y} + D(t).
$$
Comparing, we see that $C = D = 0$ will work, so we obtain
$f(t,y) = \frac 12 t^2 e^y + te^{2y}$.  Hence, the
general solution is 
$$\frac 12 t^2 e^y + te^{2y} = C.$$

Notice that in this case the integrating factor neither vanishes
nor has any singularities, so we don't have to worry about
adding extraneous solutions or losing solutions.
\endexample

Note that in the above analysis we did not worry too much about
the geometry of the domain.  We know that it is not generally
true that a closed form is exact, so even if we find an
integrating factor, we may still not be able to find a
$f$ defined on the entire domain of the original differential
form.   However, this issue is not usually worth worrying about
because there are so many other things which can go wrong in
applying the method.   In any case, we know that {\it locally\/}
closed forms are exact, so that in any sufficiently small 
neighborhood of a point,  we can always find a $f$ which
works in that neighborhood.   Since, in general,
solution methods for differential equations only 
give local solutions, this may be the best we can hope for.
Extending local solutions to a global solution
is often a matter of great interest and great difficulty.

\bigskip
\input chap6.ex2
\bigskip


\nextsec{First Order Linear Equations}
\head \sn.  First Order Linear Equations \endhead
The simplest first order equations are the {\it linear equations\/},
\outind{linear first order differential equation}%
those of the form
\nexteqn
$$
\frac{dy}{dt} + a(t)y = b(t), \tag\eqn
$$
where $a(t)$ and $b(t)$ are known functions.  Equation (\eqn)
may also be written in the form $\dfrac{dy}{dt} = f(t,y)$ by taking
$f(t,y) = b(t) - a(t)y$.

\example{Examples}
$$\gather
\frac{dy}{dt} = ky \\
\frac{dy}{dt} - ky = j \\
\frac{dy}{dt} + \frac 1t y = \frac 1{t^2}
\endgather $$
The first equation is that governing exponential growth or decay,
and it was solved in Chapter II.  The general solution is
$y = Ce^{kt}$.

An example of a non-linear first order equation is
$$
\frac{dy}{dt} + y^2 = t.
$$
\endexample

Equation (\eqn) can be put in the form
$$
   (a(t)y - b(t))dt + dy = 0
$$
and solved by the methods of the previous section.  You should try it
as an exercise to see if you understand those methods.   However,
we shall use another method which is easier to generalize to
second order linear equations.

First consider the so-called {\it homogeneous\/} case  where  $b(t) = 0$,
i.e., we want to solve
$$
\frac{dy}{dt} + a(t)y = 0.
$$
\outind{homogeneous linear first order differential equation}%
This may be done fairly easily by separation of variables.
$$\gather
\frac{dy}y = -a(t)dt \\
\ln |y| = -\int a(t) dt + c \\
\intertext{where $\int a(t) dt$ denotes any antiderivative}
|y| = e^{-\int a(t) dt +c} = e^{-\int a(t) dt} e^c.
\endgather
$$
Put $C = \pm e^c$, depending on the sign of $y$, so the
{\it general solution\/}  takes the 
the form
\nexteqn
 $$
    y = Ce^{-\int a(t) dt},\tag\eqn
$$
where the constant $C$ is determined as usual by specifying an
initial condition.
Of course, you could rederive this in each specific case if you
remember the method, but you will probably save yourself considerable
time by memorizing formula (\eqn).

\nextex
\example{Example \en}  Consider
$$
\frac{dy}{dt} + \frac 1t y = 0\qquad\text{for } t > 0.
$$
Note that in this case $t = 0$ is a singularity of the coefficient
$a(t)$.  Hence, we would not expect the solution for $t < 0$
to have anything to do with the solution for $t > 0$.

We have
$$
   \int a(t) dt = \int \frac {dt}t = \ln t.
$$
(Of course, you could find the most general indefinite integral
or antiderivative by adding `$+ c$', but we only need {\it one\/}
antiderivative.)   Hence, according to formula (\eqn), the
general solution is
$$
  y = C e^{-\ln t} = \frac C t.
$$
The graphs for some values of $C$ are sketched below
\medskip
\centerline{\epsfbox{s6-8.ps}}
\medskip
%D   graphs
Note how one specific solution may be picked out by specifying
 an initial condition $y(t_0) = y_0$
\endexample

Consider next the {\it inhomogeneous case\/} where $b(t) \not= 0$,
i.e.,
$$
\frac{dy}{dt} + a(t) y = b(t).
$$
Denote by
$$
h(t) = e^{-\int a(t) dt}
$$
the solution of the corresponding homogeneous equation with $C = 1$.
We try to find a solution of the inhomogeneous equation of
the form $ y = u(t)h(t)$ where $u(t)$ is a function to be determined.
(This method is called ``variation of parameters''.)
We have
$$
  \gather
\frac{d(uh)}{dt} + auh = b,\\
\frac{du}{dt}h + u\frac{dh}{dt} + auh = b, \\
\frac{du}{dt}h + u\left(\frac{dh}{dt} + ah\right) = b.
\endgather $$
Since $h$ was chosen to be a solution of the homogeneous equation,
the quantity in parentheses vanishes.  Hence, the above
equation reduces to
$$
\frac{du}{dt} = \frac bh.
$$
 This is easy to solve.  The general solution is
$$
u = \int \frac{b(t)}{h(t)} dt + C
$$
where again the indefinite integral notation means that one
specific antiderivative is selected.  Since $y = hu$, we
obtain the following general solution of the inhomogeneous
equation
\outind{inhomogeneous linear first order differential equation}%
\nexteqn
\xdef\VarPar{\eqn}
$$
y = h(t)\int\frac{b(t)}{h(t)} dt + C\,h(t).\tag\eqn
$$
Note the form of this equation.  The term $Ch(t) = C e^{-\int a(t) dt}$
is a general solution of the homogeneous equation.  The first term
is the {\it particular solution\/} of the inhomogeneous equation
\outind{particular solution, linear first order equation}%
obtained by setting $C = 0$.  This is a theme which will be
repeated often in what follows
\smallskip
{\narrower\it

A general solution of  the inhomogeneous equation is obtained by
adding a general solution of the homogeneous equation to a particular
solution of the inhomogeneous equation.
}
\smallskip
\nextex
\example{Example \en}  Consider
$$
\frac{dy}{dt} + \frac 1t y = \frac 1{t^2}\qquad\text{for } t > 0.
$$
We determined $h(t) = \dfrac 1t$ in the previous example where we
solved the homogeneous equation.
Moreover,
$$
\int \frac{1/t^2}{1/t} dt = \int \frac1t dt = \ln t
$$
where as suggested we pick one antiderivative.  Then
from equation (\eqn), the general solution is
$$
 y = \frac 1t \ln t + \frac Ct.
$$
\endexample

It is important to note that equation (\eqn)
provides us, at
least in principle, with
a {\it complete solution\/} to the problem, a situation which will not
arise often in our study of differential equations,

\subhead Sometimes Guessing is Okay \endsubhead
In general, the solution of a first order differential equation
is uniquely determined if an initial condition is specified.
(We have seen how this works in several examples, and we shall
discuss the theory behind it later.)  Hence, if you are able
to guess a solution that works, that is a perfectly acceptable
way to proceed.   We can adapt that principle to finding the
general solution of a first order linear equation as
follows.   We may write the general solution (\eqn) in the
form
$$
    y = p(t) + Ch(t)
$$
where $p(t) = h(t)\int (b(t)/h(t)) dt$ is the particular
solution obtained by setting $C = 0$.   Suppose we can guess
some  particular solution $p_1(t)$  (which might be different 
from $p(t)$.)   Then, for some value of the constant $C_1$,
$$
    p_1(t) = p(t) + C_1h(t)\qquad\text{or}\qquad p(t) = p_1(t) - C_1h(t).
$$
Hence, the general solution may be written
$$\align
    y &= p_1(t) - C_1h(t) + Ch(t) = p_1(t) + (C-C_1)h(t) \\
      &= p_1(t) + C'h(t)
\endalign
$$
where $C' = C - C_1$ is also an arbitrary constant.  What this says
is that the general solution of the inhomogeneous equation is
the sum of {\it any\/} particular solution and the general solution
of the homogeneous equation.

\nextex
\example{Example \en}  Consider the equation
\nexteqn
$$
\frac{dy}{dt} - k y = I\tag\eqn
$$
where $k$ and $I$ are constants.   The homogeneous equation
$$
\frac{dy}{dt} -k y = 0 \qquad\text{or}\qquad \frac{dy}{dt} = k y
$$
has the general solution  $y = Ce^kt$.   We try to find a
particular solution of the inhomogeneous equation by guessing.
The simplest thing to try is a constant solution $y = A$.
Substituting in (\eqn) yields
$$
0 - kA = I\qquad\text{or}\qquad A = -\frac Ik.
$$
Hence,
$$
y = -\frac Ik + Ce^{kt}
$$
is a general solution of the inhomogeneous equation.   Note that this
is a bit simpler than using equation (\VarPar).

\subhead Using Definite Integrals in the Formulas \endsubhead
The notation $\int a(t)\,dt$ stands for {\it any\/} antiderivative
of the function $a(t)$.  Thus
$$
\int t\,dt = \frac 12 t^2\qquad\text{and}\qquad \int t\,dt = \frac 12 t^2 + 1
$$
are both true statements.  This ambiguity can lead to confusion.  To
avoid this, we may on occasion use definite integrals with variable
upper limits.  Thus
$$
\int_{t_0}^t a(s)\,ds
$$
is definitely an antiderivative for $a(t)$ because the fundamental
theorem of calculus tells us that
$$
\frac{d}{dt} \int_{t_0}^t a(s)\,ds = a(t).
$$
With this notation, we may write
$$
 h(t) = e^{-\int_{t_0}^t a(s)ds}.
$$
Note that $h(t_0) = e^0 = 1$.   In addition, we may write the
general solution of the inhomogeneous equation
$$
y = h(t)\int_{t_0}^t \frac{b(s)}{h(s)} ds + Ch(t).
$$
Note that $y(t_0) = 0 + Ch(t_0) = C$, so it may also be
written
$$
y = h(t)\int_{t_0}^t \frac{b(s)}{h(s)} ds + y_0h(t)
$$
where $y_0 = y(t_0)$.

Note that we have been careful to use a `dummy variable' $s$ in
\outind{dummy variable}%
the integrand.
Sometimes people are a bit sloppy and just use $t$ both for
the integrand and the upper limit, but
that is of course wrong.


\bigskip
\input chap6.ex3
\bigskip

\nextsec{Applications to Population Problems}
\head \sn.  Applications of First Order Equations to Population
Problems \endhead

With the little you now know, it is possible to solve a large
number of problems both in natural and social sciences.   To
do so, you need to construct a {\it model\/} of reality which
\outind{model}%
can be expressed in terms of a differential equation.  While
the mathematics that follows may be impeccable, the conclusions
will still only be as good as the model.  If the model is
based on established physical law, as for example the theory
of radioactive decay, the predictions will be quite accurate.
On the other hand, in some other applications,
there is little or no reason to accept the model,
so the results will be of questionable value.

\subhead Population Growth with Immigration \endsubhead
Let $p = p(t)$ denote the number of individuals in a population.
(The individuals could be people, bacteria, radioactive atoms,
etc.)  One common assumption about population growth is that
the number of births per unit time is proportional to the size
of the population, i.e., the rate of change of $p(t)$ due to
births is of the form $bp(t)$ for some constant $b$ called
the {\it birth rate}.  Similarly, it is assumed that the
number of deaths per unit time is of the form $dp(t)$ where
$d$ is another constant called the {\it death rate}.  We
may combine these and write symbolically
$$
  \frac{dp}{dt} = bp - dp = (b - d)p = kp
$$
where $k = b - d$ is a constant combining both the birth and
death rates.   The solution to this equation is
$p = p_0e^{kt}$ where $p_0 = p(0)$.  Thus according to this
model, population grows exponentially if $k > 0$ and
declines exponentially if $k < 0$.   Suppose in addition
to natural growth due to births and deaths, we also have
immigration taking place at a constant rate $I$.  (We
can assume $I$ is the net effect of immigration---into
the population---and emigration---out of the population.
It could even be negative in the case of net outflow.)  The
differential equation
becomes
$$\gather
\frac{dp}{dt} = kp + I\\
\text{or}\qquad \frac{dp}{dt} - kp = I.
\endgather $$
We solved this problem (with slightly different notation) in
the previous section.  The general solution is
$$
  p = -\frac Ik + Ce^{kt}.
$$
If $p = p_0$ at $t = 0$, we have
$$
  p_0 = -\frac Ik + C\qquad\text{or}\qquad C = p_0 +\frac Ik.
$$
Hence, the solution can also be written
$$
  p = -\frac Ik + (p_0 + \frac Ik)e^{kt}.
$$
Note that {\it in this model\/} if $k > 0$ and $p_0 + I/k > 0$,
then population grows exponentially, even if $I < 0$, i.e.,
even if there is net emigration.  
%Ex What would it mean for p_0 + I/k = 0?
  
The idea that natural populations grow exponentially was
first popularized by Thomas Malthus (1766-1834), and it
\outind{Malthus}%
is a basic element of modern discussions of population
growth.  Of course, the differential equation model ignores
many complicated factors.  Populations come in discrete
chunks and cannot be governed by a differential equation
in a continuous variable $p$.  In addition, most biological
populations involve two sexes, only one of which produces
births, and individuals vary in fertility due to age
and many other factors.  However, even when all these
factors are taken into consideration, the Malthusian model
seems quite accurate as long as the birth rate less the
death rate is constant.

\subhead The Logistic Law \endsubhead
The Malthusian model (with $k > 0$)
 leads to population growth which fills
all available space within a fairly short time.  That clearly
doesn't happen, so limiting factors somehow come into play
and change the birth rate so it no longer is constant.
Various models have been proposed to describe real populations
more accurately.    One is the so-called {\it logistic law}.
Assume $I = 0$---i.e., there is
no net immigration or emigration---and that
the `birth rate' has the form $k = a - bp$ where $a$ and $b$
are positive constants.  Note that there is very little thought behind
this about how populations behave.  We are just choosing the
simplest non-constant mathematical form for $k$.  About the only
thought involved is the decision to make the constant term positive
and the coefficient of $p$ negative.  That is, we assume that
as population grows, the birth rate goes down, although no
mechanism is suggested for why that might occur.  With
these assumptions, the differential equation becomes
$$
  \frac{dp}{dt} = (a - bp)p.
$$
\outind{logistic law}%
This is not a linear equation, but it can be solved by
separation of variables.
$$
  \frac{dp}{(a - bp)p} = dt \qquad\text{or}\qquad
\int\frac{dp}{(a - bp)p} = t + c.
$$
Assume $p > 0$ and  $a - bp > 0$.  In effect, that means we
are looking for solution curves in the $t,p$-plane between the
lines $p = 0$ and  $p = a/b$.  
If we find any solutions which start in this region and leave it,
we will be in trouble, so we will
have to go back and redo the analysis.
Under these assumptions, the integration on the left yields
(by the method of partial fractions) 
$$
-\frac 1a \ln(a - bp) + \frac 1a \ln p = \frac 1a\ln\left(\frac p{a - bp}
\right).
$$
(I tried to do this using Mathematica, but it kept getting the
sign of one of the factors wrong.  You should review the method
of partial fractions and make sure you understand where the
answer came from!  If you can't do it, ask for help from your
instructor.)  Hence, we obtain
$$
\ln\left(\frac p{a - bp}
\right) = at + ac = at + c'
$$
which after exponentiating yields
\nexteqn
$$
  \frac p{a - bp} = e^{at}e^{c'} = Ce^{at}.\tag\eqn
$$
If $p(0) = p_0$, we have
$$
  \frac {p_0}{a - bp_0} = C.
$$
Equation (\eqn) may be solved with some simple algebra to obtain
$$\align
p &= \frac {Cae^{at}}{1 + Cbe^{at}} =
\frac{\dfrac {p_0}{a - bp_0}ae^{at}}{1 + \dfrac {p_0b}{a - bp_0}e^{at}}\\
\intertext{which simplifies to}
p &= \frac{ap_0e^{at}}{a - bp_0 + bp_0e^{at}}\\
  &= \frac{p_0e^{at}}{1 + \dfrac ba p_0(e^{at} - 1)}.
\endalign
$$
To see what happens for large $t$, let $t \to \infty$.  For this
divide both numerator and denominator by $e^{at}$ as in
$$
p = \frac {p_0}{e^{-at} + \dfrac bap_0(1 - e^{-at})} \to \frac{p_0}
{\dfrac ba p_0} = \frac ab.
$$
Hence, the solution approaches the line $p = a/b$ asymptotically
from below, but it never crosses that line.
See the graph below which illustrates the population
growth under the logistic law.  Initially, it does appear
to grow exponentially, but eventually population growth
slows down and population approaches the equilibrium value
$p_e = a/b$ asymptotically.
%Diagram 6-9
\medskip
\centerline{\epsfbox{s6-9.ps}}
\medskip
%Ex   Work out the case  a - bp < 0.  
%Ex   What happens if   a = bp_0.  Note that in this case
%Ex  separation of variables fails because it involves division by 0.
%Ex   Similarly for $p = 0$, not interesting case.
If you want to read a bit more about population models,
see Section 1.5 of {\it Differential Equations and Their Applications\/}
by M. Braun.   However, you should take this with a grain of
salt, since it is all based on assuming a particular model and
then fiddling with parameters in the model to fit observation.

Note that the graphs of the solutions are also asymptotic to the
line $p = 0$ as $t \to -\infty$.   It may not be clear how to
interpret this in terms of populations, but there is no problem
with the mathematics.  Thus, each of the solutions obtained
above is defined for
$-\infty < t < \infty$, and its graph is contained entirely
in the strip $0 < p < a/b$.   Moreover, the lines $p = 0$ and
$p = a/b$ are also solutions, and there are solutions above
and below these lines.  (See the Exercises.)  Indeed,
by  suitable
choice of the arbitrary constants, you can find a solution curve passing
through every point in the plane, and moreover these solution
curves {\it never cross}.  

The fact that solution curves don't cross (except in very special
circumstances) is a consequence of the basic uniqueness theorem and
will be discussed in the next section.
However, we can use it here to justify our contention that
that $0 < p(t) < a/b$  if $0 < p_0 < a/b$.   
What we have to worry about is the possibility
 that some other solution curve (not one
arrived at above) could start inside
the desired strip and later leave it.   However, to do so, it would
have to cross one of the bounding lines, and we saw that they
also are solution curves, so that never happens.

\bigskip
\input chap6.ex4
\bigskip

\nextsec{Existence and Uniqueness of Solutions}
\head \sn. Existence and Uniqueness of Solutions \endhead

I mentioned earlier that the first order equations
$$
  \frac{dy}{dt} = f(t,y)
$$
often cannot be solved explicitly.    In such cases, one must resort
to graphical or numerical techniques in order to get
approximate solutions.  Before attempting that, however, one
should be confident that there actually is a solution.  Clearly,
if we can't find the solution, we have to use other methods to
convince ourselves it exists.   In this section, we describe
some of what is known in this area.

Consider the basic {\it initial value problem}: find $y = y(t)$
such that
$$
\frac{dy}{dt} = f(t,y)\qquad\text{where } y(t_0) = y_0.
$$
There are two basic questions one may ask.
\roster
\item  {\it Existence of a solution}.   Under what circumstances
can we be sure there is a solution $y = y(t)$ defined for some
$t$-interval containing $t_0$?   In this connection, we may also
ask what is the largest possible domain for such a solution.
\item  {\it Uniqueness of solutions}.  Under what circumstances
is the solution unique?  That is, can there be two or more
solutions satisfying the same initial condition?
\endroster
It would be foolish to devote a lot of time and energy to
finding a solution without first knowing the answers to these
questions.

\nextex
\example{Example \en}  Consider
$$
\frac{dy}{dt} = y^2\qquad\text{where } y(0) = y_0,
$$
and we assume $y_0 > 0$.
This equation is easy to solve by separation of variables.
$$\gather
\frac{dy}{y^2} = dt \\
-\frac 1y = t + C
\intertext{so for $t = 0$}
-\frac 1{y_0} = C\\
\text{and}\qquad -\frac 1y = t - \frac 1{y_0}\\
\text{so } y = \frac{y_0}{1 - y_0t}.
\endgather $$
The graph of this solution is a hyperbola asymptotic horizontally to the
$t$-axis and asymptotic vertically to the line $t = 1/y_0$.
\medskip
\centerline{\epsfbox{s6-10.ps}}
\medskip
Thus the solution has a singularity at $t = 1/y_0$, and there is
no coherent relationship between the solution to the right and
left of the singularity.  We conclude that there is a solution
defined on the interval $-\infty < t < 1/y_0$, but that solution
can't be extended smoothly to a solution on a larger interval.
Note that the differential equation itself provides no hint that
the solution has to have a singularity.  The function $f(t,y) = y^2$ is
smooth in the entire $t,y$-plane.  The lesson to be learned from
this is that at best we can be sure that the initial value problem 
has a solution
{\it locally\/} in the vicinity of the initial
point $(t_0, y_0)$.

The above solution is unique.  The reasoning for that is a little
tricky.   Basically, the solution was deduced by a valid argument from
the differential equation, so if there is a solution,
it must be the one we found.  Unfortunately, the solution process
involved division by $y^2$, so there is a minor complication.  
We have to worry about the possibility that there might be a
solution with $y(t) = 0$ for some $t$.   To eliminate that
we argue as follows. If $y(t_1) \not=0$ for a given $t_1$, by continuity
$y(t) \not= 0$ for all $t$ in an interval centered at $t_1$.  Hence,
the above reasoning is valid and the solution satisfies
$-1/y(t) = t + C$ in that interval, which is to say its graph is part
of a hyperbola.  However, none of these hyperbolas intersect the
$t$-axis, so no solution which is non-zero at one point can vanish
at another.    Notice, however, that $y(t) = 0$ for all $t$
does define a solution.  It satisfies the initial condition $y(0) = 0$.
\endexample

\nextex
\example{Example \en}  Consider 
$$
\frac{dy}{dt} = \frac yt.
$$
I leave it to you to work out the general solution by separation of
variables.  It is $y = Ct$.  For every 
$t_0 \not= 0$ and any $y_0$, there is a unique such solution satisfying
the initial condition $y_(t_0) = y_0$.
(Take $C = y_0/t_0$.)  If $t_0 = 0$, there is no such solution
satisfying $y(0) = y_0$ if $y_0 \not= 0$.
For $y_0 = 0$, every one of these solutions satisfies
the initial condition
$y(0) = 0$, so  
  the solution is not unique.  This is
not entirely surprising since $f(t,y) = y/t$  is not continuous for
$t = 0$.  Note however, that all the solutions $y = Ct$ are defined
and continuous at $t = 0$.
\endexample

\nextex
\example{Example \en}  Consider 
$$
\frac{dy}{dt} = 3y^{2/3}\qquad\text{where } y(0) = 0.
$$
We can solve this by separation of variables
$$
\gather
\frac{dy}{3y^{2/3}} =  dt \\
\frac{y^{1/3}}{3/3} = t + C \\
\intertext{so putting $y = 0$ when $t = 0$ yields}
0 = 0 + C \qquad \text{or} \qquad C = 0.\\
\text{Hence, }\qquad y^{1/3} = t\qquad\text{or } y = t^3
\endgather $$
is a solution.  (You should check that it works!)   Unfortunately,
the above analysis would exclude solutions which vanish, and it
is easy to see that
$$
y(t) = 0\qquad\text{for all } t
$$
also defines a solution.  Hence, there are (at least) two solutions
satisfying the initial condition $y(0) = 0$.
\medskip
\centerline{\epsfbox{s6-11.ps}}
\medskip
Note that in this case $f(t,y) = 3y^{2/3}$ is continuous for all
$t,y$.   However, it is not terribly smooth since
$f_y(t,y) =2y^{-1/3}$ blows up for $y = 0$.
\endexample

\subhead Propagation of Small Errors in the Solution \endsubhead
The above example suggests that the uniqueness of solutions
may depend on how smooth the function $f(t,y)$ is.  We shall
present an argument showing how uniqueness may be related to
the behavior of the
 partial derivative $f_y(t,y)$.   You might want to skip
this discussion
your first time through the subject since the reasoning is quite intricate.

Suppose $y_1(t)$ and $y_2(t)$ are two solutions of $dy/dt = f(t,y)$
which happen to be close for one particular value $t = t_0$.
(The extreme case of `being close' would be `being equal'.)
Let $\epsilon(t) = y_1(t) - y_2(t)$.   We want to see how far apart
the solutions can get as $t$ moves away from $t_0$, so
we
try to determine an upper bound on the function $\epsilon(t)$.   We have
$$\align
\frac{dy_1}{dt} &= f(t,y_1(t)) \\
\frac{dy_2}{dt} &= f(t,y_2(t)) \\
\endalign
$$
Subtracting yields
$$
\frac{d(y_1 - y_2)}{dt}
= 
\frac{dy_1}{dt}-\frac{dy_2}{dt}
 = f(t,y_1(t)) -   f(t,y_2(t)).
$$
However, if the difference $\epsilon = y_1 - y_2$ is {\it small enough\/}
we have approximately
$$
 f(t,y_1) -   f(t,y_2) \approx f_y(t, y_2(t))(y_1-y_2) = f_y(t, y_2) \epsilon.
$$
  That means that $\epsilon(t)$ {\it approximately\/} satisfies the
differential equation
$$
\frac{d\epsilon}{dt} = f_y(t,y_2(t))\epsilon.
$$
This equation is linear, and it has the solution
\nexteqn
$$
\epsilon = Ce^{\int f_y(t,y_2(t))dt},\tag\eqn
$$
so we can take the right had side as an {\it estimate\/}
of the size of the error as a function of $t$.
Of course, the above reasoning is a bit weak.  First of all,
we haven't defined what we mean  by an `approximate'
solution of a
differential equation.   Moreover, the approximation is only
valid for $\epsilon$ small, but we want to
use the
result to see if $\epsilon$ is small.  That is certainly circular
reasoning.
However, it is possible to make all this precise by judicious use
of inequalities, so the method can give you an idea of
how the difference of two solutions propagates.   

\emar{s6-12.ps}{100}
There are two important conclusions to draw from the estimate
in (\eqn).  First, assume the function $f_y$ is continuous,
so the exponential on the right will be well behaved.  In that
case
if the difference between two solutions $\epsilon$ is initially
0, the constant $C$ will be 0, so the difference $\epsilon$
will vanish for all $t$, at least in the range where all
the approximations are valid.  Second, assume the function
$f_y(t,y_2(t))$ has a singularity at $t_0$, and
the exponent in (\eqn) diverges to $-\infty$ as $t \to t_0$.
In that case, the exponential would approach zero, meaning that
we could have $\epsilon = y_1(t) - y_2(t) \not = 0$ for
$t \not = t_0$ but have it approach 0 as $t \to t_0$.  In
other words, the graphs of the two solutions might cross at
$(t_0, y_0)$.  In Example \en, that is exactly what occurs.
We have $f_y = 2y^{-1/3}$, so if we take $y_2 = t^3$
and $y_1 = 0$, we have
$$\gather
\int f_y(t,y_2(t))dt = \int 2(t^3)^{-1/3}dt
  = 2\int t^{-1}dt =  \ln t^2 \\
\text{so}\qquad e^{\int f_y(t,y_2(t))dt} = e^{\ln t^2} = t^2,
\endgather
$$
and this does in fact approach 0 as $t \to 0$.
\medskip
\mar{s6-13.ps}
\subhead The Basic Existence and Uniqueness Theorem \endsubhead

\nextthm
\proclaim{Theorem \cn.\tn}
Assume $f(t,y)$ is continuous on an open set $D$ in the $t,y$-plane.
Let $(t_0,y_0)$ be a point in $D$.   Then there is a function
$y(t)$ defined on some interval $(t_1, t_2)$  containing the
point $t_0$  such that
$$\gather
\frac{dy(t)}{dt} =  f(t, y(t))\qquad\text{for every $t$ in } (t_1, t_2)\\
\text{and }\qquad y(t_0) = y_0.
\endgather
$$

Moreover, if $f_y(t,y)$ exists and is continuous on $D$, then
there is at most one such solution defined on any interval
containing $t_0$.
\endproclaim
\outind{existence theorem, first order differential equations}%
\outind{uniqueness theorem, first order differential equations}%
\medskip
\centerline{\epsfbox{s6-16.ps}}
\medskip
The proof of this theorem is quite involved, so it is better left
for a course in Real Analysis.  However, most good books on
differential equations include a proof of some version of this
theorem.  See for example, Section 1.10 of {\it Braun} which uses
the method of {\it Picard iteration}.

\nextex
\example{Example \en}  Consider  the logistic equation
$$
\frac{dp}{dt} = ap - bp^2.
$$
Here $f(t,p) = ap - bp^2$ is continuous for all $(t,p)$,
so a solution exists satisfying any possible initial condition
$p(t_0) = p_0$.   Similarly $f_p(t,p) = a - 2bp$ is
continuous everywhere, so every solution is unique.  Thus,
graphs of solutions $p = p(t)$ never cross.
\endexample

\nextex
\example{Example \en}  Consider
$$
\frac{dy}{dt} = (y - t)^{1/3}.
$$
Here $f(t,y) = (y - t)^{1/3}$ is continuous for all $(t,y)$ so
a solution always exists satisfying any given initial condition.
However, $f_y(t,y) = \dfrac 13\dfrac 1{(y - t)^{2/3}}$ has
singularities along the line $y = t$.  Hence, we cannot be
sure of the uniqueness of solutions satisfying initial
conditions of the form $y_0 = t_0$.
\endexample

\mar{s6-15.ps}
Let us return to the second part of the {\it existence\/} question:
how large a $t$-interval can we choose for the domain of definition
of the solution.  We saw in Example 1
%Ref to first example
that this cannot be determined simply by examining the
domain of $f(t,y)$.  Here is another example.  


\nextex
\example{Example \en}
Consider the initial value problem $\dfrac{dy}{dt} = 1 + y^2$
with $y(0) = 0$.   The equation can be solved by separation of
variables, and the solution is  $y = \tan t$.  (Work it out
yourself.)  The domain of the continuous branch of this function
passing through $(0,0)$ is  the interval $-\pi/2 < t < \pi/2$.
\endexample

\emar{s6-19.ps}{-150}
Even in cases where the differential equation can't be
solved explicitly, it is {\it sometimes\/} possible to
determine an interval on which there is a solution.
We illustrate how to do this in the above example
(without solving the equation), but 
the reasoning 
is quite subtle, and you may not want to go through it at
this time .

%%Advice to instructor: skip this discussion
%First, we consider the general problem of determining bounds
%    Suppose that we know 
%$$
%           m \le f(t,y) \le M
%$$
%in the domain $D$ of $f(t,y)$.
%First, assume $t > t_0$.  Putting $y'(t) = \dfrac{dy}{dt} = f(t,y)$,
%we may integrate
%$$
%m \le y' \le M
%$$
%from $t$ to $t_0$ to obtain
%\nexteqn
%$$\gather
%\int_{t_0}^t m\,ds \le \int_{t_0}^t y'(s)\,ds \le \int_{t_0}^t M\, ds
%\qquad\text{or}\\
%m(t - t_0) \le y(t) - y_0 \le M(t - t_0)\qquad\text{or}\\
%y_0 + m(t - t_0) \le y(t) \le y_0 + M(t - t_0).\tag\eqn
%\endgather
%$$
%You have to be careful with the inequalities for $t < t_0$, but
%a similar argument  works in that case except that $m$ and $M$
%get switched.  This says that {\it the solution
%passing through $(t_0, y_0)$  lies between the two lines with slopes
%$m$ and $M$}.

\example{Example \en, continued}
  We consider the case $0 \le t$.  (The reasoning for
$t < 0$ is similar.)  The problem is to find a value $a$ such
that when $0 \le t \le a$, the solution won't grow so fast that
it will become asymptotic to some line $t = t_1$ with $0< t_1 < a$.
Since $y' = 1 + y^2 \ge 1$, there is certainly no problem with
the solution going to $-\infty$ in such an interval, so we only
concern ourselves with positive asymptotes.  Suppose that
we want to be sure that $0\le y \le b$.   Then we would have
$y' = 1 + y^2 \le 1 + b^2$, i.e., the graph would lie under the
a line (starting at the initial point $(0,0)$) with slope
$1 + b^2$.   That line has equation $y = (1 + b^2)t$
so it intersects the horizontal line
$y = b$ in the point $(\dfrac{b}{1 + b^2}, b)$.  Thus, for a given
$b$, we could take $a = \dfrac{b}{1 + b^2}$, and the solution 
 could not go to $\infty$ for $0 < t < a$.

The trick is to maximize $a$ by 
finding the maximum value of $b/(1 + b^2)$ over all possible
$b > 0$.   This may be done by
elementary calculus. Setting
$$
\frac{d}{db}\frac b{1 + b^2} =  \frac{(1 + b^2) - b(2b)}{(1 + b^2)^2}
  = \frac{1 - b^2}{(1 + b^2)^2} = 0
$$
yields $b = 1$ (since we know $b > 0$).  It is easy to see that this
is indeed a maximum point and the maximum value is $1/(1 + 1^2) = 1/2$.
It follows that we can be sure there is a solution for $0 \le t < 1/2$.
Similar reasoning (or a symmetry argument) shows that there is also
a solution for $-1/2 < t \le 0$.

Note that this reasoning gave us an interval $(-1/2,1/2)$ somewhat
smaller than that obtained above for the exact solution.
\endexample

\centerline{\epsfbox{s6-20.ps}}
\bigskip
The analysis in the above example involves
`lifting oneself by one's bootstraps' in that one uses `$b$' (a bound
on $y$) to find `$a$' (a bound on $t$), whereas what we really
want is the reverse.  It only
works well if $f(t,y)$ is
independent of $t$ or is bounded by functions independent
of $t$.  In general,  there may be no effective way to determine
an interval on which we can be sure there is a solution, although
the existence theorem assures there is such an interval. 

\bigskip
\input chap6.ex5
\bigskip

\nextsec{Graphical and Numerical Methods}
\head \sn. Graphical and Numerical Methods \endhead

What if the initial value problem for
a first order differential equation cannot be solved
explicitly?   In that case one may try to use graphical
or numerical techniques to find an approximate solution.
Even if one can't get very precise values, it is
sometimes possible to get a qualitative understanding
of the solution.

\subhead The Direction Field \endsubhead
For each point $(t,y)$ in $\R^2$, $f(t,y)$ specifies the slope
that the graph of a solution passing through that point should
have.   One may draw small line segments at selected points,
and it is sometimes possible to sketch in a good approximation
to the solution. 
\outind{direction field}%

\nextex
\example{Example \en}   
Consider  
$$
\frac{dy}{dt} = t\sqrt{1 + y^3}\qquad\text{where } y(0) = 1.
$$
We can make an attempt to solve this by separation of variables
to obtain
$$
\gather
\frac{dy}{\sqrt{1 + y^3}} = t\,dt \\
\text{so}\qquad \int \frac{dy}{\sqrt{1 + y^3}} = \frac 12 t^2 + c.
\endgather
$$
Unfortunately, the left hand side cannot be integrated explicitly
in terms of known functions, so there is no way to get an
explicit solution $y = y(t)$.  In the diagram below, I sketched
some line segments with the proper slope  $t\sqrt{1 + y^3}$
at points in the first quadrant.   Also, starting at the point
$(0,1)$, I attempted to sketch a curve which is 
tangent to the  line segment  at
each point it passes through.  I certainly didn't get it
exactly right, but the general characteristics of the solution
seem clear.  It increases quite rapidly and it may even have
an asymptote around $t = 2$.
\medskip
\centerline{\epsfbox{s6-21.ps}}
\medskip
We may verify analytically that the graph has an asymptote as follows.
We have
$$
\frac{dy}{dt} = t\sqrt{1 + y^3} > t\sqrt{y^3} = t y^{3/2}
\qquad\text{for } t >0, y > 0.
$$
Hence, in the first quadrant
$$
\frac{dy}{y^{3/2}} > t\,dt
$$
so 
$$
\int_1^y \frac{du}{(u)^{3/2}} > \int_0^t t'\,dt' = \frac 12 t^2.
$$
Note that we have to use definite integrals in order to preserve
the inequality, and we also used that $y > 1$ along the solution
curve.  The latter is apparent from the direction field, although
a detailed proof might be tricky.   Integrating on the left,
we obtain
$$
\gather
\left . -\frac 2{u^{1/2}}\right|_1^y > \frac 12 t^2 \\
-2(\frac 1{\sqrt y} - 1) > \frac 12 t^2 \\
\text{or}\qquad \sqrt y > \frac 1 {1 - \frac 14 t^2} \\
\text{or}\qquad y > \frac 1{(1 - \frac 14 t^2)^2}, 
\endgather
$$
and the expression on the right has an asymptote at $t = 2$.

\subhead Euler's Method \endsubhead

There are a variety of {\it numerical methods\/} for solving
differential equations.  Such a method can generate a table
\outind{numerical methods}%
of values  $y_i$ which are approximations to the true solution
$y(t_i)$ for selected values $t_i$ of the independent variable.
Usually, the values $t_i$ are obtained by dividing up a
specified interval  $[t_0, t_f]$  into $n$ equally spaced
points.
\medskip
\centerline{\epsfbox{s6-22.ps}}
\medskip
The simplest method is called {\it Euler's method}.  It is
\outind{Euler's method}%
based on the linear approximation
$$
y(t + h) = y(t) + y'(t)h + o(h)
$$
where as in Chapter III, Section 4,
%%Back reference to Chapter III, Section 4.
 the notation `$o(h)$' indicates
a quantity small compared to $h$.  ($\lim_{h\to 0} o(h)/h = 0$.)
Since  $y'(t) = f(t,y(t))$, we have approximately
$$
y(t + h) \approx y(t) + f(t,y)h.
$$
Using this, we may describe Euler's method by the following
Pascal-like pseudo-code, where $t_0$ and $y_0$ denote the
initial values of $t$ and $y$,  $t_f$ is the `final' value
of $t$, $n$ is the number of steps from $t_0$ to $t_f$,
and $h = (t_f - t_0)/n$ is the step size.
\smallskip
\def\hs{\hskip 20pt}
{\obeylines

read $t_0, y_0, t_f, n$;
$h$ :$=$ $(t_f - t_0)/n$;
$t$ :$=$ $t_0$;
$y$ :$=$ $y_0$;
while ($t \le t_f + h/2$) do
\hs begin
\hs\hs $yy$ :$=$ $y + f(t,y)*h$;
\hs\hs $t$ :$=$ $t + h$;
\hs\hs $y$ := $yy$;
\hs\hs show($t$, $y$);
\hs end;
end.
}
\smallskip
\mar{s6-23.ps}
Note the construction `$t \le t_f + h/2$'.   This complication
is needed
 because of the way numerical
calculations are done in a computer. 
 Namely, the calculation of $h$ will involve either
a round-off error or truncation error in almost all cases.  As
a result,  $n$ iterations of the step $t$ :$=$ $t + h$
could yield a value for $t$ slightly greater than $t_f$, but it
will certainly yield a value short of $t_f + h/2$. 
Adding the $h/2$ will insure that the loop stops exactly where we
want it to.

Euler's method is very simple to program, but it is not too
accurate.  It is not hard to see why.  At each step, we make
an error due to our use of the linear approximation rather
than the true solution at that point.  In effect that moves us
onto a neighboring solution curve.  After many steps, these
errors will {\it compound\/}, i.e., the errors from all previous
 steps will add additional errors at the current step.  Hence,
the cumulative error after a large number of steps can get
very large.   
See Figure 1 for the graphs of some
solutions of the initial value problem  $dy/dt = 1 - y,
y(0) = 0$ by Euler's method with different values of $n$.
The exact solution in this case is $y = 1 - e^{-t}$.
  In this case, the approximate solutions all lie above
 the exact solution.  Can you see why?
\medskip
\centerline{\epsfbox{euler.ps}}
\medskip
\centerline{Figure 1.  Euler's method for $n = 5, 10, 25$ and
exact solution}
\bigskip
One can estimate the error due to the approximation in
Euler's method.  It turns out that the error in the
calculation of the last value $y(t_f)$ is $O(h)$.  That means
if you double the number of steps, you will generally halve
the size of the error.  See {\it Braun\/} Section 1.13.1 for
\outind{Euler's method, error}%
a discussion of the error analysis.  Thus, in principle,
you ought to be able to get any desired degree of accuracy
simply by making the step size small enough.  However, the
error estimate $O(h)$ is based on the assumption of exact arithmetic.
Numerical considerations such as round-off error become
increasingly important as the step size is made smaller.
(For example, in the extreme case, if the number of steps
$n$ is chosen large enough, the computer may decide that the
step size $h$ is zero, and the while loop won't terminate!)
 As a result, there is a point of diminishing returns where
the answer actually starts getting worse as the step size
is decreased.   To get better answers one needs a more
sophisticated method.

\subhead The Improved Euler Method \endsubhead
Euler's method may be improved by incorporating a feedback
mechanism which tends to correct errors.  We start off as
\outind{Euler's method (improved)}%
before with the tangent approximation
$$
yy \text{:}= y + f(t,y)*h;
$$
but then we use this provisional value $yy$ at $t + h$
to calculate the putative slope  $f(t + h, yy)$ at
$(t + h, yy)$.   This is now averaged with the slope
$f(t,y)$ at $(t,y)$, and the result is used to determine
a new value $yyy$ at $t + h$
$$
yyy \text{:}= y + ((f(t,y) + f(t+h,yy))/2)*h;
$$

\emar{s6-24.ps}{-50}
Here is some Pascal like pseudo code implementing this
algorithm.
{\obeylines

read $t_0, y_0, t_f, n$;
$h$ :$=$ $(t_f - t_0)/n$;
$t$ :$=$ $t_0$;
$y$ :$=$ $y_0$;
while ($t \le t_f + h/2$) do
\hs begin
\hs\hs $m$ :$=$ $f(t,y)$;
\hs\hs $yy$ :$=$ $y + m*h$;
\hs\hs $t$ :$=$ $t + h$;
\hs\hs $yyy$ :$=$ $y + ((m + f(t,yy))/2)*h$;
\hs\hs $y$ :$=$ $yyy$;
\hs\hs show($t$, $y$);
\hs end;
end.
}

See Figure 2 for graphs showing the behavior of the improved
Euler method for the initial value problem $y' = 1- y, \, y(0) = 0$ for 
various $n$.   Note that the approximations lie below the
exact solution in this case.  Can you see why?    Note also
that the improved Euler method does give better accuracy
than the Euler method for the same value of $n$.
\medskip
\centerline{\epsfbox{ieuler.ps}}
\medskip
\centerline{Figure 2.  Improved Euler for $n = 5, 10, 25$ versus
exact solution.}
\bigskip

One can also estimate the error due to the approximation in
the improved Euler method.  It turns out that the error in the
\outind{Euler's method (improved), error}%
calculation of the last value $y(t_f)$ is $O(h^2)$.  That means
if you double the number of steps, you will generally reduce
the size of the error by a factor
of $1/4$.  See {\it Braun\/} Section 1.13.1 for
a discussion of the error analysis.

\bigskip
There are many methods for wringing out the last bit of accuracy
when attempting to solve a differential equation numerically.
One very popular method is the {\it Runge-Kutte method\/}
\outind{Runge-Kutte method}%
which tries to use quadratic approximations rather than linear
approximations.   There are people who have devoted a considerable
part of their professional
careers to the study of numerical solutions of
differential equations, and their results are embodied in a variety
of software packages.

\bigskip
\input chap6.ex6
\endchapter
\closeseg{chap6}
\enddocument
