\documentstyle{book}
\input extra.tex
\input le2.sty
\input epsf.tex
\input chap11.lnk
\maketoctrue
\indextrue
\mardiagtrue
\openseg{chap12}
\Monograph
\topmatter
\nextchap{More about Linear Systems}
\title
\chapter{\chapno}More about Linear Systems
\endtitle
\endtopmatter
\document
\nextsec{The Fundamental Solution Matrix}
\head \sn.  The Fundamental Solution Matrix \endhead

Let $A$ be an $n\times n$ matrix and consider the problem of solving
\nexteqn
$$
\frac{dX}{dt} = AX\tag\eqn
$$
where $X = X(t)$ is an $n\times n$ {\it matrix valued\/} function
of the real variable $t$.   If $X = \bm \x_1(t) & \x_2(t) & \hdots
&\x_n(t) \em$, then (\eqn) amounts to the simultaneous consideration
of $n$ equations, one for each column of $X$,
$$
\frac{d\x_1}{dt} = A\x_1,\quad \frac{d\x_2}{dt}= A\x_2,\quad\dots,
 \frac{d\x_n}{dt} = A\x_n.
$$
Much of our previous discussion of systems still applies.  In particular,
if the entries of $A$ are continuous functions on an interval 
$a < t <b$, then there is a unique solution of (\eqn) defined on that
interval which assumes a specified initial value $X(t_0)$ at some
point $t_0$ in the interval.

This formalism gives us a way to discuss a basis
 $\{\x_1(t), \x_2(t), \dots, \x_n(t)\}$
for the vector space of solutions of the homogeneous linear system
\nexteqn
$$
\frac{d\x}{dt} = A\x\tag\eqn
$$
in one compact notational package.   Clearly, if we want the columns of $X$ to
form such a basis, we need to assume that they constitute a linearly
independent set.   An $X = X(t)$ with these properties is called
a {\it fundamental solution matrix\/} for the system (\eqn).  Finding
a fundamental solution matrix is equivalent to finding a basis for the
\outind{fundamental solution matrix}%
solution space of (\eqn).

A fundamental solution matrix may also be used to express the
general solution
$$\align
\x &= \x_1(t)c_1 + \x_2(t)c_2 + \dots + \x_n(t)c_n
 = \bm \x_1(t)&\x_2(t)&\hdots &\x_n(t) \em\bm c_1\\c_2\\ \vdots \\c_n\em \\
\intertext{or}
\x &= X(t)\c\qquad\text{where } \c = \bm c_1\\c_2\\\vdots\\c_n\em.
\endalign$$

\nextex
\example{Example \en}
As in Example 1a of Section 8 of Chapter XI,
%Back reference
consider the system
$$
\frac{d\x}{dt} = \bm 1 & 2 & 0 \\
                      2 & 1 & 0 \\
                      0 & 1 & 3 \em \x.
$$
Form the fundamental solution matrix by putting together the
basic solutions in a $3\times 3$ matrix
$$\align
X(t) &= \bm e^{3t}\bm 1\\ 1\\ t\em & e^{3t}\bm 0\\ 0 \\ 1 \em &
e^{-t}\bm\format\r\\ 4\\-4\\1\em \em \\
 &= \bm\format\r&\quad\r&\quad\r\\
     e^{3t} & 0\hphantom{e} & 4e^{-t} \\
     e^{3t} & 0\hphantom{e} & -4e^{-t} \\
     te^{3t}& e^{3t} & e^{-t} \em.
\endalign
$$
Suppose we want a solution $\x(t)$ satisfying 
$$
\x(0) = \bm\format\r\\ 1\\ 0\\ 0\em.
$$
This amounts to solving  $X(0)\c = \x(0)$ or
$$
\bm\format\r&\quad\r&\quad\r\\
           1 & 0 & 4 \\
           1 & 0 & -4\\
           0 & 1 & 1  \em \c = \bm 1\\ 0\\ 0 \em.
$$
for $\c$.   We leave the details of the solution to you.  The
solution is $c_1 = 1/2, c_2 = -1/8$, and $c_3 = 1/8 $.   The desired solution
is
$$
\x = \frac 12 e^{3t}\bm 1\\ 1\\ t\em - \frac 18  e^{3t}\bm 0\\ 0 \\ 1 \em
+ \frac 18 e^{-t}\bm\format\r\\ 4\\-4\\1\em.
$$
\endexample

If $A$ is a constant $n\times n$ matrix, 
then
$$
X = e^{At}.
$$
is always a fundamental solution matrix.
Indeed, one way to characterize the exponential $e^{At}$
is as the unique solution of $\dfrac{dX}{dt} = AX$ satisfying
\outind{exponential of a matrix}%
$X(0) = I$.  However, the exponential matrix is defined as the sum of
a series of matrices which is not usually easy to compute.   Instead, 
it is usually easier to find
a fundamental solution matrix $X(t)$ by some other method, and then
use $X(t)$ to find $e^{At}$.  

\nextthm
\proclaim{Theorem \cn.\tn}  Let $A$ be an $n\times n$
matrix.  If $X(t)$ is a fundamental solution matrix for
the system $\dfrac{d}{dt}\x = A\x$, then for any initial
value $t_0$,
$$
   X(t) = e^{A(t - t_0)}X(t_0).
$$
In particular, for $t_0 = 0$,
\nexteqn
\xdef\Exp{\eqn}
$$
  X(t) = e^{At}X(0)\qquad\text{or}\qquad e^{At} = X(t)X(0)^{-1}.\tag\eqn
$$
\endproclaim

\demo{Proof}
By assumption,
$Y = X(t)$ satisfies the matrix equation $\dfrac{dY}{dt} = AY$.
However,  $Y = e^{A(t - t_0)}X(t_0)$ also satisfies that equation
since
$$
\frac{dY}{dt} = \frac{d}{dt}e^{A(t - t_0)}X(t_0)
              = Ae^{A(t - t_0)}X(t_0) = AY.
$$
Moreover, at $t = t_0$, these two functions agree since
$$
  e^{A(t_0 - t_0)}X(t_0) = I X(t_0) = X(t_0).
$$
Hence, by the uniqueness theorem,  $X(t) = e^{A(t - t_0)}X(t_0)$
for all $t$.
\enddemo

\example{Example \en, continued}
Let 
$$
 X = \bm\format\r&\quad\r&\quad\r\\
     e^{3t} & 0\hphantom{e} & 4e^{-t} \\
     e^{3t} & 0\hphantom{e} & -4e^{-t} \\
     te^{3t}& e^{3t} & e^{-t} \em
$$
be the fundamental solution matrix obtained above.
We may calculate by the usual method
$$
X(0)^{-1} =
\bm\format\r&\quad\r&\quad\r\\
           1 & 0 & 4 \\
           1 & 0 & -4\\
           0 & 1 & 1  \em^{-1} 
= \bm\format\r&\quad\r&\quad\r\\
     \frac 12 & \frac 12 & 0 \\
    -\frac 18 & \frac 18 & 1 \\
   \frac 18 &-\frac 18 & 0 \em
$$
so
$$\align
e^{At} &= \bm\format\r&\quad\r&\quad\r\\
     e^{3t} & 0\hphantom{e} & 4e^{-t} \\
     e^{3t} & 0\hphantom{e} & -4e^{-t} \\
     te^{3t}& e^{3t} & e^{-t} \em
\bm\format\r&\quad\r&\quad\r\\
     \frac 12 & \frac 12 & 0 \\
    -\frac 18 & \frac 18 & 1 \\
   \frac 18 &-\frac 18 & 0 \em\\
&= \bm \frac 12e^{3t}+\frac 12e^{-t} & \frac12e^{3t} -\frac12e^{-t}
      & 0 \\
    \frac 12e^{3t} - \frac12e^{-t} & \frac12e^{3t} + \frac12e^{-t}
     & 0 \\
    \frac 12te^{3t} -\frac 18e^{3t} +\frac 18e^{-t}
    & \frac12te^{3t} + \frac 18e^{3t} -\frac 18e^{-t}
    & e^{3t} \em.
\endalign$$
\endexample

\nextex
\example{Example \en}  Consider the system
$$
\frac{d\x}{dt} = \bm\format\r&\quad\r\\
                           0 & 1 \\
                          -1 & 0 \em \x.
$$
The characteristic equation of the coefficient matrix is
$\lambda^2 + 1 = 0$ which has roots $\lambda = \pm i$.   That
means that we start by finding complex valued solutions.  The
eigenvalues are distinct in this case, so the eigenvalue-eigenvector
method will suffice; 
we don't need to use generalized eigenvectors.

For $\lambda = i$, we need to solve $(A - iI)\v = 0$.  
We have
$$
\bm\format\r&\quad\r\\
        -i & 1 \\
         -1 & -i \em
\to \bm\format\r&\quad\r\\
         1 & i \\ 0 & 0 \em
$$
so the general solution is $v_1 = -iv_2$ with $v_2$ free.  The
general solution vector is
$$
\v = v_2 \bm\format\r\\ -i\\ 1\em,
$$
so 
$$
\v_1 = \bm\format\r\\ -i\\ 1\em
$$
is a basic eigenvector.   The corresponding solution of the
differential equation is
$$\align
\x_1 &= e^{it}\v_1 = \bm \format\r\\ -ie^{it}\\ e^{it}\em\\
     &= \bm\format\r\\ \sin t - i\cos t \\
                   \cos t + i\sin t \em.
\endalign$$
To find independent real solutions, take the real and imaginary
parts of this complex solution.  They are
$$
\u= \bm \sin t \\ \cos t \em,\qquad \v = \bm -\cos t \\ \sin t \em.
$$
Hence, a fundamental solution matrix for this system is
$$
X(t) = \bm\format\r&\quad\r\\
         \sin t & -\cos t \\
         \cos t & \sin t \em.
$$
Thus,
$$\align
e^{\bm \format\r&\quad\r\\ 0 & t\\ -t & 0 \em}
&= \bm\format\r&\quad\r\\
         \sin t & -\cos t \\
         \cos t & \sin t \em
 \bm\format\r&\quad\r\\
      0 & -1 \\
      1 & 0 \em^{-1}\\
&= \bm\format\r&\quad\r\\
         \sin t & -\cos t \\
         \cos t & \sin t \em
 \bm\format\r&\quad\r\\
      0 & 1 \\
      -1 & 0 \em\\
&= \bm\format\r&\quad\r\\
             \cos t & \sin t \\
             -\sin t & \cos t \em.
\endalign$$
We computed this earlier in Chapter XI, Section 7, Example 2
by adding up the series.
\endexample

\subhead  Wronskians \endsubhead
%Advice to Instructor.  Skip this.
Let $\{\x_1(t), \x_2(t), \dots, \x_n(t) \}$ be a set of solutions
of the system $d\x/dt = A(t)\x$ where $A(t)$ is an $n\times n$
matrix which is not necessarily constant.  Let
$$
X(t) = \bm \x_1(t)&\x_2(t) & \dots & \x_n(t) \em.
$$
The quantity $W(t) = \det X(t)$ is called the {\it Wronskian\/},
and it generalizes the Wronskian for second order linear
equations in one variable.
\outind{Wronskian of a system}%
It is not hard to see that $W(t)$ never vanishes if $X(t)$
is a fundamental solution matrix.  For, if it vanished for a
given $t = t_0$, the columns of $X(t_0)$ would form a dependent
set of vectors,
and this in turn would imply that the columns of
$X(t)$ form a  
dependent set of functions of $t$.  

It is possible to show that the Wronskian satisfies the first
order differential equation
$$
\frac{dW}{dt} = a(t)W,
$$
where
$$
a(t) = \sum_{i=1}^n a_{ii}(t)
$$
is the sum of the diagonal entries of $A(t)$. (The sum of
the diagonal entries of an $n\times n$ matrix $A$ is
called the {\it trace\/} of the matrix.) 

We shall not use the Wronskian, but you may encounter it
if you do further work with linear systems of differential
equations.
\bigskip
\input chap12.ex1
\bigskip
\nextsec{Inhomogeneous Systems}
\head \sn.  Inhomogeneous Systems \endhead 

Having thoroughly explored methods for solving homogeneous
systems, we now consider inhomogeneous systems
$$
\frac{d\x}{dt} = A(t)\x + \f(t)
$$
where $A(t)$ is a given $n\times n$ matrix, $\f(t)$ is a given 
vector function, and $\x = \x(t)$ as before is an
vector solution to be found.  
\outind{inhomogeneous system}%
\outind{system, inhomogeneous}%

The analysis is similar to that we
went through for  second order inhomogeneous linear equations 
$y'' + p(t)y' + q(t)y = f(t)$.
We proceed by
 first finding the general solution of the
homogeneous equation
$$
\frac{d\x}{dt} = A\x
$$
to which is added a {\it particular\/} solution of the
inhomogeneous equation.
\outind{particular solution of a system}%
(Indeed, second order inhomogeneous linear equations may be reformulated
as $2\times 2$ inhomogeneous systems in the usual way.)  

 To find a particular
solution of the inhomogeneous equation, we appeal to methods
that worked for second order equations.

The simplest method, if it works, is guessing.

\nextex
\example{Example \en}  Consider the system
\nexteqn
$$
\frac{d\x}{dt} = \bm\format\r&\quad\r\\
                    0 & 1 \\
                   -1 & 0 \em\x + \bm 1\\ 0 \em.\tag\eqn
$$
In Example 2 in the previous section, we found a general solution
of the corresponding homogeneous system.  Using the fundamental
solution matrix $e^{At}$, it may be expressed
$$
\x_h = \bm\format\r&\quad\r\\ \cos t & \sin t \\ -\sin t & \cos t \em
\bm c_1 \\ c_2 \em.
$$
To find a particular solution of (\eqn), try a constant solution
$$
\x_p = \bm a_1\\a_2 \em
$$
where $a_1$ and $a_2$ are to be determined.   Putting this in
the differential equation, we have
$$
0 = \bm\format\r&\quad\r\\
                    0 & 1 \\
                   -1 & 0 \em \bm a_1\\a_2 \em + \bm 1\\ 0 \em
$$
or
$$
\bm\format\r&\quad\r\\
                    0 & 1 \\
                   -1 & 0 \em \bm a_1\\a_2 \em = \bm \format\r\\-1\\ 0 \em.
$$
This is a $2\times 2$ algebraic system which is not very difficult
to solve.  The solution is $a_1 = 0, a_2 = -1$.   Hence, the
general solution of the inhomogeneous equation is
$$
\align
\x &= \bm\format\r\\ 0\\-1\em + \bm\format\r&\quad\r\\
          \cos t & \sin t \\ -\sin t & \cos t\em\bm c_1\\c_2\em\\
&= \bm\format\r\\ 0\\-1\em + c_1\bm\format\r\\ \cos t\\ -\sin t\em
+ c_2\bm \sin t\\ \cos t \em.
\endalign$$
\endexample

There is another method for finding a
particular solution  which is based on the
method of `variations of parameters'. 
 Unfortunately, it usually leads to extremely complex
calculations, even when the answer is relatively simple.
However, it is useful in many theoretical discussions.
To apply the system version of {\it variation of parameters\/}
\outind{variation of parameters for systems}%
we look for a particular solution 
of the form
$$
\x = X(t)\u(t)
$$
where $X(t)$ is a fundamental solution matrix of the
corresponding homogeneous system and $\u(t)$ is a
 vector valued function to be determined.
Substituting in the inhomogeneous equation yields
$$
\gather
\frac{d}{dt}(X\u) = AX\u + \f \\
  \frac{dX}{dt}\u + X\frac{d\u}{dt} = AX\u + \f
\endgather$$
However, $\dfrac{dX}{dt} = AX$ so the first term on each side
may be canceled, and we get
$$
\gather
X\frac{d\u}{dt} = \f\\
\frac{d\u}{dt} = X^{-1}\f.
\endgather$$ 
We may now determine $\u$ by integrating both sides with respect
to $t$.   This could be done using indefinite integrals, but
it is usually done with a dummy variable
 as follows.  Let $t_0$ be an initial value of $t$.
$$
\gather
    \frac{d\u}{ds} = X(s)^{-1}\f(s)\\
\left. \u(t)\right|_{t_0}^t = \int_{t_0}^t X(s)^{-1}\f(s)\,ds \\
\u(t) - \u(t_0) = \int_{t_0}^t X(s)^{-1}\f(s)\,ds\\
\u(t) = \u(t_0) + \int_{t_0}^t X(s)^{-1}\f(s)\,ds.
\endgather$$
If we multiply this by $X(t)$ to obtain $\x$, we get the
particular solution
$$
\x = X(t)\u(t) =  X(t)\u(t_0) + X(t) \int_{t_0}^t X(s)^{-1}\f(s)\,ds.
$$
Since we only need one particular
solution, we certainly should be free to choose $\u(t_0)$
any way we want.  If we set it equal to 0, then the second
term gives us the desired particular solution
\nexteqn
\xdef\VarPar{\eqn}
$$
\x_p =  X(t) \int_{t_0}^t X(s)^{-1}\f(s)\,ds.\tag\eqn
$$
On the other hand, we may also write 
$$
\u(t_0) = \c = \bm c_1 \\c_2 \em.
$$
where $\c$ is a vector of arbitrary constants.  
Then the above equation becomes
\nexteqn
$$
\x =  X(t)\c + X(t)\int_{t_0}^t X(s)^{-1}\f(s)\,ds
\tag\eqn
$$
which is the {\it general solution\/} of the inhomogeneous
equation.

\example{Example \en, again}   Use the same fundamental solution
$$
X(t)  = \bm \format\r&\quad\r\\ \cos t & \sin t \\ -\sin t & \cos t \em
$$
as before, and take $t_0 = 0$.   Then
$$\align
X(s)^{-1} &= \bm\format\r&\quad\r\\
 \cos s & \sin s \\ -\sin s & \cos s \em^{-1} = 
\frac 1{\cos^2 s + \sin^2 s}\bm\format\r&\quad\r\\
                          \cos s & -\sin s \\ \sin s & \cos s \em\\
 &= \bm\format\r&\quad\r\\
                          \cos s & -\sin s \\ \sin s & \cos s \em.
\endalign$$
Thus,
$$
\align
\x_p  &= \bm \format\r&\quad\r\\ \cos t & \sin t \\ -\sin t & \cos t \em
\int_0^t  \bm\format\r&\quad\r\\
                          \cos s & -\sin s \\ \sin s & \cos s \em
\bm 1\\ 0 \em ds.
\endalign$$
The integral is
$$\align 
\int_0^t \bm \cos s\\ \sin s \em ds 
&= \left. \bm \format\r\\ \sin s \\ -\cos s \em\right|_0^t \\
&= \bm\format\r\\ \sin t \\  -\cos t \em - \bm\format\r\\ 0\\ -1 \em.
\endalign
$$
Multiplying this by $X(t)$ yields
$$
\align
\x_p &= \bm \format\r&\quad\r\\ \cos t & \sin t \\ -\sin t & \cos t \em
\left(\bm\format\r\\ \sin t \\  -\cos t \em - \bm\format\r\\ 0\\ -1 \em
\right)\\
&= \bm \cos t \sin t - \sin t \cos t \\ -\sin^2 t - \cos^2 t \em
   + \bm -\sin t \\ -\cos t\em \\
&= \bm\format\r\\ 0 \\ -1 \em - \bm \sin t \\ \cos t \em.
\endalign$$ 
The second term is a solution of the homogeneous equation.  (In
fact, except for the sign, it is the second column of the
fundamental solution matrix.)   Hence, we can 
drop that term, and we are left with
$$
\x_p = \bm\format\r\\0\\ -1\em,
$$
which is the same particular solution we obtained previously
by guessing.
\endexample

If $A$ is a constant $n\times n$ matrix, then
$X(t) = e^{At}$ is always a fundamental solution matrix for
$d\x/dt = A\x$.   Also,
$$
X(s)^{-1} = (e^{As})^{-1} = e^{-As}.
$$
Hence, the general solution of the inhomogeneous equation
is
$$\align
\x &= e^{At}\c + e^{At}\int_{t_0}^t e^{-As}\f(s)\,ds \\
   &= e^{At}\c +\int_{t_0}^t e^{At}e^{-As}\f(s)\,ds \\
   &= e^{At}\c + \int_{t_0}^t e^{A(t - s)}\f(s)\,ds.
\endalign
$$
Moreover, if $\x(t)$ assumes the initial value $\x(t_0)$
at $t = t_0$, we have
$$
\x(t_0) = e^{At_0}\c +\int_{t_0}^{t_0}(\dots)\,ds = e^{At_0}\c,
$$
so $\c = e^{-At_0}\x(t_0)$.   Thus, 
$$\align
\x &= e^{At}e^{-At_0}\x(t_0) +  \int_{t_0}^t e^{A(t - s)}\f(s)\,ds \\
   &= e^{A(t - t_0)}\x(t_0) +  \int_{t_0}^t e^{A(t - s)}\f(s)\,ds
\endalign
$$
is a solution of the inhomogeneous equation satisfying the
desired initial condition at $t = t_0$.   This formula
sums everything up in a neat package, but it is not specially
easy to use for a variety of reasons.  First, $e^{At}$ is
not usually easy to calculate, and in addition, the integration
may not be specially easy to do.  

(See if you can simplify the calculations
in the previous example  by exploiting the fact that we were
using $e^{At}$ as our fundamental solution matrix.)
\bigskip
\input chap12.ex2
\bigskip

\nextsec{Normal Modes}
\head \sn. Normal Modes \endhead

\nextex
\xdef\ExA{\en}
\example{Example \en}
Recall the {\it second order system\/}
$$
\align
m\frac{d^2x_1}{dt^2} &= -2kx_1 + kx_2 \\
m\frac{d^2x_2}{dt^2} &= kx_1 - 2kx_2
\endalign$$
which was discussed in
Chapter X, Section 1, Example 2.
This system arose from the configuration of particles and springs
indicated below, where $m$ is the common mass of the two particles
and $k$ is the common spring constant of the three springs.
\medskip
\centerline{\epsfbox{s10-1.ps}}
\medskip
The system  may also be rewritten in
matrix form
$$
m\frac{d^2\x}{dt^2} = \bm\format\r&\quad\r\\
                        -2k & k \\ k & -2k \em\x.
$$
\endexample

Systems of this kind abound in nature.   For example,
a molecule may be modeled as a system of particles connected
by springs {\it provided one assumes all the displacements from
equilibrium are small}.  One is often very interested in determining the
ways in which such a molecule may oscillate and in particular
what the oscillatory frequencies are.  These will tell us
something about the spectral response of the molecule to
\outind{molecular oscillation}%
infrared radiation.   This {\it classical\/} model of a molecule 
 is only an approximation, of course.
and one must use quantum mechanics to get a more accurate
picture of what happens.   However, the classical model often
illustrates important features of the problem, and it is usually
more tractable mathematically.
 
\nextex
\xdef\ExB{\en}
\example{Example \en}  A CO$_2$ molecule may be represented
as two Oxygen atoms connected by springs to a Carbon atom.
In reality, the interatomic forces are quite complicated,
but to a first approximation, they may be thought of as
linear restoring forces produced by imaginary springs.
Of course, the atoms in a real CO$_2$ molecule may be oriented
relative to each other in space in quite complicated ways,
but for the moment we consider only configurations
in which all three atoms lie in a line.
\medskip
\centerline{\epsfbox{s12-2.ps}}
\medskip
If $m$ is the mass of each Oxygen atom and $m'$ is the mass of
the Carbon atom then we have  $m'/m \approx 12/16 = 3/4$.
As in the diagram, let $x_1$ and $x_3$ denote the linear displacements
of the two Oxygen atoms from some equilibrium position and
let $x_2$ be the linear displacement of the Carbon atom.  Then
Newton's Second Law and an analysis of the forces yields the
equations
$$\align
m\frac{d^2x_1}{dt^2} &= -k(x_1 - x_2) = -kx_1 + k x_2 \\
m'\frac{d^2x_2}{dt^2} &= -k(x_2 - x_1) - k(x_2 - x_3) = kx_1 - 2kx_2 + k x_3 \\
m\frac{d^2x_3}{dt^2} &= -k(x_3 - x_2) = kx_2 - k x_3 
\endalign$$
which may be put in the following matrix form
$$
\bm m & 0 & 0 \\
    0 & m' & 0 \\ 
    0 & 0 & m \em
\frac{d^2}{dt^2}
\bm x_1\\x_2\\x_3\em
= \bm\format\r&\quad\r&\quad\r\\
    -k & k & 0 \\
    k & -2k & k \\
    0 & k & -k \em
\bm x_1\\x_2\\x_3\em.
$$
This may be written more compactly as
\nexteqn
\xdef\Prob{\eqn}
$$
M\x'' = K\x\tag\eqn
$$
where
$$
M =
\bm m & 0 & 0 \\
    0 & m' & 0 \\ 
    0 & 0 & m \em
$$
is a {\it diagonal matrix\/} of masses, and
$$
K =
 \bm\format\r&\quad\r&\quad\r\\
    -k & k & 0 \\
    k & -2k & k \\
    0 & k & -k \em
$$
is a matrix of spring constants.   Note that $K$ is a {\it symmetric
matrix\/}, i.e. $K$ is equal to its transpose $K^t$.
\endexample

It will be the object of this and ensuing sections to solve second
order systems of the form (\eqn).   Of course, we already have
a method for doing that: convert to a first order system of twice
the degree and solve that by the methods we developed in the
previous chapter.   It is more enlightening, however, to start
from the beginning and apply the same principles directly to
the second order system.  As in the eigenvalue-eigenvector method
for first order systems, we proceed by looking for complex
vector valued solutions of
the form
\nexteqn
$$
\x = e^{i\omega t}\v\tag\eqn
$$
where $\omega$ and $\v\not= 0$ are to be determined.   (The rationale for
replacing $\lambda$ by $i\omega$ is that because of the nature of
the physical problem it makes sense to look for oscillatory
real solutions, and we know from our previous study of simple harmonic
motion that the complex expression of such solutions will involve
exponentials of the form $e^{i\omega t}$.)   Then
$$
\frac{d\x}{dt} = i\omega e^{i\omega t}\v\qquad\text{and}
\frac{d^2\x}{dt} = (i\omega)^2e^{i\omega t}\v = -\omega^2e^{i\omega t}\v.
$$
Hence,
 putting (\eqn) in (\Prob) yields
$$
M( -\omega^2e^{i\omega t}\v) =  Ke^{i\omega t}\v.
$$
Factoring out the (non-zero) term $e^{i\omega t}$
yields in turn $-\omega^2M\v = K\v$, which may be rewritten
\nexteqn
$$
K\v = \mu M\v\qquad\text{where}\quad \mu = -\omega^2\quad\text{and }
 \v \not= \bold 0. \tag\eqn
$$
This equation should look familiar.
 The quantity
$\mu = -\omega^2$ looks like an eigenvalue for $K$, and the
vector $\v$ looks like an eigenvector, except of course
for the presence of the diagonal matrix $M$.  As previously, (\eqn) may
be rewritten as a system
\nexteqn
\xdef\EqnC{\eqn}
$$
(K -\mu M)\v = 0\qquad\tag\eqn
$$
and since we need to find non-zero solutions, we need to
require
\nexteqn
$$
\det(K - \mu M) = 0.\tag\eqn
$$
This equation is similar to the characteristic equation for
$K$ except that the identity matrix $I$ has been replaced
by the diagonal matrix $M$.
It is called the {\it secular equation\/} of the system.
\outind{secular equation}%

The strategy then is first to find the possible values of
$\mu$ by solving (\eqn), and for each such $\mu$ to find
the possible $\v\not=0$ by solving the system (\EqnC).
The corresponding oscillatory (complex) solutions will be
$e^{i\omega t}\v$ where
$\omega = \sqrt{|\mu|}$. 

\example{Example \ExA, continued}  We have
$$
M = m I = \bm m & 0 \\  0 & m \em
\qquad K = k\bm\format\r&\quad\r\\
                       -2 & 1 \\
                       1 & -2 \em = \bm\format\r&\quad\r\\
                       -2k & k \\
                       k & -2k\em
$$
so the secular equation is
$$\align
\det \bm -2k - m\mu & k \\
          k  & -2k - m\mu \em
&= (-2k - m\mu)^2 - k^2 \\
 &= m^2\mu^2 + 2km\mu + 4k^2 - k^2\\
&=  m^2\mu^2 + 4km\mu + 3k^2 \\
  &= (m\mu + k)(m\mu + 3k) = 0.
\endalign
$$
Hence, the roots are 
$\mu = -k/m$ ($\omega = \sqrt{k/m}$)
 and $\mu = -3k/m$ ($\omega = \sqrt{3k/m}$).   

For $\mu = -k/m$  ($\omega = \sqrt{k/m}$), we need to solve
$(K + (k/m)M)\v = 0$.  But   
$$
K + \frac km M = \bm -2k +k & k \\
                     k & -2k + k \em =
          k\bm\format\r&\quad\r\\ -1 & 1 \\ 1 & -1 \em
\to \bm\format\r&\quad\r\\ 1 & -1 \\0 & 0 \em.
$$
Hence, the solution is $v_1 = v_2$ with $v_2$ free.
A basic solution vector for the subspace of solutions is
$$
\v_1 = \bm 1\\ 1  \em.
$$
The corresponding complex solution is
$$
 e^{i\sqrt{k/m}\,t}\bm 1 \\ 1\em.
$$
If we take the real and imaginary parts, we get {\it two\/}
real solutions.
$$
\cos \sqrt{k/m}\,t\bm 1 \\ 1 \em\qquad\text{and}\qquad
\sin \sqrt{k/m}\,t \bm 1 \\ 1 \em.
$$
If we write the components out explicitly, we get
$$
x_1 = \cos \sqrt{k/m}\,t,\quad x_2 =  \cos \sqrt{k/m}\,t
$$
for the first real solution, and
$$
x_1 = \sin \sqrt{k/m}\,t,\quad x_2 =  \sin \sqrt{k/m}\, t
$$
for the second real solution.  In either case, we have
$x_1(t) = x_2(t)$ for all $t$, and the two particles move together
in tandem
with the same angular frequency $\sqrt{k/m}$.   Note the 
behavior of the particles is a consequence of the fact
that the components of the basic vector $\v_1$ are equal.
Indeed, the same would be true for any linear combination
$$\multline
c_1\cos \sqrt{k/m}\,t\bm 1 \\ 1 \em + c_2\sin \sqrt{k/m}\,t \bm 1 \\ 1 \em
 \\ =
(c_1\cos \sqrt{k/m}\,t + c_2\sin \sqrt{k/m}\,t) \bm 1 \\ 1 \em
\endmultline
$$
of the two real solutions obtained above.  This two dimensional
real subspace of solutions is called the {\it normal mode\/}
of angular frequency $\sqrt{k/m}$.
\outind{normal mode}%

Similarly, for $\mu = -3k/m$ ($\omega = \sqrt{3k/m}$), we have 
$$
K + 3k/m M = \bm -2k +3k & k \\ k & -2k +3k \em 
= k\bm 1 & 1\\ 1 & 1\em \to \bm 1 & 1 \\ 0 & 0\em.
$$
The solution is $v_1 = -v_2$ with $v_2$ free, and a basic solution
vector for the system is
$$
\v_2 = \bm\format\r\\ -1\\1\em.
$$
The corresponding solution of the differential equation
is
$$
e^{i\sqrt{3k/m}\,t}\bm\format\r\\-1\\1\em.
$$
The corresponding normal mode is encompassed by the set of
all real solutions of the form
$$
(c_3\cos \sqrt{3k/m}\,t
+c_4\sin \sqrt{3k/m}\,t)\bm\format\r\\-1\\1\em
$$
which oscillate with angular frequency  $\sqrt{3k/m}$.

The general real solution of the differential equation has
the form
$$
\multline
\x = c_1\cos \sqrt{k/m}\,t\bm 1 \\ 1 \em + c_2\sin \sqrt{k/m}\,t \bm 1 \\ 1 \em
\\+ c_3\cos \sqrt{3k/m}\,t\bm\format\r\\-1\\1\em
+c_4\sin \sqrt{3k/m}\,t\bm\format\r\\-1\\1\em.
\endmultline
$$
\endexample

This example illustrates some features which need further
discussion.  First, we assumed implicitly in writing out the
general solution that the 4 functions
$$
\cos \sqrt{k/m}\,t\bm 1 \\ 1 \em,
\sin \sqrt{k/m}\,t \bm 1 \\ 1 \em,
\cos \sqrt{3k/m}\,t\bm\format\r\\-1\\1\em,
\sin \sqrt{3k/m}\,t\bm\format\r\\-1\\1\em
$$
constitute a basis for the vector space of solutions.   For this
to be true we need to know first of all that the space of solutions
is 4 dimensional, and secondly that the above functions form
a linearly independent set.   The first conclusion is clear if
we recall that the second order system in 2 variables which we
are considering  
  is equivalent to
a first order system of twice the size.  Hence,  the dimension of
the solution space is 4.  (In general, for a normal mode problem,
 the solution space should have
dimension $2n$ where $n$ is the number of variables.)   It is not
so obvious that the functions form a linearly independent set.
We shall address this question in detail at the end of this
section.  Note, however, that the rule which worked for
first order systems does not work here.  If we evaluate the
above vector functions at $t = 0$, we don't get a linearly
independent set; in fact, two of the vectors so obtained are zero.

Another point is that we could have determined the two vectors
$\v_1$ and $\v_2$ by inspection.  The first corresponds to motion
in which the particles move in tandem and
the spring between them experiences no net change in length.  
The second corresponds to motion in which the particles move back
and forth
equal amounts in opposite directions but with the same
frequency.  In fact, it is often true that careful consideration
of the physical arrangement of the particles, with particular
attention to any symmetries that may be present, may suggest
possible normal modes with little or no calculation.
\medskip
\centerline{\epsfbox{s12-3.ps}}
\medskip
\example{Example \en, continued}
In this case, the secular equation is
$$
\align
\det(K -\mu M)
&= \det\bm -k -m\mu & k & 0 \\
          k & -2k -\mu m' & k \\
         0 & k & -k-\mu m \em \\
 &= (-m\mu - k)((-m'\mu - 2k)(-m\mu -k) -k^2)
   -k(k(-m\mu  -k)) \\
 &= (-m\mu  - k)((-m'\mu - 2k)(-m\mu - k) - 2k^2)\\
 &= (-m\mu - k)(mm'\mu^2 + k(2m + m')\mu + 2k^2 - 2k^2)\\
 &= -(m\mu + k)(mm'\mu + k(2m + m'))\mu = 0.
\endalign$$
This has 3 roots.  (Don't worry about multiplicities at this point.)
They are
$$
\mu = -\frac km, \quad \mu = -\frac{k(2m + m')}{mm'},\quad \mu = 0
$$
with corresponding angular frequencies
$$
\omega = \sqrt{\frac km},\quad \omega = \sqrt{\frac{k(2m + m')}{mm'}},
\quad\omega = 0.
$$

Let's find a pair of real solutions for each of these.  That should
provide an independent set of 6 basic real solutions, which is what
we should expect since the solution space is 6 dimensional.

Start with $\mu = -k/m$ ($\omega = \sqrt{k/m}$).   If we make the
approximation $\dfrac {m'}m = \dfrac 34$,  the coefficient matrix
of the system $(K +k/mM)\v = 0$ becomes
$$\align
\bm -k + k & k & 0 \\
      k & -2k+(k/m)m' & k \\
     0 & k & 0 \em
&= \bm 0 & k & 0 \\
    k & -2k +k(3/4) & k \\
     0 & k & 0 \em\\
 &= \bm 0 & k & 0 \\
   k  &-(5/4)k & k \\
   0 & k & 0 \em 
\to \bm 1 & 0 & 1 \\
      0 & 1 & 0 \\
     0 & 0 & 0 \em.
\endalign
$$
The general solution is $v_1 = - v_3, v_2 = 0$ with $v_3$ free.
A basic solution is
$$
\v_1 = \bm\format\r\\ -1\\ 0 \\ 1 \em.
$$
The corresponding complex solution is
$$
e^{i\sqrt{k/m}\, t}\bm\format\r\\ -1\\ 0 \\ 1 \em.
$$
The corresponding normal mode is describe in real terms by
$$
(c_1\cos \sqrt{k/m}\,t + c_2\sin \sqrt{k/m}\,t)
\bm\format\r\\ -1\\ 0 \\ 1 \em.
$$
The physical interpretation is clear.  The two Oxygen atoms move
in equal and opposite directions while the Carbon atom stays fixed. 
\medskip
\centerline{\epsfbox{s12-4.ps}}
\medskip
For $\mu = - k(2/m' + 1/m)$ ($\omega = \sqrt{k(2/m' + 1/m)}$),
 the coefficient matrix
of the relevant system is
$$
\multline
\bm -k +mk(2/m' + 1/m) & k & 0 \\
   k  & -2k + m'k(2/m' + 1/m) & k \\
   0 & k & -k +mk(2/m' + 1/m) \em  \\
 = \bm -k + k(8/3 + 1) & k & 0 \\
      k  & -2k + k(2 + 3/4) & k \\
      0 & k & -k + k(8/3 + 1) \em  \\
= k\bm 8/3 & 1 & 0 \\
     1 & 3/4 & 1 \\
      0 & 1 & 8/3 \em \\
\to \bm 1 & 3/8 & 0 \\
         0 & 3/8 & 1 \\
         0 & 0 & 0 \em
\to \bm 1 & 0 & -1 \\
        0 & 1 & 8/3 \\
        0 & 0 & 0 \em.
\endmultline
$$
The corresponding solution is $v_1 = v_3, v_2 = -(8/3)v_3$
with $v_3$ free.  The corresponding basic vector is
$$
\v_2 = \bm 1 \\ -8/3 \\ 1 \em,
$$
and the corresponding normal mode is described in real terms
by
$$
(c_3\cos \sqrt{k(2/m' +1/m)}\,t + c_4\sin \sqrt{k(2/m' +1/m)}\,t)
\bm 1 \\ -8/3 \\ 1 \em.
$$
The physical interpretation is that the two Oxygen atoms move
together in tandem while the Carbon atom moves in the opposite
direction in such a way that the center of mass always stays
fixed.
\medskip
\centerline{\epsfbox{s12-5.ps}}
\medskip

Finally, consider $\mu = 0$.  This does not correspond to
an oscillatory solution at all since in this case
$\omega = 0$ and  $e^{i\omega t} = 1$.
Let's solve the system
$(K -\mu M)\v = K\v = 0$ in any case, although it
is not exactly clear what the physical interpretation should be.
$$
\bm \format\r&&\quad\r\\ -k & k & 0 \\
     k & -2k & k \\
     0 & k & -k \em
\to \bm\format\r&&\quad\r\\ -1 & 1 & 0 \\
                            1 & -2 & 1 \\
                            0 & 1 & -1 \em
\to \bm  \format\r&&\quad\r\\1 & -1 & 0 \\
        0  & -1 & 1 \\
        0 & 1 & -1 \em
\to \bm \format\r&&\quad\r\\ 1 & 0 & -1 \\
        0 & 1 & -1 \\
        0 & 0 & 0 \em.
$$
The corresponding solution is $v_1 = v_3, v_2 = v_3$ with $v_3$
free.  The corresponding basic vector is
$$
\v_3 = \bm 1\\1\\1\em,
$$
but all we get this way for a real solution is
$$
c_5\bm 1\\1\\1\em.
$$
What does this mean and where is the second real solution in this
case?  Since
all three displacements are equal, it appears that the particles
are displaced an equal distance in the same direction and there
is no oscillation.   A little thought suggests that what this
corresponds to is a uniform motion of the center of mass and
no relative motion of the individual particles about the center
of mass.   This tells us that we should add the
additional solution
$t\v_3$,
and the corresponding `normal mode' is
$$
(c_5 + c_6t)\bm 1\\1\\1\em,
$$    
which is a mathematical description of such uniform motion.
We now have
a set of 6 independent real solutions.
\endexample

This example also illustrates the principle that understanding
the physical nature of the problem and the underlying
symmetries can often lead to appropriate
guesses for the vectors $\v$.  Note also that once you have
picked out such a vector, you can check if you are right and also
determine the corresponding root $\mu = -\omega^2$ of the secular equation
by using the relation
$$
K\v = \mu M\v.
$$

\subhead Some General Remarks \endsubhead
The whole method depends on the fact that the roots $\mu$
of the secular equation
$$
\det(K - \mu M) = 0
$$
can be represented $\mu = -\omega^2$ where $\omega \ge 0$.   This
is the same as assuming all the roots $\mu$ are {\it negative\/}
or at worst zero.
However, if you pick a random symmetric $n\times n$ matrix and
a random diagonal matrix $M$---even assume the diagonal entries of
$M$ are positive---there is no way to be sure that 
some of the roots $\mu$ may not be positive.  Hence, for the problem
to be a bona-fide normal mode problem, we must impose this as
an additional assumption.   This is not entirely arbitrary,
however, because it can be shown from energy considerations that
if some of the roots are positive, then the physical configuration
will tend to fly apart instead of oscillating about a stable
equilibrium.  

As in Example 2, solutions associated with the root zero
correspond to non-oscillatory uniform motions.  However, if we
allow more than one spatial dimension, the situation is much
more complicated.   Consider, for example, plane
motions of the CO$_2$ molecule.  For each of the three
particles, there are two spatial coordinates, and so there
are altogether 6 displacement variables.  Thus the vector
space of all solutions is 12 dimensional,
and it has 6 possible basic `normal modes'.  Some
of these will involve two dimensional oscillations---see the
diagram---but others will be uniform motions corresponding to
a zero root of the secular equation.   
Some non-oscillatory solutions will consist of
  motion of the center of mass in some
direction at a constant velocity with no relative motion of the
particles about the center of mass.  
The problem is that other solutions will correspond to uniform
{\it rotations\/} about the center of mass, but that won't be apparent from
the mathematical representation.   The point is that in our analysis
of the problem, we assumed that the components of the vector $\x(t)$
are small, since we were concentrating on oscillations.  Consider
for example the motion in which the two Oxygen atoms rotate at a constant
rate about the Carbon atom which stays fixed at the center of mass
of the system.  {\it For small displacements\/}, this is not distinguishable
from a solution in which each Oxygen atom starts moving 
perpendicular to the line between the two Oxygen atoms 
(passing through  the Carbon atom) but in opposite directions. 
This is
what the mathematical solution $\x = (a + bt)\v$ will appear to
describe, but it is only valid `infinitesimally'.
\medskip
\centerline{\epsfbox{s12-6.ps}}
\medskip

\subhead Relation to Eigenvectors and Eigenvalues \endsubhead
The normal mode problem may be restated as follows.  Solve
$$
\det(K - \mu M) = 0
$$
and, for each root $\mu$, find all solutions of the system
\nexteqn
$$
K\v = \mu M\v.\tag\eqn
$$
($\mu = -\omega^2$, but that doesn't matter here.)   We noted that
this looks very
much like an eigenvalue-eigenvector problem, and by an appropriate
change of variables, we can reduce it exactly to such a problem.
\outind{normal mode, relation to eigenvectors}%
Let
$$
M = \bm m_1 & 0 & \hdots & 0 \\
         0 & m_2 & \hdots & 0 \\
        \vdots & \vdots &\hdots & \vdots \\
          0 & 0 & \hdots & m_n \em.
$$
Let 
$$v_j = \frac 1{\sqrt{m_j}}u_j\qquad\text{for }j = 1, 2, \dots, n.$$
Thus, $u_j  = \sqrt{m_j}v_j$
is weighted by the mass of the corresponding particle.
This may be written in compact matrix form as
$$
\v = (\sqrt M)^{-1} \u
$$
where $\sqrt M$ is the matrix with diagonal entries $\sqrt{m_j}$.
Putting this in (\eqn) yields
$$
K(\sqrt M)^{-1} \u = \mu M(\sqrt M)^{-1}\u = \sqrt M \mu \u
$$
or
$$
(\sqrt M)^{-1}K(\sqrt M)^{-1} \u = \mu \u.
$$
This says that $\u$ is an eigenvector for the matrix
$A = (\sqrt M)^{-1}K(\sqrt M)^{-1}$ with $\mu$ as the eigenvalue.
It is not hard to see that $A$ is also a real symmetric matrix,
so we see that the normal mode problem is really a special case
of the problem of finding eigenvalues and eigenvectors of real symmetric
matrices. 
\nextex
\example{Example \en}
Consider a normal mode problem similar to that in Example 1
except that the second particle has mass $4m$ rather than $m$.
Then the matrix $K$ is the same, but 
$$
M = \bm m & 0 \\ 0 & 4m \em.
$$
Hence,
$$
\sqrt M = \bm \sqrt m & 0 \\ 0 & 2\sqrt{m} \em
$$
and
$$
\align
A &= (\sqrt M)^{-1}K(\sqrt M)^{-1} \\
    &= \bm 1/\sqrt m & 0 \\ 0 & 1/(2\sqrt{m}) \em
 \bm\format\r&&\quad\r\\-2k & k \\ k & -2k \em
\bm 1/\sqrt m & 0 \\ 0 & 1/(2\sqrt{m}) \em \\
 &=  \bm\format\r&&\quad\r\\-2k/m & k/(2m) \\ k/(2m) & -k/(2m)\em
\endalign
$$
\medskip
\subhead Linear Independence of the Solutions \endsubhead
Suppose that in solving the $n\times n$ normal mode problem
$M\x'' = K\x$
 we obtained a basis $\{\v_1, \v_2, \dots, \v_n\}$ for
$\R^n$ such that 
$$
K\v_j = -\omega_j{}^2\v_j\qquad j = 1, 2, \dots, n,
$$
where each angular frequency $\omega_j$ is a non-negative root of
the secular equation.  We don't assume that the $\omega_j$ are
distinct, and, as in Example \en, some of them may be zero.
We want to show that the set of $2n$ real solutions of the normal
mode problem 
$$
\gather
\cos \omega_jt\,\v_j,\quad \sin\omega_jt\,\v_j\qquad\text{if }\omega_j\not= 0\\
\text{or }\qquad
\v_j,\quad t\v_j\qquad\text{if } \omega_j = 0 \\
\text{for } j \ \text{between } 1\ \text{and } n
\endgather
$$
is  linearly independent.

Suppose not.  Then there is a dependence relation of some sort.
By transposing, we may assume this has the form
\nexteqn
$$
\sum_{j = 1}^n (a_j\cos \omega_jt + b_j\sin \omega_jt)\v_j = 0,\tag\eqn
$$
and at least one of the coefficients $a_1, b_1, a_2, b_2, \dots,
a_n, b_n$ is 1.  (If $\omega_j = 0$, the
appropriate term in the sum should be $(a_j + b_jt)\v_j$, but,
as you will see, that will not affect the nature of the argument.)

Put $t = 0$ in (\eqn).  We obtain
$$
\sum_{j=1}^n a_j\v_j = 0.
$$
However, the set $\{\v_1, \v_2, \dots, \v_n\}$ is linearly independent
so the only such relation is the trivial one, i.e., we must have
$a_j = 0$ for $j = 1, 2, \dots, n$.   Rewrite (\eqn)
\nexteqn
$$
\sum_{j=1}^n b_j\sin \omega_jt\, \v_j = 0.
$$
(Again, if $\omega_j = 0$, the corresponding term will be $b_jt\v_j$
instead.)   Differentiate (\eqn) and set $t = 0$.  We obtain
$$
\sum_{j=1}^n b_j\omega_j\cos\omega_jt\, \v_j = 0.
$$
(Note that if $\omega_j = 0$, the differentiated term would be
$b_j\v_j$ without the factor of $\omega_j$.)   Again by linear
independence, all the coefficients are zero.  It follows that
$b_j = 0$ for $j = 1, 2, \dots, n$.   (Either $\omega_j \not = 0$
or the factor $\omega_j$ is not there.)   This contradicts the
assumption that we had a dependence relation in the first place.
\bigskip
\input chap12.ex3
\bigskip

\nextsec{Real Symmetric and Complex Hermitian Matrices}
\head \sn.  Real Symmetric and Complex Hermitian Matrices \endhead

We saw in the previous section that finding the normal modes of
a system of particles is mathematically a special case of finding
the eigenvalues and eigenvectors of a real symmetric matrix.
\outind{symmetric matrices, eigenvalues of}%
Many other physical problems reduce mathematically to the
same problem.
In this section we investigate that problem in greater detail.

Let $A$ be a real symmetric matrix.   The first thing we want
to show is that {\it the roots of its characteristic equation\/}
$$
\det(A - \lambda I) = 0
$$
{\it are all real}.  This is important in modern physics because
we have to boil the predictions of a theory down to some numbers
which can be checked against experiment, and it is easier if
these are real.   Many physical theories generate such numbers
as the eigenvalues of a matrix.

It is not true that the characteristic
equation of a real matrix must have real roots.  (Try
$A = \bm\format\r&\quad\r\\0 & -1 \\ 1 & 0 \em$, for example,
as in Chapter XI, Section 6.) Hence, the fact that the
matrix is symmetric must play an important role.   
In order to see how this comes into play, we need a short digression.

The {\it dot product in $\R^n$\/} was defined by the formula
\nexteqn
\xdef\EqReal{\eqn}
$$
(\u, \v) = \u^t\v = \bm u_1 & u_2 & \hdots & u_n \em
              \bm v_1\\v_2\\\vdots\\v_n \em =
\sum_{j=1}^nu_jv_j.\tag\eqn
$$
(Note that we introduced a new notation $(\u, \v)$ for the dot
product $\u\cdot \v$.  Another notation you will often see is
$\lb \u, \v \rb$.)  In order to discuss complex eigenvalues,
even if only to show there aren't any, we have to allow
the possibility of complex eigenvectors, i.e., we need to
work in $\C^n$, so we want to generalize the notion of
dot product to that domain.  One's first thought is to
just use the same formula (\eqn), but there is a problem
with that.   In $\R^n$, the length of a vector is
given by
$$
|\v|^2 = (\v, \v) = \sum_{j=1}^m v_j{}^2
$$
and this has all the properties you expect from a good `length
function'.   Unfortunately, in the complex case, the quantity
$\sum_{i=1}^nv_i{}^2$ can vanish {\it without\/} $\v$ being
zero.  For example, with $n = 2$, for
$$
\v = \bm i\\ 1\em,\quad\text{we have } v_1{}^2 + v_1{}^2 =
  i^2  + 1 = 0.
$$
Clearly, it would be much better to use the formula
\nexteqn
\xdef\EqLen{\eqn}
$$
|\v|^2 = \sum_{j=1}^n |v_j|^2\tag\eqn
$$
where $|v_j|^2 = \overline v_j v_j$ is the square of the absolute
value of the complex number $v_j$.   Unfortunately, this is not
consistent with the definition (\EqReal) of the dot product, but it
is easy to remedy that.  Define
\nexteqn
\xdef\EqHerm{\eqn}
$$
(\u, \v) = \overline\u^t\v = \sum_{j=1}^n\overline u_jv_j\tag\eqn
$$
for two vectors $\u$ and $\v$ in $\C^n$.   If the vectors 
are real this gives the same dot product as before, but it
also gives
$$
|\v|^2 = (\v, \v)
$$
if the left hand side is defined by (\EqLen).
\outind{dot product}%
\outind{inner product in $\noexpand C^n$}%


We may now use this extended dot product to derive the promised
result.

\nextthm
\proclaim{Theorem \cn.\tn}  Let $A$ be a real symmetric matrix.
The roots of $\det(A - \lambda I) = 0$ are all real.
\endproclaim
\demo{Proof}
Let $\lambda$ be a possibly complex eigenvalue for $A$, i.e.,
assume there is a non-zero $\v$ in $\C^n$ such that
$A\v = \lambda \v$.  
Consider the expression 
$$(A\v,\v)  = \overline{A\v}^t\v.
$$
We have
\nexteqn
$$
\overline{(A\v)}^t = (\overline A\overline\v)^t =
 \overline\v^t\overline A^t,\tag\eqn
$$
but since $A$ is real and symmetric, we have
\nexteqn
$$
 \overline A^t = A^t = A.\tag\eqn
$$
Hence,
$$
\overline{(A\v)}^t\v = \overline\v^t A\v.
$$
Now put $A\v = \lambda\v$ in the last equation to get
$$\gather
\overline{(\lambda\v)}^t\v = \overline\v^t\lambda\v \\
       \overline\lambda\overline\v^t\v =\lambda\overline\v^t\v
\endgather$$
However, $\overline\v^t\v = (\v, \v) = |\v|^2 \not = 0$ since
$\v\not= 0$.  Hence,
$$
\overline\lambda = \lambda.
$$
That tells us $\lambda$ is real.

Note that paradoxically we have to consider the {\it possibility\/}
that $\lambda$ is complex in order to show it is real.   It is
possible to prove this result without mentioning $\C^n$, but
the argument is much more difficult.   We will mention it again
when we discuss the subject of Lagrange multipliers.
\qed \enddemo

One crucial step in the above argument was in (\eqn)
where we concluded that
$$
\overline A^t  = A
$$
from the fact that $A$ is real and symmetric.   However, the
proof would work just as well if $A$ were complex and satisfied
$\overline A^t  = A$.  Such matrices are called {\it Hermitian\/}
\outind{Hermitian matrix}%
(after the 19th century French mathematician Hermite).  Thus,
we may extend the previous result to

\nextthm
\proclaim{Theorem  \cn.\tn}  Let $A$ be a complex Hermitian
$n\times n$ matrix.  Then the eigenvalues of $A$ are real.
\endproclaim

\nextex
\example{Example \en}  The matrix
$$
A = \bm\format\r&\quad\r\\0&i\\ -i&0\em
$$
which is used in quantum mechanics is Hermitian since
$$
\overline A = \bm\format\r&\quad\r\\0 & -i \\i & 0 \em,
\qquad
\overline A^t = \bm\format\r&\quad\r\\0 & i \\-i & 0 \em = A.
$$
Its eigenvalues are the roots of
$$
\det\bm -\lambda & i \\ -i& -\lambda \em = \lambda^2 - 1 = 0,
$$
which are $\lambda = \pm 1$ so they are certainly real.
\endexample

Note that a real $n\times n$  matrix $A$ is  Hermitian matrix 
if and only if it is symmetric.  (The conjugation has no effect
so the condition becomes $\overline A^t = A^t = A$.)
\medskip

\subhead More about the dot product in $\C^n$ \endsubhead
The new dot product $(\u, \v) = \overline\u^t\v$
 has all the usual properties we expect
of a dot product except that because of the complex conjugation
\outind{dot product}%
\outind{inner product in $\noexpand\C^n$}%
of the first factor, it obeys the rule
\nexteqn
\xdef\FirstFactor{\eqn}
$$
(c\u, \v) = \overline c (\u, \v).\tag\eqn
$$
However, for the second factor it obeys the usual rule
$(\u, c\v) = c(\u, \v)$.   It is also not commutative,
but obeys the following rule when the factors are switched.
$$
(\u, \v) = \overline{(\v, \u)}.
$$
(These formulas follow easily from the definition $(\u, \v) = 
\overline\u^t\v$.  See the Exercises.)

In $\R^3$ we chose the basis vectors $\e_1, \e_2, \e_3$
(formerly called $\i, \j, \k$) by picking unit vectors
along the coordinate axes.   It is usual in $\R^3$ to pick
mutually perpendicular coordinate axes, so the basis vectors
are {\it mutually perpendicular unit vectors}.   It makes sense to
do the same thing in $\R^n$ (or in the complex case
$\C^n$).  That is, we should attach special significance to
bases $\{\u_1, \u_2, \dots, \u_n\}$ satisfying
$$\align
(\u_i, \u_j) &= 0\qquad\text{for } i\not= j \\
|\u_i|^2 &= (\u_i, \u_i) = 1\qquad\text{otherwise}.
\endalign$$   
Such a basis is called an {\it orthonormal basis}.  The `ortho' part
\outind{orthonormal basis}%
\outind{basis, orthonormal}%
of `orthonormal' refers to the fact that the vectors are
mutually {\it orthogonal\/}, i.e., perpendicular, and the `normal'
part refers to the fact that they are unit vectors.

Orthogonality plays an important role for eigenvectors of
Hermitian matrices.
\nextthm
\proclaim{Theorem \cn.\tn} Let $A$ be a real symmetric
$n\times n$ matrix
or, in the complex case,  a Hermitian matrix.  Then eigenvectors
of $A$ associated with different eigenvalues are perpendicular.
\endproclaim

\demo{Proof} Assume
$$
A\u = \lambda\u\qquad\text{and}\qquad A\v = \mu\v\qquad\text{where }
\lambda\not=\mu.
$$
We have
$$
   \overline{(A\u)}^t\v = \overline\u^t\overline A^t \v = 
      \overline\u^t (A\v) 
$$
since $\overline A^t = A$.   Hence,
$$\gather
\overline{\lambda\u}^t\v = \overline\u^t(\mu\v) \\
  \overline\lambda\overline\u^t\v = \mu \overline\u^t\v .
\endgather$$
But the eigenvalues of $A$ are real, so $\overline\lambda =
\lambda$, and
$$
\gather
\lambda\overline\u^t\v = \mu \overline\u^t\v \\
 (\lambda - \mu)\overline u^t\v = 0 \\
 (\lambda - \mu)(\u, \v) = 0.
\endgather
$$
Since, $\lambda \not= \mu$, this implies that $(\u, \v) = 0$
as required.
\qed\enddemo

Because of the above theorems, orthogonality plays an important
role in finding the eigenvectors of a real symmetric 
(or a complex Hermitian) $n\times n$ matrix $A$.  Suppose first of all
that $A$ has $n$ distinct eigenvalues.  Then we know in any case
that we can choose a basis for $\R^n$ (or $\C^n$ in the complex case)
consisting of eigenvectors for $A$.   However, by Theorem \cn.\tn,
we know in addition in the symmetric (Hermitian) case
 that these basis vectors are mutually
perpendicular.  To get an orthonormal basis, it suffices to
{\it normalize\/} each basic eigenvector by {\it dividing it by
its length}. 

\example{Example \en, continued}
We saw that the eigenvalues of 
$$
A = \bm\format\r&\quad\r\\0&i\\ -i&0\em
$$
are $\lambda = \pm 1$.

For $\lambda = 1$, we find the eigenvectors by reducing
$$
A -I = \bm\format\r&\quad\r\\-1&i\\ -i&-1\em
\to 
\bm\format\r&\quad\r\\1&-i\\ 0&0\em.
$$
The general solution is $v_1 = iv_2$ with $v_2$ free, and
a basic eigenvector is
$$
\v_1 = \bm\format\r\\i\\ 1\em.
$$
To get a unit vector, divide this by its length
$|\v_1| = \sqrt{|i|^2 + 1^2} = \sqrt 2$; this gives
$$
\u_1 = \frac 1{\sqrt 2}\bm i \\ 1 \em.
$$

Similarly, for $\lambda = -1$, reduce
$$
A+I = \bm\format\r&\quad\r\\1&i\\ -i&-\em
\to \bm\format\r&\quad\r\\1&i\\ 0&0\em
$$
which yields as above the unit basic eigenvector
$$
\u_2 = \frac 1{\sqrt 2}\bm\format\r\\ -i \\ 1 \em.
$$
Note that 
$$(\u_1, \u_2) = \frac 12 ((-i)(-i) + (1)(1)0 = 0
$$
as expected.
\endexample
\bigskip
\input chap12.ex4
\bigskip

\nextsec{The Principal Axis Theorem}
\head \sn.  The Principal Axis Theorem \endhead

One of the most important results in linear algebra asserts that
{\it if $A$ is a real symmetric (a complex Hermitian) $n\times n$ matrix 
then there is an
orthonormal basis for $\R^n$ (\/$\C^n$) consisting of eigenvectors
for $A$}.   This is usually called the {\it Principal Axis Theorem}.
The reason for the name is that the case $n = 2,3$ may be used to
find the orientation or `principal axes' of an arbitrary conic in
the plane or  quadric
\outind{Principal Axis Theorem}%
\outind{principal axes}%
surface in space.   There is an important generalization of the result to
infinite dimensional spaces which is called the {\it Spectral
Theorem\/}, so the Principal Axis Theorem is also called the
`finite dimensional Spectral Theorem'.

In this section we shall work an example and also explore
some concepts related to the use of the theorem.   The proof will
be deferred for the moment.

Let $A$ be a complex Hermitian $n\times n$ matrix.   (If it is
real it will automatically be symmetric, so we don't need to discuss
the real case separately.)  As we saw in the previous section,
if the eigenvalues of $A$ are distinct, then we already know
that there is a basis consisting of eigenvectors and they are
automatically perpendicular to one another.   Hence, the
Principal Axis Theorem really only tells us something new
in case of repeated eigenvalues.

\nextex
\example{Example \en}
Consider
$$
A = \bm \format\r&\quad\r&\quad\r\\
              -1 & 1 & 1 \\
              1 & -1 & 1 \\
              1 & 1 & -1 \em.
$$
This example is real, so we shall work in $\R^3$.

The characteristic equation is
$$\multline
\det \bm  -1-\lambda & 1 & 1 \\
              1 & -1-\lambda & 1 \\
              1 & 1 & -1-\lambda \em\\
= -(1 + \lambda)((1 + \lambda)^2 - 1)
    - 1(-1 - \lambda - 1) + 1(1 + 1 + \lambda) \\
= -(1 + \lambda)(\lambda^2 + 2\lambda) + 2(\lambda + 2)\\
= -(\lambda^3 + 3\lambda^2 - 4) = 0.
\endmultline$$
Using the method suggested at the end of Chapter XI, Section 5,
we may find the roots of this equation by trying the factors of
the constant term.  The roots are $\lambda = 1$, which has multiplicity 1,
and $\lambda = -2$, which has multiplicity 2.

For $\lambda = 1$, we need to reduce
$$
A - I = \bm\format\r&\quad\r&\quad\r\\
              -2 & 1 & 1 \\
              1 & -2 & 1 \\
              1 & 1 & -2 \em
 \to \bm\format\r&\quad\r&\quad\r\\
              1 & 1 & -2 \\
              0 & -3 & 3 \\
              0 & 3 & -3 \em 
\to 
\bm\format\r&\quad\r&\quad\r\\
              1 & 0 & -1 \\
              0 & 1 & -1 \\
              0 & 0 & 0 \em.
$$
The general solution is $v_1 = v_3, v_2 = v_3$ with $v_3$ free.
A basic eigenvector is
$$
\v_1 = \bm 1\\1\\1\em
$$
but we should normalize this by dividing it by 
$|\v_1| = \sqrt 3$.   This gives
$$
\u_1 = \frac 1{\sqrt 3}\bm 1\\1\\1\em.
$$

For $\lambda = -2$, the situation is more complicated.  Reduce
$$
A + 2I = \bm 1 & 1 & 1\\1 & 1 & 1 \\1 & 1 & 1 \em
\to \bm 1 & 1 & 1\\ 0 & 0 & 0 \\0 & 0 & 0 \em
$$
which yields the general solution $v_1 = -v_2 - v_3$ with
$v_2, v_3$ free.   This gives basic eigenvectors
$$
\v_2 = \bm\format\r\\-1\\1\\0\em,\qquad
\v_3 = \bm\format\r\\-1\\ 0 \\ 1\em.
$$
Unfortunately, $\v_2$ and $\v_3$ are {\it not perpendicular\/},
but this is easy to remedy.  All we have to do is pick
{\it another basis\/} for the subspace spanned by $\{\v_2, \v_3\}$.
The eigenvectors with eigenvalue $-2$ are exactly the non-zero
vectors in this subspace, so any basis will do as well.

   It is easy to construct the new basis.  Indeed
we need only replace one of the two vectors.
Keep $\v_2$, and let $\v_3' = \v_3 - c\v_2$ where $c$ is chosen
so that  
$$
(\v_2, \v_3') = (\v_2, \v_3) - c(\v_2, \v_2) = 0, 
$$
i.e., take $c = \dfrac{(\v_2,\v_3)}{(\v_2,\v_2)}$.  (See the
diagram to get some idea of the geometry behind this calculation.)
We have
$$
\gather
\frac{(\v_2,\v_3)}{(\v_2,\v_2)} = \frac 12 \\
\v_3' = \v_3 - \frac 12 \v_2 = \bm\format\r\\-1\\0\\1\em
      -\frac 12\bm\format\r\\-1\\1\\0\em
= \bm\format\r\\-\frac 12\\-\frac 12\\1\em.
\endgather$$

\mar{s12-6a.ps}
We should also normalize this basis by choosing
$$
\u_2 = \frac 1{|\v_2|}\v_2 = \frac 1{\sqrt 2} \bm\format\r\\-1\\1\\0\em,
\qquad
\u_3 =\frac 1{|\v_3'|}\v_3'
 = \sqrt{\frac 23}\bm\format\r\\-\frac 12\\-\frac 12\\1\em.
$$

Putting this all together, we see that 
$$
\u_1 =  \frac 1{\sqrt 3}\bm 1\\1\\1\em,\qquad
\u_2 = \frac 1{\sqrt 2} \bm\format\r\\-1\\1\\0\em,
\qquad
\u_3  = \sqrt{\frac 23}\bm\format\r\\-\frac 12\\-\frac 12\\1\em
$$
form an orthonormal basis for $\R^3$ consisting of eigenvectors
for $A$.   Notice that $\u_1$ is automatically perpendicular
to $\u_2$ and $\u_3$ as the theory predicts.
\endexample
\subhead The Gram--Schmidt Process \endsubhead
In Example \en, we used a special case of a more general
algorithm in order to construct an orthonormal basis of
eigenvectors.   The algorithm, called the {\it Gram--Schmidt
Process\/} works as follows.   Suppose 
\outind{Gram--Schmidt Process}%
$$
\{\v_1, \v_2, \dots, \v_k\}
$$
is a {\it linearly independent set\/} spanning a certain
subspace $W$.  We construct an orthonormal basis for $W$
as follows.  Let
$$\align
\v_1' &= \v_1 \\
\v_2' &= \v_2 - \frac{(\v_1',\v_2)}{(\v_1',\v_1')}\v_1'\\
\v_3' &= \v_3 - \frac{(\v_1',\v_3)}{(\v_1',\v_1')}\v_1'
 - \frac{(\v_2',\v_3)}{(\v_2',\v_2')}\v_2' \\
 &\vdots \\
\v_k' &= \v_k - \sum_{j=1}^{k-1}\frac{(\v_j',\v_k)}{(\v_j',\v_j')}\v_j'.
\endalign$$

It is not hard to see that each new $\v_j'$ is perpendicular
to those constructed before it.   For example,
$$
(\v_1',\v_3') = (\v_1',\v_3) - \frac{(\v_1',\v_3)}{(\v_1',\v_1')}
(\v_1',\v_1') - \frac{(\v_2',\v_3)}{(\v_2',\v_2')}(\v_1',\v_2'). 
$$
However, we may suppose that we already know that $(\v_1',\v_2') = 0$
(from the previous stage of the construction), so the above becomes
$$
(\v_1',\v_3') = (\v_1',\v_3) - (\v_1',\v_3) = 0.
$$
The same argument works at each stage.

It is also not hard to see that at each stage, replacing
$\v_j$ by $\v_j'$ 
in
$$\{\v_1', \v_2', \dots, \v_{j-1}', \v_j\}$$
does not change the subspace spanned by the set.  
Hence, for $j = k$, we conclude that $\{\v_1', \v_2', \dots,
\v_k'\}$ is a basis for $W$ consisting of mutually perpendicular
vectors.   Finally, to complete the process simply divide
each $\v_j'$ by its length
$$
\u_j = \frac 1{|\v_j'|}\v_j'.
$$
Then $\{\u_1,\dots, \u_k\}$ is an orthonormal basis for $W$.

\nextex
\example{Example \en}  Consider the subspace of $\R^4$ spanned by
$$
\v_1 = \bm\format\r\\-1\\1\\0\\1\em,
\v_2 =  \bm\format\r\\-1\\1\\1\\0\em,
\v_3 =  \bm\format\r\\1\\0\\0\\1\em.
$$
Then
$$\align
\v_1' &= \bm\format\r\\-1\\1\\0\\1\em \\
\v_2' &= \bm\format\r\\-1\\1\\1\\0\em - \frac 23\bm\format\r\\-1\\1\\0\\1\em
      = \bm\format\r\\ -\frac 13\\ \frac 13 \\ 1\\ -\frac 23 \em \\
\v_3' &= \bm\format\r\\1\\0\\0\\1\em - \frac 03\bm\format\r\\-1\\1\\0\\1\em
   - \frac {-1}{\frac {15}9}\bm\format\r\\ -\frac 13\\ \frac 13 \\ 1\\ -\frac 23 \em
    = \bm\format\r\\ \frac 45\\ \frac 15\\ \frac 35\\ \frac 35 \em.
\endalign$$
Normalizing, we get
$$\align
\u_1 &= \frac 1{\sqrt 3}\bm\format\r\\-1\\1\\0\\1\em \\
\u_2 &= 
\frac 3{\sqrt{15}}\bm\format\r\\ -\frac 13\\ \frac 13 \\ 1\\ -\frac 23 \em 
=
\frac 1{\sqrt{15}}\bm\format\r\\ -1\\ 1 \\ 3\\ -2 \em \\
\u_3 &= \frac 5{\sqrt {35}}
\bm\format\r\\ \frac 45\\ \frac 15\\ \frac 35\\ \frac 35 \em
= \frac 1{\sqrt {35}}
\bm\format\r\\ 4\\ 1\\ 3\\ 3 \em.
\endalign$$
\bigskip
\input chap12.ex5
\bigskip

\nextsec{Change of Coordinates and the Principal Axis Theorem}
\head \sn.  Change of Coordinates and the Principal Axis
Theorem \endhead

One way to understand the Principal Axis Theorem and other such theorems
about a special choice of basis is to think of how a given problem
 would be expressed relative to that basis.
For example, if we look at the linear operator $L$ defined by
$L(\x) = A\x$, then if $\{\v_1,\v_2,\dots,\v_n\}$ is a basis
of eigenvectors for $A$, we have by definition $L(\v_i) = \lambda_i\v_i$.
Thus, for any vector $\x$, its coordinates
$x_1', x_2', \dots, x_n'$ with respect to this basis
are the coeffcients in 
$$
\x = \v_1x_1' +\v_2x_2' + \dots +\v_nx_n' =
\bm \v_1 & \v_2 & \hdots &\v_n\em \bm x_1'\\x_2'\\\vdots\\x_n'\em,
$$
so we have
$$
\align
L(\x) &= L(\v_1)x_1' + L(\v_2)x_2' + \dots + L(\v_n)x_n'\\
&= \v_1\lambda_1x'_1 + \v_2\lambda_2x'_2 + \dots + \v_n\lambda_nx'_n
=\bm \v_1 &\v_2 &\hdots &\v_n\em \bm \\\lambda_1x'_1\\\lambda_2x'_2\\
\vdots\\\lambda_nx'_n\em.
\endalign
$$
Thus, the effect of $L$ on the {\it coordinates\/} of a vector
{\it with respect to such a basis\/} is quite simple: each coordinate is
just multiplied by the corresponding eigenvalue.  (See Chapter X,
Section 8
%Back reference
to review the concept of coordinates with respect to a basis.)
\outind{coordinates, change of}%

To study this in greater detail, we need to talk a bit more about
changes of coordinates.  Although the theory is quite general, we
shall concentrate on $\R^n$ and $\C^n$.  In either of these vector
spaces, we start implicitly with the standard basis
$\{\e_1, \e_2, \dots, \e_n\}$.   The entries 
 in a vector
$\x$ may be thought of as the coordinates $x_1, x_2, \dots, x_n$
of the vector with respect to that basis.  Suppose  $\{\v_1,\v_2,\dots,
\v_n\}$ is another basis.  
As above, the coordinates of $\x$ with respect to
the new basis are obtained by solving
\nexteqn
$$
\x = \bm \v_1 & \v_2 & \hdots &\v_n \em \x'\tag\eqn
$$
for
$$
\x' = \bm x_1' \\ x_2' \\ \vdots \\ x_n' \em.
$$
Let 
$$
P = \bm \v_1 &\v_2 & \hdots &\v_n\em.
$$
Then the relation (\eqn) becomes
\nexteqn
$$
\x = P \x'\tag\eqn
$$
which may be thought of as a rule relating the `old' coordinates
of a vector to its `new' coordinates.  $P$ is called the
{\it change of coordinates matrix}, and its $j$th column is
$\v_j$ which may also be thought of as {\it the set of
`old' coordinates of the $j$th `new' basis vector}.

 (\eqn) is backwards in
that the `old' coordinates are expressed in terms of the `new'
coordinates.   However, it is easy to turn this around.  Since the
columns of $P$ are linearly independent, $P$ is invertible and
we may write instead
\nexteqn
$$
\x' = P^{-1}\x.
$$
These rules have been stated for the case in which we change from
the standard basis to some other basis, but they work quite
generally for any change of basis.  (They even work in cases
where there is no obvious `standard basis'.)   Just use
the rule enunciated above: the $j$th column of $P$ is
the  set of
`old' coordinates of the $j$th `new' basis vector.

\nextex
\example{Example \en}  Suppose in $\R^2$ we pick a new set of
coordinate axes  by rotating each of the old axes through angle
$\theta$ in the counterclockwise direction.  Call the old coordinates
$(x_1, x_2)$ and the new coordinates $(x_1', x_2')$.   According
to the above discussion, the columns of the 
change of basis matrix $P$  come from the old coordinates of the
new basis vectors, i.e., of unit vectors along the new axes.
From the diagram, these are
$$
\bm \cos\theta\\ \sin\theta \em\qquad \bm\format\r\\-\sin\theta\\cos\theta\em.
$$
\medskip
\centerline{\epsfbox{s12-9.ps}}
\medskip
Hence,
$$
\bm x_1\\ x_2 \em = 
\bm\format\r&\quad\r\\
       \cos\theta & -\sin\theta \\
       \sin\theta & \cos\theta \em 
\bm x_1'\\ x_2'\em.
$$
The change of basis matrix is easy to invert in this case.
(Use the special rule which applies to $2\times 2$ matrices.)
$$
\bm\format\r&\quad\r\\
       \cos\theta & -\sin\theta \\
       \sin\theta & \cos\theta \em ^{-1}
= \frac 1{\cos^2\theta + \sin^2\theta}
\bm\format\r&\quad\r\\
       \cos\theta & \sin\theta \\
       -\sin\theta & \cos\theta \em 
=
\bm\format\r&\quad\r\\
       \cos\theta & \sin\theta \\
       -\sin\theta & \cos\theta \em 
$$
(You could also have obtained this by using the matrix for
rotation through angle $-\theta$.)
Hence, we may express the `new' coordinates in terms of the `old'
coordinates through the relation
$$
\bm x'_1\\ x'_2 \em = 
\bm\format\r&\quad\r\\
       \cos\theta & \sin\theta \\
       -\sin\theta & \cos\theta \em 
\bm x_1\\ x_2\em.
$$
\endexample
\medskip
The significance of the Principal Axis Theorem is clarified
somewhat by thinking in terms of changes of coordinates.
Suppose  $A$ is diagonalizable and
$\{\v_1, \v_2, \dots, \v_n\}$ is a basis of
eigenvectors for $A$.  Suppose
$P = \bm \v_1 &\v_2 & \hdots &\v_n\em$ is the
corresponding change of basis matrix.  We
%have
%$$
%\align
%A\v_1 &= \v_1\lambda_1 
%= \bm \v_1 &\v_2 &\hdots &\v_n\em \bm\lambda_1 \\ 0\\\vdots\\0\em\\
%A\v_2 &= \v_2\lambda_2
%= \bm \v_1 &\v_2 &\hdots &\v_n\em \bm 0 \\\lambda_2 \\\vdots\\0\em\\
%&\vdots \\
%A\v_n &= \v_n\lambda_n
%= \bm \v_1 &\v_2 &\hdots &\v_n\em \bm 0 \\ 0\\\vdots\\\lambda_n\em
%\endalign
%$$
%where some eigenvalues $\lambda_j$   for different eigenvectors
%might be repeated.  These equations can be written in a single
%matrix equation
%$$
%A\bm \v_1 & \v_2 & \hdots &\v_n\em
% = \bm \v_1 & \v_2 & \hdots &\v_n\em\bm \lambda_1 & 0 & \hdots & 0\\
%0 & \lambda_2 &\hdots & 0 \\ \vdots & \vdots &\hdots &\vdots \\
%   0 & 0 & \hdots & \lambda_n \em
%$$
%or
%$$
%AP = PD
%$$
%where $D$ is a diagonal matrix with eigenvalues (possibly repeated)
%on the diagonal.  This may also be written
showed in Chapter 11, Section 8 that
\nexteqn
$$
P^{-1}AP = D\tag\eqn
$$
where  $D$ is a diagonal matrix  with the eigenvalues of $A$ on
the diagonal.
To see how this might be used,  consider 
a second order system of the form
$$
\frac{d^2\x}{dt^2} = A\x.
$$
Assume we make the change of coordinates
$$
\x = P\x'.
$$
Then
$$\gather
\frac{d^2P\x'}{dt^2} = AP\x'\\
P\frac{d^2\x'}{dt^2} = AP\x'\\
\frac{d^2\x'}{dt^2} = P^{-1}AP\x' = D\x'.
\endgather$$
However, since $D$ is diagonal, this last equation may be written
as $n$ scalar equations
$$
\frac{d^2x_j'}{dt^2} = \lambda_j x_j'\qquad j = 1, 2, \dots, n.
$$
In the original coordinates, the motions of the particles
are `coupled' since the motion of each particle
may affect the motion of the other particles.   In the new coordinate
system,
these motions are `decoupled'. 
If we do this for a normal modes problem,
  the new coordinates are called {\it normal\/}
\outind{normal mode}%
\outind{normal coordinates}%
\outind{coordinates, normal}%
coordinates. Each $x_j'$ may be thought of as the displacement
of one of $n$ fictitious particles, each of which oscillates
independently of the others in one of
$n$  mutually perpendicular directions.  The physical significance
in terms of the original particles of each normal coordinate is
a but murky, but they presumably represent underlying structure
of some importance. 

\nextex
\example{Example \en}  Recall the normal modes problem in
Section 3, Example 1.
%Back reference
\medskip
\centerline{\epsfbox{s10-1.ps}}
\medskip
Since the masses are equal, the problem
 can be reformulated as
$$
\frac{d^2\x}{dt^2} = \frac km \bm \format\r&\quad\r\\-2 & 1\\ 1 & -2 \em\x.
$$
 This doesn't change anything in the solution process, and a basis
of eigenvectors for the coefficient matrix is as before
$$
\left\{\v_1 = \bm 1\\1\em, \, \v_2 =\bm\format\r\\ -1\\1\em\right\}.
$$
If we divide the vectors by their lengths, we obtain the orthonormal
basis
$$
\left\{\frac 1{\sqrt 2}\bm 1\\1\em,\, 
\frac 1{\sqrt 2}\bm\format\r\\ -1\\1\em\right\}.
$$
This in turn leads to the change of basis matrix
$$
P = \bm\format\r&\quad\r\\
      \frac 1{\sqrt 2} & -\frac 1{\sqrt 2}\\
      \frac 1{\sqrt 2} & \frac 1{\sqrt 2} \em
$$
\medskip
\centerline{\epsfbox{s12-11.ps}}
\medskip
If you look carefully, you will see this represents a rotation of
the original $x_1, x_2$-axes through an angle $\pi/4$.  However,
this has nothing to do with the original geometry of the problem.
$x_1$ and $x_2$ stand for displacements of two different particles
along the same one dimensional axis.   The $x_1,x_2$ plane is
a fictitious configuration space in which a single point represents
the pair of particles.   It is not absolutely clear what a rotation
of axes means for this plane, but the new normal coordinates $x_1', x_2'$
obtained thereby give us a formalism in which the normal modes appear
as decoupled oscillations.
\endexample

\medskip
\subhead Orthogonal and Unitary Matrices \endsubhead
You may have noticed that the matrix $P$ obtained in Example \en\ %
has the property $P^{-1} = P^t$.   This is no accident.  It is
a consequence of the fact that its columns are mutually perpendicular
unit vectors.

\nextthm
\proclaim{Theorem \cn.\tn}  Let $P$ be an $n\times n$ real matrix.
Then the columns of $P$ form an orthonormal basis for $\R^n$
if and only if $P^{-1} = P^t$.  Similarly if $P$ is an $n\times n$
complex matrix, its columns form an orthonormal basis for
$\C^n$ if and only if $P^{-1} = \overline P^t$.
\endproclaim

A matrix with this property is called {\it orthogonal\/} in
the real case and {\it unitary\/} in the complex case.  The complex
case subsumes the real case since a real matrix is unitary if and
only if it is orthogonal.
\outind{orthogonal matrix}%
\outind{unitary matrix}%

\demo{Proof of the Theorem}  We consider the real case.
(The argument in the complex case is similar except that dot
products need a complex conjugation on the first factor.)  
 Let
$$
P = \bm \v_1 & \v_2 & \hdots & \v_n \em.
$$
Then
$$
 P^t = \bm \v_1{}^t\\
                    \v_2{}^t \\
                     \vdots \\
                    \v_n{}^t \em.
$$
Hence, the $j,k$-term of the product $P^tP$ is
$$
 \v_j{}^t\v_k = (\v_j, \v_k)
$$
Thus, $P^tP = I$ if and only if
$$
(\v_j, \v_k) = \delta_{jk}
$$
where $\delta_{jk}$, the `Kronecker $\delta$', gives the entries of
the identity matrix.   However, this just says that the vectors
are mutually perpendicular (for $j\not=k$) and have length
1 (for $j = k$).
\qed\enddemo

Since the Principal Axis Theorem asserts that there is an
{\it orthonormal\/} basis consisting of eigenvectors
for the Hermitian matrix $A$, that means that the corresponding
change of basis matrix is always unitary (orthogonal in the
real case).  Putting this in (\eqn), we get the following
equivalent form of the Principal Axis Theorem.   

\nextthm
\proclaim{Theorem \cn.\tn} 
If $A$ is a complex Hermitian $n\times n$ matrix, there is
a unitary matrix $P$ such that
$$
	\overline P^tAP = P^{-1}AP = D
$$
is diagonal.  If $A$ is real symmetric, $P$ may be chosen
to be real orthogonal. 
\endproclaim

\subhead The Proof of the Principal Axis Theorem \endsubhead
\demo{}
We know that we can always find a basis for $\C^n$ of generalized
eigenvectors for a complex $n\times n$ matrix $A$.    The point of the
Principal Axis Theorem is that if $A$ is Hermitian, ordinary
eigenvectors suffice.   The issue of orthogonality may be dealt with
\outind{Principal Axis Theorem, proof}%
separately  since, for a Hermitian matrix,
 eigenvectors for different eigenvalues are
perpendicular and the Gram--Schmidt Process is available for repeated
eigenvalues.   Unfortunately there does not seem to be a simple
direct way to eliminate the possibility of generalized eigenvectors
which are not eigenvectors.   The proof we shall give proceeds by
induction, and it shares with many inductive proofs the feature that,
while you can see that it is correct, you may not find it too
enlightening. 
  You might want to skip the proof the first time you
study this material.

We give the proof in the real case.  The only difference in the
complex case is that you  need to put complex conjugates over
the appropriate terms in the formulas.

Let $A$ be an $n\times n$ symmetric matrix.   We shall show that
there is a real orthogonal $n\times n$ matrix $P$ such that
$$
AP = PD\qquad\text{or equivalently}\qquad P^tAP = D
$$
where $D$ is a diagonal matrix with the eigenvalues of $A$
(possibly repeated) on its diagonal.

If $n = 1$ there really isn't anything to prove.  (Take $P = \bm 1 \em$.)
Suppose the theorem has been proved for $(n-1)\times(n-1)$ matrices.
Let $\u_1$ be a unit eigenvector for $A$ with eigenvalue $\lambda_1$.
Consider the subspace $W$ consisting of all vectors perpendicular
to $\u_1$.  It is not hard to see that $W$ is an $n-1$ dimensional
subspace.   Choose (by the Gram--Schmidt Process) an orthonormal
basis $\{\w_2, \w_2\dots, \w_n\}$ for $W$.   Then
$\{\u_1,\w_2, \dots, \w_n\}$ is an orthonormal basis for $\R^n$,
and
$$
A \u_1 = \u_1\lambda_1 =
\undersetbrace{P_1}\to{\bm \u_1 &\w_2 &\hdots &\w_n\em}
\bm\lambda_1\\0\\\vdots\\0\em.
$$
This gives the first column of $AP_1$, and we want to say something
about its remaining columns 
$$
A\w_2, \quad A\w_2,\dots, A\w_n.
$$
To this end, note that if $\w$ is any vector in $W$, then
$A\w$ is also a vector in $W$.  For, starting with the
{\it self adjoint property\/} (Section 4, Problem A3),
%Reference to HW problem
we have
$$
   (\u_1, A\w) = (A\u_1, \w) = (\lambda_1\u_1, \w) = \lambda_1(\u_1, \w)
       = 0,
$$
which is to say, $A\w$ is perpendicular to $\u_1$ if $\w$ is
perpendicular to $\u_1$.    It follows that each $A\w_j$ is a
linear combination just of $\w_2, \w_3, \dots, \w_n$, i.e.,
$$
A\w_j = \bm \u_1 & \w_2 & \hdots & \w_n\em
        \bm 0 \\ * \\ \vdots \\ * \em
$$
where `$*$' denotes some unspecified entry.  Putting this all
together, we see that
$$ 
AP_1 = P_1\undersetbrace{A_1}\to{
\bm \lambda_1 & 0 & \hdots & 0 \\
    0 & {} & {} & {} \\
    \vdots &{} & A' & {}\\
     0 & {}& {}& {} \em}
$$
where $A'$ is an $(n-1)\times(n-1)$ matrix.   $P_1$ is orthogonal
(since its columns form an orthonormal basis) so
$$
P_1{}^tA P_1 = A_1,
$$
and it  is not hard to derive from this the fact that $A_1$
is symmetric.   Because of the structure of $A_1$, this
implies that $A'$ is symmetric.   Hence, by induction we
may assume there is an $(n-1)\times(n-1)$ orthogonal matrix
$P'$ such that
$A'P' = P'D'$ with $D'$ diagonal.   It follows that
$$\align
A_1
\undersetbrace{P_2}\to{\bm 1 & 0 & \hdots & 0 \\
    0 & {} & {} & {} \\
    \vdots &{} & P' & {}\\
     0 & {}& {}& {} \em}
&= \bm \lambda_1 & 0 & \hdots & 0 \\
    0 & {} & {} & {} \\
    \vdots &{} & A' & {}\\
     0 & {}& {}& {} \em
\bm 1 & 0 & \hdots & 0 \\
    0 & {} & {} & {} \\
    \vdots &{} & P' & {}\\
     0 & {}& {}& {} \em \\
&= \bm \lambda_1 & 0 & \hdots & 0 \\
    0 & {} & {} & {} \\
    \vdots &{} & A'P' & {}\\
     0 & {}& {}& {} \em 
 = \bm \lambda_1 & 0 & \hdots & 0 \\
    0 & {} & {} & {} \\
    \vdots &{} & P'D' & {}\\
     0 & {}& {}& {} \em \\
&= 
\undersetbrace{P_2}\to{\bm 1 & 0 & \hdots & 0 \\
    0 & {} & {} & {} \\
    \vdots &{} & P' & {}\\
     0 & {}& {}& {} \em}
\undersetbrace{D}\to{\bm \lambda_1 & 0 & \hdots & 0 \\
    0 & {} & {} & {} \\
    \vdots &{} & D' & {}\\
     0 & {}& {}& {} \em}
= P_2 D.
\endalign
$$
Note that $P_2$ is orthogonal and $D$ is diagonal.
Thus,
$$
\gather
A\undersetbrace{P}\to{P_1P_2} = P_1A_1P_2 = 
\undersetbrace{P}\to{P_1P_2}D\\
\text{or}\qquad AP = PD.
\endgather
$$
However, a product of orthogonal matrices is orthogonal---see
the Exercises---so $P$ is orthogonal as required.
\qed\enddemo
\bigskip
\input chap12.ex6
\bigskip

\nextsec{Classification of Conics and Quadrics}
\head \sn.  Classification of Conics and Quadrics \endhead

As mentioned earlier, the Principal Axis Theorem derives its name
from its relation to classifying conics, quadric surfaces,
and their higher dimensional analogues.

A level curve in $\R^2$ defined by an equation of the form
$$
f(\x) = a_{11}x_1{}^2 + 2a_{12}x_1x_2 + a_{22}x_2{}^2 = C
$$
is called a {\it central conic\/}.   (The reason for the 2
will be clear shortly.)  
   As we shall see, a central conic
is either an ellipse or a hyperbola  (for which the principal axes
need not be the coordinate axes) or a degenerate `conic'
consisting of a pair of lines.
\outind{conic, central}%
\outind{central conic or quadric}%
\outind{quadric, central}%

The most general conic
is the locus of an arbitrary quadratic equation which may have
linear as well as quadratic terms.   Such
curves may be studied by applying the methods
discussed
in this section to the
quadratic terms and then completing squares to
eliminate linear terms.   Parabolas are included  in the
theory in this way.

  To study a central conic,
it is convenient to express the function $f$ as follows.
$$\align
f(\x) &= (x_1a_{1,1} + x_2a_{21})x_1 + (x_1a_{12} + x_2a_{22})x_2\\
      &= x_1(a_{11}x_1 + a_{12}x_2) + x_2(a_{21}x_1 + a_{22}x_2),
\endalign$$
where we have introduced $a_{21} = a_{12}$.   
The above expression may also be written in matrix form
$$
  f(\x) = \sum_{j,k=1}^2 x_ja_{jk}x_k = \x^tA\x
$$
where $A$ is the symmetric matrix of coefficients.

This may be generalized to $n > 2$ in a rather obvious manner.
Let $A$ be a real symmetric $n\times n$ matrix, and define  
$$
f(\x) = \sum_{j, k = 1}^n x_ja_{jk}x_k = \x^tA\x. 
$$
For $n=3$ this may be written explicitly 
$$\align
f(\x) &= (x_1a_{11} + x_2a_{21} + x_3a_{31})x_1 \\
      &\hphantom{=} + (x_1a_{12} + x_2a_{22} + x_3a_{32})x_2\\
      &\hphantom{=} + (x_1a_{13} + x_2a_{23} + x_3a_{33})x_3\\
  &= a_{11}x_1{}^2 + a_{22}x_2{}^2 + a_{33}x_3{}^2 \\ 
  &\hphantom{=} + 2a_{12}x_1x_2 + 2a_{13}x_1x_3 + 2a_{23}x_2x_3.
\endalign$$

The level set defined by 
$$
f(\x) = C
$$
is called a {\it central hyperquadric}.  It should be visualized
\outind{hyperquadric}%
as an $n-1$ dimensional curved object in $\R^n$.  For $n = 3$
it will be an ellipsoid or a hyperboloid (of one or two sheets)
or perhaps a degenerate `quadric' like a cone.   (As in the case
of conics, we must also allow linear terms 
to encompass paraboloids.)

If the above contentions are true, we expect the locus of
the equation $f(\x) = C$ to have certain axes of symmetry
which we shall call its {\it principal axes}.   It turns out
that these axes are determined by an {\it orthonormal
basis of eigenvectors\/} for the coefficient matrix $A$.
To see this, suppose $\{\u_1, \u_2, \dots, \u_n\}$ is
such a basis and $P = \bm \u_1 & \u_2 & \hdots & \u_n \em$
is the corresponding orthogonal matrix.  By the Principal
Axis Theorem,  $P^tAP = D$ is diagonal with the eigenvalues,
$\lambda_1, \lambda_2, \dots, \lambda_n$,
of $A$
 appearing on the
diagonal.    Make the change of coordinates  $\x = P\x'$
where $\x$ represents the `old' coordinates and
$\x'$ represents the `new' coordinates.   Then
$$
f(\x) = \x^tA\x = (P\x')^tA(P\x') = (\x')^tP^tAP\x' = (\x')^tD\x'.
$$
Since $D$ is diagonal, the quadratic expression on the right has
no cross terms, i.e.
$$
\align
(\x')^tD\x' &= \bm x_1'& x_2' & \cdots x_n' \em
 \bm \lambda_1 & 0 & \cdots & 0 \\
    0 & \lambda_2 & \cdots & 0 \\
    \vdots & \vdots & \cdots & \vdots \\
     0 & 0 & \cdots & \lambda_n \em \bm x_1' \\ x_2' \\ \vdots \\ x_n'\em \\
   &= \lambda_1(x_1')^2 + \lambda_2(x_2')^2 + \dots + \lambda_n(x_n')^2.
\endalign $$ 
In the new coordinates, the equation takes the form
$$
\lambda_1(x_1')^2 + \lambda_2(x_2')^2 + \dots + \lambda_n(x_n')^2 = C
$$
and its graph is usually quite easy to describe.

\nextex
\xdef\ConEx{\en}
\example{Example \en} 
We shall determine the level curve  $f(x,y) = x^2 + 4xy + y^2 = 1$.  
First rewrite the equation
$$
\bm x & y \em \bm 1 & 2 \\ 2 & 1 \em \bm x\\ y \em = 1.
$$
Next,
 find the eigenvalues of the coefficient matrix by solving
$$
\det \bm 1 - \lambda & 2 \\ 2 & 1 - \lambda \em
= (1 - \lambda)^2 - 4 = \lambda^2 - 2\lambda -3 = 0.
$$
This equation is easy to factor, and the roots are
$\lambda = 3, \lambda = -1$.

For $\lambda = 3$, to find the eigenvectors,  we need to solve
$$
\bm\format\r&\quad\r\\-2 & 2 \\ 2 & -2 \em\bm v_1\\ v_2 \em = 0.
$$
Reduction of the coefficient matrix yields 
$$
\bm\format\r&\quad\r\\-2 & 2 \\ 2 & -2 \em \to
\bm\format\r&\quad\r\\1 & -1 \\ 0 & 0 \em 
$$
with the general solution $v_1 = v_2$,  $v_2$ free.   
A basic {\it normalized\/} eigenvector is
$$
\u_1 = \frac 1{\sqrt 2}\bm 1 \\ 1 \em.
$$

For $\lambda = -1$,  a similar calculation (which you should
make) yields the basic normalized eigenvector
$$
\u_2 = \frac 1{\sqrt 2}\bm\format\r\\ -1 \\ 1 \em.
$$
(Note that $\u_1 \perp \u_2$ as expected.)

From this we can form the corresponding orthogonal
matrix $P$ and make the change of coordinates
$$
\bm x \\ y \em = P\bm x'\\ y' \em,
$$
and, 
according to the above analysis, the equation of the level curve
in the new coordinate system
is
$$
3(x')^2 - (y')^2 = 1.
$$
It is clear that this is a hyperbola with principal axes pointing along
the new axes.  
\medskip
\centerline{\epsfbox{s12-16h.ps}}
\medskip
\endexample
\nextex
\example{Example \en}
Consider the quadric surface defined by
$$
x_1{}^2 + x_2{}^2 + x_3{}^2 - 2x_1x_3 = 1.
$$
We take 
$$
f(\x) = 
x_1{}^2 + x_2{}^2 + x_3{}^2 - 2x_1x_3 
 = \bm x_1 & x_2 & x_3 \em
 \bm\format\r&\quad\r&\quad\r\\
             1 & 0 & -1\\ 
             0 & 1 & 0 \\
             -1 & 0 & 1 \em
\bm x_1\\ x_2 \\ x_3 \em.
$$
The characteristic equation of the coefficient matrix is
$$
\det 
 \bm 1-\lambda & 0 & -1\\
     0 & 1-\lambda & 0 \\
    -1 & 0 & 1-\lambda \em
= (1 - \lambda)^3 - (1 - \lambda) = -(\lambda - 2)(\lambda - 1)\lambda = 0
$$
Thus, the eigenvalues are $\lambda = 2, 1, 0$.   

For $\lambda = 2$, reduce
$$
\bm\format\r&\quad\r&\quad\r\\
     -1 & 0 & -1 \\
     0 & -1 & 0 \\
    -1 &  0 & -1 \em 
\to \bm\format\r&\quad\r&\quad\r\\
     1 & 0 & 1 \\
     0 & 1 & 0 \\
    0 & 0 & 0 \em
$$
to obtain $v_1 = -v_3, v_2 = 0$ with $v_3$ free.
Thus,
$$
\v_1 = \bm\format\r\\ -1 \\ 0 \\ 1\em
$$
is a basic eigenvector for $\lambda = 2$, and
$$
\u_1 = \frac 1{\sqrt 2}\bm\format\r\\ -1 \\ 0 \\ 1\em
$$
is a basic unit eigenvector.

Similarly, for $\lambda = 1$ reduce
$$
\bm\format\r&\quad\r&\quad\r\\
     0 & 0 & -1 \\
     0 & 0 & 0 \\
    -1 &  0 & 0 \em 
\to \bm\format\r&\quad\r&\quad\r\\
     1 & 0 & 0 \\
     0 & 0 & 1 \\
    0 & 0 & 0 \em
$$
which yields $v_1 = v_3 = 0$ with $v_2$ free.  Thus a basic unit
eigenvector for $\lambda = 1$ is
$$
\u_2 = \bm 0\\1\\ 0\em.
$$

Finally, for $\lambda = 0$, reduce
$$
\bm\format\r&\quad\r&\quad\r\\
     1 & 0 & -1 \\
     0 & 1 & 0 \\
    -1 &  0 & 1 \em 
\to \bm\format\r&\quad\r&\quad\r\\
     1 & 0 & -1 \\
     0 & 1 & 0 \\
    0 & 0 & 0 \em.
$$
This yields $v_1 = x_3, v_2 = 0$ with $v_3$ free.  Thus, a
basic unit eigenvector for $\lambda = 0$ is
$$
\u_3 = \frac 1{\sqrt 2}\bm 1\\ 0 \\ 1\em.
$$

The corresponding orthogonal change of basis matrix is   
$$
P = \bm \u_1 &\u_2 & \u_3 \em = 
\bm\format\r&\quad\r&\quad\r\\
     -\frac 1{\sqrt 2} & 0 & \frac 1{\sqrt 2} \\
          0         & 1 &   0 \\
      \frac 1{\sqrt 2} & 0 & \frac 1{\sqrt 2}\em.
$$
Moreover, putting
$\x = P\x'$,
we can express the
equation of the quadric surface in the new coordinate system
\nexteqn
$$
  2x_1'{}^2 + 1x_2'{}^2 + 0x_3'{}^2
   =  2x_1'{}^2 + x_2'{}^2 = 1.\tag\eqn
$$
Thus it is easy to see what the level surface is: an elliptical cylinder
perpendicular to the $x_1', x_2'$ plane.    The three `principal
axes' in this case are the two axes of the ellipse in the
$x_1', x_2'$ plane and the $x_3'$ axis, which is the central
axis of the cylinder.

Representing the graph in the new coordinates makes it easy to
understand its geometry.   Suppose, for example, that we want
to find the points on the graph which are closest to the origin.
These are the points at which the $x_1'$-axis intersects the
surface.   
These are the points with new coordinates
$x_1' = \pm \dfrac 1{\sqrt 2}, x_2' = x_3' = 0$. 
If you want the coordinates of these points in the original
coordinate system, use the change of coordinates formula
$$
\x = P\x'.
$$
Thus, the old coordinates of the minimum point with
new coordinates  $(1/\sqrt 2, 0, 0)$ are given by
$$
\bm\format\r&\quad\r&\quad\r\\
     -\frac 1{\sqrt 2} & 0 & \frac 1{\sqrt 2} \\
          0         & 1 &   0 \\
      \frac 1{\sqrt 2} & 0 & \frac 1{\sqrt 2}\em
\bm \frac 1{\sqrt 2}\\ 0 \\ 0 \em
= \bm\format\r\\ -\frac 12\\ 0\\ \frac 12\em.
$$
\endexample

% These examples are typical.   Let  $A$ be a real symmetric $n\times n$
% matrix, $\{\u_1,\u_2,\dots,\u_n\}$ an orthonormal basis of
% eigenvectors, and $P = \bm \u_1 &\u_2 &\hdots &\u_n\em$.  
% Make the change of coordinates
% $\x = P\x'$. In the new coordinate system,
%  the level set defined by $f(\x) = \x^tA\x = C$ has the equation
% $$
% \lambda_1x_1'{}^2 + \lambda_2x_2'{}^2 + \dots +\lambda_nx_n'{}^2 = C.
%  $$
% The coefficients are the eigenvalues associated with the basis vectors
% $$\u_1, \u_2, \dots, \u_n,$$
% in the order given.  In the new coordinate system, it is 
%  easy to understand the geometry of the conic or quadric or
% hyperquadric.  Finally, you may use the equation $\x = P^t\x'$
% to express information so obtained in the original coordinates.

\bigskip
\input chap12.ex7
\bigskip

\nextsec{A Digression on Constrained Maxima and Minima}
\head \sn.  A Digression on Constrained Maxima and Minima \endhead

There is another approach to finding the principal axes of a
conic, quadric, or hyperquadric.   Consider for an example
an ellipse in $\R^2$ centered at the origin.  One of the
principal axes intersects the conic in the two points
at greatest distance from the origin, and the other
intersects it in the two points at least distance from the origin.
Similarly, two of the three principal axes of a central ellipsoid in
$\R^3$ may be obtained in this way.   Thus, if we didn't know about
eigenvalues and eigenvectors, we might try to find the
principal axes by
 maximizing (or minimizing) the
function  giving the distance to the origin
 {\it subject to\/} the quadratic equation
defining the conic or quadric.   In such a problem, we need to
minimize a function assuming there are one or more relations
or {\it constraints\/} among the variables.   In this section
\outind{constraint}%
we shall consider problems of this kind in general.   


We start by considering the case of a single constraint.
  Suppose we want to maximize (minimize)
\outind{maxima and minima with constraints}%
the real valued function  $f(\x) = f(x_1,x_2,\dots,x_n)$
subject to the constraint $g(\x) = g(x_1,x_2,\dots,
x_1) = c$.   For $n = 2$, this has a simple geometric interpretation.
The locus of the equation $g(x_1, x_2) = c$ is a level curve of the function $g$,
and we want to maximize (minimize) the function $f$ {\it on that
curve.}   Similarly, for $n = 3$, the level set  $g(x_1,x_2.x_3) = c$
is a surface in $\R^3$, and we want to
maximize (minimize) $f$ {\it on that surface}.  In $\R^n$, we call the
level set defined by $g(x_1,x_2,\dots, x_n) = c$ a 
{\it hypersurface\/}, and the problem is to maximize (minimize) the function
\outind{hypersurface}%
$f$ {\it on that hypersurface}.
\medskip
\centerline{\epsfbox{s12-12.ps}}
\medskip
\example{Examples}
Maximize $f(x,y) = x + 3y$ on the hyperbola  $g(x,y) = x^2 - y^2 = 1$.

Maximize $f(x,y) = x^2 + y^2$ on the ellipse $g(x,y) = x^2 + 4y^2 = 3$.  (This is
easy if you draw the picture.) 

Minimize $f(x,y,z) = 2x^2 + 3xy + y^2 + xz - 4z^2$ on the sphere
$g(x,y,z) = x^2 + y^2 + z^2 = 1$.

Minimize $f(x,y,z,t) = x^2 + y^2 + z^2 - t^2$ on the hypersphere
$g(x,y,z,t) = x^2 + y^2 + z^2 + t^2 = 1$.
\endexample


We shall concentrate on the case of $n = 3$ variables,
but the reasoning for any $n$ is similar.
We want to maximize (or minimize) $f(\x)$
on a level set  $g(\x) = c$ in $\R^3$, where as usual we 
abbreviate 
$\x = (x_1, x_2, x_3)$.
 Assume that both $f$ and $g$ are smooth functions
defined on open sets in $\R^3$,
and that the level set $g(\x) = c$ has a well defined
tangent plane at a potential maximum point.
The latter assumption means that the normal vector $\nabla g$
does not vanish at the point.   It follows from this
assumption that every vector $\v$
perpendicular to $\nabla g$ at the point is a tangent vector
for some curve in the level set passing through the point.  (Refer back
to the discussion of tangent planes and the implicit function theorem
in Chapter III, Section 8.)
% Back reference
  Suppose such a curve is given by the parametric
representation $\x = \x(t)$.
\medskip
\centerline{\epsfbox{s12-13.ps}}
\medskip
   By the chain rule we have
$$
\frac{df}{dt} = \nabla f\cdot\frac{d\x}{dt} = \nabla f\cdot\v
$$
where $\v = d\x/dt$.   If the function attains a maximum on the level set
at the given point, it also attains a maximum along this curve, so
we conclude that
$$
\frac{df}{dt} = \nabla f\cdot \v = 0.
$$
As above, we can arrange that the vector $\v$ is any possible
vector in the tangent plane at the point.    Since there is a
unique direction perpendicular to the tangent plane, that of $\nabla g$,
we conclude that $\nabla f$ is parallel to $\nabla g$, i.e.,
\nexteqn
$$
\nabla f(\x) = \lambda \nabla g(\x)\tag\eqn
$$
for some scalar $\lambda$.
\medskip
\centerline{\epsfbox{s12-14.ps}}
\medskip
   (\eqn) is a {\it necessary condition\/}
which must hold at any maximum point where $f$ and $g$ are smooth
and $\nabla g \not= 0$.  (It doesn't by itself guarantee
that there is a maximum at the point.   There could be a minimum
or even no extreme value at all at the point.)  Taking components, 
we obtain 3 scalar equations
for the 4 variables  $x_1, x_2, x_3, \lambda$.   We would not expect,
even in the best of circumstances to get a unique solution from this,
but the defining equation for the level surface
\nexteqn
$$
g(\x) = c
$$
provides a 4th equation.   We still won't generally get a unique
solution, but we will usually get at most a finite number of possible
solutions.   Each of these can be examined further to see if $f$
attains a maximum (or minimum) at that point in the level set.
Notice that the variable $\lambda$ plays an auxiliary role since
we really only want the coordinates of the point $\x$.  (In some
applications, $\lambda$ has some significance beyond that.)
This method
is due to the 19th century French mathematician Lagrange and
$\lambda$ is called a {\it Lagrange multiplier}.
\outind{Lagrange multiplier}%

\nextex
\example{Example \en}
Suppose we want to
maximize the function $f(x,y,z) = x + y - z$ on the
sphere $x^2 + y^2 + z^2 = 1$.  We take $g(x,y,z) = x^2 + y^2 + z^2$.
Then, $\nabla f = \lb 1, 1, -1 \rb$ and $\nabla g = \lb 2x, 2y, 2z \rb$,
so the relation $\nabla f = \lambda \nabla g$ yields
$$
\align
1 &= \lambda(2x)\\
1 &= \lambda (2y) \\
-1 &= \lambda (2z)\\
\intertext{to which we add the equation}
x^2 + y^2 + z^2 &= 1. 
\endalign$$
From the first three equations, we obtain
$$\gather
x =\frac 1{2\lambda} \quad y = \frac 1{2\lambda}
\quad z = -\frac 1{2\lambda} \\
\frac 1{4\lambda^2} + \frac 1{4\lambda^2} + \frac 1{4\lambda^2} = 1 \\
   \frac 34 = \lambda^2 \\
  \lambda = \pm \frac{\sqrt 3}2.
\endgather$$
Thus we have two possible solutions.  For $\lambda = \sqrt 3/2$,
we obtain  the point 
$$(1/\sqrt 3, 1/\sqrt 3, -1/\sqrt 3)
\qquad\text{at which }f = x + y -z = \sqrt 3.$$
   For $\lambda = -\sqrt 3/2$,
we obtain the point 
$$(-1/\sqrt 3, -1/\sqrt 3, 1/\sqrt 3)
\qquad\text{at which }f = -\sqrt 3.$$
\medskip
\centerline{\epsfbox{s12-15.ps}}
\medskip
    Since the level
set $x^2 + y ^2 + z^2 = 1$ is a closed bounded set, and since
the function $f$ is continuous, both maximum and minimum values
must be attained somewhere on the level set.  The only two candidates
we have come up with are the two points given above, so it is
clear the first is a maximum point and the second is a minimum
point.
\endexample

The method of Lagrange multipliers often leads to a set of equations
which is difficult to solve.  
Sometimes a great deal of ingenuity is required, so you should
treat each problem as unique and expect to have to be creative
about solving it.

\nextex
\xdef\Eex{\en}
\example{Example \en} 
Suppose we want to minimize the function $f(x,y) = x^2 + 4xy + y^2$  
on the circle $x^2 + y^2 = 1$.  For this problem $n = 2$,
and the level set is a curve.   Take $g(x,y) = x^2 + y^2$.
Then $\nabla f = \lb 2x + 4y, 4x + 2y \rb$, $\nabla g
= \lb 2x, 2y \rb$, and $\nabla f = \lambda \nabla g$
yields the equations
$$\align
2x + 4y &= \lambda(2x) \\
 4x + 2y &= \lambda(2y)\\
\intertext{to which we add}
x^2 + y^2 = 1.
\endalign$$
After canceling a common factor of
2, the first two equations may be written in matrix form
$$
\bm 1 & 2 \\ 2 & 1 \em \bm x\\y\em
 = \lambda \bm x\\ y\em
$$
which says that 
$$
\bm x \\ y \em
$$
is an {\it eigenvector\/} for the 
 eigenvalue $\lambda$, and the 
equation $x^2 + y^2 = 1$ says it is a {\it unit
eigenvector}.   You should know how to
solve such problems, and we leave it to you to make the
required calculations.
(See also Example \ConEx\ in the previous section where we made 
these  calculations in another context.)
The eigenvalues are
$\lambda = 3$ and $\lambda = -1$.
For $\lambda = 3$,  
a basic unit eigenvector is
$$
\u_1 = \frac 1{\sqrt 2}\bm 1\\ 1\em,
$$
and every other eigenvector is of the form $c\u_1$.   The latter
will be a {\it unit\/} vector if and only $|c| = 1$, i.e.,
$c = \pm 1$.
We conclude that $\lambda = 3$ yields 
 two solutions of the Lagrange mulitplier problem:
 $(1/\sqrt 2, 1/\sqrt 2)$ and
 $(-1/\sqrt 2, -1/\sqrt 2)$.   At each of these points
$f(x,y) = x^2 + 4xy + y^2 = 3$.

For $\lambda = -1$,  we obtain the basic unit eigenvector
$$
\u_2 = \frac 1{\sqrt 2} \bm\format\r\\-1\\ 1\em,
$$
and a similar analysis (which you should
do) yields
the two points:  $(1/\sqrt 2, -1/\sqrt 2)$ and
 $(-1/\sqrt 2, 1/\sqrt 2)$.   At each of these points
$f(x,y) = x^2 + 4xy + y^2 = -1$. 
\medskip
\centerline{\epsfbox{s12-16.ps}}
\medskip
Hence, the function attains its maximum value at  the first
two points and its minimum value at  the second two.
\endexample
\nextex
\example{Example \en} 
Suppose we want to minimize the function 
$g(x,y) = x^2 + y^2$ (which is the square of the distance to the origin)
on the conic
$f(x,y) = x^2 + 4xy + y^2 = 1$. 
Note that this is basically the same as the previous example
except that the roles of the two functions are reversed.   The
Lagrange mulitplier condition
$\nabla g = \lambda \nabla f$ is the same
as the condition $\nabla f = (1/\lambda)\nabla g$
provided $\lambda \not=0$.  ($\lambda \not= 0$ in this
case since otherwise $\nabla g = 0$,
which yields  $x = y = 0$.  However, $(0, 0)$ is
not a point on the conic.)   We just solved that problem
and found  eigenvalues $1/\lambda = 3$ or $1/\lambda = -1$. In this
case, we don't need unit eigenvectors, so to avoid square roots
we choose 
basic eigenvectors 
$$
\v_1 = \bm 1\\ 1\em\qquad\text{and}\qquad\bm\format\r\\ -1\\ 1 \em
$$
corresponding respectively to $\lambda = 3$ and $\lambda = -1$.
The endpoint of $\v_1$ does not lie on the conic, but any other
eigenvector for $\lambda = 3$ is of the form $c\v_1$, so
all we need to do is adjust $c$ so that the point satisfies
the equation
$f(x,y) = x^2 +4xy + y^2 = 1$.   Substituting $(x, y) = (c, c)$
yields
$6c^2 = 1$  or $c = \pm 1/\sqrt 6$.   Thus, we obtain
 the two points
$(1/\sqrt 6, 1/\sqrt 6)$ and
$(-1/\sqrt 6, -1/\sqrt 6)$.
For $\lambda = -1$, substituting $(x,y) = (-c, c)$ in
the equation yields
 $-2c^2 = 1$ which has no solutions.

Thus, the only candidates for a minimum (or maximum) are the
first pair of points:  $(1/\sqrt 6, 1/\sqrt 6)$ and
$(-1/\sqrt 6, -1/\sqrt 6)$.   A simple calculation shows these
are both
 $1/\sqrt 3$ units from the origin, but without further analysis,
we can't tell if this is the maximum, the minimum,
or neither.
However, it is not hard to classify this conic---see the previous
section---and discover that it is a hyperbola.  Hence, the two
points are minimum points.
\medskip

\subhead The Rayleigh-Ritz Method \endsubhead
   Example \Eex\ above is typical of a certain class of
Lagrange multiplier problems.
Let  $A$ be
 a real symmetric $n\times n$ matrix, and consider the problem
of maximizing (minimizing)
  quadratic
function  $f(\x) = \x^tA\x$ subject to
the constraint $g(\x) = |\x|^2 = 1$.
This is called the {\it Rayleigh--Ritz problem}.
\outind{Rayleigh--Ritz problem}%
For $n = 2$ or $n = 3$, the level set $|\x|^2 = 1$ is a circle or
sphere, and for $n > 3$, it is called a {\it hypersphere}.

Alternatvely, we could reverse
the roles of the functions $f$ and $g$, i.e., we could try to maximize
(minimize)
the square of the distance to the origin
$g(\x) = |\x|^2$
on the level set
$f(\x) = 1$.
Because the Lagrange multiplier condition in either case asserts
that the two
gradients $\nabla f$ and $\nabla g$  are parallel,
these two problems are very closely related.
The latter problem---finding the points on a conic, quadric, or
hyperquadric furthest from (closest to) the origin---is easier
 to visualize, but the
former
 problem---maximizing or minimizing the quadratic function
 $f$ on the hypersphere $|\x| = 1$---is easier to compute with. 

Let's go about applying the Lagrange Multiplier
method to the Rayleigh--Ritz
problem.
The components
of $\nabla g$ are easy:
$$\frac{\d g}{\d x_i} =  2x_i,\qquad i = 1, 2, \dots n.$$
The calculation of $\nabla f$ is harder.   
First write
$$
f(\x) = \sum_{j=1}^nx_j(\sum_{k=1}^n a_{jk}x_k)
$$
and then carefully apply the product rule together with
$a_{jk} = a_{kj}$.   The result is
$$
\frac{\d f}{\d x_i} = 2\sum_{j=1}^n a_{ij}x_j\qquad i = 1, 2, \dots, n.
$$
(Work this out explicitly in the cases $n = 2$ and $n = 3$ if you
don't believe it.)
Thus, the Lagrange multiplier condition $\nabla f = \lambda \nabla g$
yields the equations
$$
2 \sum_{j=1}^n a_{ij}x_j = \lambda(2x_i)\qquad i = 1, 2, \dots, n
$$
which may be rewritten in matrix form (after canceling the 2's)
\nexteqn
$$
A\x = \lambda \x.\tag\eqn
$$
To this we must add the equation of the level set
$$
g(\x) = |\x|^2 = 1.
$$
Thus, any potential solution $\x$ is a {\it unit\/} eigenvector for the
matrix $A$ with eigenvalue $\lambda$.   
Note also that for such a unit eigenvector, we have
$$
f(\x) = \x^tA\x = \x^t(\lambda\x) = \lambda\x^t\x = \lambda|\x|^2 =
\lambda.
$$
Thus the eigenvalue is the extreme value of the quadratic function at the
point on the (hyper)sphere given by the unit eigenvector. 

The upshot of this discussion is that for a real symmetric matix
$A$, the Rayleigh--Ritz problem is equivalent to the problem
of finding an orthonormal basis of eigenvectors for $A$.

\medskip
The Rayleigh--Ritz method may be used to show that a real
symmetric matrix has real eigenvalues without invoking the use
of complex vectors as we did previously in Section 4.  (See
Theorem 12.1.)
Here is an outline of the argument.
The hypersphere $g(\x) = |\x|^2 = 1$ is a closed bounded set in
$\R^n$ for any $n$. 
It follows from a basic theorem in analysis that any continuous
function, in particular the quadratic function
 $f(\x)$, must attain both maximum and minimum values on the
hypersphere.
Hence, the Lagrange multiplier problem always has solutions,
which by the above algebra amounts to the
assertion that the real symmetric matrix
$A$ must have at least one eigenvalue.   This suggests a general
procedure for showing that all the eigenvalues are real.
First find the largest eigenvalue by maximizing the quadratic
function $\f(\x)$ on the set
$|\x|^2 = 1$.   Let $\x = \u_1$ be the corresponding eigenvector.  
Change coordinates by choosing an orthonormal basis
starting with $\u_1$.  Then the additional basis elements will span
the subspace perpendicular to $\u_1$ and we may obtain a lower dimensional
quadratic function by restricting $f$ to that subspace.   We can now
repeat the process to find the next smaller real eigenvalue.
Continuing in this way, we will obtain an orthonormal basis of
eigenvectors for $A$ and each of the corresponding eigenvalues
will be real.

    The Rayleigh--Ritz Method generalizes
nicely for complex Hermitian matrices
and also for  infinite dimensional analogues.   In quantum mechanics,
for example, one considers complex valued functions $\psi(x,y,z)$
defined on $\R^3$ satisfying the condition
$$
\iiint_{\R^3}|\psi(x,y,z)|^2 dV < \infty.
$$
Such functions are called {\it wave functions\/}, and the
set of all such functions form a complex vector space.  Certain
operators $A$ on this vector space represent observable
quantities, and the eigenvalues of these operators represent
the possible results of measurements of these observables.
Since the vector space is infinite dimensional, one can't represent
these operators by finite matrices, so the usual method of
determining eigenvalues and eigenvectors breaks down.  However,
one can generalize many of the ideas we have developed here.
For example, one may define the inner product of two wave functions
by the formula
$$
     \langle\psi | \phi\rangle = \iiint_{\R^3}\overline{\psi(x,y,z)}
\phi(x,y,z)\, dV.
$$
Then, one may determine the eigenvalues of a Hermitian operator
$A$ by the studying the optimization problem for the quantity
$$
     \langle\psi | A\psi\rangle 
$$
subject to the condition $\langle \psi | \psi \rangle = 1$.


\subhead Lagrange Multipliers with More Than One Constraint \endsubhead
 In $\R^n$, suppose we want to maximize (minimize) a function 
$f(\x)$ subject to $m$ constraints
$$
g_1(\x)  = c_1,\quad g_2(\x) = c_2, \dots, g_r(\x) = c_m.
$$
We can make the scalar functions $g_i(\x)$ the components of
a {\it vector\/} function $\g:\R^n \to \R^m$.   Then the $m$
constraining equations may be summarized by a single vector
constraint
$$
\g(\x) = \c = \lb c_1,c_2,\dots, c_m\rb.
$$
In this way, we may view the constraint as defining a level set
(for $\g$) in $\R^n$, and the problem is to maximize $f$ on this
level set.   The level set may also be viewed as the {\it intersection\/}
of the $m$ hypersurfaces in $\R^n$ which are  level sets of the
component scalar functions $g_i(\x) = c_i$.

\nextex
\example{Example \en}
Consider the problem of finding the highest point on the curve of
intersection of the plane $x + 2y + z = 1$ with the sphere
$x^2 + y^2 + z^2 = 21$.   
\medskip
\centerline{\epsfbox{s12-20.ps}}
\medskip
Here we take 
$f(x,y,z) = z$ and
$$\gather
\g(x,y,z)= \bm g_1(x,y,z)\\ g_2(x,y,z)\em
= \bm x + 2y + z \\ x^2 + y^2 + z^2 \em\\
  \c = \bm 1\\ 21\em.
\endgather$$
\endexample

If we assume that $f$ and $\g$ are smooth, then
just as before  we obtain 
$$
\frac{df}{dt} = \nabla f\cdot\v = 0
$$
for every vector $\v$ tangent to a curve in the level set through the
maximum point.      Every such $\v$, since it is tangent to the
level set, will be perpendicular to each of the normal vectors
$\nabla g_i$ at the point, i.e.,
$$\align
\nabla g_1\cdot \v &= 0 \\
\nabla g_2\cdot \v &= 0 \\
 &\vdots \\
\nabla g_m\cdot \v &= 0. 
\endalign$$
If we make the gradient vectors into the rows of a matrix, this system
may be rewritten
\nexteqn
$$
\bm \nabla g_1 \\ \nabla g_2 \\ \vdots \\ \nabla g_m\em \v = 0.\tag\eqn
$$
   In the case of one constraint, we
assumed that $\nabla g \not= 0$ so there
would be a well defined tangent plane at the maximum point. 
  Now,  we need
a more stringent condition:   the gradients
at the potential maximum point 
$$
\nabla g_1, \nabla g_2, \dots, \nabla g_m
$$
should form a {\it linearly independent set}.   This means that the 
$m\times n$ system 
(\eqn) has rank $m$.
Hence, the solution space of all vectors $\v$ satisfying (\eqn)
is $n - m$-dimensional.  This solution space is called the
{\it tangent space\/} to the level set at the point.
In these circumstances, it is possible
to show (from higher dimensional analogues of the implicit function
theorem) that  every vector $\v$ in this tangent space is in fact
tangent to a curve lying in the level set.  Using this, we may conclude
that, at a maximum point,
$$
\nabla f\cdot \v = 0
$$
for every vector $\v$ in the tangent space. 
 Consider then the  $(m +1)\times n$ system
$$ 
\bm \nabla g_1\\
\nabla g_2 \\
 \vdots \\
\nabla g_m \\
\nabla f \em \v =  0.
 $$
This cannot have rank $m+1$ since it has exactly the same solution
space as the system (\eqn).  Hence, it has rank $m$, and the only way
that could happen is if the last row is dependent on the $m$ previous
rows, i.e.,
\nexteqn
$$
\nabla f = \lambda_1\nabla g_1 + \lambda_2\nabla g_2 + \dots +
\lambda_m\nabla g_m.\tag\eqn
$$
The scalars $\lambda_1, \lambda_2, \dots, \lambda_m$ are called
Lagrange multipliers.    
%Exer  Some exercise mentioning the normal space??

\example{Example \en, continued}
We have
$\nabla f = \bm  0 & 0 & 1 \em$ and
$$
\bm \nabla g_1 \\ \nabla g_2 \em = \bm 1 & 2 & 1
 \\ 2x & 2y  & 2z \em.
$$
Hence, the Lagrange multiplier condition $\nabla f = \lambda_1\nabla g_1
+ \lambda_2\nabla g_2$ amounts to
$$
 \bm 0 & 0 & 1 \em = \lambda_1 \bm 1 & 2 & 1\em +\lambda_2
\bm  2x & 2y  & 2z \em
$$
which yields
$$\align
\lambda_1 + 2\lambda_2x &= 0\\
2\lambda_1 + 2\lambda_2y & = 0 \\
\lambda_1 + 2\lambda_2z &= 1.
\endalign
$$
To this we must add the constraints
\nexteqn
$$
\aligned
x+ 2y + z &= 1 \\
x^2 + y^2 + z^2 &= 21.
\endaligned\tag\eqn$$
In total, this gives 5 equations for the 5 unknowns
$x, y, z, \lambda_1, \lambda_2$.   We can solve these equations
by being sufficiently ingenious, but there is a short cut.
The multiplier condition just amounts to the assertion that
$\{\nabla g_1, \nabla g_2, \nabla f\}$ is a dependent set.
(But, it is assumed that $\{\nabla g_1, \nabla g_2\}$ is independent.)
However, this set is dependent if and only if
$$\gather
\det \bm \nabla g_1 \\ \nabla g_2 \\ \nabla f \em
   = \det \bm 0 & 0 & 1 \\ 1 & 2 & 1\\ 2x & 2y & 2z \em = 0 \\
\vspace{10pt}
\text{i.e. }\quad
 1(2(2x) - 2y) = 2(2x - y) = 0 \\
\text{i.e. }\quad y = 2x.
\endgather$$
Putting this in (\eqn) yields
$$\gather
5x + z = 1\qquad 5x^2 + z^2 = 21 \\ 
5x^2 + (1-5x)^2 = 30x^2 -10x + 1 = 21 \\
30x^2 -10x - 20 = 0 \\
  x = 1, -\frac 23.
\endgather
$$
Using $z = 1-5x, y = 2x$ yields the following two points as 
possible maximum points:
$$
(1, 2, -4)
\qquad \text{and} \qquad
(-2/3,-4/3,13/3).
$$
It is clear that the maximum value of $f(x,y,z) =z$ is attained
at the second point.

There is one minor issue that was ignored in the above calculations.
The reasoning is only valid at points at which the `tangent space'
to the level set is well defined as defined above.  In this case
the level set is a curve (in fact, it is a circle), and the
tangent space is 1 dimensional, i.e., it is a line. 
 The two gradients $\nabla g_1,
\nabla g_2$ generally span the plane perpendicular to the
tangent line, but it could happen at some point that one of
the
gradients is a multiple of the other.   In that case the two
level surfaces
$g_1(\x) = c_1$ and $g_2(\x) = c_2$ are tangent to one another
at the given point, so we would expect some problems.   For example,
consider the intersection of the hyperbolic paraboloid
$z - x^2 + y^2 = 0$ with its tangent plane at the origin 
$z = 0$.   This `curve'
consists two straight lines which intersect at the origin. At any
point other than the origin, there is a well defined tangent line,
i.e., whichever of the two lines is appropriate, but at the origin there
is a problem.  

In general, there is no way to know that a maximum or minimum
does not occur at a point where the tangent space is not well
defined.  Hence, all such points must be considered possible
candidates for maximum or minimum points.   In Example \en,
however, it is fairly clear geometrically that there are no
such points.   This can also be confirmed analytically by seeing
that $\{\nabla g_1, \nabla g_2 \}$ is independent at every
point of the set.  For, since $\nabla g_1 \not = 0$,
the only way the pair could be dependent is by
a relation of the form $\nabla g_2 = c\nabla g_1$.  This 
yields
$$\gather
\bm 2x & 2y & 2z \em =  
c\bm 1 & 2 & 1\em\\
2x =c,\, 2y =c 2c,\, 2z = c \\
 x = z, y = x/2 
 \endgather$$
and it is easy to see these equations are not consistent with
$x + 2y + z = 1, x^2 + y^2 + z^2 = 21$.    
\endexample
\bigskip
\input chap12.ex8
\bigskip

\endchapter
\closeseg{chap12}
\enddocument
