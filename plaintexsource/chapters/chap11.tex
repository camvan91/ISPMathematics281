\documentstyle{book}
\input extra.tex
\input le2.sty
\input epsf.tex
\input chap10.lnk
\maketoctrue
\indextrue
\mardiagtrue
\openseg{chap11}
\Monograph
\topmatter
\nextchap{Determinants and Eigenvalues}
\title
\chapter{\chapno}Determinants and Eigenvalues
\endtitle
\endtopmatter
\document
\nextsec{Homogeneous Linear Systems of Differential Equations}
\head \sn. Homogeneous Linear Systems of Differential Equations \endhead

Let $A = A(t)$ denote an $n\times n$ matrix with entries
$a_{ij}(t)$  which are functions defined and continuous
on some real interval $a < t < b$.   In general, these functions
may be complex valued functions.   Consider the $n\times n$ system of
differential equations
$$
\frac{d\x}{dt} = A(t)\x
$$
where $\x = \x(t)$ is a vector valued function also defined on
$a < t < b$ and taking values in $\C^n$.  (If $A(t)$ happens to
have real entries, then we may consider solutions
$\x = \x(t)$ with values in $\R^n$, but even in that case it is often
advantageous to consider complex valued solutions.)   The most interesting
case is that in which $A$ is a {\it constant\/} matrix, and we shall
devote almost all our attention to that case.
\outind{homogeneous system of differential equations}%

\nextex
\example{Example \en}  Consider
$$
\frac{d\x}{dt} = \bm 1 & 2 \\ 2 & 1 \em\x
\qquad\text{where }
\x = \bm x_1(t)\\x_2(t)\em
$$
on the interval $-\infty < t < \infty$.
Note that the entries in the coefficient matrix are constants.  
\endexample

 The set of solutions
of such a homogeneous system forms a (complex) vector space.
To see this, consider the operator
$$
L = \frac{d}{dt} - A
$$
which is defined on the vector space of all differentiable vector
\outind{linear operator}%
valued functions.  It is not hard to see that $L$ is a linear
operator, and its null space is the desired set of solutions since
$$
L(\x) = 0\quad\text{means } \frac{d\x}{dt} - A\x = 0.
$$

We shall see shortly that {\it this vector space is $n$-dimensional}.
Hence, solving the system amounts to finding $n$ solutions 
$\x_1(t), \x_2(t), \dots, \x_n(t)$ which form a basis for the
solution space.   That means that any solution can be written
uniquely
$$
  \x = c_1\x_1(t) + c_2\x_2(t) + \dots + c_n\x_n(t).
$$
Moreover, if the solution has a specified initial value
$\x(t_0)$, then the $c_i$ are determined by solving 
$$
\x(t_0) = c_1\x_1(t_0) + c_2\x_(t_0) + \dots + c_n\x_n(t_0).
$$
(If you look closely, you will see that this is in fact a
linear system of algebraic equations with unknowns
$c_1, c_2, \dots , c_n$.)


\example{Example \en, continued}  We shall try to solve
$$
\frac{d\x}{dt} = \bm 1 & 2 \\ 2 & 1 \em\x
\qquad \text{given } \x(0) = \bm\format\r\\ 1\\ -2\em.
$$

The first problem is to find a linearly independent pair
solutions.  Assume for the moment that you have a method
for generating such solutions, and it
tells
you to try
$$
\x_1 = \bm\format\r\\e^{3t}\\e^{3t}\em = e^{3t}\bm 1\\1\em,\qquad
\x_2 = \bm\format\r\\-e^{-t}\\e^{-t}\em = e^{-t}\bm\format\r\\ -1\\ 1\em.
$$
(We shall develop such methods later in this chapter.)
 It is not hard to see that these are solutions:
$$
\alignat2
\frac{d\x_1}{dt} &= 3e^{3t}\bm 1\\1\em\qquad&\text{and}\qquad 
\bm 1 & 2 \\ 2 & 1 \em \x_1 &= e^{3t}\bm 1 & 2 \\ 2 & 1 \em\bm 1\\1\em
    = e^{3t}\bm 3\\ 3\em \\
\frac{d\x_2}{dt} &= -e^{-t}\bm\format\r\\-1\\1\em\qquad&\text{and}\qquad 
\bm 1 & 2 \\ 2 & 1 \em \x_2 &= e^{-t}\bm 1 & 2 \\ 2 & 1 \em\bm\format\r\\-1\\1\em
    = e^{-t}\bm\format\r\\ 1\\ -1\em .
\endalignat$$

Also, $\x_1(t)$ and $\x_2(t)$ form a linearly independent pair.
For, otherwise one would be a scalar multiple of the other,
say $\x_1(t) = c\x_2(t)$ for all $t$.  Then the same thing would be
true of their first components, e.g., $e^{3t} = ce^{-t}$ for all
$t$, and we
know that to be false.

Thus, $\{\x_1, \x_2\}$ is a  basis for the solution space, so any
solution may be expressed uniquely
$$
\x = c_1\x_1(t) + c_2\x_2(t) = \x_1(t)c_1 + \x_2(t)c_2
   = \bm \x_1(t) & \x_2(t) \em\bm c_1\\ c_2 \em.
$$
Thus putting $t = 0$ yields
$$
\x(0) = \bm \x_1(0) & \x_2(0) \em \bm c_1\\c_2 \em = \bm\format\r\\ 1\\-2\em.
$$
However,
$$
\x_1(0) = e^{3(0)}\bm 1\\ 1\em = \bm 1\\1\em\qquad
\x_2(0) = e^{-(0)}\bm\format\r\\-1\\1\em = \bm\format\r\\ -1\\ 1\em,
$$
so the above system becomes
$$
\bm\format\r&\quad\r\\
    1 & -1\\ 1 & 1 \em \bm c_1\\ c_2 \em = \bm\format\r\\ 1\\-2\em.
$$
This system is easy to solve by Gauss-Jordan reduction
$$
\bm \format\r&\quad\r&\quad\c\quad&\r\\
       1 & -1 &|& 1\\ 1 & 1 &|&-2\em \to
\bm \format\r&\quad\r&\quad\c\quad&\r\\
       1 & -1 &|& 1\\ 0 & 2 &|&-3\em \to
\bm \format\r&\quad\r&\quad\c\quad&\r\\
       1 & 0 &|& -1/2\\ 0 & 1 &|&-3/2\em. 
 $$
Hence, the solution is $c_1 = -1/2, c_2 = -3/2$ and the desired
solution of the system of differential equations is
$$
\x = -\frac 12 e^{3t}\bm 1\\ 1\em - \frac32 e^{-t}\bm\format\r\\ -1\\ 1\em
   = \bm -\frac 12 e^{3t} + \frac 32 e^{-t} \\
         -\frac 12 e^{3t} - \frac 32 e^{-t} \em.
$$
\endexample

The general situation is quite similar.   Suppose we have some method
for generating $n$ vector valued functions $\x_1(t), \x_2(t), \dots, \x_n(t)$
which together constitute a linearly independent set of solutions
of the $n\times n$ system
$$
\frac{d\x}{dt} = A\x.
$$
Then, the general solution takes the form
$$
\x = \x_1(t)c_1 + \x_2(t)c_2 + \dots \x_n(t)c_n
= \bm \x_1(t) & \x_2(t) & \dots &\x_n(t) \em \bm c_1\\ c_2\\ \vdots \\ c_n \em.
$$
To match a given initial condition $\x(t_0)$
at $t = t_0$, we have to solve the
$n\times n$ algebraic system
\nexteqn
\xdef\AlgSys{\eqn}
$$
 \bm \x_1(t_0) & \x_2(t_0) & \dots &\x_n(t_0) \em 
\bm c_1\\ c_2\\ \vdots \\ c_n \em
= \x(t_0).\tag\eqn
$$
Note that the coefficient matrix is just a specific $n\times n$ matrix
with scalar entries and the quantity on the right is a specific
$n\times 1$ column vector.  (Everything in sight is evaluated at
$t_0$, so there are no variable quantities in this equation.)  We
now have to rely on the results of the previous chapter.   Since
in principle the given initial value vector $\x(t_0)$ could be
anything whatsoever, we hope that this algebraic system can be
solved for any possible $n\times 1$ column vector on the right.
However, we know this is possible only in the case that the
coefficient matrix is non-singular, i.e., if it has rank $n$.
How can we be sure of this?  It turns out to be a consequence
of the basic uniqueness theorem for systems of differential equations.

\medskip
\subhead Existence and Uniqueness for Systems \endsubhead
We state the basic theorem for complex valued functions.  There
is a corresponding theorem in the real case.

\nextthm
\xdef\ThA{\tn}
\proclaim{Theorem \cn.\tn}  Let $A(t)$ be an $n\times n$ complex
matrix with entries continuous functions defined on a real interval
$a < t < b$, and suppose  $t_0$  is a point in that interval.   Let
$\x_0$ be a given $n\times 1$ column vector in $\C^n$.
Then there exists a unique solution $\x = \x(t)$ 
of the
system
$$
\frac{d\x}{dt} = A(t)\x
$$
defined on the interval $a < t < b$
and satisfying  $\x(t_0) = \x_0$.
\endproclaim
\outind{existence theorem for systems}%

We shall not try to prove this theorem in this course.

The uniqueness part of the theorem has the following important
consequence.

\nextthm
\proclaim{Corollary \cn.\tn}  With the notation as above, suppose
$\x_1(t), \x_2(t), \dots,\x_n(t)$ are  $n$ solutions of
the system  $\dfrac{d\x}{dt} = A(t)\x$ on the interval
$a < t < b$.   Let $t_0$ be any point in that interval.  Then the
set $\{\x_1(t), \x_2(t), \dots, \x_n(t)\}$ is a linearly
independent set of functions if and only if the set
$\{\x_1(t_0),\x_2(t_0),\dots, \x_n(t_0) \}$ is a linearly independent
set of column vectors in $\C^n$.
\endproclaim

\example{Example 1, again}
The set of functions is
$$
\left\{ e^{3t}\bm 1\\ 1\em, e^{-t}\bm\format\r\\-1\\1\em\right\}
$$
and the set of column vectors (obtained by setting $t = t_0 = 0$) is
$$
\left\{ \bm 1\\ 1\em, \bm\format\r\\-1\\1\em\right\}.
$$
\endexample

\demo{Proof of the Corollary}
We shall prove that one set is linearly {\it dependent\/} if and
only if the other is.   First suppose that the set of functions
is dependent.  That means that one of them may be expressed as
a linear combination of the others. For the sake of argument suppose
the notation is arranged so that
\nexteqn
\xdef\EqA{\eqn}
$$
\x_1(t) = c_2\x_2(t) + \dots + c_n\x_n(t)\tag\eqn
$$
holds for all $t$ in the interval.
Then, it holds for $t = t_0$ and we have
\nexteqn
\xdef\EqB{\eqn}
$$
\x_1(t_0) = c_2\x_2(t_0) + \dots + c_n\x_n(t_0).\tag\eqn
$$
This in turn tells us that the set of column vectors at $t_0$ is dependent.

Suppose on the other hand that the set of column vectors at $t_0$ is dependent.
Then we may assume that there is a relation of the form (\EqB)
with appropriate scalars $c_2, \dots , c_n$.   But that means that
the solutions $\x_1(t)$ and $c_2\x_2(t) + \dots + c_n\x_n(t)$
agree at $t = t_0$.   According to the uniqueness part of
Theorem \cn.\ThA, this means that they agree for all $t$, which means
that (\EqA) is true as a relation among functions.
 This in turn implies that the set of
functions is dependent.
\qed\enddemo

The corollary gives us what we need to show that the $n\times n$ matrix
$$
\bm \x_1(t_0)& \x_2(t_0) &\dots & \x_n(t_0) \em
$$
is non-singular. Namely, if
$\{\x_1(t), \x_2(t), \dots, \x_n(t)\}$ is a linearly independent
set of solutions, then the columns of the above matrix form a
linearly independent set in $\C^n$.  It follows that they
form a basis for the column space of the matrix which must then
have rank $n$, and so it is non-singular.  As noted above, that
means that we can always solve the algebraic system of equations
(\AlgSys)
specifying initial conditions at $t_0$.
\bigskip
\input chap11.ex1
\bigskip
\nextsec{Finding Linearly Independent Solutions}
\head \sn.  Finding Linearly Independent Solutions \endhead
\xdef\DiagExSec{\sn}

\subhead Existence of a Basis \endsubhead
The basic existence and uniqueness theorem stated in the previous
section ensures that a system of the form
$$
\frac{d\x}{dt} = A(t)\x
$$
always has $n$ solutions $\u_1(t), \u_2(t), \dots, \u_n(t)$
which form a basis for the vector space of all solutions. 
% Advice to instructor:  Skip this argument in class
 To see this, fix $t_0$ in the interval
$a < t < b$, and define the $i$th solution $\u_i(t)$ to be the {\it
unique\/} solution satisfying the initial condition
$$
\u_i(t_0) = \e_i
$$
where as before $\e_i$ is the $i$th vector in the {\it standard basis\/}
for $\C^n$, i.e., it is the $i$th column of the $n\times n$ identity
matrix.    Since $\{\e_1, \e_2, \dots, \e_n\}$ is an independent set,
it follows from Corollary  11.2 
%  Back reference to corollary. 
that $\{\u_1(t), \u_2(t), \dots, \u_n(t) \}$ is an independent set
of solutions.   It also spans the subspace of solutions.   To see this,
let $\x(t)$ denote any solution.  At $t = t_0$ we have
$$
\align
\x(t_0)  = \bm x_1(t_0)\\ x_2(t_0)\\ \vdots \\ x_n(t_0) \em
    &= x_1(t_0)\e_1 + x_2(t_0)\e_2 + \dots + x_n(t_0)\e_n \\
    &= c_1\u_1(t_0) + c_2\u_2(t_0) + \dots + c_n(\u_n(t_0),
\endalign
$$
where $c_1 = x_1(t_0), c_2 = x_2(t_0), \dots, c_n = x_n(t_0)$.
Thus, by the uniqueness theorem, we have for all $t$ in the interval
$$
\x(t)  =   c_1\u_1(t) + c_2\u_2(t) + \dots + c_n\u_n(t).
$$
\medskip
\subhead Case of Constant Coefficients \endsubhead
It is reassuring to know that in principle we can always find a basis
for the solution space, but that doesn't
help us find it.   In this section we outline a method for generating
a basis for the solutions of $\dfrac{d\x}{dt} = A\x$ in case
$A$ is {\it constant}.   This method will work reasonably well for
$2\times 2$ systems, but we shall have to develop the theory of
$n\times n$ determinants to get it to work for general $n\times n$
systems.

\nextex
\xdef\EE{\en}
\example{Example \en}
Consider the $2\times 2$ system
\nexteqn
\xdef\SysX{\eqn}
$$
\frac{d\x}{dt} = \bm \format\r&\quad\r\\0 & -2 \\ 1 & 3 \em \x.\tag\eqn
$$
I chose this because it is just the system version of the second
order equation  
\nexteqn
\xdef\SecOrd{\eqn}
$$
y'' - 3y' + 2y = 0\tag\eqn
$$
which we already know how to solve.
Namely, put $x_1 = y$ and $x_2 = y'$, so 
$$
\alignat2
x_1'&= y'  &&=  \hphantom{-2x_1 + 3}x_2 \\
x_2'&= y'' = -2y + 3y' &&=  -2x_1 + 3x_2,
\endalignat
$$ 
which when put in matrix form is (\SysX).
To solve the second order equation (\SecOrd), proceed in the
usual manner.   The roots of
$$
r^2 - 3r + 2 = (r -1)(r - 2) = 0
$$
are $r_1 = 1, r_2 = 2$.   Hence, $y_1 = e^t$ and $y_2 = e^{2t}$
constitute a linearly independent pair of solutions.   The
corresponding {\it vector\/} solutions are
$$
\x_1 = \bm \format\c\\y_1\\y_1'\em = \bm\format\c\\ e^t\\ e^t\em
   = e^t\bm 1\\ 1\em\qquad
\x_2 = \bm \format\c\\y_2\\y_2'\em = \bm\format\c\\ e^{2t}\\ 2e^{2t}\em
   = e^{2t}\bm 1\\ 2\em.
$$
\endexample

The above example suggests that we {\it look for solutions of the
form}
\nexteqn
$$
\x = e^{\lambda t}\v\tag\eqn
$$
where $\lambda$ is a scalar and $\v$ is a vector, both to be determined
by the solution process.  Note also that we want $\v \not= \bold 0$ since
otherwise the solution of the differential equation would be identically
zero and hence not very interesting.

Substitute (\eqn) in $\dfrac{d\x}{dt} = A\x$ to obtain
$$
\frac{d\x}{dt} = \lambda e^{\lambda t}\v = A e^{\lambda t} \v.
$$
The factor $e^{\lambda t}$ is a scalar and non-zero for all $t$, so
we cancel it from the above equation.   The resulting equation may
be rewritten
\nexteqn
$$
A\v = \lambda \v\qquad\text{where } \v \not= \bold 0.\tag\eqn
$$
We introduce special terminology for the situation described by
this equation.   If  (\eqn) has a non-zero solution for a given
scalar $\lambda$, then  $\lambda$ is called an {\it eigenvalue\/}
of the matrix $A$, and any {\it non-zero\/} vector $\v$ which works for
that eigenvalue is called an {\it eigenvector\/} of $A$ corresponding
to $\lambda$.   These related concepts are absolutely essential for
understanding systems of differential equations, and they arise in
fundamental ways in a wide variety of applications of linear algebra.
\outind{eigenvalue}%
\outind{eigenvector}%

Let's analyze the problem of finding eigenvalues and eigenvectors
for $A$ a
 $2\times 2$ matrix.   Then, (\eqn) may be rewritten
$$
\bm a_{11} & a_{12} \\ a_{21} & a_{22} \em \bm v_1\\ v_2 \em
= \bm \lambda v_1 \\ \lambda v_2 \em
$$
or
$$
\align
a_{11}v_1 + a_{12}v_2 &= \lambda v_1 \\
a_{21}v_1 + a_{22}v_2 &= \lambda v_1,
\endalign $$
which, after transposing, becomes
$$
\align
(a_{11} - \lambda)v_1 + a_{12}v_2 &= 0 \\
a_{21}v_1 + (a_{22} - \lambda)v_2 &= 0.
\endalign $$
This is a homogeneous system which may be put in matrix form
\nexteqn
\xdef\EigVecEq{\eqn}
$$
\bm a_{11} - \lambda & a_{12} \\ a_{21} & a_{22} - \lambda \em
\bm v_1 \\ v_2 \em = \bold 0.\tag\eqn
$$
This system will have a {\it non-zero\/} solution $\v$, as required,
if and only if the coefficient matrix has rank less
than $n = 2$.  Unless the matrix consists of zeroes, this means
it must have rank one.   That, in turn, amounts to saying
that one of the rows
is a multiple of the other, i.e., that
 the ratios of corresponding
components are the same, or, in symbols,
$$
\frac{a_{11} - \lambda}{a_{21}} = \frac{a_{12}}{a_{22} - \lambda}.
$$
Cross multiplying and transposing yields the quadratic equation
\nexteqn
$$
(a_{11} - \lambda)(a_{22} - \lambda) - a_{12}a_{21} = 0.\tag\eqn
$$
Our strategy then is to solve this equantion for $\lambda$ to
find the possible {\it eigenvalues\/} and then
for each eigenvalue $\lambda$ to find the non-zero solutions
of (\EigVecEq)  which are the eigenvectors corresponding to that
eigenvalue.   In this way, each eigenvalue and eigenvector pair
will generate a solution  $\x = e^{\lambda t}\v$ of the original
system of differential equations.      

Note that  (\eqn) may be rewritten using $2\times 2$ determinants
as 
\nexteqn
\xdef\CharEq{\eqn}
$$
\det
\bm a_{11} - \lambda & a_{12} \\ a_{21} & a_{22} - \lambda \em
 = 0.\tag\eqn
$$
This equation is called the {\it characteristic equation\/} of the matrix.
\outind{characteristic equation}%

\nextex
\example{Example \en}
\xdef\DiagEx{\en}
Consider (as in Section 1) the system
\nexteqn
\xdef\SysDD{\eqn}
$$
\frac{d\x}{dt} = \bm \format\r&\quad\r\\1 & 2 \\ 2 & 1 \em \x.\tag\eqn
$$
Try 
$\x = e^{\lambda t}\v$ as above.
As we saw, this comes down to solving
the eigenvalue--eigenvector problem for the coefficient matrix
$$A = \bm 1 & 2\\ 2& 1 \em.$$
To do so, we first solve the characteristic equation
$$
\align 
\det \bm 1 - \lambda & 2\\ 2& 1  - \lambda\em &= (1 - \lambda)^2 -4 = 0\\
\text{or}\qquad
1 - 2\lambda + \lambda^2 - 4 &= \lambda^2 -2\lambda - 3  =
(\lambda -3)(\lambda + 1) = 0.
\endalign
$$
The roots of this equation are $\lambda = 3$ and $\lambda = -1$.
First consider  $\lambda = 3$.    Putting this in 
(\EigVecEq) yields 
\nexteqn
\xdef\SysHH{\eqn}
$$
\bm 1 - 3 & 2 \\
2 & 1 - 3  \em \v
 =
\bm\format\r&&\qquad\r\\ -2 & 2  \\
2  & - 2 \em \v  = \bold 0.\tag\eqn
$$
Gauss-Jordan reduction yields 
 the solution $v_1 = v_2$ with $v_2$ free.  A general solution
vector has the form
$$
\v = \bm v_1\\ v_2 \em = \bm v_2\\ v_2 \em  = v_2\bm 1\\ 1 \em.
$$
Put $v_2 = 1$ to obtain
$$
\v_1 = \bm 1\\ 1 \em
$$
which will form a basis for the solution space of
(\SysHH).
Thus, we obtain as one solution of  (\SysDD)
$$
\x_1 = e^{\lambda t}\v_1 =  e^{3t}\bm 1\\1\em.
$$
Note that any other eigenvector for $\lambda = 3$ is a non-zero
mulitple of the basis vector $\v_1$, so choosing another
eigenvector in this case will result in a solution of the differential
equation which is just a constant multiple of $\x_1$.

To find a second solution, consider the root $\lambda = -1$.  Put
this in (\EigVecEq) to obtain
\nexteqn
$$
\bm 1 - (-1) & 2 \\
2 & 1 - 3  \em \v
 =
\bm 2 & 2  \\
2  &  2 \em \v  = \bold 0.\tag\eqn
$$
Gauss-Jordan reduction yields
the general solution $v_1 = - v_2$ with $v_2$ free.  The
general solution vector is
$$
\v = \bm v_1\\v_2\em = 
v_2\bm\format\r\\-1\\1\em.
$$
Putting $v_2 = 1$ yields a the basic eigenvector
$$
\v_2 = \bm\format\r\\-1\\1\em
$$
and the corresponding solution of the differential equation
$$
\x_2 = e^{-t}\bm\format\r\\-1\\1\em.
$$

Note that these are the solutions we used to form a basis in
Example 1 in the
previous section.
\endexample

The above procedure appears similar to what we did to solve a
second order equation $y'' + py' + qy = 0$  with constant
coefficients.   This is no accident!.   The quadratic equation
$$
r^2 + pr + q = 0
$$
is  just the characteristic equation (with $r$ replacing
$\lambda$) of the $2\times 2$ matrix you obtain when you reformulate
the problem as a first order system.   You should check this
explicitly in Example \EE.   (The general case is the topic of
an exercise.)


The method for $n\times n$ systems is very similar to what
we did above.    However, the analogue of (\CharEq), i.e.,
the characteristic equation, takes the form
$$
\det\bm a_{11} - \lambda & a_{12} & \hdots & a_{1n} \\
 a_{21} & a_{22} - \lambda & \hdots & a_{2n} \\
\vdots & \vdots & \hdots & \vdots \\
a_{n1} & a_{n2} & \hdots &  a_{nn} - \lambda \em  = 0
$$
which requires the use of $n\times n$ determinants.
   So far in this course we have only discussed
$2\times 2$ determinants and briefly $3\times 3$ determinants.  Hence, to
develop the general theory, we need to define and study the
properties of $n\times n$ determinants.

\bigskip
\input chap11.ex2
\bigskip
\nextsec{Definition of the Determinant}
\head \sn. Definition of the Determinant \endhead

Let $A$ be an $n\times n$ matrix.

By definition
$$
\alignat2
\text{for } n &=1&\qquad \det \bm a \em &= a \\
\text{for } n &= 2 &\qquad \det\bm a_{11} & a_{12} \\ a_{21} & a_{22} \em
         &= a_{11}a_{22} - a_{12}a_{21}.
\endalignat
$$
For $n > 2$, the definition is much more complicated.  It is a
sum of many terms formed as follows.  Choose any entry from
the first row of $A$; there are $n$ possible ways to do that.
Next, choose any entry from the second row which is not in the
same column as the first entry chosen; there are $n - 1$ possible
ways to do that.   Continue in this way until you have chosen one
entry  from each row in such a way that no column is repeated;
there are $n!$ ways to do that.  Now multiply all these entries
together to form a typical term.  If that were all, it would be
complicated enough, but there is one further twist.  The products
are divided into two classes of equal size according to a rather complicated
rule and then the sum is formed with the terms in one class multiplied
by $+1$ and those in the other class multiplied by $-1$.

Here is the definition for $n = 3$ arranged to exhibit the signs.
$$\multline
\det \bm a_{11} & a_{12} & a_{13} \\
         a_{21} & a_{22} & a_{23} \\
         a_{31} & a_{32} & a_{33} \em = \\
           a_{11}a_{22}a_{33} + a_{12}a_{23}a_{31} + a_{13}a_{21}a_{32} \\
           - a_{11}a_{23}a_{32} - a_{12}a_{21}a_{33} - a_{13}a_{22}a_{31}.
\endmultline
$$
The definition for $n=4$ involves $4! = 24$ terms, and I won't
bother to write it out.
\outind{determinant, definition of}%

A better way to develop the theory is {\it recursively}.  That is,
we assume that determinants have been defined for all $(n-1)\times (n-1)$
matrices, and then use this to define determinants for $n\times n$
matrices.   Since we have a definition for $1\times 1$ matrices, this
allows us in principle to find the determinant of any $n\times n$
matrix by recursively invoking the definition.   This is less
explicit, but it is  easier to work with.

Here is the recursive definition.
Let $A$ be an $n\times n$ matrix, and let $D_j(A)$ be the determinant
of the $(n-1)\times(n-1)$ matrix obtained by {\it deleting\/} the
$j$th {\it row\/} and the {\it first column\/} of $A$.   Then, define
$$
\det A = a_{11}D_1(A) - a_{21}D_2(A) + \dots + (-1)^{j+1}a_{j1}D_j(A)
 + \dots + (-1)^{n+1}a_{n1}D_n(A).
$$
In words: take each entry in the first column of $A$, multiply it
by the determinant of the $(n-1)\times(n-1)$ matrix obtained by
deleting the first column and that row, and then add up these
entries alternating signs as you do.  

\example{Examples}
$$
\align
\det \bm \format\r&\quad\r&\quad\r\\
2 & -1 & 3 \\
1 & 2 & 0 \\
0 & 3 & 6 \em
&=
2\det \bm 2 & 0 \\
        3 & 6 \em
- 1\det \bm \format\r&\quad\r\\
             -1 & 3 \\
             3 & 6 \em
+ 0\det \bm \format\r&\quad\r\\
             -1 & 3 \\
             2 & 0 \em\\
&= 2 (12 - 0) - 1(-6 -9) + 0(\dots) = 24 + 15 = 39.
\endalign $$
Note that we didn't bother evaluating the $2\times 2$ determinant
with coefficient 0.
You should check that the earlier definition gives the same
result.

$$
\multline
\det \bm \format\r&\quad\r&\quad\r&\quad\r\\
          1 & 2 & -1 & 3 \\
          0 & 1 & 2 & 0 \\
          2 & 0 & 3 & 6 \\
          1 & 1 & 2 & 1 \em \\
= 1\det \bm \format\r&\quad\r&\quad\r\\
             1 & 2 & 0 \\
              0 & 3 & 6 \\
             1 & 2 & 1 \em
 - 0\det \bm \format\r&\quad\r&\quad\r\\
             2 & -1 & 3 \\
              0 & 3 & 6 \\
             1 & 2 & 1 \em \\
 + 2 \det \bm \format\r&\quad\r&\quad\r\\
             2 & -1 & 3 \\
              1 & 2 & 0 \\
             1 & 2 & 1 \em
 - 1 \det \bm \format\r&\quad\r&\quad\r\\
             2 & -1 & 3 \\
              1 & 2 & 0 \\
             0 & 3 & 6 \em.
\endmultline
$$
Each of these $3\times 3$ determinants may be evaluated recursively.
(In fact we just did the last one in the previous example.)  You should
work them out for yourself.  The answers yield
$$
\det \bm \format\r&\quad\r&\quad\r&\quad\r\\
          1 & 2 & -1 & 3 \\
          0 & 1 & 2 & 0 \\
          2 & 0 & 3 & 6 \\
          1 & 1 & 2 & 1 \em
 = 1 (3) - 0(\dots) + 2(5) - 1(39) = -26.
$$
\endexample

Although this definition allows one to compute the determinant
of any $n\times n$ matrix {\it in principle\/}, the number of operations
grows very quickly with $n$.  In such
calculations one usually keeps track only of the multiplications
since they are usually the most time consuming operations.  Here
are some values of $N(n)$, the number of multiplications
needed for a recursive calculation of the determinant of
an $n\times n$ determinant.  We also tabulate $n!$ for
comparison.
$$
\matrix\format\quad\c\quad&\quad\c\quad&\quad\c\quad\\
                n &   N(n) & n! \\
                2  &   2   & 2  \\
                3  &   6  &  6   \\
                4  &   28 &  24 \\
                5  &  145 & 120 \\
                6  &  876 & 720 \\
                \vdots & \vdots & \vdots \endmatrix
$$
Thus, we clearly need a more efficient method to calculate
determinants.   As is often the case in linear algebra, elementary
row operations provide us with such a method.  This is based on
the following rules relating such operations to determinants.
\medskip

\noindent
  Rule (i): {\it If $A'$ is obtained from $A$ by adding a multiple of
 one row of $A$ to another, then $\det A' = \det A$}.

\nextex
\example{Example \en}
$$
\align
\det\bm \format\r&\quad\r&\quad\r\\
    1 & 2 & 3 \\ 2 & 1 & 3 \\ 1 & 2 & 1 \em
  &= 1(1-6) - 2(2 - 6) + 1(6 - 3) = 6\\
\det\bm \format\r&\quad\r&\quad\r\\
    1 & 2 & 3 \\ 0 & -3 & -3 \\ 1 & 2 & 1 \em
  &= 1(-3+6) - 0(2 - 6) + 1(-6 + 9) = 6.
\endalign$$
\medskip
\noindent Rule (ii):
{\it if $A'$ is obtained from $A$ by multiplying one row by
a scalar $c$, then $\det A' = c\det A$}.

\nextex
\example{Example \en}
$$
\align
\det\bm \format\r&\quad\r&\quad\r\\
    1 & 2 & 0 \\ 2 & 4 & 2 \\ 0 & 1 & 1 \em
  &= 1(4-2) - 2(2 - 0) + 0(\dots) = -2\\
2\det\bm \format\r&\quad\r&\quad\r\\
    1 & 2 & 0 \\ 1 & 2 & 1 \\ 0 & 1 & 1 \em
  &= 2(1(2-1) - 1(2 - 0) + 0(\dots)) = 2(-1) = -2.
\endalign$$
One may also state this rule as follows: {\it any common factor
of a row of $A$ may be `pulled out' from its determinant}.
\medskip
\noindent Rule (iii):
{\it If $A'$ is obtained from $A$ by interchanging two rows, then
$\det A' = - \det A$}.

\nextex
\example{Example \en}
$$
\det\bm 1 & 2 \\ 2 & 1 \em = -3\qquad\qquad
\det\bm 2 & 1 \\ 1 & 2 \em = 3.
$$
\endexample
The verification of these rules
is a bit involved, so we relegate it to an appendix.  
\medskip
The rules allow us to compute the determinant of any $n\times n$
matrix with specific numerical entries.

\nextex
\example{Example \en}   We shall calculate the determinant of
a $4\times 4$ matrix.  You should make sure you keep track of
which elementary row operations have been performed at each stage.

$$
\align
\det\bm\format\r&\quad\r&\quad\r&\quad\r\\
   1 & 2 & -1 & 1 \\
   0 & 2 & 1 & 2 \\
   3 & 0 & 1 & 1 \\
   -1 & 6 & 0 & 2 \em
&=
\det\bm\format\r&\quad\r&\quad\r&\quad\r\\
   1 & 2 & -1 & 1 \\
   0 & 2 & 1 & 2 \\
   0 & -6 & 4 & -2 \\
   0 & 8 & -1 & 3 \em
=
\det\bm\format\r&\quad\r&\quad\r&\quad\r\\
   1 & 2 & -1 & 1 \\
   0 & 2 & 1 & 2 \\
   0 & 0 & 7 & 4 \\
   0 & 0 & -5 & -5 \em\\
&=
-5\det\bm\format\r&\quad\r&\quad\r&\quad\r\\
   1 & 2 & -1 & 1 \\
   0 & 2 & 1 & 2 \\
   0 & 0 & 7 & 4 \\
   0 & 0 & 1 & 1 \em
=
+5\det\bm\format\r&\quad\r&\quad\r&\quad\r\\
   1 & 2 & -1 & 1 \\
   0 & 2 & 1 & 2 \\
   0 & 0 & 1 & 1 \\
   0 & 0 & 7 & 4 \em \\
&=
+5\det\bm\format\r&\quad\r&\quad\r&\quad\r\\
   1 & 2 & -1 & 1 \\
   0 & 2 & 1 & 2 \\
   0 & 0 & 1 & 1 \\
   0 & 0 & 0 & -3 \em.
\endalign$$
We may now use the recursive definition to calculate the last
determinant.  In each case there is only one non-zero entry in the first
column.
$$
\multline
\det\bm\format\r&\quad\r&\quad\r&\quad\r\\
   1 & 2 & -1 & 1 \\
   0 & 2 & 1 & 2 \\
   0 & 0 & 1 & 1 \\
   0 & 0 & 0 & -3 \em.
=1\det\bm\format\r&\quad\r&\quad\r\\
        2 & 1 & 2 \\
        0 & 1 & 1 \\
        0 & 0 & -3 \em \\
  = 1\cdot 2\det\bm\format\r&\quad\r\\
                   1 & 1 \\
                   0 & -3 \em
   = 1\cdot 2 \cdot 1\det\bm -3\em \\ = 1\cdot 2 \cdot 1 \cdot (-3) = -6.
\endmultline
$$
Hence, the determinant of the original matrix is $5(-6) = -30$.
\endexample

The last calculation is a special case of a general fact which is
established in much the same way by repeating the recursive
definition.

$$
\det\bm a_{11} & a_{12}& a_{13} &\hdots & a_{1n} \\
         0  & a_{22} & a_{23} & \hdots & a_{2n} \\
         0  & 0   & a_{33} & \hdots & a_{3n} \\
       \vdots & \vdots & \vdots & \hdots & \vdots \\
        0 & 0 & 0 & \hdots & a_{nn} \em =
a_{11}a_{22}a_{33}\hdots a_{nn}.
$$
In words, {\it the determinant of an upper triangular matrix is
the product of its diagonal entries}.
\medskip
It is important to be able to tell when the determinant of
an $n\times n$ matrix $A$ is zero.  Certainly, this will be the
case if the first column consists of zeroes, and indeed it turns
out that the determinant vanishes if any row or any column
consists only of zeroes.   More generally, if either the set of
rows or the set of columns is a linearly {\it dependent\/} set, then
the determinant is zero.   (That will be the case if the rank
$r < n$ since the rank is the dimension of both the row space and
the column space.)  This follows from the following important
theorem.
  
\nextthm
\proclaim{Theorem \cn.\tn}  Let $A$ be an $n\times n$ matrix.
Then $A$ is singular if and only if $\det A = 0$.
Equivalently, $A$ is invertible, i.e., has rank $n$, if and
only if $\det A \not= 0$.
\endproclaim

\demo{Proof}  If $A$ is invertible, then Gaussian reduction leads
to an upper triangular matrix with non-zero entries on its
diagonal, and the determinant of such a matrix is the
product of its diagonal entries, which is also non-zero. 
No elementary row operation can make the determinant
zero.   For,  type (i) operations don't
change the determinant, 
type (ii) operations multiply by
non-zero scalars, and type (iii) operations change its sign.
Hence, $\det A \not= 0$.

If $A$ is singular, then Gaussian reduction also leads to an
upper triangular matrix, but one in which at least the
\outind{upper triangular matrix, determinant of}%
\outind{determinant of upper triangular matrix}%
last row consists of zeroes.   Hence, at least one diagonal
entry is zero, and so is the determinant.
\qed\enddemo

\nextex
\example{Example \en}
$$
\det \bm 1 & 1 & 2 \\ 
         2 & 1 & 3\\
         1 & 0 & 1 \em
= 1(1 - 0) - 2(1 - 0) + 1(3 - 2) = 0
$$
so the matrix must be singular.  To confirm this, we reduce
$$
 \bm 1 & 1 & 2 \\ 
     2 & 1 & 3\\
     1 & 0 & 1 \em
\to
 \bm\format\r&\quad\r&\quad\r\\
      1 & 1 & 2 \\ 
     0 & -1 & -1\\
     0 & -1 & -1 \em
\to
 \bm\format\r&\quad\r&\quad\r\\
      1 & 1 & 2 \\ 
     0 & -1 & -1\\
     0 & 0 & 0 \em
 $$
which shows that the matrix is singular.
\endexample

In the previous section, we encountered $2\times 2$ matrices
with symbolic non-numeric entries.   For such a matrix,
Gaussian reduction doesn't work very well because we
don't know whether the non-numeric expressions are zero
or not.

\nextex
\example{Example \en}  Suppose we want to know whether or not the
matrix
$$
\bm
\format\r&\quad\r&\quad\r&\quad\r\\  
 -\lambda & 1 & 1 & 1 \\
   1 & -\lambda & 0 & 0 \\
  1 & 0 & -\lambda & 0 \\
  1 & 0 & 0 & -\lambda \em
$$
is singular.   We could try to calculate its rank, but since we
don't know what 
$\lambda$ is, it is not clear how to proceed.  Clearly, the
row reduction works differently if $\lambda = 0$ than if
$\lambda \not= 0$.   However, we can calculate the determinant
by the recursive method.
$$
\align
\det\bm\format\r&\quad\r&\quad\r&\quad\r\\  
 -\lambda & 1 & 1 & 1 \\
   1 & -\lambda & 0 & 0 \\
  1 & 0 & -\lambda & 0 \\
  1 & 0 & 0 & -\lambda \em \\
 &= (-\lambda)
\det\bm\format\r&\quad\r&\quad\r\\
              -\lambda & 0 & 0 \\
               0 & -\lambda & 0 \\
               0 & 0 & -\lambda \em
  - 1
\det\bm\format\r&\quad\r&\quad\r\\
               1 & 1 & 1 \\
               0 & -\lambda & 0 \\
               0 & 0 & -\lambda \em\\
 &\qquad+ 1
\det\bm\format\r&\quad\r&\quad\r\\
               1 & 1 & 1 \\
              -\lambda & 0 & 0 \\
               0 & 0 & -\lambda \em
 - 1
\det\bm\format\r&\quad\r&\quad\r\\
               1 & 1 & 1 \\
              -\lambda & 0 & 0 \\
               0 & -\lambda & 0 \em \\
&= (-\lambda)(-\lambda^3) -(\lambda^2) + (-\lambda^2) -(\lambda^2)\\
&= \lambda^4 - 3 \lambda^2 = \lambda^2(\lambda -\sqrt 3)(\lambda + \sqrt 3).
\endalign
$$
Hence, this matrix is singular just in the cases $\lambda = 0,
\lambda = \sqrt 3$, and $\lambda = -\sqrt 3$.
\endexample

\medskip
\subhead  Appendix.  Some Proofs \endsubhead
We now establish the basic rules relating determinants to
elementary row operations.   If you are of a skeptical turn of
mind, you should study this section, since 
the relation between the recursive definition
and rules (i), (ii), and (iii) is not at all obvious. 
  However, if you have a trusting
nature, you might want to skip this section since the
proofs are quite technical and not terribly enlightening.

   The idea behind the proofs is to assume that the
rules---actually,  modified forms of the rules---have
 been established for $(n-1)\times (n-1)$ determinants,
and then to prove them for $n\times n$ determinants.  To start
it all off, the rules  must be checked
explicitly for $2\times 2$ determinants.   I leave that step
for you in the Exercises.

%Exer
%  Prove that the following rules hold for $2\times 2$ determinants.
%
%   Rule (ia): $\det\bm a' + a'' & b' + b''\\ c & d \em
%     = \det\bm a' & b' \\ c & d \em + \det  a'' & b'' \\ c & d \em$.
%
%    Rule (ii): $\det\bm sa & sb \\ s & b \em = s\det \bm a & b \\ c & d \em$.
%
%    Rule (iii): $\det\bm a & b \\ c & d\em = - \det \bm c & d \\ a & b \em$.
%
%    Note that rules (ia) and (ii) need only be proved for the first row
%    since by means of rule (iii), those rules for the second row may be
%    reduced to the corresponding rules for the first row.

We start with the hardest case, rule (iii).   First we consider the
special case that $A'$ is obtained from $A$ by switching two {\it adjacent\/}
rows, the $i$th row and the $(i+1)$st row.  Consider the recursive
definition
$$
\multline
\det A'  = a'_{11}D_1(A') - \dots + (-1)^{i+1}a'_{i1}D_i(A') \\
     + (-1)^{i+2}a'_{i+1,1}D_{i+1}(A') + \dots + (-1)^{n+1}a'_{n1}D_n(A').
\endmultline
$$
Look at the subdeterminants occurring in this sum.   For $j \not= i, i+1$,
we have
$$
D_j(A') = - D_j(A)
$$
since deleting the first column and $j$th row of $A$ and then
switching two rows---neither of which was deleted---changes
the sign by rule (iii) for $(n-1)\times(n-1)$ determinants.
The situation for $j = i$ or $j=i+1$ is different; in fact, we have
$$
D_i(A') = D_{i+1}(A)\qquad\text{and}\qquad D_{i+1}(A')= D_i(A).
$$
The first equation follows because  switching
rows $i$ and $i+1$ and then deleting row $i$ is the same as
deleting row $i+1$ without touching row $i$.   A similar argument
establishes the second equation.   Using this together
with $a'_{i1} = a_{i+1,1}, a'_{i+1,1} = a_{i1}$ yields
$$\align
(-1)^{i+1}a'_{i1}D_i(A') &= (-1)^{i+1}a_{i+1,1}D_{i+1}(A) = - (-1)^{i+2}
a_{i+1,1}D_{i+1}(A) \\
(-1)^{i+2}a'_{i+1,1}D_{i+1}(A') &= (-1)^{i+2}a_{i1}D_i(A) = - (-1)^{i+1}
a_{i1}D_i(A).
\endalign$$
In other words, all terms in the recursive definition of $\det A'$
are negatives of the corresponding terms of $\det A$ {\it except\/}
those in positions $i$ and $i+1$ which get reversed with
signs changed.  Hence, the effect of switching adjacent rows is to
change the sign of the sum.

Suppose instead that non-adjacent rows in positions $i$ and $j$
are switched, 
and  suppose for the sake of argument
that $i < j$.  One way to do this is as follows.  First
move row $i$ past each of the rows between row $i$ and row $j$.  This
involves some number of switches of adjacent rows---call that
number $k$.  ($k = j - i - 1$, but
it that doesn't matter in the proof.)   Next, move
row $j$ past row $i$ and then past the $k$ rows just mentioned, 
all in their new positions.   That requires  $k + 1$ switches
of adjacent rows.  All told, to switch rows $i$ and $j$ in this
way requires $2k + 1$ switches of adjacent rows.  The net effect
is to multiply the determinant by $(-1)^{2k +1} = -1$ as required.

There is one important consequence of rule (iii) which we shall use
later in the proof of rule (i).
\medskip
\noindent
Rule (iiie):  {\it  If an $n\times n$ matrix has two equal rows,
then $\det A = 0$}.
\medskip
This is not too hard to see.   Interchanging two rows changes the
sign of $\det A$, but if the rows are equal, it doesn't change
anything.  However, the only number with the property that it
isn't changed by changing its sign is the number 0.  Hence,
$\det A = 0$.
\medskip
We next verify rule (ii).   Suppose $A'$ is obtained from
$A$ by multiplying the $i$th row by $c$.   Consider the
recursive definition
\nexteqn
$$
\det A'
= a'_{11}D_1(A') + \dots + (-1)^{i+1}a'_{i1}D_i(A') + \dots
+ (-1)^{n+1}a_{n1}D_n(A).\tag\eqn
$$
For any $j \not= i$, $D_j(A') = c D_j(A)$ since one of the
rows appearing in that determinant is multiplied by $c$.
Also, $a'_{j1} = a_{j1}$ for $j \not = i$.   On the other hand,
$D_i(A') = D_i(A)$ since the $i$th row  is
deleted in calculating these quantities, and, except for the $i$th row,
$A'$ and $A$ agree. In addition, $a'_{i1} = ca_{i1}$ so we
pick up the extra factor of $c$ in any case.  It follows that
every term on the right of (\eqn) has a factor $c$, so
$\det A' = c\det A$.
\medskip
Finally, we attack the proof of rule (i).  It turns out to
be necessary to verify the following stronger rule.
\medskip

\noindent
Rule (ia):  {\it Suppose $A, A'$, and $A''$ are three $n\times n$
matrices which agree except in the $i$th row.  Suppose moreover
that the $i$th row of $A$ is the sum of the $i$th row of $A'$
and the $i$th row of $A''$.  Then  $\det A = \det A' + \det A''$}.
\medskip
Let's first see why rule (ia) implies rule (i).   We can add
$c$ times the $j$th row of $A$ to its $i$ row as follows.
Let $B' = A$, let $B''$ be the matrix obtained from $A$ by
replacing its $i$th row by $c$ times its $j$th row, and let
$B$ be the matrix obtained form $A$ by adding $c$ times its $j$th row
to its $i$th row.
 Then according to rule (ia), we have
$$
\det B = \det B' + \det B'' = \det A + \det B''.
$$
On the other hand, by rule (ii), $\det B'' = c\det A''$
where $A''$ has both $i$th and $j$th rows equal to the
$j$th row of $A$.  Hence, by rule (iiie), $\det A'' = 0$,
and $\det B = \det A$.
\medskip
Finally, we establish rule (1a).  Assume it is known to be true
for $(n-1)\times(n-1)$ determinants.  We have
\nexteqn
$$
\det A = a_{11}D_1(A) - \dots + (-1)^{i+1}a_{i1}D_i(A) +
\dots + (-1)^{n+1}a_{n1}D_n(A).\tag\eqn
$$
For $j\not= i$, the the sum rule (ia) may be applied to the
determinants $D_i(A)$ because the appropriate submatrix
has one row which breaks up as a sum as needed.  Hence,
$$
D_j(A) = D_j(A') + D_j(A'').
$$
Also, for $j\not = i$, we have $a_{j1} = a_{j1}'  = a''_{j1}$ since
all the matrices agree in any row except the $i$th row.
Hence, for $j\not=i$,
$$
a_{i1}D_i(A) = a_{i1}D_i(A') + a_{i1}D_i(A'')
= a'_{i1}D_i(A') + a''_{i1}D_i(A'').
$$
On the other hand, $D_i(A) = D_i(A') = D_i(A'')$ because
in each case the $i$th row was deleted.
But $a_{i1} = a'_{i1} + a''_{i1}$, so
$$
a_{i1}D_i(A) = a'_{i1}D_i(A) + a''_{i1}D_i(A)
= a'_{i1}D_i(A') + a''_{i1}D_i(A'').
$$
It follows that every term in (\eqn) breaks up into a sum
as required, and $\det A = \det A' + \det A''$.
\bigskip
\input chap11.ex3
\bigskip

\nextsec{Some Important Properties of Determinants}
\head \sn.  Some Important Properties of Determinants \endhead

\nextthm
\proclaim{Theorem \cn.\tn{\rm\enspace (The Product Rule)}}
Let $A$ and $B$ be $n\times n$ matrices.   Then
$$
\det (AB) = \det A\, \det B.
$$
\endproclaim

\demo{Proof}  First assume that $A$ is non-singular.   
Then there is a sequence of row operations which reduces $A$
to the identity
$$
A \to A_1 \to A_2 \to \hdots \to A_k = I.
$$
Associated with each of these operations will be a multiplier
$c_i$ which will depend on the particular operation, and
$$
\det A = c_1\det A_1 = c_1c_2\det A_2 = \dots = 
c_1c_2\dots c_k\det A_k = c_1c_2\dots c_k
$$
since $A_k =I$ and $\det I = 1$.   Now apply exactly these
row operations to the product $AB$
$$
AB \to A_1B \to A_2B \to \hdots \to A_kB = IB = B.
$$
The same multipliers contribute factors at each stage, and
$$
\det AB = c_1 \det A_1B = c_1c_2\det A_2B =
\dots = \undersetbrace{\det A}\to{c_1c_2\dots c_k}\det B
      = \det A\, \det B.
$$

Assume instead that $A$ is singular.   Then, $AB$ is also
singular.   (This follows from the fact that the rank of $AB$
is at most the rank of $A$, as mentioned in the Exercises for
Chapter X, Section 6.
However, here is a direct proof for the record.  Choose   a sequence
of elementary row operations for $A$, the end result of which is a matrix
$A'$ with at least one row of zeroes.   Applying the same
operations to $AB$ yields $A'B$ which also has to have at least
one row of zeroes.)   It follows that both $\det AB$ and
$\det A\, \det B$ are zero, so they are equal.
\qed\enddemo
\medskip
\subhead Transposes \endsubhead
Let $A$ be an $m\times n$ matrix.   The {\it transpose\/} of
$A$ is the $n\times m$ matrix for which the columns are the
rows of $A$.   (Also, its rows are the columns of $A$.)
It is usually denoted $A^t$, but other notations
\outind{transpose of a matrix}%
are possible.

\example{Examples}
$$
\alignat2
A & = \bm 2 & 0 & 1 \\ 2 & 1 & 2 \em&\qquad
A^t &= \bm 2 & 2 \\ 0 & 1 \\ 1 & 2 \em \\
A &= \bm 1 & 2 & 3 \\ 0 & 2 & 3 \\ 0 & 0 &3 \em&\qquad
A^t &= \bm 1 & 0 & 0 \\ 2 & 2 & 0 \\ 3 & 3 & 3 \em \\
\a &= \bm a_1\\ a_2 \\ a_3 \em&\qquad \a^t &= \bm a_1 & a_2 & a_3 \em.
\endalignat$$
\endexample

The following rule follows almost immediately from the
definition.  Assume $A$ is an $m\times n$ matrix and $B$ is
an $n\times p$ matrix.   Then
$$
   (AB)^t = B^tA^t.
$$
Note that {\it the order on the right is reversed}.   Unless the matrices are
square, the shapes won't even match if the order is not reversed.

\nextthm
\proclaim{Theorem \cn.\tn}
Let $A$ be an $n\times n$ matrix.   Then
$$
\det A^t = \det A.
$$
\endproclaim

\nextex
\example{Example \en}
$$
\align
\det \bm 1 & 0 & 1 \\
         2 & 1 & 2 \\
         0 & 0 & 1 \em
   &= 1(1-0) - 2(0 - 0) + 0(\dots) = 1 \\
\det \bm 1 & 2 & 0 \\
         0 & 1 & 0 \\
         1 & 2 & 1 \em
   &=  1(1 - 0) - 0(\dots) + 1(0 - 0) = 1.
\endalign$$
\endexample

The importance of this theorem is that it allows us to go freely
from statements about determinants involving rows of the matrix
to corresponding statements involving columns and vice-versa.

%Advice to instructor.  You may want to skip this proof in class.

\demo{Proof of the Theorem}   If $A$ is singular, then $A^t$ is
also singular and vice-versa.
For, the rank may be characterized as either the
dimension of the row space or the dimension of the column space,
and an $n\times n$ matrix is singular if its rank is less than $n$.
Hence, in the singular case, $\det A = 0 = \det A^t$. 

Suppose then that $A$ is non-singular.  Then there is a sequence
of elementary row operations
$$
A \to A_1 \to A_2 \to \dots \to A_k = I.
$$
Recall from Chapter X, Section 4 
% Back reference
that each elementary row operation may be accomplished by multiplying
by an appropriate elementary matrix.   Let $C_i$ denote the elementary
matrix needed to perform the $i$th row operation.  Then,
$$
A \to A_1 = C_1A \to A_2 = C_2C_1A \to \dots \to
 A_k = C_kC_{k-1}\dots C_2C_1A = I.
$$
In other words,
$$
A = (C_k \dots C_2C_1)^{-1} = C_1{}^{-1}C_2{}^{-1}\dots C_k{}^{-1}.
$$
To simplify the notation, let $D_i = C_i{}^{-1}$.  The inverse  $D$
of an elementary matrix $C$ is also an elementary matrix; its effect
is the row operation which reverses the effect of $C$.   Hence,
we have shown that {\it any non-singular square matrix $A$
may be expressed as a product of elementary matrices}
$$
A = D_1D_2 \dots D_k.
$$
Hence, by the product rule
$$
\det A = (\det D_1)(\det D_2)\dots(\det D_k).
$$
On the other hand, we have by rule for the transpose of a product
$$
A^t = D_k{}^t\dots D_2{}^t D_1{}^t,
$$
so by the product rule
$$
\det A^t = \det(D_k{}^t)\dots \det(D_2{}^t)\det(D_1{}^t).
$$
Suppose we know the rule  $\det D^t = \det D$ {\it for any
elementary matrix $D$}.  
Then,
$$\align
\det A^t &= \det(D_k{}^t)\dots \det(D_2{}^t)\det(D_1{}^t)\\
       &= \det(D_k)\dots \det(D_2)\det(D_1) \\
       &= (\det D_1)(\det D_2)\dots(\det D_k) = \det A.
\endalign$$
(We used the fact that the products on the right are products
of scalars and so can be rearranged any way we like.)

It remains to establish the rule for elementary matrices.
If $D = E_{ij}(c)$ is obtained from the identity matrix by
adding $c$ times its $j$th row to its $i$th row, then
$D^t = E_{ji}(c)$ is a matrix of exactly the same type.
In each case, $\det D = \det D^t = 1$.  If $D = E_i(c)$ is
obtained by multiplying the $i$th row of the identity matrix
by $c$, then $D^t$ is exactly the same matrix $E_i(c)$.  Finally,
if $D = E_{ij}$ is obtained from the identity matrix by interchanging
its $i$th and $j$th rows, then  $D^t$ is
$E_{ji}$ which in fact is just $E_{ij}$ again.  Hence, in
each case $\det D^t = \det D$ does hold.
\qed\enddemo

Because of this rule, we may use {\it column operations\/} as
well as row operations to calculate determinants.   For, performing
a column operation is the same as transposing the matrix, performing
the corresponding row operation, and then transposing back.  The
two transpositions don't affect the determinant.

\nextex
\example{Example}
$$\align
\det \bm 1 & 2 & 3 & 0 \\
         2 & 1 & 3 & 1 \\
         3 & 3 & 6 & 2 \\
         4 & 2 & 6 & 4 \em
&=
\det \bm 1 & 2 & 2 & 0 \\
         2 & 1 & 1 & 1 \\
         3 & 3 & 3 & 2 \\
         4 & 2 & 2 & 4 \em
\qquad\text{operation } (-1)c1 + c3\\
&= 0.
\endalign
$$
The last step follows because the 2nd and 3rd columns are equal,
which implies that the rank (dimension of the column space) is
less than 4.  (You could also subtract the third column from the
second and get a column of zeroes, etc.)
\endexample
\medskip
\subhead Expansion in Minors or Cofactors \endsubhead
There is a generalization of the formula used for the recursive
definition.   Namely, for any $n\times n$ matrix $A$,  let
$D_{ij}(A)$ be the determinant of the $(n-1)\times(n-1)$ matrix
obtained by deleting the $i$th row and $j$th column of $A$.
\outind{cofactor}%
\outind{minor}%
Then,
\nexteqn
\xdef\LapCol{\eqn}
$$\align
\det A &= \sum_{i=1}^n(-1)^{i+j}a_{ij}D_{ij}(A)\tag\eqn \\
   &= (-1)^{1+j}a_{1j}D_{1j}(A) + \dots + (-1)^{i+j}a_{ij}D_{ij}(A)
   + \dots + (-1)^{n+j}a_{nj}D_{nj}(A).
\endalign$$
The special case $j = 1$ is the recursive definition given in the
previous section.   The more general rule is easy to derive
from the special case $j = 1$ by means of column interchanges.
Namely, form a new matrix $A'$
 by moving the $j$th column to the first position by successively
interchanging it with columns $j-1, j-2, \dots, 2, 1$.   There
are $j-1$ interchanges, so the determinant is changed by the
factor $(-1)^{j-1}$.  
Now apply the rule for the first column.  The first column of
$A'$ is the 
$j$th column of $A$, and deleting it has the same effect
as deleting the $j$th column of $A$.  Hence, $a'_{i1} = a_{ij}$
and $D_i(A') = D_{ij}(A)$.     Thus,
$$\align
\det A =
 (-1)^{j-1}\det A' &= (-1)^{j-1}\sum_{i=1}^n(-1)^{1+i}a'_{i1}D_i(A')\\
    &=\sum_{i=1}^n(-1)^{i+j}a_{ij}D_{ij}(A).
\endalign$$
\medskip
            
Similarly, there is a corresponding rule for any {\it row\/}
of a matrix
\nexteqn
\xdef\LapRow{\eqn}
$$\align
\det A &= \sum_{j=1}^n(-1)^{i+j}a_{ij}D_{ij}(A) \tag\eqn \\
&= (-1)^{i+1}a_{i1}D_{i1} + \dots + (-1)^{i+j}a_{ij}D_{ij}(A)
    + \dots + (-1)^{i+n}a_{in}D_{in}(A).
\endalign$$
This formula is obtained from (\LapCol) by transposing, applying
the corresponding column rule, and then transposing back.

\example{Example}
Expand the following determinant using its second row.
$$\multline
\det \bm 1 & 2 & 3 \\
         0 & 6 & 0 \\
         3 & 2 & 1 \em
 = (-1)^{2+3}0(\dots) + (-1)^{2+2}6\det \bm 1 & 3\\ 3 & 1 \em
         + (-1)^{2+3}0(\dots)\\ = 6(1 - 9)  = -48.
\endmultline
$$
\endexample

There is some terminology which you may see used in connection with
these formulas.   The determinant  $D_{ij}(A)$ of the
 $(n-1)\times(n-1)$ matrix obtained by deleting the $i$th row
and $j$th column is called the {\it $i,j$-minor\/} of $A$.    
The quantity $(-1)^{i+j}D_{ij}(A)$ is called the {\it $i,j$-cofactor}.
Formula (\LapCol) is called expansion in minors (or cofactors) of
the $j$th column and formula (\LapRow) is called expansion in
minors (or cofactors) of the $i$th row.   It is not necessary
to remember the terminology as long as you remember the formulas
and understand how they are used.
\medskip
\subhead Cramer's Rule \endsubhead
One may use determinants to derive a formula for the solutions of
 a {\it non-singular\/}
 system of $n$ equations in $n$ unknowns
$$
\bm a_{11} & a_{12} & \hdots & a_{1n} \\
   a_{21} & a_{22} & \hdots & a_{2n} \\
\vdots & \vdots & \hdots & \vdots \\
a_{n1} & a_{n2} & \hdots & a_{nn} \em
     \bm x_1\\ x_2\\ \vdots \\ x_ n \em
= \bm b_1\\ b_2 \\ \vdots \\ b_n \em.
$$
The formula is called {\it Cramer's rule\/}, and here it is.
\outind{Cramer's rule}%
{\it For the $j$th unknown $x_j$, take the determinant of the
matrix formed by replacing the $j$th column of the coefficient
matrix $A$ by $\b$, and divide it by $\det A$}.  In symbols,
$$
x_j =
\frac {\det\bm 
 a_{11} & \hdots & b_1 & \hdots & a_{1n} \\
   a_{21} & \hdots & b_2 & \hdots & a_{2n} \\
\vdots &\hdots & \vdots & \hdots & \vdots \\
a_{n1} &\hdots & b_n & \hdots & a_{nn} \em}
{\det\bm
a_{11} & \hdots & a_{1j} & \hdots & a_{1n} \\
   a_{21} & \hdots & a_{2j} & \hdots & a_{2n} \\
\vdots &\hdots & \vdots & \hdots & \vdots \\
   a_{n1} &\hdots & a_{nj} & \hdots & a_{nn} \em}
$$

\example{Example}
Consider
$$
\bm  1 & 0 & 2 \\
     1 & 1 & 2 \\
     2 & 0 & 6 \em
\bm x_1\\x_2\\x_3\em
=
\bm 1 \\ 5 \\ 3 \em.
$$
We have
$$
\det\bm
     1 & 0 & 2 \\
     1 & 1 & 2 \\
     2 & 0 & 6 \em = 2.
$$
(Do you see a quick way to compute that?)   Hence,
$$\align
x_1 &=
\frac{\det\bm
     1 & 0 & 2 \\
     5 & 1 & 2 \\
     3 & 0 & 6 \em}2
 = \frac 02 = 0 \\
x_1 &=
\frac{\det\bm
     1 & 1 & 2 \\
     1 & 5 & 2 \\
     2 & 3 & 6 \em}2
 = \frac 82 = 4 \\
x_3 &=
\frac{\det\bm
     1 & 0 & 1 \\
     1 & 1 & 5 \\
     2 & 0 & 3 \em} 2
= \frac 12.
\endalign
$$
You should try to do this by Gauss-Jordan reduction.

Cramer's rule is not too useful for solving specific numerical
systems of equations.   The only practical method for calculating
the needed determinants for $n$ large is to use row (and possibly
column) operations.  It is usually easier to use row operations
to solve the system without resorting to determinants.   However,
if the system has non-numeric symbolic coefficients, Cramer's
rule is sometimes useful.  Also, it is often valuable as a theoretical
tool.

Cramer's rule is related to expansion in minors.   You can find
further discussion of it and proofs in Section 5.4 and 5.5
of  {\it Introduction to
Linear Algebra\/} by Johnson, Riess, and Arnold.  (See also
Section 4.5 of {\it Applied Linear Algebra\/} by Noble and
Daniel.)
\bigskip
\input chap11.ex4
\bigskip

\nextsec{Eigenvalues and Eigenvectors}
\head \sn.  Eigenvalues and Eigenvectors for $n\times n$ Matrices \endhead

As in Section 2,
we want to solve an $n\times n$ system
\nexteqn
$$
\frac{d\x}{dt} = A\x\tag\eqn
$$
by finding a basis $\{\x_1(t), \x_2(t), \dots, \x_n(t)\}$ of the
solution space.   In case $A$ {\it is constant\/}, it was suggested
that we {\it look for\/} solutions of the form
$$
 \x = e^{\lambda t}\v
$$
where $\lambda$ and $\v \not= 0$ are to be determined by the process.
Such solutions form a linearly independent set
as long as the corresponding $\v$'s form a linearly independent
set.   For, suppose 
$$
\x_1 = e^{\lambda_1 t}\v_1, \x_2 = e^{\lambda_2 t}\v_2,
\dots, \x_k = e^{\lambda_k t}\v_k 
$$
are $k$ such solutions.   We know that the
set of solutions $\{\x_1(t), \dots, \x_k(t)\}$ is linearly independent
if and only if the set of vectors 
obtained by evaluating the functions at $t = 0$ is linearly
independent.  However, this set of vectors is just
$\{\v_1, \dots, \v_k\}$.


We discovered in Section 2 that trying a solution of the form
$\x = e^{\lambda t}\v$ leads to the eigenvalue--eigenvector
problem
\outind{eigenvalue}%
\outind{eigenvector}%
\nexteqn
$$
  A\v = \lambda \v.\tag\eqn
$$

We redo some of the algebra in Section 2 as follows.
Rewrite equation (\eqn) as 
$$
\gather
  A\v = \lambda \v \\
  A\v - \lambda \v = 0 \\
  A \v - \lambda I\v = 0 \\
  (A - \lambda I)\v = 0.
\endgather
$$
The last equation is the homogeneous $n\times n$ system 
with $n\times n$ coefficient matrix 
$$
A - \lambda I = 
\bm a_{11} - \lambda & a_{12} & \hdots & a_{1n} \\
 a_{21} & a_{22} - \lambda & \hdots & a_{2n} \\
\vdots & \vdots & \hdots & \vdots \\
a_{n1} & a_{n2} & \hdots &  a_{nn} - \lambda \em.  
$$
It has
a {\it non-zero\/} solution vector $\v$ if and only if
the coefficient matrix has rank less than $n$,
i.e., if and only if it is {\it singular\/}.  By
Theorem 11.3, 
% Back reference
this will be true if and only if $\lambda$ satisfies the
{\it characteristic equation\/}
\nexteqn
\xdef\CharEqnII{\eqn}
$$
   \det (A - \lambda I) 
 = \det
\bm a_{11} - \lambda & a_{12} & \hdots & a_{1n} \\
 a_{21} & a_{22} - \lambda & \hdots & a_{2n} \\
\vdots & \vdots & \hdots & \vdots \\
a_{n1} & a_{n2} & \hdots &  a_{nn} - \lambda \em  
= 0.\tag\eqn
$$
\outind{characteristic equation}%

As in Section 2, the strategy for finding
eigenvalues and eigenvectors is
as follows.  First find the roots of the characteristic
equation.   These are the eigenvalues.  Then for each root $\lambda$, find
a general solution for the system
\nexteqn
$$
(A - \lambda I)\v = 0.\tag\eqn
$$
This gives us all the eigenvectors for that eigenvalue.

\nextex
\example{Example \en}  Consider the matrix
$$
 A = \bm 1 & 4 & 3 \\
         4 & 1 & 0 \\
         3 & 0 & 1 \em.
$$
The characteristic equation is
$$\align
\det(A - \lambda I) &= \det \bm 1 - \lambda & 4 & 3 \\
                                4 &  1 - \lambda & 0 \\
                                3 & 0 & 1 - \lambda \em \\ 
&= (1-\lambda)((1-\lambda)^2- 0) - 4(4(1 - \lambda)- 0)
 + 3(0 - 3(1-\lambda) \\
 &= (1-\lambda)^3 - 25(1 - \lambda) = (1 - \lambda)((1- \lambda)^2 -25)\\
  &= (1 - \lambda)(\lambda^2 - 2\lambda - 24) = 
(1-\lambda)(\lambda - 6)(\lambda + 4) = 0.
\endalign
$$
Hence, the eigenvalues are $\lambda = 1, \lambda = 6$, and
$\lambda = -4$.  We proceed to find the eigenvectors for each
of these eigenvalues, starting with the largest.

First, take $\lambda = 6$, and  put it in (\eqn) to obtain the system
$$
 \bm 1 - 6 & 4 & 3 \\
         4 & 1- 6 & 0 \\
         3 & 0 & 1-6 \em
\bm v_1\\ v_2\\ v_3 \em = 0
\qquad\text{or}\qquad
 \bm\format\r&\quad\r&\quad\r\\
        -5 & 4 & 3 \\
         4 & -5 & 0 \\
         3 & 0 & -5 \em
\bm v_1\\ v_2\\ v_3 \em = 0.
$$
To solve, use Gauss-Jordan reduction
$$\align
 \bm\format\r&\quad\r&\quad\r\\
        -5 & 4 & 3 \\
         4 & -5 & 0 \\
         3 & 0 & -5 \em
&\to
 \bm\format\r&\quad\r&\quad\r\\
        -1 & -1 & 3 \\
         4 & -5 & 0 \\
         3 & 0 & -5 \em
\to
 \bm\format\r&\quad\r&\quad\r\\
        -1 & -1 & 3 \\
         0 & -9 & 12 \\
         0 & -3 & 4 \em \\
&\to
 \bm\format\r&\quad\r&\quad\r\\
        -1 & -1 & 3 \\
         0 & 0 & 0 \\
         0 & -3 & 4 \em
\to
 \bm\format\r&\quad\r&\quad\r\\
        -1 & -1 & 3 \\
         0 & 3 & -4 \\
         0 & 0 & 0 \em \\
&\to
 \bm\format\r&\quad\r&\quad\r\\
        1 & 1 & -3 \\
         0 & 1 & -4/3 \\
         0 & 0 & 0 \em 
\to
 \bm\format\r&\quad\r&\quad\r\\
        1 & 0 & -5/3 \\
         0 & 1 & -4/3 \\
         0 & 0 & 0 \em .
\endalign$$
Note that the matrix is singular, and the rank is smaller than 3.
This must be the case because the condition $\det(A - \lambda I) = 0$
guarantees it.  If  the coefficient matrix
were non-singular, you would know that there was a mistake: either
the roots of the characteristic equation are wrong or the
row reduction was not done correctly.

The general solution is
$$\align
v_1 &= (5/3)v_3 \\
v_2 &=  (4/3)v_3 
\endalign$$
with $v_3$ free.   The general solution vector is
$$
\v = \bm (5/3)v_3\\ (4/3)v_3\\ v_3 \em
 = v_3\bm 5/3\\ 4/3\\ 1 \em.
$$
Hence, the solution space is 1-dimensional.  A basis may be obtained
by setting $v_3 = 1$ as usual, but it is a bit neater to
put $v_3 = 3$ so as to avoid fractions.
Thus,
$$
\v_1 = \bm 5 \\ 4 \\ 3 \em
$$
constitutes a basis for the solution space.
Note that we have now found all eigenvectors for the eigenvalue
$\lambda = 6$.   They are all the non-zero vectors in the 1-dimensional
solution subspace, i.e., all non-zero multiples of $\v_1$.

Next take $\lambda = 1$ and put it in (\eqn) to obtain the system
$$
\bm 0 & 4 & 3 \\
    4 & 0 & 0 \\
    3 & 0 & 0 \em \bm v_1\\ v_2\\v_3 \em = 0.
$$
Use Gauss-Jordan reduction
$$
\bm 0 & 4 & 3 \\
    4 & 0 & 0 \\
    3 & 0 & 0 \em 
\to
\bm 1 & 0 & 0 \\
    0 & 1 & 3/4 \\
    0 & 0 & 0 \em.
$$
The general solution is
$$\align
v_1 &= 0 \\
v_2 &= -(3/4)v_3
\endalign$$
with $v_2$ free.
Thus the general solution vector is
$$
\v = \bm 0 \\ -(3/4)v_3 \\ v_3 \em  = v_3\bm 0\\ -3/4 \\ 1 \em.
$$
Put $v_3 = 4$ to obtain a single basis vector
$$
\v_2 = \bm 0 \\ -3 \\ 4 \em.
$$
The set of eigenvectors for the eigenvalue $\lambda = 1$
is the set of non-zero multiples of $\v_2$.

Finally, take $\lambda = -4$, and put this in (\eqn) to obtain the
system
$$
\bm 5 & 4 & 3 \\
    4 & 5 & 0 \\
    3 & 0 & 5 \em
\bm v_1\\ v_2\\ v_3 \em = 0.
$$
Solve this by Gauss-Jordan reduction.
$$\align
\bm 5 & 4 & 3 \\
    4 & 5 & 0 \\
    3 & 0 & 5 \em
&\to
\bm\format\r&\quad\r&\quad\r\\
    1 & -1 & 3 \\
    4 & 5 & 0 \\
    3 & 0 & 5 \em
\to
\bm\format\r&\quad\r&\quad\r\\
    1 & -1 & 3 \\
    0 & 9 & -12 \\
    0 & 3 & -4 \em\\
&\to
\bm\format\r&\quad\r&\quad\r\\
    1 & -1 & 3 \\
    0 & 3 & -4 \\
    0 & 0 & 0 \em
\to
\bm\format\r&\quad\r&\quad\r\\
    1 & 0 & 5/3 \\
    0 & 1 & -4/3 \\
    0 & 0 & 0 \em.
\endalign$$
The general solution is
$$\align
v_1 &= -(5/3)v_3 \\
v_2 &= (4/3)v_3 
\endalign$$
with $v_3$ free.  The general solution vector is
$$
\v = \bm -(5/3)v_3\\ (4/3)v_3 \\ v_3 \em
   = v_3\bm -5/3 \\ 4/3 \\ 1 \em.
$$
Setting $v_3 = 3$ yields the basis vector
$$
\v_3 = \bm \format\r\\ -5 \\ 4\\ 3 \em.
$$
The set of eigenvectors for the eigenvalue $\lambda = -4$ consists
of all non-zero multiples of $\v_3$.
\endexample

The set  $\{\v_1, \v_2, \v_3\}$ obtained in the previous
example  is  linearly independent.
 To see this apply Gaussian reduction to the matrix with
these vectors as columns:
$$
\bm \format\r&\quad\r&\quad\r\\
     5 & 0 & -5 \\
     4 & -3 & 4 \\
     3 & 4 & 3 \em
\to
\bm \format\r&\quad\r&\quad\r\\
     1 & 0 & -1 \\
     0 & -3 & 8 \\
     0 & 4 & 6 \em
\to 
\bm \format\r&\quad\r&\quad\r\\
     1 & 0 & -1 \\
     0 & 1 & -8/3 \\
     0 & 0 & 50/3 \em.
$$
The reduced matrix has rank 3, so the columns of the original
matrix form an independent set.

It is  no accident that a set so obtained is linearly
independent.   The following theorem tells us that this will
always be the case.

\nextthm
\proclaim{Theorem \cn.\tn}  Let $A$ be an $n\times n$ matrix.
Let $\lambda_1, \lambda_2, \dots, \lambda_k$ be different
eigenvalues of $A$, and let $\v_1, \v_2, \dots, \v_k$
be corresponding eigenvectors.   Then 
$$\{\v_1, \v_2, \dots, \v_k\}$$
is a linearly independent set.
\endproclaim
%Advice to instructor.  Just do the proof for $k = 3$.
\demo{Proof}  Assume $\{\v_1, \v_2,\dots, \v_k\}$ is not a linearly
independent set, and  try to derive a contradiction.
In this case, one of the vectors in the set can be
expressed as a linear combination of the others.   If we number
the elements appropriately, we may assume that
\nexteqn
\xdef\EqA{\eqn}
$$
\v_1 = c_2\v_2 + \dots + c_k\v_r,\tag\eqn
$$
where $r \le k$.   (Before renumbering,
leave out any vector $\v_i$ on the right
if it appears  with coeficient $c_i = 0$.)
Note that we may also assume that no vector which appears on the
right is a linear combination of the others because otherwise
we could express it so and after combining terms delete
it from the sum.  Thus we may assume the vectors which appear
on the right form a linearly independent set.
Multiply (\EqA)  on the left by $A$.  We get
\nexteqn
\xdef\EqnB{\eqn}
$$\align
A\v_1 &= c_2A\v_2 + \dots + c_kA\v_k\\
\lambda_1\v_1 &= c_2\lambda_2\v_2 + \dots c_k\lambda_k\v_k\tag\eqn
\endalign$$
where in (\eqn) we used the fact that each $\v_i$ is an eigenvector
with eigenvalue $\lambda_i$.  Now multiply (\EqA) by $\lambda_1$
and subtract from (\EqnB).  We get
\nexteqn
$$
0 = c_2(\lambda_2 - \lambda_1)\v_2 + \dots + c_k(\lambda_k - \lambda_1)\v_k.
\tag\eqn
$$
Not all the coefficients on the right in this equation are zero.
For at least one of the $c_i \not = 0$ (since $\v_1 \not=0$),
and none of the quantities $\lambda_2 - \lambda_1, \dots 
\lambda_k - \lambda_1$ is zero.   It follows that (\eqn)
may be used to express one
of the vectors $\v_2, \dots, \v_k$ as a linear combination of the
others.  However, this contradicts the assertion that the set
of vectors appearing on the right is linearly independent.
Hence, our initial assumption that the set
$\{\v_1, \v_2, \dots, \v_k\}$ is dependent must be false,
and the theorem is proved. 

You should try this argument out on a set $\{\v_1, \v_2, \v_3\}$
of three eigenvectors to see if you understand it.
 \qed\enddemo

\subhead Historical Aside \endsubhead
The concepts discussed here and in Section 2
 were invented by the 19th century English mathematicians Cayley
and Sylvester, but they used the terms `characteristic vector'
and `characteristic value'.   These were translated into German
as `Eigenvector' and `Eigenwerte', and then partially translated
back into English---largely by physicists---as `eigenvector'
and `eigenvalue'.   Some English and American mathematicians tried
to retain the original English terms, but they were overwhelmed
by extensive use of the physicists' language in applications.  Nowadays
everyone uses the German terms.   The one exception is that we
still call
$$
\det(A - \lambda I) = 0
$$
the characteristic equation and not some strange German-English
name. 
\medskip
\subhead Application to Homogeneous Linear Systems of Differential Equations
\endsubhead
What lessons can we learn for solving systems of differential equations
from the previous discussion of eigenvalues?   First, Theorem \cn.\tn\ 
assures us that if the $n\times n$ matrix $A$ has distinct eigenvalues
$\lambda_1, \lambda_2, \dots, \lambda_k$ corresponding to
eigenvectors $\v_1, \v_2, \dots, \v_k$, then the 
functions
$$
\x_1= e^{\lambda_1t}\v_1,\,\x_2 = e^{\lambda_2t}\v_2, \dots, 
\x_k = e^{\lambda_kt}\v_k
$$
form a linearly independent set of solutions of
$\dfrac{d\x}{dt} = A\x$.   If $k=n$ then
this set will be a basis for the space of solutions of
$\dfrac{d\x}{dt} = A\x$.  (Why?)

\example{Example \en a}  Consider the system
$$
\frac{d\x}{dt} =
     \bm 1 & 4 & 3 \\
         4 & 1 & 0 \\
         3 & 0 & 1 \em \x.
$$
We found that $\lambda_1 = 6, \lambda_2 = 1, \lambda_3 = -4$
are eigenvalues of the coefficient matrix corresponding to
eigenvectors
$$
\v_1 = \bm\format\r\\5\\4\\3\em,\,
\v_2 = \bm\format\r\\0\\-3\\4\em,\,
\v_3 = \bm\format\r\\-5\\4\\3\em.
$$
Since $k = 3$ in this case, we conclude that the general solution
of the system of differential equations is
$$
\x =
c_1e^{6t}
 \bm\format\r\\5\\4\\3\em +
c_2e^{t}
 \bm\format\r\\0\\-3\\4\em +
c_3e^{-4t}
 \bm\format\r\\-5\\4\\3\em.
$$
\endexample

The above example illustrates that 
we should ordinarily look for a linearly independent set of eigenvectors,
as large as possible,
for the $n\times n$ coefficient matrix $A$.  If we can find such a set with $n$
elements,
then we may write out a complete solution 
as in the example.  The condition that there is a linearly independent set
of $n$ eigenvectors for $A$, i.e., that {\it there is a basis for
$\R^n$ ($\C^n$ in the complex case) consisting of eigenvectors for
$A$\/}, will certainly be verified if there are $n$ distinct
eigenvalues  (Theorem \cn.\tn).   We shall see later that there are 
other circumstances in which it holds.  On the other hand, it is easy to find
examples where it fails.

\nextex
\example{Example \en}
Consider the $2\times 2$ system
$$
\frac{d\x}{dt} = \bm 2 & 3 \\ 0 & 2 \em \x.
$$
The eigenvalues are found by solving the characteristic equation
of the coefficient matrix
$$
\det \bm 2 - \lambda & 3 \\ 0 & 2- \lambda \em
 = (2 - \lambda)^2 = 0.
$$
Hence there is only one (double) root $\lambda = 2$.  To find
the corresponding eigenvectors, solve
$$
\bm 0 & 3 \\ 0 & 0 \em \v = 0.
$$
This one is easy, (but you can make it hard for yourself if
you get confused about Gauss-Jordan reduction).  The general solution
is
$$
v_2 = 0,\qquad v_1\quad\text{free}.
$$
Hence, the general solution vector is
$$
\v = \bm v_1\\ 0\em = v_1\bm 1\\ 0 \em.
$$
Hence, a basic eigenvector for $\lambda = 2$ is 
$$
\v_1 = \bm 1\\ 0 \em = \e_1.
$$
$\{\v_1\}$ is certainly not a basis for $\R^2$. 
\endexample

In the sections that follow, we shall be concerned with these
two related questions.  Given an $n\times n$ matrix $A$, when 
can we be sure that there is a basis
for $\R^n$ ($\C^n$ in the complex case) consisting of eigenvectors
for $A$?   If there is no such basis, is there another way to
solve the system $d\x/dt = A\x$?

\subhead Solving Polynomial Equations \endsubhead
To find the eigenvalues of an $n\times n$ matrix, you have
to solve a polynomial equation.  You all know how to solve
quadratic equations, but you may be stumped by cubic or
higher equations, particularly if there are no obvious
ways to factor.  You should review what you learned in high
school about this subject, but here are a few guidelines
to help you.
\outind{polynomial equations}%

First, it is not generally possible to find a simple solution
in closed form for an algebraic equation.   For most equations
you might encounter in practice, you would have to use
some method to approximate a solution.  (Many such methods exist.
One you may have learned in your calculus course is
{\it Newton's Method\/}.)   Unfortunately, an approximate
solution of the characteristic equation isn't much good for
finding the corresponding eigenvectors.  After all, the
system
$$ (A - \lambda I)\v = 0 $$
must have rank smaller than $n$ for there to be non-zero solutions.
If you replace the exact value of $\lambda$  by an approximation,
the chances are that the new system will have rank $n$.  Hence,
the textbook method we have described for finding eigenvectors
won't work.  There are in fact many alternative methods for
finding eigenvalues and eigenvectors approximately when exact
solutions are not available.   Whole books are devoted to
such methods.  (See {\it Johnson, Riess, and Arnold\/} or
{\it Noble and Daniel\/} for some discussion of these matters.)  

Fortunately, textbook  exercises and examination questions
 almost always involve characteristic equations for
which exact solutions exist, but it is not always obvious what
they are.   Here is one  fact (a consequence of an
important result called {\it Gauss's Lemma\/}) which helps us
find such exact solutions when they exist.  Consider an
equation of the form
$$
\lambda^n + a_1\lambda^{n-1} + \dots + a_{n-1}\lambda + a_n = 0
$$
where all the coefficients are {\it integers}.   (The
characteristic equation of a matrix always has leading coefficient
1 or $-1$.  In the latter case, just imagine you have
multiplied through by $-1$
to apply the method.)   Gauss's Lemma tells us that if this
equation has any roots which are {\it rational numbers\/},
i.e., quotients of integers, then any such root is actually
an integer, and, moreover, it must divide the constant term
$a_n$.  Hence, the first step in solving such an equation should
be checking all possible factors (positive and negative)
of the constant term.  Once, you know a root $r_1$, you can divide through
by $\lambda - r_1$ to reduce to a lower degree equation.
If you know the method of synthetic division, you will find
checking the possible roots and the polynomial long division 
much simpler.

\nextex
\xdef\ExBB{\en}
\example{Example \en}
Solve 
$$
\lambda^3 -3\lambda  + 2 = 0.
$$
If there are any rational roots, they must be factors of the
constant term 2.  Hence, we must try $1, -1, 2, -2$.  Substituting
$\lambda = 1$ in the equation yields 0, so it is a root.
Dividing $\lambda^3 - 3\lambda + 2$ by $\lambda - 1$ yields
$$
\lambda^3 - 3\lambda + 2 = (\lambda - 1)(\lambda^2 + \lambda -2)
$$
and this may be factored further to obtain
$$\lambda^3 - 3\lambda^2 + 2 = (\lambda - 1)(\lambda -1)(\lambda + 2)
 = (\lambda - 1)^2(\lambda + 2).
$$
Hence, the roots are $\lambda = 1$ which is a double root and
$\lambda = -2$.
\medskip
\subhead Eigenvalues and Eigenvectors for Function Spaces \endsubhead
The concepts of eigenvalue and eigenvector make sense for
a linear operator $L$ defined on any vector space $V$, i.e.,
$\lambda$ is an eigenvalue for $L$ with eigenvector $\v$
if
$$
L(\v) = \lambda \v\qquad\text{with } \v \not= \bold 0.
$$
If the vector space $\V$ is not finite dimensional, then the
use of the characteristic equation and the other methods introduced
in this section do not apply, but the concepts are still very
useful, and other methods may be employed to calculate them.

In particular, eigenvalues and eigenvectors arise naturally
in the function spaces which occur in solving
differential equations, both ordinary and partial.  Thus, if
you refer back to the analysis of the vibrating drum
problem in Chapter IX, Section 2, you will recall that the
process of separation of variables led to equations of
the form
$$
\align
\Theta'' &= \mu \Theta \\
R'' + \frac 1r R' - \frac{m^2}{r^2} &= \gamma R
\qquad\text{where } \mu = -m^2.
\endalign $$
Here I took some liberties with the form of the equations in
order to emphasize the relation with eigenvalues and eigenvectors.
In each case, the equation has the form  $L(\psi) = \lambda\psi$
where $\psi$ denotes a function, and $L$ is an appropriate
differential operator:  
$$
\align
L &=  \frac{d^2}{d\theta^2}\\
L &= \frac{d^2}{dr^2}+ \frac 1r\frac d{dr} -\frac{m^2}{r^2}.
\endalign 
$$ 
There is one subtle but crucial point here.  The allowable
functions  $\psi$ in the domains of these operators are not
arbitrary but have {\it other conditions\/} imposed on them
by their interpretation in the underlying physical problem.
Thus, we impose the periodicity condition $\Theta(\theta + 2\pi)
= \Theta(\theta)$ because of the geometric meaning of the
variable $\theta$, i.e.,  the domain of the operator
$L = \dfrac{d^2}{d\theta^2}$ is restricted to such periodic
functions.   The eigenvalue--eigenvector condition
$L(\Theta) = \mu \Theta$ amounts to a differential equation
which is easy to solve---see Chapter IX, Section 2---but the
periodicity condition limits the choice of the
eigenvalue $\mu$ to numbers
of the form $\mu = -m^2$ where $m$ is a non-negative integer.
The corresponding eigenvectors (also called appropriately
eigenfunctions) 
\outind{eigenfunction}%
are the corresponding solutions of the differential equation
given by
$$\Theta(\theta) = c_1\cos m\theta + c_2\sin m\theta.$$
Similarly, the allowable functions $R(r)$ must satisfy
the boundary condition $R(a) = 0$.
Solving the
eigenvalue--eigenvector problem $L(R) = \gamma R$ 
in this case amounts to  solving Bessel's equation and
finding the eigenvalues $\gamma$ comes down to finding roots
of Bessel functions.

This approach is commonly used in the study of partial differential
equations, and you will go into it thoroughly in your course
on Fourier series and boundary value problems.  It is also
part of the formalism used to describe
quantum mechanics.   In that theory, linear operators correspond to
observable quantities like position, momentum, energy, etc.,
and the eigenvalues of these operators are the possible
results of measurements of these quantities.

\bigskip
\input chap11.ex5
\bigskip

\nextsec{Complex Roots}
\head \sn.  Complex Roots \endhead
Let $A$ be an $n\times n$ matrix.
The characteristic equation
$$
\det(A  - \lambda I) = 0
$$
is a polynomial equation of degree $n$ in $\lambda$.   The 
{\it Fundamental Theorem of Algebra\/} tells us that
\outind{fundamental theorem of algebra}%
such an equation has $n$ {\it complex\/} roots, at least
if we count repeated roots with proper multiplicity.  
Some  or all of these roots may be real, but even if $A$ is a real
matrix, some of the roots may be non-real complex numbers.   For example,
the characteristic equation of 
$$A = \bm\format\r&\quad\r\\ 0 & -1 \\ 1 & 0 \em
\qquad\text{is } \det\bm\format\r&\quad\r\\ -\lambda & -1 \\ 1 & \lambda\em
= \lambda^2 +1 = 0
$$
which has roots $\lambda = \pm i$.
 We want to see in general how the
nature of these roots affects the calculation of the eigenvectors
of $A$.
\outind{complex eigenvalues}%
\outind{eigenvalues, complex}%

First, suppose that $A$ has some non-real complex entries, that
is, not all its entries are real.

\nextex
%Advice to instructor Skip this example in class
\example{Example \en}
Consider
$$
A = \bm 0 & i \\ -i & 0 \em.
$$
The characteristic equation is
$$
\det \bm -\lambda & i \\ -i & -\lambda \em = \lambda^2 -(-i^2)
   = \lambda^2 - 1 = 0.
$$
Thus the eigenvalues are $\lambda = 1$ and $\lambda = -1$.
For $\lambda = 1$, we find the eigenvectors by solving
$$
\bm\format\r&\quad\r\\ -1 & i \\ -i & -1 \em
\v = 0.
$$
Gauss-Jordan reduction yields
$$
\bm\format\r&\quad\r\\ -1 & i \\ -i & -1 \em
\to 
\bm\format\r&\quad\r\\ 1 & -i \\ 0 & 0 \em.
$$
(Multiply the first row by $i$ and add it to the second row; then
change the signs in the first row.)  Thus the general solution is
$$
v_1 = i v_2,\qquad\qquad v_2\  \text{free}.
$$
A general solution vector is
$$
\v = \bm\format\r\\ v_2 i\\ v_2 \em = v_2 \bm\format\r\\ i\\ 1 \em
$$
Thus a basic eigenvector for $\lambda = 1$ is 
$$\v_1 = \bm i \\ 1 \em.$$

A similar calculations shows that a basic eigenvector for the
eigenvalue $\lambda = -1$ is
$$
\v_2 = \bm\format\r\\ -i\\ 1\em.
$$
\endexample

The above example shows that when some of the entries are non-real
complex numbers,
we should expect complex eigenvectors.   That is, the proper domain
to consider is the complex vector space $\C^n$.

Suppose instead that $A$ has only real entries.  It may still be
the case that some of the roots of the characteristic equation
are not real.  
We have two choices.   We can consider only {\it real\/}
roots as possible eigenvalues.  For such roots $\lambda$, we
may consider only {\it real\/} solutions of the
system
$$
(A - \lambda I)\v = 0.
$$
That is, we choose as our domain of attention the {\it real\/}
vector space $\R^n$.  In effect, we act as if we don't know about
complex numbers.
Clearly, we will be missing something this way.  We will have a better
picture of what is happening if we also consider the non-real complex
roots of the characteristic equation.  Doing that will ordinarily
lead to complex eigenvectors, i.e., to  the complex vector vector space
$\C^n$.

\nextex
\example{Example \en}
Consider 
$$
A = \bm\format\r&\quad\r\\ 2 & 1 \\ -2 & 0 \em.
$$
The characteristic equation is
$$
\det
\bm 2-\lambda & 1 \\ -2 & -\lambda \em = \lambda^2 - 2\lambda +2 = 0.
$$
The roots of this equation are
$$
\frac{2 \pm \sqrt{4 - 8}}2 = 1 \pm i.
$$

Neither of these roots are real, so considering this a purely real
problem in $\R^n$ will yield {\it no\/} eigenvectors.

Consider it instead as a complex problem.   The eigenvalue $\lambda = 1 + i$
yields the system
$$
\bm 1 - i & 1 \\ -2 & -1 - i \em\v = 0.
$$
Gauss-Jordan reduction (done carefully to account for the complex
entries) yields
$$\align
\bm 1 - i & 1 \\ -2 & -1 - i \em
&\to  \bm 1 & (1 + i)/2 \\ 1 - i & 1 \em
\qquad\text{switch rows, divide by }-2\\
&\to \bm1 & (1 +i)/2 \\ 0 & 0 \em.
\endalign$$
(The calculation of the last $2,2$-entry is
$1 - (1 -i)\dfrac{(1 +i)}2 = 1 - 1 = 0$.)
The general solution  is
$$
v_1 = -\frac{(1 + i)}2v_2\qquad v_2\quad\text{free}.
$$
The general solution vector is
$$
\v = v_2\bm -(1 + i)/2 \\ 1 \em.
$$
Putting $v_2 = 2$ to avoid fractions yields a basic eigenvector 
$$
\v_1 = \bm -1 - i\\ 2 \em
$$
for the eigenvalue $\lambda = 1 + i$.

A similar calculation may be used to determine the eigenvectors for
the eigenvalue $1 - i$.    However, there is a shortcut
based on the fact that the second eigenvalue $1 - i$ is the
{\it complex conjugate\/} $\overline\lambda$
 of the first eigenvalue $1 + i$.
To see how this works requires a short digression.
\medskip
Suppose $\v$ is an eigenvector with eigenvalue $\lambda$.  This means
that
$$
A\v = \lambda \v.
$$
Now take the complex conjugate of everything in sight on both sides
of this equation.  This yields
$$
\overline A\overline\v = \overline\lambda \overline\v.
$$
(Here, putting a `bar' over a matrix means that you should take the
complex conjugate of every entry in the matrix.)  Since $A$
is {\it real\/}, we have $\overline A = A$.  Thus, we have
$$
A\overline\v = \overline\lambda\overline\v.
$$
In words, {\it for a real $n\times n$ matrix,
the complex conjugate of an eigenvector is also an eigenvector,
and the eigenvalue corresponding to the latter is the complex
conjugate of the eigenvalue corresponding to the former}.

\medskip
Applying this principle in Example \en\ yields
$$
\v_2 = \overline\v_1 = \bm -1 + i \\ 2 \em
$$
as a basic eigenvector for eigenvalue $\overline\lambda = 1 - i$.
\endexample

\subhead Application to Homogeneous Linear Systems of Differential
Equations \endsubhead
Given a system of the form  $\dfrac{d\x}{dt} = A\x$ where $A$
is a real $n\times n$ matrix, we know from our previous work
with second order differential equations that it may be
useful to consider complex valued solutions $\x = \x(t)$.

\example{Example \en, expanded}   Consider the system
$$
\frac{d\x}{dt} = \bm\format\r&\quad\r\\   2 & 1 \\ -2 & 0 \em \x.
$$
Since the roots of the characteristic equation of
the coefficient matrix are complex, if we look for solutions
$\x(t)$ taking values in $\R^2$, we won't get anything by
the eigenvalue-eigenvector method.   Hence, it makes sense to
look for solutions with values in $\C^2$.   Then, according to
our previous calculations, the general solution will be
$$
\align
\x &=c_1e^{\lambda t}\v_1 + c_2e^{\overline\lambda t}\overline\v_1 \\
   &=
c_1 e^{(1+i) t}
\bm  -1 - i\\ 2 \em + c_2e^{(1 - i)t}\bm -1 + i \\ 2 \em.
\endalign
$$
As usual, the constants $c_1$ and $c_2$ are arbitrary complex
scalars.

It is often the case in applications that the complex solution
is meaningful in its own right, but there are also occasions
where one wants a real solution.   For this, we adopt the
same strategy we used when studying second order linear
differential equations:  {\it take the real and imaginary parts
of the complex solution}.  This will be valid if $A$ is real
since if $\x(t) = \u(t) + i\v(t)$ is a solution
 with $\u(t)$ and $\v(t)$ real, then we have
$$\gather
\frac{d\x}{dt} = A\x \\
\frac{d\u}{dt} + i\frac{d\v}{dt} = A\u + iA\v
\endgather$$
so comparing real and imaginary parts on both sides, we
obtain
$$
\frac{d\u}{dt} = A\u\qquad\text{and}\qquad \frac{d\v}{dt} = A\v.
$$
Note that we needed to know $A$ is real in order to know that
$A\u$ and $A\v$ on the right are real.

Let's apply this in Example \en.   One of the basic complex
solutions is
$$\align
\x(t) &=  e^{(1+i) t}
\bm  -1 - i\\ 2 \em = e^t(\cos t + i\sin t)
\bm  -1 - i\\ 2 \em \\
&= e^t\bm (\cos t + i\sin t)(-1 - i) \\ 2\cos t + i2\sin t \em \\
 &=
e^t\bm -\cos t  + \sin t -i(\sin t + \cos t)\\
    2\cos t + i2\sin t \em \\
&= e^t\bm -\cos t  + \sin t\\ 2\cos t \em
 + ie^t\bm -(\sin t + \cos t)\\ 2\sin t\em.
\endalign$$
Thus, the real and imaginary parts are
$$\align
\u(t) &=  e^t\bm -\cos t  + \sin t\\ 2\cos t \em\\
\v(t) &= e^t\bm -(\sin t + \cos t)\\ 2\sin t\em.
\endalign$$
These form a linearly independent set since putting $t = t_0 = 0$
yields
$$\align
\u(0) &= \bm \format\r\\ -1\\ 2 \em \\
\v(0) &= \bm \format\r\\ -1 \\ 0 \em
\endalign$$
and these form a linearly independent pair in $\R^2$.
Hence, the general {\it real\/} solution of the system is
$$
\x = c_1 e^t\bm -\cos t  + \sin t\\ 2\cos t \em
+
c_2 e^t\bm -(\sin t + \cos t)\\ 2\sin t\em
$$
where $c_1$ and $c_2$ are arbitrary real scalars.

Note that if we had used the eigenvalue $\overline\lambda = 1 - i$
and the corresponding basic 
complex solution $\overline\x(t) = \u(t) - i\v(t)$ instead, we would
have obtained the same thing except for the sign of
one of the basic real solutions.
\endexample

The analysis in the example illustrates what happens in general.
If $A$ is a {\it real\/} $n\times n$ matrix, then the roots of
its characteristic equation are either real or come in {\it conjugate
complex pairs\/} $\lambda, \overline\lambda$.    For real
roots $\mu$, we can always find basic eigenvectors 
in $\R^n$.
For non-real complex roots $\lambda$, we need
to look for basic eigenvectors in $\C^n$, but we may obtain the
basic eigenvectors for $\overline\lambda$ by taking the conjugates of
the basic eigenvectors for $\lambda$.  Hence, for each pair of
conjugate complex roots, we need only consider one root in the
pair in order to generate an independent pair of real solutions.

We already know that if the eigenvalues are distinct then the
corresponding set of eigenvectors will be linearly independent.
Suppose that, for each pair of conjugate complex roots,
we choose one root of the pair and take the real and imaginary
parts of the corresponding basic eigenvectors.   If we throw
in the basic real eigenvectors associated with the real
roots, then the set obtained in this way is always a linearly
independent subset of $\R^n$.   The proof of this fact is
not specially difficult, but we shall skip it.   (See the
Exercises for special cases.)
\bigskip
\input chap11.ex6
\bigskip

\nextsec{Repeated Roots and the Exponential of a Matrix}
\head \sn. Repeated Roots and the Exponential of a Matrix \endhead

We have noted that the eigenvalue-eigenvector method for solving
the $n\times n$ system $\dfrac{d\x}{dt} = A\x$ succeeds
if we can find a set of eigenvectors for $A$ which forms a basis
for $\R^n$ or where necessary for $\C^n$.   Also, according to
Theorem \cn.6,
% Back reference
this will always be the case if there are $n$ distinct eigenvalues.
Unfortunately, we still have to figure out what to do if the
characteristic equation has repeated roots.

First, {\it we might be lucky\/}, and there might be a basis of
eigenvectors of $A$.

\nextex
\example{Example \en}
Consider the system
$$
\frac{d\x}{dt} = \bm\format\r&\quad\r&\quad\r\\
                     1 & 1 & -1 \\
                     -1 & 3 & -1 \\
                     -1 & 1 & 1 \em \x.
$$
First solve the characteristic equation
$$\multline
\det \bm 1 - \lambda & 1 & -1 \\
          -1 & 3 - \lambda & -1 \\
          -1 & 1 & 1 - \lambda \em
= \\
(1 - \lambda)((3 - \lambda)(1 - \lambda) + 1)
   + (1 - \lambda +1) - (-1 + 3 - \lambda)\\
 = (1 - \lambda)(3 - 4\lambda + \lambda^2 + 1)  + 2 - \lambda
  - 2 + \lambda \\
 = (1 - \lambda)(\lambda^2 - 4\lambda + 4) \\ 
= (1 - \lambda)(\lambda - 2)^2 =0.
\endmultline$$
Note that 2 is a repeated root.
We find the eigenvectors for each of these eigenvalues.

For $\lambda = 2$ we need to solve  $(A - 2I)\v = 0$.
$$
 \bm\format\r&\quad\r&\quad\r\\
                     -1 & 1 & -1 \\
                     -1 & 1 & -1 \\
                     -1 & 1 & -1 \em
\to
 \bm\format\r&\quad\r&\quad\r\\
                     1 & -1 & 1 \\
                     0 & 0& 0 \\
                     0 & 0& 0 \em.
$$
The general solution of the system  is
$v_1 = v_2 - v_3$ with $v_2, v_3$ free.   The general solution
vector for that system is
$$
\v = \bm v_2 - v_3 \\ v_2\\ v_3 \em = v_2\bm\format\r\\ 1\\ 1 \\ 0 \em
+ v_3 \bm\format\r\\ -1\\ 0 \\ 1 \em.
$$
The solution space is {\it two\/} dimensional.
Thus, for the eigenvalue $\lambda = 2$ we obtain {\it two\/}
basic eigenvectors
$$
\v_1 = \bm\format\r\\ 1\\ 1 \\ 0 \em,\qquad \v_2
=  \bm\format\r\\ -1\\ 0 \\ 1 \em,
$$
and any eigenvector for $\lambda = 2$ is a non-trivial linear
combination of these.

For $\lambda = 1$, we need to solve $(A - I)\v = 0$.
$$
 \bm\format\r&\quad\r&\quad\r\\
                     0 & 1 & -1 \\
                     -1 & 2 & -1 \\
                     -1 & 1 & 0 \em
\to
 \bm\format\r&\quad\r&\quad\r\\
                     1 & -1 & 0 \\
                     0 & 1 & -1 \\
                     0 & 0 & 0 \em
\to
 \bm\format\r&\quad\r&\quad\r\\
                     1 & 0 & -1 \\
                     0 & 1 & -1 \\
                     0 & 0 & 0 \em.
$$
The general solution of the system is
$v_1 = v_3, v_2 = v_3$ with $v_3$ free.  The general solution
vector 
is
$$
\v = \bm v_3\\ v_3\\ v_3 \em = v_3\bm 1\\ 1\\ 1\em.
$$
The solution space is one dimensional, and a basic eigenvector for
$\lambda = 1$ is
$$
\v_3 = \bm 1 \\ 1 \\ 1 \em.
$$
It is not hard to check that the set of these basic eigenvectors
$$
\left\{ \v_1 = \bm\format\r\\ 1\\ 1 \\ 0 \em,\, \v_2
=  \bm\format\r\\ -1\\ 0 \\ 1 \em,\,\v_3 = \bm 1 \\ 1 \\ 1 \em\right\}
$$
is linearly independent, so it is a basis for $\R^3$.

We may now write out the general solution of the system of
differential equations
$$
\x = c_1e^{2t}\bm\format\r\\ 1\\ 1 \\ 0 \em + c_2e^{2t}\bm\format\r\\ -1\\ 0 \\ 1 \em
+ c_3e^t \bm 1 \\ 1 \\ 1 \em.
$$
\endexample

Of course, we may not always be so lucky when we have
repeated roots of the characteristic equation.   (See for example
Example 2 in Section 5.)   Hence, we need some other method.
It turns out that there is a generalization of the 
eigenvalue-eigenvector method which always works, but it requires
a digression.
\medskip
\subhead The Exponential of a Matrix \endsubhead
Let $A$ be a constant $n\times n$ matrix.  We could try to
solve $d\x/dt = A\x$ by the following nonsensical calculations
$$
\gather
\frac{d\x}{dt} = A\x \\
 \frac{d\x}{\x} = A\,dt \\
\ln \x = At + c \\
\x = e^{At}e^c = e^{At}C.
\endgather$$
Practically every line in the above calculation contains some
undefined quantity.  For example, what in the world is $\dfrac{d\x}{\x}$?
($\x$ is an $n\times 1$ column vector, so it isn't an invertible matrix.)
Strangely enough something like this actually works, but one must
first make some proper definitions. 
   We start with
the definition of `$e^{At}$'.

Let $B$ be any $n\times n$ matrix.  We define
$$
e^B = I + B + \frac 12 B^2 + \frac 1{3!}B^3 + \dots + \frac 1{j!}B^j + \dots
$$
A little explanation is necessary.   Each term on the right is
an $n\times n$ matrix.   If there were only a finite number of such
terms, there would be no problem, and the sum would also be an
$n\times n$ matrix.   In general, however, there are infinitely
many terms, and we have to worry about whether it makes sense
to add them up.
\outind{exponential of a matrix}%
\outind{matrix, exponential of}%

\nextex
\example{Example \en}  Let
$$
B = t\bm \format\r&\quad\r\\
              0 & 1 \\
             -1 & 0 \em.
$$
Then
$$
\align
B^2 &= t^2\bm\format\r&\quad\r\\
                   -1 & 0 \\ 
                  0 & -1 \em \\
B^3 &= t^3\bm\format\r&\quad\r\\
                0 & -1 \\
                1 & 0 \em \\
B^4 &= t^4\bm\format\r&\quad\r\\
                1 & 0 \\ 0 & 1 \em \\
B^5 &= t^5\bm\format\r&\quad\r\\
                0 & 1 \\
              -1 & 0 \em \\
&\vdots 
\endalign$$
Hence,
$$\align
e^B &= \bm 1 & 0 \\ 
          0 & 1 \em
+t\bm\format\r&\quad\r\\
         0 & 1 \\ -1 & 0 \em 
+\frac 12
t^2\bm\format\r&\quad\r\\
                 -1 & 0 \\
                  0 & -1 \em 
+
 \frac 1{3!}t^3\bm\format\r&\quad\r\\
                0 & -1 \\
                1 & 0 \em 
 + \dots \\
&= \bm 1 -\frac{t^2}2 +\frac{t^4}{4!} -\dots &
      t - \frac{t^3}{3!} + \frac{t^5}{5!} - \dots \\
   -t + \frac{t^3}{3!} - \frac{t^5}{5!} + \dots & 1 - \frac{t^2}2
+ \frac{t^4}{4!} - \dots \em  \\
&= \bm \cos t & \sin t \\
      -\sin t & \cos t \em.
\endalign$$
\endexample

As in the example, a series of $n\times n$ matrices yields a
separate  series for
each of the $n^2$ possible entries. 
  We shall say that such a series of matrices converges
if the series it yields for each entry converges.   With this rule, it
\outind{matrix series}%
\outind{series of matrices}%
is possible to show that the series defining $e^B$  converges
for any $n\times n$ matrix $B$, but the proof is a bit involved.
 Fortunately, as we shall see presently,
we can usually avoid worrying about convergence by a trick.  
In what follows we shall generally ignore such matters and act as if
the series
 were finite sums.

The exponential function for matrices obeys the usual rules you
expect an exponential function to have, but sometimes you have
to be careful.

\roster
\item  If $0$ denotes the $n\times n$ zero matrix, then $e^0 = I$.
\item  The law of exponents holds if the matrices commute, i.e.,
if $B$ and $C$ are $n\times n$ matrices such that 
$BC = CB$, then $e^{B + C} = e^Be^C$.
\item  If $A$ is an $n\times n$ constant matrix, then
$\dfrac{d}{dt}e^{At} = Ae^{At} = e^{At}A$.   (It is worth writing
this in both orders because products of matrices don't automatically commute.)
\endroster

Here are the proofs of these facts.  
\medskip
(1)  $e^0 = I + 0 + \dfrac 12 0^2 + \dots = I$.
\medskip
(2)  See the Exercises.
\medskip
(3)  Here we act as if the sum were finite (although the argument
would work in general if we knew enough about convergence of
series of matrices.)
$$\align
\frac{d}{dt}e^{At} &=
\frac{d}{dt}\left(I + tA + \frac 12 t^2 A^2 + \frac 1{3!}t^3A^3 +
\dots + \frac 1{j!}t^jA^j + \dots\right) \\
 &= 0 + A + \frac 12(2t) A^2  + \frac 1{3!}(3t^2)A^3  + \dots
+ \frac 1{j!}(jt^{j-1})A^j  + \dots \\
 &= A + tA^2 + \frac 12 t^2A^3 + \dots + \frac 1{(j-1)!}t^{j-1}A^j + \dots \\
 &= A(I + tA + \frac 12 t^2A^2 + \dots + \frac 1{(j-1)!}t^{j-1}A^{j-1}
+ \dots)\\
 &= Ae^{At}.
\endalign$$
Note that  in the next to last step $A$ could just as well
have been factored out on the right, so it doesn't matter which
side you put it on.
\medskip
Rule (3) gives us a formal way to solve the system $\dfrac{d\x}{dt}
= A\x$ when $A$ is a constant $n\times n$ matrix.  Namely, if
$\v$ is any constant $n\times 1$ column vector, then
$\x = e^{At}\v$ is a solution.  For,
$$
\frac{d\x}{dt} = \frac{d}{dt}e^{At}\v = Ae^{At}\v = A\x.
$$
Suppose $\{\v_1, \v_2, \dots, \v_n\}$ is any basis for
$\R^n$ (or in the complex case for $\C^n$).   That gives us
$n$ solutions
$$
\x_1 = e^{At}\v_1,\quad \x_2 = e^{At}\v_2,\quad \dots, \x_n =e^{At}\v_n.
$$
Moreover, these solutions form a linearly independent set of
solutions (hence a basis for the vector space of all solutions)
since when we evaluate at $t= 0$, we get
$$
\x_1(0) = e^0\v_1 = \v_1, \dots, x_n(0)=e^0\v_n = \v_n.
$$
By assumption, these form a linearly independent set of vectors
in $\R^n$ (or $\C^n$ in the complex case).

The simplest choices for basis vectors are the standard basis
vectors
$$
\v_1 = \e_1, \v_2 = \e_2, \dots, \v_n = \e_n,
$$
(which you should recall are just the columns of the identity
matrix).
In this case, we have $\x_i(t) = e^{At}\e_i$, which is
the $i$th column of the $n\times n$ matrix $e^{At}$.  Thus,
\smallskip
{\narrower\it
the columns of $e^{At}$ always form a basis for the vector
space of all solutions.}
\smallskip
\example{Example \en, revisited}  For the system
$$
\frac{d\x}{dt} = \bm\format\r&&\quad\r\\ 0 & 1 \\ -1 & 0 \em \x
$$
we have
$$
e^{At} = \bm\format\r&&\quad\r\\ \cos t & \sin t \\ -\sin t & \cos t \em
$$
so 
$$
\bm\format\r\\ \cos t \\ -\sin t \em\qquad\text{and}\qquad
\bm\format\r\\ \sin t \\ \cos t \em
$$
form a basis for the solution space of the system.
\endexample

There is one serious problem with the above analysis.   Adding
up the series for $e^{At}$ is usually not very easy.  Hence,
the facts mentioned in the previous paragraphs are usually not
very helpful if you want to write out an explicit solution.
To get around this problem, 
we rely on the observation that a proper choice of $\v$ can make the series
$$\align
e^{At}\v &= (I + tA \frac 12 t^2A^2 + \frac 1{3!}t^3A^3 + \dots )\v\\
&= \v + t(A\v) + \frac 12 t^2(A^2\v) + \frac 1{3!}t^3(A^3\v) + \dots
\endalign$$
easier to calculate. 
The object then is to pick $\v_1, \v_2, \dots, \v_n$ with this
strategy in mind.

For example, suppose $\v$ is an eigenvector for $A$ with
eigenvalue $\lambda$.  Then
$$\align
A\v &= \lambda\v \\
A^2\v &=A(A\v) = A(\lambda\v) = \lambda(A\v) = \lambda^2\v \\
A^3\v &= \dots = \lambda^3\v \\
 &\vdots
\endalign$$
In fact, if $\v$ is an eigenvector with eigenvalue $\lambda$,
it follows that $A^j\v = \lambda^j\v$ for any $j = 0, 1, 2. \dots$.
Thus,
$$
\align
e^{At}\v &= \v + t(A\v) + \frac 12 t^2(A^2\v) + \frac 1{3!}t^3(A^3\v) + \dots\\
&= \v + t(\lambda\v) + \frac 12 t^2(\lambda^2\V) +
 \frac 1{3!}t^3(\lambda^3\v) + \dots \\
&= (1 + t\lambda + \frac 12 t^2\lambda^2 + \frac 1{3!}t^3\lambda^3 + \dots)\v\\
&= e^{\lambda t}\v.
\endalign$$
Thus, if $\v$ is an eigenvector with eigenvalue $\lambda$,  the series
essentially reduces to the scalar series for $e^{\lambda t}$, and
$$
\x = e^{At}\v = e^{\lambda t}\v
$$
is exactly the solution obtained by the eigenvalue-eigenvector method.

Even where we don't have enough eigenvectors, we may
exploit this strategy as follows.  Let $\lambda$ be an
eigenvalue, and write
$$
  A = \lambda I + (A - \lambda I).
$$
Then, since $A$ and $A - \lambda I$ commute, the law of exponents
tells us that
$$
e^{At} = e^{\lambda I t}e^{(A - \lambda I)t}.
$$
However, as above,
$$\align
e^{\lambda I t} &= I + \lambda t I + \frac 12 (\lambda t)^2 I^2 +
\frac 1{3!}(\lambda t)^3 I^3 + \dots \\
&= e^{\lambda t}I.
\endalign
$$
   Hence,
$$
e^{At} =  e^{\lambda t}Ie^{(A - \lambda I)t} = 
e^{\lambda t}e^{(A - \lambda I)t}
$$
which means that calculating $e^{At}\v$ can be reduced to calculating
the {\it scalar multiplier\/}  $e^{\lambda t}$ and
the quantity $e^{(A - \lambda I)t}\v$.   However,
\nexteqn
$$
\align
e^{(A - \lambda I)t}\v &=
(I +  t(A - \lambda I) + \frac 12 t^2(A - \lambda I)^2
 + \dots + 
\frac 1{j!}t^j (A - \lambda I)^j + \dots)\v \\
&= \v +  t(A - \lambda I)\v + \frac 12 t^2(A - \lambda I)^2\v
      + \dots +
 \frac 1{j!}t^j (A - \lambda I)^j\v + \dots, \tag\eqn
\endalign
$$
so it makes sense to try to choose $\v$ so that
 $(A - \lambda I)^j\v$  vanishes for all $j$ beyond
a certain point.    Then the series  (\eqn) will reduce to a finite
sum. 
 In the next section, we shall explore  a systematic method to
do this.

\bigskip
\input chap11.ex7
\bigskip

\nextsec{Generalized Eigenvectors}
\head \sn.  Generalized Eigenvectors \endhead

Let $A$ be an $n\times n$ matrix, and let $\lambda$ be an
eigenvalue for $A$.   Suppose moreover that $\lambda$ has
multiplicity $m$ as a root of the characteristic equation
of $A$.  Call any solution of the
system
$$
(A - \lambda I)\v = 0.
$$
a {\it level 1 generalized
eigenvector}.    These include all the usual eigenvectors plus
the zero vector.   Similarly, consider the system
\outind{eigenvector, generalized}%
\outind{generalized eigenvector}%
$$
(A - \lambda I)^2\v = 0,
$$
and call any solution of that system a {\it level 2 generalized
eigenvector}.   Continuing in this way, call any solution of the system
$$
(A - \lambda I)^j\v = 0
$$
a {\it level $j$ generalized eigenvector}.   We will also
sometimes just use
the term {\it generalized eigenvector\/} without
explicitly stating the level.

If  $\v$ is a level $j$ generalized eigenvector, then
$$(A - \lambda I)^{j+1}\v = (A - \lambda I)(A - \lambda I)^j\v = 0
$$
so it is also a level $j + 1$ generalized eigenvector, and similarly
for all higher levels.  Thus, one may envision first finding the
level 1 generalized eigenvectors (i.e., the ordinary
eigenvectors), then finding the level 2 vectors, which may constitute
a larger set, then finding the level 3 vectors, which may consititute
a still larger set, etc.   That we need not continue this process
indefinitely is guaranteed by the following theorem.    

\nextthm
\proclaim{Theorem \cn.\tn}  Let $A$ be an $n\times n$ matrix,
and suppose $\lambda$ is an eigenvalue for $A$ with multiplicity
$m$.

   (a) The solution space of $(A - \lambda I)^j\v = 0$
for $j > m$ is identical with the solution space of
$(A - \lambda I)^m\v = 0$.

   (b) The solution space of $(A - \lambda I)^m\v = 0$
has dimension $m$.
\endproclaim

Part (a) tells us that we need go no further than level $m$ in
order to obtain {\it all\/} generalized eigenvectors of any
level whatsover.   Part (b) tells us that in some sense there
are `sufficiently many' generalized eigenvectors.   This will
be important when we need to find a basis consisting of
such vectors.

We shall not attempt to prove this theorem here.   The proof is
quite deep and closely related to the theory of the so-called
{\it Jordan Canonical Form}.   You will
probably encounter this theory
if you take a more advanced course in linear algebra.

\nextex
\example{Example \en}
Let
$$
A = \bm 1 & 2 & 0 \\
        2 & 1 & 0 \\
        0 & 1 & 3 \em.
$$
The eigenvalues of $A$ are obtained by solving
$$\align
\det \bm 1 -\lambda & 2 & 0 \\
        2 & 1 -\lambda & 0 \\
        0 & 1 & 3 -\lambda \em
&= (1 - \lambda)((1 - \lambda)(3 - \lambda) - 0) - 2(2(3 - \lambda) - 0) + 0 \\
&= (3 - \lambda)((1 - \lambda)^2 - 4) \\
&= (3 - \lambda)(\lambda^2 - 2\lambda - 3) = -(\lambda - 3)^2(\lambda + 1).
\endalign
$$
Hence, $\lambda = 3$ is a root of multiplicity 2 and $\lambda = -1$
is a root of multiplicity 1.

Let's find the generalized eigenvectors for each eigenvalue.

For $\lambda = 3$, we need only go to level 2
and solve $(A - 3 I)^2\v = 0$.
We have
$$
(A - 3 I)^2 = 
\bm\format\r&\quad\r&\quad\r\\
      -2 & 2 & 0 \\
      2 & -2 & 0 \\
      0 & 1 & 0 \em^2 =
\bm\format\r&\quad\r&\quad\r\\
     8 &  -8 & 0 \\
    -8 & 8 & 0 \\
    2 & -2 & 0 \em,
$$
and Gauss-Jordan reduction yields
$$
\bm\format\r&\quad\r&\quad\r\\
     8 &  -8 & 0 \\
    -8 & 8 & 0 \\
    2 & -2 & 0 \em
\to
\bm\format\r&\quad\r&\quad\r\\
     1 &  -1 & 0 \\
    0 & 0 & 0 \\
    0 & 0 & 0 \em.
$$
The general solution is $v_1 = v_2$ with $v_2, v_3$ free.
The general solution vector is
$$
\v = \bm v_2\\v_2\\ v_2\em = v_2\bm 1\\ 1\\ 0 \em + v_3\bm 0\\ 0\\ 1\em.
$$
Hence, a basis for the subspace of generalized eigenvectors for the
eigenvalue $\lambda = 3$ is
$$
\left\{\v_1 = \bm 1\\ 1\\ 0 \em,\, \v_2 = \bm 0\\ 0\\ 1\em\right\}.
$$

Theorem \cn.\tn\ tells us that we don't need to go further
than level 2  for the eigenvalue $\lambda = 3$, but it is
reassuring to check that explicitly in this case.
Look for level 3 vectors by solving
$(A - 3I)^3\v = 0$  ($j = m +1 = 3$).  We have
$$
(A - 3I)^3\v = \bm\format\r&\quad\r&\quad\r\\
     -32 &  32 & 0 \\
    32& -32 & 0 \\
    -8 & 8 & 0 \em,
$$
and it is clear that solving this system gives us
exactly the same solutions as solving $(A - 2I)^2\v = 0$.

For $\lambda = -1$, the multiplicity is 1, and we need to solve
$(A - (-1) I)^1\v = 0$.   Hence, finding the generalized
eigenvectors for $\lambda = -1$ just amounts to finding the
usual eigenvectors.
$$
\bm\format\r&\quad\r&\quad\r\\
 2 & 2 & 0 \\
 2 & 2 & 0 \\
 0 & 1 & 4 \em
\to
\bm\format\r&\quad\r&\quad\r\\
 1 & 0 & -4 \\
 0 & 1 & 4 \\
 0 & 0 & 0 \em.
$$
 The general solution is $v_1 = 4v_3, v_2 = -4v_3$, with $v_3$ free.
The general solution vector is
$$
\v = \bm\format\r\\ 4v_3\\ -4v_3\\ v_3 \em
= v_3\bm\format\r\\ 4\\ -4\\ 1 \em.
$$
Thus,
$$
\v_3 = 
\bm\format\r\\ 4\\ -4\\ 1 \em
$$
forms a basis for the subspace of (generalized) eigenvectors for
$\lambda = -1$.

Put these basic generalized eigenvectors together in a set
$$
\left\{\v_1 = \bm 1\\ 1\\ 0 \em,\, \v_2 = \bm 0\\ 0\\ 1\em,
\,
\v_3 = 
\bm\format\r\\ 4\\ -4\\ 1 \em
\right\},
$$
It is not hard to check by the usual means that we get a basis for
$\R^3$.
\endexample


What we observed in this example always happens.   If we find a basis
for the subspace of generalized eigenvectors for each eigenvalue
and put them together in a set, the result is always a
linearly independent set.  (See the first appendix to this section if
you are interested in a proof.)  If we are working in $\C^n$
and using complex scalars, then the Fundamental Theorem of
Algebra tells us that the 
multiplicities of the roots of the characteristic
equation  add up to $n$, the degree of the equation.
Thus, the linearly independent set of basic generalized
eigenvectors has the right number of elements for a basis, so
it is a basis.  If we are working in $\R^n$ using real scalars,
we will also get a basis in this way provided all the
(potentially complex) roots of the characteristic equation
are real.   However, if there are any non-real complex
roots, we will miss the corresponding generalized eigenvectors
by sticking strictly to $\R^n$.

%% Material on diagonalizability added.
\medskip
\subhead Diagonalizable Matrices \endsubhead
In the simplest case, that in which all generalized eigenvectors
are level one, the matrix $A$ is said to be {\it diagonalizable\/}.
\outind{diagonalizable}%
In this case, there is a basis for
$\C^n$ consisting of eigenvectors for $A$.   (If the eigenvalues
and eigenvectors are all real, we could replace  $\C^n$ by
$\R^n$ in this statement.)   This is certainly
the easiest case to deal with, so it is not surprising that we
give it a special name.   However, why we use the term
`diagonalizable' requires an explanation.

Suppose  $\{\v_1, \v_2,\dots,\v_n\}$ is a basis for $\C^n$ consisting
of eigenvectors for  $A$.  Then we have
$$
A\v_i = \lambda_i \v_i = \v_i\lambda_i,\qquad i = 1,2,\dots, n
$$
where $\lambda_i$ is the eigenvalue associated with $\v_i$.
(These eigenvalues need not be distinct.)   We may write this
as a single matrix equation
$$
\align
A\bm \v_1 & \v_2 & \dots &\v_n\em
&= 
\bm \v_1\lambda_1  & \v_2\lambda_2 & \dots &\v_n\lambda_n\em \\
&=
\bm \v_1 & \v_2 & \dots &\v_n\em
\bm \lambda_1 & 0 & \cdots & 0 \\
     0 & \lambda_2 & \cdots & 0 \\
     \vdots & \vdots & \cdots & \vdots \\
     0 & 0 & \cdots & \lambda_n \em.
\endalign
$$
If we put
$$
  P = \bm \v_1 & \v_2 & \hdots & \v_n\em
$$
then this becomes
$$
AP = PD
$$
where $D$ is {\it a diagonal matrix with the eigenvalues of $A$ appearing
on the diagonal}.  $P$ is an invertible matrix since its columns
form a basis for $\C^n$, so the last equation can be written in turn
$$
P^{-1}AP = D \qquad\text{where $D$ is a diagonal matrix}.
$$
\example{Example}  In Example \DiagEx\ of Section \DiagExSec, we considered
the matrix
$$
A = \bm 1 & 2 \\ 2 & 1 \em.
$$
The eigenvalues are $\lambda = 3$ and $\lambda = -1$.   Corresponding
eigenvectors are
$$
\v_1 = \bm\format\r\\ 1\\ 1 \em, \qquad \v_2 = \bm\format\r\\ -1\\ 1\em
$$
and these form a basis for $\R^2$ so $A$ is diagonalizable.  Take
$$
  P = \bm\format\r&&\quad\r\\
                1 & -1 \\ 1 & 1 \em.
$$
The theory predicts that  $P^{-1}AP$ should be diagonal.   Indeed,
$$
\align
   \bm\format\r&&\quad\r\\
             1 & -1 \\ 1 & 1 \em^{-1} 
 \bm 1 & 2 \\ 2 & 1 \em
   \bm\format\r&&\quad\r\\
             1 & -1 \\ 1 & 1 \em 
 &=  \frac 12
   \bm\format\r&&\quad\r\\
             1 & 1 \\ -1 & 1 \em 
 \bm 1 & 2 \\ 2 & 1 \em
   \bm\format\r&&\quad\r\\
             1 & -1 \\ 1 & 1 \em 
 \\
 &= \frac 12 \bm\format\r&\quad\r\\ 3 & 3 \\ 1 & -1\em
   \bm\format\r&&\quad\r\\
             1 & -1 \\ 1 & 1 \em \\ 
&= \frac 12
   \bm\format\r&&\quad\r\\
             6 & 0 \\ 0 & -2 \em 
=   \bm\format\r&&\quad\r\\
             3 & 0 \\ 0 & -1 \em.
\endalign
$$
Note that the eigenvalues appear on the diagonal as predicted.
 
\medskip
\subhead Application to Homogeneous Linear Systems of Differential
Equations \endsubhead

Let $A$ be an $n\times n$ matrix and let $\v$ be a generalized
eigenvector for the eigenvalue $\lambda$.  In this case,
$e^{At}\v$ is specially easy to calculate.   Let $m$ be the
multiplicity of $\lambda$.   Then we know that
$(A - \lambda I)^j\v = 0$ for $j \ge m$ (and perhaps also for
some lesser powers).   Then, as in the previous section,
$$
e^{At} = e^{\lambda t}e^{(A - \lambda I)t},
$$
but
$$
e^{(A - \lambda I)t}\v= \v + t(A - \lambda I)\v+
\frac 12 t^2(A - \lambda I)^2\v + \dots +
\frac 1{(m-1)!}t^{m-1}(A - \lambda I)^{m-1}\v
$$
since {\it all other terms in the series vanish}.   Hence,
\nexteqn
$$
e^{At}\v
 = e^{\lambda t}(\v +  t(A - \lambda I)\v+
\frac 12 t^2(A - \lambda I)^2\v + \dots +
\frac 1{(m-1)!}t^{m-1}(A - \lambda I)^{m-1}\v).\tag\eqn
$$
This gives us a method for solving a homogeneous system
$\dfrac{d\x}{dt} = A\x$.  First find a basis
$\{\v_1, \v_2, \dots, \v_n\}$
 consisting of
generalized eigenvectors.   (This may require working in $\C^n$
rather than $\R^n$
if some of the eigenvalues are non-real complex numbers.)   Then, the solutions
$\x_i = e^{At}\v_i$ may be calculated by formula
(\eqn), and together form a basis for the solution space.

\example{Example \en a}  Consider the system
$$
\frac{d\x}{dt} = \bm 1 & 2 & 0 \\
                      2 & 1 & 0 \\
                      0 & 1 & 3 \em \x.
$$
Then, as we determined above,
$$
\left\{\v_1 = \bm 1\\ 1\\ 0 \em,\, \v_2 = \bm 0\\ 0\\ 1\em,\,
\v_3 =\bm\format\r\\ 4\\ -4\\ 1 \em\right\}
$$
is a basis of $\R^3$ consisting of generalized eigenvectors of
the coefficient matrix.  The first two correspond to the eigenvalue
$\lambda = 3$ with multiplicity 2,
 and $\v_3$ corresponds to the eigenvalue $\lambda = 1$ with multiplicity 1.
For $\lambda = 3$, $m = 2$, so we only need terms up to the {\it first\/}
power of $(A - 3I)$ in computing $e^{(A - 3I)t}\v$
for $\v = \v_1$ or $\v = \v_2$.
$$
(A - 3 I)\v_1 =
\bm\format\r&\quad\r&\quad\r\\
      -2 & 2 & 0 \\
      2 & -2 & 0 \\
      0 & 1 & 0 \em  \bm 1\\ 1\\ 0 \em = \bm 0\\ 0\\ 1\em. 
$$
Thus
$$
\x_1 = e^{At}\v = e^{3t}\left(
 \bm 1\\ 1\\ 0 \em + t\bm 0\\ 0\\ 1\em \right)
  = e^{3t}\bm 1\\ 1\\ t\em.
$$

Similarly,
$$
(A - 3 I)\v_2 =
\bm\format\r&\quad\r&\quad\r\\
      -2 & 2 & 0 \\
      2 & -2 & 0 \\
      0 & 1 & 0 \em  \bm 0\\ 0\\ 1 \em = \bm 0\\ 0\\ 0\em 
$$
so $\v_2$ turns out to be
an eigenvector.  Thus,
$$
\x_2 = e^{At}\v_2 = e^{3t}\v_2 = e^{3t}\bm 0\\ 0 \\ 1 \em.
$$


For the eigenvalue $\lambda = -1$, $\v_3$ is also an eigenvector,
so the method also just gives the expected solution
$$
\x_3 = e^{At}\v_3 = e^{-t}\v_3 = e^{-t}\bm\format\r\\ 4\\-4\\1\em.
$$

It follows that the general solution of the system is
$$
\x = c_1 e^{3t}\bm 1\\ 1\\ t\em + c_2  e^{3t}\bm 0\\ 0 \\ 1 \em 
+ c_3e^{-t}\bm\format\r\\ 4\\-4\\1\em.
$$

The fact that 
$\v_2$ and $\v_3$ are
 eigenvectors  
 simplifies the calculation of
the solutions $\x_2$ and $\x_3$.  {\it This sort of
simplification  often
happens, so you should be on the lookout for it}.
$\v_3$ is an eigenvector because the
associated eigenvalue has multiplicity one, but it is
a bit mysterious why $\v_2$ should be
an eigenvector.
However, this may be clarified somewhat
if you note that $\v_2$  turns up
 (in the process of finding $\x_1 = e^{At}\v_1$) as
 $\v_2 = (A - 3I)\v_1$.  Thus, it is an eigenvector because
 $(A - 3I)\v_2 = (A - 3I)(A - 3I)\v_1 = (A - 3I)^2\v_2 = 0$.
\endexample

\nextex
\example{Example \en}
Consider the linear system
$$
\frac{d\x}{dt} = 
\bm\format\r&\quad\r\\
 -3 & -1 \\ 1 & -1 \em \x.
$$
The characteristic equation is
$$
(3 + \lambda)(1 + \lambda) +1 = \lambda^2 + 4\lambda + 4
 = (\lambda + 2)^2 = 0.
$$
Since $\lambda = -2$ is the only eigenvalue and its multiplicity
is 2, it follows that {\it every element\/} of $\R^2$ is a generalized
eigenvector for that eigenvalue.   Hence, $\{\e_1, \e_2\}$
is a basis consisting of generalized eigenvectors.
The first vector $\e_1$  leads to  the basic solution
$$\align
\x_1 &= e^{At}\e_1 = e^{-2t}e^{t(A + 2I)}\e_1
   = e^{-2t}(I + t(A + 2I))\e_1\\
    &= e^{-2t}(\e_1 + t(A + 2I)\e_1) =
e^{-2t}\left(\bm 1\\ 0\em + 
t\bm\format\r&\quad\r\\-1 & -1\\ 1 & 1\em\bm 1\\0\em\right) \\
&= e^{-2t}\left(\bm 1\\ 0\em + t\bm\format\r\\ -1\\ 1\em\right)\\
&= e^{-2t}\bm 1 -t\\ t\em.
\endalign$$
A similar calculation gives a second independent
solution 
$$
e^{At}\e_2 = e^{-2t}\left(\bm 0\\1\em + t\bm\format\r\\ -1\\ 1\em\right) 
 = e^{-2t}\bm -t\\ 1 + t\em.
$$
However,
we may simplify things somewhat as follows. 
Let
$$
  \v_2 = (A + 2I)\e_1 = \bm\format\r\\ -1\\ 1\em.
$$
Then, it is not hard to see that
$$
\left\{\e_1 = \bm 1\\ 0\em,\, \v_2 =  \bm\format\r\\ -1\\1\em \right\}
$$
is an  independent pair of vectors. Hence, it must also necessarily be
a basis for $\R^2$.
Using the basis $\{\e_1, \v_2\}$ rather than $\{\e_1, \e_2\}$
seems superficially to make things harder,
but we gain something by using it
since 
$$
 (A + 2I)\v_2 = (A + 2I)(A + 2I)\e_1 = (A + 2I)^2\e_1 = 0.
$$
That is, $\v_2$ is an
eigenvector for $\lambda = -2$.  
Hence, we may use the {\it simpler\/} second solution
$$
\x_2 = e^{-2t}\v_2 = e^{-2t}\bm\format\r\\ -1\\1\em.
$$ 
Thus, 
$$
\align
\x_1 &= e^{-2t}(\e_1 + t\v_2) = \bm \e_1 & \v_2 \em\bm e^{-2t}\\ te^{-2t}\em \\
\x_2 &= e^{-2t}\v_2 = \bm \e_1 & \v_2 \em \bm 0\\ e^{-2t}\em
\endalign
$$
form a basis for the solution space of the linear system of differential
equations.
\endexample

\nextex
\example{Example \en} Consider the system
$$
\frac{d\x}{dt} = \bm 2 & 0 & 0 \\
                     1 & 2 & 0 \\
                     0 & 1 & 2 \em\x.
$$
It is apparent that the characteristic equation is
$-(\lambda - 2)^3 = 0$, so $\lambda = 2$ is the only root and
has multiplicity 3.  Also,
$$
(A - 2I)^2 = \bm 0 & 0 & 0 \\ 1 & 0 & 0 \\ 0 & 1 & 0\em^2
  = \bm 0 & 0 & 0 \\ 0 & 0 & 0 \\ 1 & 0 & 0 \em,
$$
and necessarily $(A - 2I)^3 = 0$.
Hence, {\it every\/} vector is a generalized eigenvector for
$A$ and
$$
\{\e_1, \e_2, \e_3\}
$$
is a perfectly good basis consisting of generalized eigenvectors.

The solutions are determined as before.
$$
\align
\x_1 &= e^{2t}\left(\e_1 + t\bm 0 & 0 & 0 \\ 1 & 0 & 0 \\ 0 & 1 & 0\em
\e_1 + \frac 12 t^2\bm 0 & 0 & 0 \\ 0 & 0 & 0 \\ 1 & 0 & 0 \em\e_1\right)\\
 &= e^{2t}\left(\bm 1\\ 0 \\ 0\em +t\bm 0 \\ 1 \\ 0 \em
+ \frac 12 t^2\bm 0 \\ 0 \\ 1\em\right) \\
   &= e^{2t}\bm 1 \\ t\\ t^2/2\em 
\endalign$$
Similarly,
$$\align
\x_2 &= e^{2t}\left(\e_2 + t\bm 0 & 0 & 0 \\ 1 & 0 & 0 \\ 0 & 1 & 0\em
\e_2 + \frac 12 t^2\bm 0 & 0 & 0 \\ 0 & 0 & 0 \\ 1 & 0 & 0 \em\e_2\right)\\
 &= e^{2t}\left(\bm 0\\ 1 \\ 0\em +t\bm 0 \\ 0 \\ 1 \em
+ \frac 12 t^2\bm 0 \\ 0 \\ 0\em\right) \\
   &= e^{2t}\bm 0 \\ 1\\ t\em 
\endalign$$
and
$$\align
\x_3 &= e^{2t}\left(\e_3 + t\bm 0 & 0 & 0 \\ 1 & 0 & 0 \\ 0 & 1 & 0\em
\e_3 + \frac 12 t^2\bm 0 & 0 & 0 \\ 0 & 0 & 0 \\ 1 & 0 & 0 \em\e_3\right)\\
 &= e^{2t}\left(\bm 0\\ 0 \\ 1\em +t\bm 0 \\ 0 \\ 0 \em
+ \frac 12 t^2\bm 0 \\ 0 \\ 0\em\right) \\
   &= e^{2t}\bm 0 \\ 0\\ 1\em. 
\endalign
$$
Note that 
$\x_2$ only needs terms of degree 1 in
$t$ and $\x_3$ doesn't even need those since $\e_3$ is actually
an eigenvector.
This is not surprising since
$$
\e_2 = (A - 2I)\e_1\qquad\text{so}\qquad (A - 2I)^2\e_2 = (A - 2I)^3\e_1 = 0
$$
and
$$
 \e_3 =(A - 2I)\e_2\qquad\text{so}\qquad (A - 2I)\e_3 = (A - 2I)^2\e_2 = 0.
$$

The general solution is
$$
\x = c_1e^{2t}\bm 1 \\ t\\ t^2/2\em
+ c_2 e^{2t}\bm 0 \\ 1\\ t\em
+ c_3 e^{2t}\bm 0 \\ 0\\ 1\em.
$$
\endexample

It should be noted that if the matrix $A$ is diagonalizable, then
for each eigenvalue $\lambda_i$,
we need only deal with genuine eigenvectors
$\v_i$, and the series expansion  $e^{(A - \lambda_i I)t}\v_i$ has
only one term and
reduces to  $I\v_i = \v_i$.   Thus, if $A$ is diagonalizable,
the generalized eigenvector method just reduces to the ordinary
eigenvector method discussed earlier.
\medskip
\subhead Appendix 1.  Proof of Linear Independence of the Set
of Basic Generalized Eigenvectors \endsubhead
You may want to skip this proof.

Let $\lambda_1, \lambda_2, \dots, \lambda_k$ be distinct eigenvalues
of the $n\times n$ matrix $A$.
Suppose that, for each eigenvalue $\lambda_i$,
we have chosen a basis for the subspace of solutions of
the system $(A - \lambda_i I)^{m_i}\v = 0$, where $m_i$ is the multiplicity
of $\lambda_i$.    Put these all together in a set $S$.   We shall
prove that $S$ is a linearly independent set.    

If not there
is a dependence relation.   Rearrange this dependence relation
so that on the left we have a non-trivial
 linear combination of the basic generalized eigenvectors
belonging to one of the eigenvalues, say it is $\lambda_1$, and on
the other side we have a linear combination of basic
generalized eigenvectors
for the other eigenvalues.  Suppose this has the form
\nexteqn
$$
\v_1 = \sum_{i=2}^k\u_i\tag\eqn
$$
where $\v_1\not=0$ is a generalized eigenvector for $\lambda_1$ and each
$\u_i$ is a generalized eigenvector for $\lambda_i$ for $i= 2, \dots, k$.

Since $\v_1$ is a generalized eigenvector for $\lambda_1$, we have
$(A - \lambda_1I)^r\v_1 = 0$ for some $r > 0$.   
Assume 
$r$ is chosen to
be the {\it least\/} positive power for which that is true.  Then
$\v_1' = (A - \lambda_1I)^{r-1}\v_1 \not=0$ and
$(A -\lambda_1I)\v_1' = (A- \lambda_1I)^r\v_1 = 0$, i.e.,
$\v_1'$ is an eigenvector with eigenvalue $\lambda_1$.  (Note that
$r = 1$ is possible, so this requires that $\v_1\not = 0$, which is true by
assumption.)  Multiply
both sides of equation (\eqn) by $(A - \lambda_1I)^{r-1}$.  On
the left we get $\v_1'$.   On the right, each term 
$\u_i' = (A - \lambda_1I)^{r-1}\u_i$ is still a generalized eigenvector
for $\lambda_i$.  For,
$$
(A - \lambda_iI)^{m_i}(A - \lambda_1I)^{r-1}\u_i
= (A - \lambda_1I)^{r-1}(A - \lambda_iI)^{m_i}\u_i = 0.
$$
(This used the rather obvious fact that polynomial expressions in
the matrix $A$ commute with one another.)    The upshot of this argument
is that we may assume that $\v_1$ in (\eqn) is an actual eigenvector
for $\lambda_1$.  (Just replace $\v_1$ by $\v_1'$ and each $\u_i$
by the corresponding $\u_i'$.)

Now multiply equation (\eqn) by the product
$$
(A - \lambda_2I)^{m_2}(A - \lambda_3I)^{m_3}\dots (A - \lambda_kI)^{m_k}.
$$
Note as above that the 
factors in the product commute with one another so the
order in which the terms are written in not important.   Each $\u_i$
on the right
is a generalized eigenvector
 for $\lambda_i$, so $(A -\lambda_iI)^{m_i}\u_i = 0$.
 That means that the effect on the right
of multiplying by the product is 0.  Consider the effect on the left.
$\v_1$ is an eigenvector for $\lambda_1$, so   
$$
\align
(A - \lambda_iI)\v_1 &= A\v_1 - \lambda_i\v_1 = \lambda_1\v_1 -\lambda_i\v_1
                 = (\lambda_1 - \lambda_i)\v_1 \\
(A - \lambda_iI)^2\v_1 &= (A - \lambda_iI)(A - \lambda_iI)\v_1 = 
(A- \lambda_iI)(\lambda_1-\lambda_i)\v_1\\
  &= (\lambda_1-\lambda_i)(A-\lambda_iI)\v_1
  = (\lambda_1=\lambda_i)^2\v_1\\
&\vdots \\
(A - \lambda_i)^{m_i}\v_1 &= (\lambda_1 - \lambda_i)^{m_i}\v_1.
\endalign$$
Thus the effect of the product on the left is
$$
(\lambda_1 - \lambda_2)^{m_2}\dots (\lambda_1-\lambda_k)^{m_k}\v_1
$$
which is non-zero since the scalar multiplier is non-zero.   This
contradicts the fact that the effect on the right is zero, so we
have a contradiction from the assumption that there is a dependence
relation among the basic generalized eigenvectors.

\medskip
\subhead Appendix 2.  Cyclic Vectors and the Jordan Form\endsubhead
You may want to come back and
 read this if you take a more advanced course in
linear algebra.

It is a bit difficult to illustrate everything that can
happen when one computes solutions $e^{At}\v = e^{\lambda t}
e^{(A- \lambda I)t}\v$ as $\v$ ranges over a basis of
generalized eigenvectors, particularly since the degree is
usually fairly small.   However, there is one phenomenon we
encountered in some of the examples.
It might happen that the system $(A - \lambda I)^2\v = 0$
has some basic solutions of the form
$$
\v_1, \v_2 = (A - \lambda I)\v_1
$$
so $\v_2$ is an eigenvector.  Similarly, it might happen
that $(A - \lambda I)^3\v = 0$ has some basic solutions of the form
$$
\v_1, \v_2 = (A - \lambda I)\v_1, \v_3 = (A - \lambda I)\v_2
 = (A - \lambda I)^2\v_1.
$$
More generally, we might be able to find
a generalized eigenvector $\v$ which satisfies 
$(A - \lambda I)^r\v = 0$, where the set formed from
\nexteqn
$$
\v, (A-\lambda I)\v, (A-\lambda I)^2\v, \dots, (A-\lambda I)^{r-1}\v\tag\eqn
$$
is linearly independent.
In this case, $\v$ is called a
{\it cyclic vector\/} for $A$ of order $r$.   Note that
a cyclic vector of {\it order 1\/} is an {\it eigenvector}.

 The theory of the {\it Jordan
Canonical Form\/} asserts that it is {\it always\/} possible to
find a basis of generalized eigenvectors formed from cyclic vectors 
\outind{Jordan canonical form}%
as in (\eqn).   The advantage of using a basis derived from cyclic
vectors is that the number of terms needed in the expansion of
$e^{(A - \lambda I)t}$ is kept to a minimum. 

It sometimes happens in solving $(A - \lambda I)^m\v = 0$ (where
$m$ is the multiplicity of the eigenvalue) that
the solution method gives  a `cyclic' basis, but as we saw in
examples it is not always the case.  The best case is
that in which one of the basic
generalized eigenvectors for $\lambda$ is cyclic of order $m$,
i.e., it
 does not satisfy
a lower order system
  $(A - \lambda I)^j\v = 0$ with $j < m$.  
 However, it is quite possible
that all cyclic vectors for a given
eigenvalue have order smaller than $m$.  In that case
it is necessary to use two or more cyclic vectors to generate
a basis.
  For example, for an eigenvalue $\lambda$ of multiplicity $m = 3$,
 we could have
a basis 
$$
\{\v_1, \v_2 = (A - \lambda I)\v_1, \v_3\}
$$
where $\v_1$ is a cyclic vector of order $2$ and $\v_3$ is
a cyclic vector of order 1, i.e., it is an eigenvector.
An even more extreme case would be a basis of eigenvectors,
i.e., each basis vector would be a cyclic vector of
order 1.  In general, it may be quite difficult to find
a basis derived from cyclic vectors.
\bigskip
\input chap11.ex8
\bigskip

\endchapter
\closeseg{chap11}
\enddocument
