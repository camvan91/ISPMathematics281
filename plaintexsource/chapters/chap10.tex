\documentstyle{book}
\input extra.tex
\input le2.sty
\input epsf.tex
\input chap9.lnk
\maketoctrue
\indextrue
\mardiagtrue
\openseg{chap10}
\Monograph
\topmatter
\nextchap{Linear Algebra, Basic Notions}
\title
\chapter{\chapno}Linear Algebra, Basic Notions
\endtitle
\endtopmatter
\document
\nextsec{Systems of Differential Equations}
\head \sn.  Systems of Differential Equations \endhead

Usually, in physical problems there are several variables, and
 the rate of change of each variable depends not only on
it, but also on the other variables.   This gives rise to
a {\it system of differential equations}. 
\outind{system of differential equations}%

\nextex
\example{Example \en}  Suppose that in a chemical reaction there
are two substances $X$ and $Y$ that we need to keep track of.
Suppose the {\it kinetics\/} of the reaction is such that
$X$ decomposes into $Y$ at a rate proportional to the amount
of $X$ present, and suppose $Y$ decomposes into
uninteresting byproducts $U$ at a rate proportional to the amount
of $Y$ present.    We may indicate this schematically by
$$
   X\overset{k_1}\to\to Y \overset{k_2}\to\to U
$$
where $k_1$ and $k_2$ are rate constants.   We may translate the
above description into mathematics as follows.  Let $x(t)$
and $y(t)$ denote the amounts of $X$ and $Y$ respectively present
at time $t$.   Then
$$
\align
\frac{dx}{dt} &= -k_1x \\
\frac{dy}{dt} &= k_1x - k_2y.
\endalign$$
This is simple example of a system of differential equations.
It is not very hard to solve.  The first equation involves only
$x$, and its solution is $x = x_0e^{-k_1t}$ where $x_0 = x(0)$
is the amount of $X$ present initially.   This may be substituted
in the second equation to obtain
$$
\frac{dy}{dt} + k_2y = k_1x = k_1x_0e^{-k_1t},
$$
which is a first order linear equation which
 may be solved by the method in Chapter VI, Section 3.
%Exer  Solve this system

Not every system is so easy to solve.  Suppose for example
that the substance $Y$ in addition to decomposing into uninteresting
substances also recombines with a substrate to form $X$ at a
rate depending on $y$.   We would then have to modify the
above system to one of the form
$$
\align
\frac{dx}{dt} &= -k_1x + k_3y \\
\frac{dy}{dt} &= k_1x - (k_2 + k_3)y.
\endalign$$
It is not immediately clear how to go about solving this system!
\endexample

\nextex
\xdef\Couple{\en}
\example{Example \en}  Consider two identical
 masses $m$ on a track connected
by springs as indicated below.   Suppose all the springs have
the same spring constant $k$.
\medskip
\centerline{\epsfbox{s10-1.ps}}
\medskip
The masses will be at rest in certain equilibrium positions,
but if they are displaced from those positions, the resulting
system will oscillate in some very complicated way.  Let
$x_1$ and $x_2$ denote the displacements of the masses from
equilibrium.   The force exerted on the first mass by the
spring to its left will be $-kx_1$ since that spring will be
stretched by $x_1$.   On the other hand, the spring in the
middle will be stretched by $x_1 - x_2$, so the force it
exerts of the first mass is $-k(x_1 - x_2)$.  Thus the
total force on the first particle is the sum, so by
Newton's Second Law
$$
m\frac{d^2x_1}{dt^2} = -kx_1 -k(x_1 - x_2).
$$
By a similar argument, we get for the second mass
$$
m\frac{d^2x_2}{dt^2} = -kx_2 -k(x_2 - x_1).
$$
(You should check both these relations to be sure you agree the
forces are being exerted in the proper directions.)   These
equations may be simplified algebraically to yield
\nexteqn
\begin{align*}
m\frac{d^2x_1}{dt^2} &= -2kx_1 + kx_2 \\
m\frac{d^2x_2}{dt^2} &= kx_1 - 2kx_2.
\tag\eqn
\label{Pair}
\end{align*}
Of course, to determine the motion completely, it is necessary
to specify the initial positions $x_1(t_0), x_2(t_0)$
and the initial velocities  $x_1'(t_0), x_2'(t_0)$ of {\it both\/}
masses.
\endexample

The above example is typical of many interesting physical systems.
For example, a molecule consists of several atoms with attractive
forces between them which to a first approximation may be treated
mathematically
as simple springs.
\outind{molecular oscillations}%

\subhead Higher Order Equations as Systems \endsubhead  There is
a simple trick which, by introducing new variables, allows us to
reduce a differential equation of {\it any order\/} to a
{\it first order system}.
\outind{higher order equations}%

\nextex
\example{Example \en}
Consider the differential equation
\nexteqn
$$
y''' + 2y'' + 3y' -4y = e^t.\tag\eqn
$$
There is an elaborate theory of such equations which generalizes
what we did in Chapter VII for
second order equations.   However, there is another
 approach which replaces the
equation by a {\it first order system\/}.
  Introduce new variables as follows:
$$
\align
x_1 &= y \\
x_2 &= y' = x_1' \\
x_3 &= y'' = x_2'.
\endalign $$
From the differential equation,
$$
x_3' = y''' = 4y - 3y' - 2y'' + e^t.
$$

Hence, we may replace the single equation (\eqn) by the
system
$$
\align
x_1' &= x_2 \\
x_2' &= x_3 \\
x_3' &= 4x_1  - 3x_2 - 2x_3 + e^t.
\endalign
$$
\endexample

The same analysis may in fact be applied to any system 
of any order in any number of variables. 

\example{Example \Couple, revisited}
Introduce additional variables $x_3 = x_1'$ and $x_4 = x_2'$.
Then from (\ref{Pair}), we have
$$
\align
x_3' &= x_1'' = -\frac{2k}m x_1 + \frac km x_2 \\
x_4' &= x_2'' = \frac km x_2 - \frac{2k}m x_2.
\endalign
$$
Putting this all together yields the first order system in
4 variables
$$
\align 
x_1' &= x_3\\
x_2' &= x_4 \\
x_3' &=  -\frac{2k}m x_1 + \frac km x_2\\
x_4' &= \frac km x_2 - \frac{2k}m x_2.
\endalign$$
To completely specify the solution, we need (in the new notation)
to specify the 4 initial values $x_1(0), x_2(0), x_3(0) = x_1'(0)$
and $x_4(0) = x_2'(0)$.
\endexample

By similar reasoning, any system may be reduced to a first order
system involving variables $x_1, x_2, \dots, x_n$ each of
which is a function of the independent variable $t$.   Using the
notation of $\R^n$, such a collection of variables may be
combined in a single {\it vector\/} variable
$$
\x = \lb x_1, x_2, \dots, x_n \rb
$$
which is assumed to be a vector valued function 
$\x(t)$  of $t$.  For each component $x_i$, its derivative
is supposed to be a function
$$
\frac{dx_i}{dt} = f_i(x_1,x_2,\dots,x_n,t) = f_i(\x,t).
$$
We may summarize this in a single vector equation
$$
\frac{d\x}{dt} = \f(\x,t)
$$
where the vector valued function $\f$ has components the
scalar functions $f_i$ for $i=1,2,\dots, n$.

The most important special case is that of {\it linear
systems}.   In this case the component functions have
the special form
$$
f_i(x_1,x_2,\dots,x_n,t) = a_{i1}(t)x_1 + a_{i2}(t)x_2 + 
\dots +a_{in}(t)x_n + g_i(t)
$$
for $i = 1,2,\dots,n$.  That is, each component function
depends {\it linearly\/} on the dependent variables
$x_1, x_2, \dots, x_n$ with coefficients  $a_{ij}(t)$
which as indicated may depend on $t$.    In this case, the
system $\dfrac{d\x}{dt} = \f(\x, t)$ may be written out
in `longhand' as
$$
\align 
\frac{dx_1}{dt} &= a_{11}(t)x_1 + a_{12}(t)x_2 + \dots + a_{1n}(t)x_n + g_1(t) \\
\frac{dx_2}{dt} &= a_{21}(t)x_1 + a_{22}(t)x_2 + \dots + a_{2n}(t)x_n  + g_2(t)\\
             &\vdots \\
\frac{dx_n}{dt} &= a_{n1}(t)x_1 + a_{n2}(t)x_2 + \dots + a_{nn}(t)x_n + g_2(t) .
\endalign$$
You should compare this with each of the examples discussed in this
section.   You should see that they are all linear systems.

The above notation is quite cumbersome, and clearly it will be
easier to follow if we use the notation of vectors in $\R^n$
as above.  To exploit this fully for linear systems we need
additional notation and concepts.   To this end, we shall study
some
{\it linear algebra\/}, one purpose of which is to make
higher dimensional problems look one-dimensional. 
  In the next sections we shall introduce
notation so that the above {\it linear system\/} 
may be rewritten
$$
\frac{d\x}{dt} =  A(t)\x + \g(t).
$$
With this notation, much of what we discovered about first
order equations in Chapter VI will apply
to systems.  
\bigskip
\input chap10.ex1
\bigskip

\nextsec{Matrix Algebra}
\head \sn. Matrix Algebra \endhead

A rectangular array
$$
\bm a_{11} & a_{12} &\dots & a_{1n} \\
a_{21} & a_{22} &\dots & a_{2n} \\
\vdots & \vdots &\dots &\vdots \\
 a_{m1} & a_{m2} &\dots & a_{mn} \em
$$
is called an $m\times n$ {\it matrix}.   It has $m$ {\it rows\/}
and $n$ columns.  The quantities $a_{ij}$ are called the {\it
entries\/} of the matrix.  The first index $i$ tells you which 
{\it row\/} it is in, and the second index $j$ tells you which
{\it column\/} it is in.
\outind{matrix}%

\example{Examples}
$$\align
\bm -2k & k \\
     k & -2k \em
\qquad &\text{is a $2\times 2$ matrix} \\
\vspace{10pt}
\bm 1 & 2 & 3 & 4 \\
    4 & 3 & 2 & 1 \em
\qquad&\text{is a $2\times 4$ matrix} \\
\vspace{10pt}
\bm x_1 & x_2 & x_3 & x_4 \em
\qquad&\text{is a $1\times 4$ matrix}\\
\bm x_1 \\ x_2 \\ x_3 \\ x_4 \em
\qquad&\text{is a $4\times 1$ matrix}\\
\endalign $$
\endexample

Matrices of various sizes and shapes arise in many situations.
For example, the first matrix listed above is the {\it matrix
of coefficients\/} on the right hand side of the system
$$
\align
m\frac{d^2x_1}{dt^2} &= -2kx_1 + kx_2 \\
m\frac{d^2x_2}{dt^2} &= kx_1 - 2kx_2.
\endalign$$
in Example \Couple\  of the previous section.

In computer programming, a matrix is called a {\it 2-dimensional array}
and the entry in row $i$ and column $j$ is usually denoted
$a[i,j]$ instead of $a_{ij}$.   As in programming, it is
useful to think of the entire array as a single entity, so
we use a single letter to denote it
$$
A =
\bm a_{11} & a_{12} &\dots & a_{1n} \\
a_{21} & a_{22} &\dots & a_{2n} \\
\vdots & \vdots &\dots &\vdots \\
a_{m1} & a_{m2} &\dots & a_{mn} \em .
$$

There are various different special arrangements which play important
roles.   A matrix with the same number of rows as columns is called
a {\it square\/} matrix.   
Matrices of coefficients for linear systems of
differential equations are usually square.
A $1\times 1$ matrix
$$
\bm a \em
$$
is not logically distinguishable from a {\it scalar\/}, so we
make no distinction between the two concepts.
A matrix with one row
$$
\a = \bm a_1 & a_2 & \dots & a_n \em
$$
is called a {\it row vector\/} and a matrix with one column
$$
\a = \bm a_1 \\ a_2 \\ \vdots \\ a_n \em
$$
is called a {\it column vector}.
Logically, either a $1\times n$ row vector or an $n\times 1$ column
with real entries
is just an $n$-tuple, i.e., an element of $\R^n$.   However, as we
shall see, operations with row vectors are sometimes different than
with column vectors.   We may identify either the set of all row
vectors with real entries {\it or\/} the set of all column vectors with
real entries with the set $\R^n$.    For reasons that will become
clear shortly, we shall usually make the latter choice.  That is, we shall
ordinarily think of an element of $\R^n$ as a column vector
$$
\x = \bm x_1\\ x_2\\ \vdots \\ x_n \em.
$$
More generally, it should be noted that the information contained in
any $m\times n$ matrix has two parts.  There are the $mn$ entries
which, in the real case, specify, in some order,
 an element of $\R^{mn}$, and
there is also the arrangement of the entries in rows and columns.

Matrices are denoted in different ways by different authors.  Most
people use ordinary (non-boldface) capital letters, e.g., $A, B, X, Q$.
However, one sometimes wants to use boldface for row or column
vectors, as above, when the relationship to $\R^n$ is being emphasized.
One may also use lower case non-boldface letters for row vectors
or column vectors.   Since there is no consistent rule about this,
you should make sure you know when a symbol represents a matrix
which is not a scalar.

Matrices may be combined in various useful ways.   Two matrices
{\it of the same size and shape\/} are added by adding corresponding
entries.   You are not allowed to add matrices with different shapes.

\example{Examples}
$$\gather
\bm\format \r&&\quad\r\\ 1 & -1 \\ 2 & 1 \\ 0 & 1 \em
+ \bm\format \r&&\quad\r\\ 1 & 1 \\ 0 & 3 \\ -1 & -2 \em
= \bm\format \r&&\quad\r\\ 2 & 0 \\ 2 & 4 \\ -1 & -1 \em \\
\\
\bm x+y \\ y \\ 0 \em + \bm -y \\ -y \\ x\em = \bm x \\ 0 \\ x \em.
\endgather$$
\endexample
The $m\times n$ matrix with zero entries is called a {\it zero matrix}
and is usually just denoted $0$.   Since zero matrices with different
shapes are not the same, it is sometimes necessary to indicate the
shape by using subscripts, as in  `$0_{mn}$', but usually the context makes it
clear which zero matrix is needed.   The zero matrix of a given shape
has the property that if you add it to any matrix $A$ of the same shape,
you get the $A$ again as the result.

A matrix may also be multiplied by a scalar by multiplying each entry
of the matrix by that scalar.   More generally, we may multiply
several matrices with the same shape by different scalars and add up
the result:  
$$
c_1A_1 + c_2A_2 + \dots + c_kA_k
$$
where $c_1, c_2, \dots, c_k$ are scalars and $A_1, A_2, \dots, A_k$
are $m\times n$ matrices with the same $m$ and $n$.   This process is
called {\it linear combination}.

\example{Example}
$$
2\bm 1 \\ 0\\ 1\\0 \em + (-1)\bm 0 \\ 1\\ 0 \\ 1 \em
+ 3\bm 1\\ 1\\ 1\\ 1 \em =
\bm 2 \\ 0\\ 2\\0 \em + \bm \format\r\\0 \\ -1\\ 0 \\ -1 \em
+ \bm 3\\ 3\\ 3\\ 3 \em = \bm 5 \\ 2 \\ 5 \\ 2 \em.
$$
\endexample

Sometimes it is convenient to put the scalar on the other side
of the matrix, but the meaning is the same: each entry of the
matrix is multiplied by the scalar.
$$
cA = Ac.
$$
\medskip

We shall also have occasion to consider matrix valued {\it
functions\/} $A(t)$ of a scalar variable $t$.  That means that
each entry $a_{ij}(t)$  is a function of $t$.   Such functions
are differentiated or integrated entry by entry.  

\example{Examples}
$$\align
\frac{d}{dt}\bm e^{2t} & e^{-t} \\ 2e^{2t} & -e^{-t} \em
 &= \bm 2e^{2t} & -e^{-t} \\ 4e^{2t} & e^{-t} \em \\
\vspace{10pt}
\int_0^1 \bm t \\ t^2 \em \,dt = \bm 1/2 \\ 1/3 \em
\endalign $$
\endexample
There are various ways to {\it multiply\/} matrices.  For example, one
sometimes multiplies matrices of the same shape by multiplying
corresponding entries.   This is useful only in very special circumstances.
Another kind of multiplication generalizes the {\it dot product\/}
of vectors.   If 
$$
\bm a_1 & a_2 & \dots & a_n \em
$$
is a row vector of size $n$, and 
$$
\bm b_1 \\ b_2 \\ \vdots \\ b_n \em
$$
is a column vector of the same size $n$, the row by column product
is defined to
be the sum of the products of corresponding entries
$$
\bm a_1 & a_2 & \dots & a_n \em
\bm b_1 \\ b_2 \\ \vdots \\ b_n \em
= a_1b_1 + a_2b_2 + \dots + a_nb_n = \sum_{i=1}^n a_ib_i.
$$
This product is of course a {\it scalar\/}, and except for the
distinction between row and column vectors, it is the same as the
notion of dot product for elements of $\R^n$ introduced in
Chapter I, Section 3.
You should be familiar with its properties.
\outind{matrix product}%
\outind{product of matrices}%

More generally, let $A$ be an $m\times n$ matrix and $B$ an
$n\times p$ matrix.  Then each row of $A$ has the same size as
each column of $B$.   The {\it matrix product\/}  $AB$ is
defined to be the $m\times p$ matrix with $i,j$ entry the
row by column product of the $i$th row of $A$ with the
$j$th column of $B$.  Thus, if $C = AB$, then $C$ has the same
number of rows as $A$, the same number of columns as $B$, and
$$
c_{ij} = \sum_{r=1}^n a_{ir}b_{rj}.
$$

\example{Examples}
$$\align
\undersetbrace{2\times 2}\to
{\bm 2 & 1 \\ 1 & 0 \em}
\undersetbrace{2\times 3}\to
{\bm 1 & 0 & 1 \\ -1 & 2 &  1 \em}
 &= \undersetbrace{2\times 3}\to
{\bm 2-1 & 0+2 & 2+1 \\ 1 - 0 & 0 + 0 & 1 + 0 \em} \\
 &= \bm 1 & 2 & 3 \\ 1 & 0 & 1 \em \\
\vspace{10pt}
\undersetbrace{3\times 2}\to
{\bm 1 & -1 \\ 1 & 0 \\  2 & 1 \em}
\undersetbrace{2\times 1}\to
{\bm x \\ y \em} &=
\undersetbrace{3\times 1}\to{\bm x - y \\ x \\ 2x + y \em}
\endalign$$
\endexample

The most immediate use for matrix multiplication is a simplification
of the notation used to describe a linear system.
We have
\smallskip
$$
\bm a_{11} & a_{12} &\dots & a_{1n} \\
a_{21} & a_{22} &\dots & a_{2n} \\
\vdots & \vdots &\dots &\vdots \\
 a_{n1} & a_{n2} &\dots & a_{nn} \em
\bm x_1 \\ x_2 \\ \vdots \\ x_n \em
=
\bm
a_{11}x_1 + a_{12}x_2 + \dots + a_{1n}x_n  \\
 a_{21}x_1 + a_{22}x_2 + \dots + a_{2n}x_n \\
             \vdots \\
 a_{n1}x_1 + a_{n2}x_2 + \dots + a_{nn}x_n \em.
$$
\smallskip
\noindent
(Note that the matrix on the right is an $n\times 1$ column vector
and each entry, although expressed as a complicated sum, is a scalar.)
With this notation, the linear system
$$
\align 
\frac{dx_1}{dt} &= a_{11}(t)x_1 + a_{12}(t)x_2 + \dots + a_{1n}(t)x_n + g_1(t) \\
\frac{dx_2}{dt} &= a_{21}(t)x_1 + a_{22}(t)x_2 + \dots + a_{2n}(t)x_n  + g_2(t)\\
             &\vdots \\
\frac{dx_n}{dt} &= a_{n1}(t)x_1 + a_{n2}(t)x_2 + \dots + a_{nn}(t)x_n + g_2(t) .
\endalign$$
may be rewritten
$$
\frac{d\x}{dt} = A(t)\x + \g(t)
$$
where
$$\align
\x = \x(t) &= \bm x_1(t) \\ x_2(t)\\ \vdots \\ x_n(t) \em \\
\vspace{10pt}
A(t) &=
\bm a_{11}(t) & a_{12}(t) &\dots & a_{1n}(t) \\
a_{21}(t) & a_{22}(t) &\dots & a_{2n}(t) \\
\vdots & \vdots &\dots &\vdots \\
 a_{n1}(t) & a_{n2}(t) &\dots & a_{nn}(t) \em \\
\vspace{10pt}
\g(t) &=  \bm g_1(t) \\ g_2(t)\\ \vdots \\ g_n(t) \em
\endalign$$
\nextex
\example{Example \en}
The system
$$
\align
x_1' &= x_2 \\
x_2' &= x_3 \\
x_3' &= 4x_1  - 3x_2 - 2x_3 + e^t.
\endalign
$$
may be rewritten
$$
\frac{d}{dt} \bm x_1 \\ x_2 \\ x_3 \em
=
\bm\format\r&&\quad\r\\ 0 & 1 & 0  \\
    0 & 0 & 1  \\
    4 & -3 & -2 \em
\bm x_1 \\ x_2 \\ x_3 \em
+ \bm 0 \\ 0 \\ e^t \em . 
$$
\endexample
\bigskip
Another important
application of matrices occurs in 
the analysis of  large systems of 
simultaneous linear algebraic equations.   We shall have much more
to say about this later in this chapter.   In addition, matrices
and linear algebra are used extensively in practically every branch
of science and engineering.   
%%%  Put in some other applications.
%A couple of such applications
%are discussed in the exercises.    
\bigskip
\input chap10.ex2
\bigskip
\nextsec{Formal Rules}
\head \sn. Formal Rules \endhead

The {\it usual rules of algebra\/} apply to matrices with a few
exceptions.   Here are {\it some\/} of these rules and
warnings about when they apply.

The {\it associative law\/} 
$$
A(BC) = (AB)C
$$
works as long as the shapes of the matrices match.  That means that
the length of each row of $A$ must be the same as the length of each
column of $B$ and the length of each row of $B$ must be the same as
the length of each column of $C$.  Otherwise, none of the products
in the formula will be defined.   The proof of the associative law
requires some fiddling with indices and is left for the Exercises.
\outind{associative law for matrices}%

For each positive integer $n$, the $n\times n$ matrix
$$
I = \bm 1 & 0 & \dots & 0 \\
        0 & 1 & \dots & 0 \\
        \vdots & \vdots & \dots & \vdots \\
        0 & 0 & \dots & 1 \em
$$
is called the {\it identity matrix\/} of degree $n$.  As in the case of the
zero matrices, we get a different identity matrix for each $n$, and
if we need to note the dependence on $n$, we shall use the notation
$I_n$.   The identity matrix of degree $n$
 has the property  $IA = A$ for any matrix $A$ with $n$ rows and
the property $BI = B$ for any matrix $B$ with $n$ columns.   The entries
of the identity matrix are usually denoted $\delta_{ij}$.
$\delta_{ij} = 1$ if $i = j$ (the {\it diagonal entries\/}) and
$\delta_{ij} = 0$ if $i\not= j$.   The indexed expression
$\delta_{ij}$ is often called the {\it Kronecker $\delta$}.
\outind{identity matrix}%

The {\it commutative law\/}  $AB = BA$ is {\it not generally true\/} for
matrix multiplication.   First of all, the products won't be defined
unless the shapes match.   Even if the shapes match on both sides,
the resulting products may have different sizes.  Thus, if $A$
is $m\times n$ and $B$ is $n\times m$, then $AB$ is $m\times m$
and $BA$ is $n\times n$.  Finally, even if the shapes match
and the products have the same sizes (if both $A$ and $B$ are
$n\times n$), it may still be true that the products are different.
\outind{commutative law, fails for matrix product}%

\nextex
\example{Example \en}
Suppose
$$
A = \bm 1 & 0 \\ 0 & 0 \em\qquad B = \bm 0 & 0 \\ 1 & 0 \em.
$$
Then
$$
AB = \bm 0 & 0 \\ 0 & 0 \em = 0 \qquad  BA = \bm 0 & 0 \\ 1 & 0 \em \not= 0
$$
so $AB \not= BA$.   Lest you think that this is a specially concocted
example, let me assure you that it is the exception rather than the
rule for the commutative law to hold for a randomly chosen pair of
square matrices.
\endexample

Another rule of algebra which
holds for scalars but {\it does not generally hold\/} for matrices is the
{\it cancellation law}.   

\nextex
\example{Example \en}
Let
$$
A = \bm 1 & 0 \\ 0 & 0 \em\qquad B = \bm 0 & 0 \\ 1 & 0 \em
\qquad C = \bm 0 & 0 \\ 0 & 1 \em.
$$
Then
$$
AB =  0\qquad\text{and}\qquad AC = 0
$$
so we cannot necessarily conclude from $AB = AC$ that $B = C$.
\endexample

The {\it distributive laws\/}
$$
\align
A(B + C) &= AB + AC \\
(A + B)C &= AC + BC
\endalign
$$
do hold as long as the operations are defined.  Note however
that since the commutative law does not hold in general, the
distributive law must be stated for both possible orders of
multiplication.

Another useful rule is
$$
c(AB) = (cA)B = A(cB)
$$
where $c$ is a scalar and $A$ and $B$ are matrices whose shapes
match so the products are defined.

The rules of calculus apply in general to matrix valued functions
except that you have to be careful about orders whenever products
are involved.   For example, we have
$$
\frac{d}{dt}(A(t)B(t)) = \frac{d A(t)}{dt}B(t) + A(t)\frac{d B(t)}{dt}
$$
for matrix valued functions $A(t)$ and $B(t)$ with matching shapes.

We have just listed {\it some\/} of the rules of algebra and calculus,
and we haven't discussed any of the proofs.   Generally, you can
be confident that matrices can be manipulated like scalars if you
are careful about matters like commutativity discussed above.
However, in any given case, if things don't seem to be working
properly, you should look carefully to see if some operation you
are using is valid for matrices.

\bigskip
\input chap10.ex3
\bigskip

\nextsec{Linear Systems of Algebraic Equations}
\head \sn.  Linear Systems of Algebraic Equations \endhead

Before studying the problem of solving a linear system of
differential equations, we tackle the simpler problem
of solving a linear system of simultaneous algebraic equations.
This problem is important in its own right, and, as we shall see,
we need to be able to solve linear algebraic systems in order
to be able to solve linear systems of differential equations.
\outind{linear system of algebraic equations}%

We start with a problem you ought to be able to solve from what
you learned in high school

\nextex
\example{Example \en}  Consider the algebraic system
\nexteqn
$$
\align
x_1 + 2x_2 - x_3 &= 1 \\
x_1 - x_2 + x_3 &= 0 \\
x_1 + x_2 + 2x_3 & = 1
\tag\eqn
\endalign $$
which is a system of 3 equations in 3 unknowns $x_1, x_2, x_3$.
This system may also be written more compactly as a matrix
equation
$$
\bm 1 & 2 & -1 \\ 1 & -1 & 1 \\ 1 & 1 & 2 \em \bm x_1 \\ x_2 \\ x_2 \em
= \bm 1 \\ 0 \\ 1 \em.
$$

The method we shall use to solve (\eqn) is the method of {\it elimination\/}
of unknowns.    Subtract the first equation from each of the
other equations to eliminate $x_1$ from those equations.
$$
\align
x_1 + 2x_2 - x_3 &= 1 \\
\hskip 30pt -3x_2 + 2x_3 &= -1 \\
\hskip 30pt -x_2  + 3x_3 &= 0 
\endalign $$
Now subtract 3 times the third equation from the second equation.
$$
\align
x_1 + 2x_2 - x_3 &= 1 \\
   -7x_3 &= -1 \\
 -x_2  + 3x_3 &= 0 
\endalign $$ 
 which may be reordered to obtain
$$
\align
x_1 + 2x_2 - x_3 &= 1 \\
 -x_2  + 3x_3 &= 0 \\
      7x_3 &= 1 .
\endalign $$ 
We may now solve as follows.  According to the last equation
$x_3 = 1/7$.  Putting this in the second equation yields
$$
-x_2 + 3/7 = 0\qquad\text{or}\qquad x_2 = 3/7.
$$
Putting $x_3 = 1/7$ and $x_2 = 3/7$ in the first equation yields
$$
x_1 +2(3/7) -1/7 = 1\qquad\text{or}\qquad x_1 = 1 -5/7 = 2/7.
$$
Hence, we get 
$$\align
x_1 &= 2/7 \\
x_2 &= 3/7 \\
x_3 &= 1/7 
\endalign$$

To check, we calculate
$$
\bm 1 & 2 & -1 \\ 1 & -1 & 1 \\ 1 & 1 & 2 \em \bm 2/7 \\ 3/7 \\ 1/7 \em
= \bm 1 \\ 0 \\ 1 \em.
$$
The above example illustrates the general procedure which may be applied
to any system of $m$ equations in $n$ unknowns
$$
\align
a_{11}x_1 + a_{12}x_2 + \dots + a_{1n}x_n  &= b_1 \\
a_{21}x_1 + a_{22}x_2 + \dots + a_{2n}x_n  &= b_2 \\
&\vdots \\
a_{m1}x_1 + a_{m2}x_2 + \dots + a_{mn}x_n & = b_m
\endalign $$
or, using matrix notation,
$$
A\x = \b
$$
with 
$$
\align
A &= \bm
a_{11}& a_{12} & \dots & a_{1n} \\
a_{21} & a_{22} & \dots & a_{2n} \\
\vdots &\vdots & \hdots & \vdots \\
a_{m1} & a_{m2} & \dots & a_{mn} \em
\\
\vspace{10pt}
\x &= \bm x_1 \\ x_2 \\ \vdots \\ x_n \em \\
\vspace{10pt}
\b &= \bm b_1 \\ b_2 \\ \vdots \\ b_m \em.
\endalign
$$

As in Example \en,  a
sequence of elimination steps
yields a set of equations each involving
at least one fewer unknowns than the one above it.   This
process is called {\it Gaussian reduction\/} after the famous
19th century German mathematician C. F. Gauss.   To complete the
\outind{Gaussian reduction}%
solution, we start with the last equation and substitute back
recursively in each of the previous equations.   This process is
called appropriately {\it back-substitution}.   The combined
\outind{back substitution}%
process will generally
lead to a complete solution, but, as we shall see later, there can
be some difficulties.

\medskip
\subhead Row Operations and Gauss-Jordan reduction\endsubhead
We now consider the general process of solving a system of
equations of the form
$$
AX = B
$$
where $A$ is an $n\times n$ matrix, $X$ is an $n\times p$
 matrix
of {\it unknowns\/}, and $B$ is an $n\times p$ matrix of known quantities.
Usually, $p$ will be 1, so $X$ and $B$ will be column vectors, but
the procedure is basically the same for any $p$.  For the moment we
emphasize the case in which the coefficient matrix $A$ is {\it square\/},
but we shall return later to the general case ($m$ and $n$ possibly
different).   
   
If you look carefully at Example \en, you will see that we employed
three basic types of operations:

\roster
\item adding or subtracting a multiple of one equation from another,
\item multiplying or dividing an equation by a non-zero scalar,
\item interchanging two equations.
\endroster

These operations correspond when using matrix notation to applying
the following operations to the matrices on both sides of the
equation $AX = B$:

\roster
\item adding or subtracting one row of a matrix to another,
\item multiplying or dividing one row of a matrix by a non-zero
scalar,
\item interchanging two rows of a matrix.
\endroster

These operations are called {\it elementary row operations\/}.
\outind{row operations, elementary}%
\outind{elementary row operations}%

An important principle about row operations that we shall use
over and over again is the following:   {\it To apply a row operation
to a product $AX$, it suffices to apply the row operation to
$A$ and then to multiply the result by $X$}.  It is easy to convince
yourself that this rule is valid by looking at examples.  Thus,
for the product,
$$
\bm a & b \\ c & d \em \bm x \\ y \em  = \bm ax + by \\ cx + dy \em
$$
adding the first row to the second yields
$$
\bm ax + by \\ ax + by + cx + dy \em =
\bm ax + by \\ (a + c)x + (b + d) y \em.
$$
On the other hand, adding the rows of the coefficient matrix yields
$$
\bm a & b \\ c & d \em \to \bm a & b \\ a + c & b + d \em,
$$
and multiplying the transformed matrix by $\displaystyle{\bm x \\ y \em}$
yields
$$
\bm ax + by \\ (a + c)x + (b + d) y \em
$$
as required.
(See the appendix to this section for a general proof.)

It is now clear how to proceed in general to solve a system of
the form 
$$
AX = B.
$$
Apply row operations to both sides until we obtain a system which
is easy to solve (or for which it is clear there is no solution.)
Because of the principle just enunciated, we may apply the row
operations on the left just to the matrix $A$ and omit reference
to $X$ since that is not changed.   For this reason, it is usual
to collect $A$ on the left and $B$ on the right 
in a so-called {\it augmented matrix}
$$
[A\,| \,B]
$$
where the `$|$' (or other appropriate divider) separates the two
matrices.    We illustrate this by considering another system
of 3 equations in 3 unknowns.
\nextex
\example{Example \en}
$$\align
 x_1 + x_2 - x_3 & = 0 \\
2 x_1 {\hskip 25pt} + x_3 & = 2 \\
x_1 - x_2 + 3x_3 &= 1 \\
\intertext{or}
\bm 1 & 1 & -1 \\ 2 & 0 & 1 \\ 1 & -1 & 3 \em
\bm x_1 \\ x_2 \\ x_2 \em &= \bm 0 \\ 2 \\  1 \em
\endalign
$$
We first do the Gaussian part of the reduction but for the
augmented matrix rather than the original set of equations.
$$
\align
\bm
\format \r&\quad\r&\quad\r&\quad\c\quad&\r\\
 1 & 1 & -1& | & 0 \\ 2 & 0 & 1 & | & 2 \\ 1 & -1 & 3 & | & 1\em
&\to
\bm
\format \r&\quad\r&\quad\r&\quad\c\quad&\r\\
 1 & 1 & -1& | & 0 \\ 0 & -2 & 3 & | & 2 \\ 1 & -1 & 3 & | & 1\em
\qquad -2[r1] + r2 \\
&\to
\bm
\format \r&\quad\r&\quad\r&\quad\c\quad&\r\\
 1 & 1 & -1& | & 0 \\ 0 & -2 & 3 & | & 2 \\ 0 & -2 & 4 & | & 1\em
\qquad -[r1] + r3 \\
&\to
\bm
\format \r&\quad\r&\quad\r&\quad\c\quad&\r\\
 1 & 1 & -1& | & 0 \\ 0 & -2 & 3 & | & 2 \\ 0 & 0 & 1 & | & -1\em
\qquad -[r2] + r3 
\endalign $$
At this point the corresponding system is
$$\align
x_1 + x_2 - x_3  &= 0 \\
     -2 x_2 + 3x_3 &= 2 \\
              x_3 &= -1 
\endalign
$$
so we could now apply back-substitution to find the solution.   However,
it is better for matrix computation to use an essentially equivalent
process.  Starting with the last row, use the leading non-zero entry
to eliminate the entries above it.  (That corresponds to substituting
the value of the corresponding unknown in the previous equations.)
This process is called {\it Jordan reduction}.
\outind{Jordan reduction}%
$$
\align
\bm
\format \r&\quad\r&\quad\r&\quad\c\quad&\r\\
 1 & 1 & -1& | & 0 \\ 0 & -2 & 3 & | & 2 \\ 0 & 0 & 1 & | & -1\em
&\to
\bm
\format \r&\quad\r&\quad\r&\quad\c\quad&\r\\
 1 & 1 & -1& | & 0 \\ 0 & -2 & 0 & | & 5 \\ 0 & 0 & 1 & | & -1\em
\qquad -3[r3] + r2 \\ 
&\to
\bm
\format \r&\quad\r&\quad\r&\quad\c\quad&\r\\
 1 & 1 & 0& | & -1 \\ 0 & -2 & 0 & | & 5 \\ 0 & 0 & 1 & | & -1\em
\qquad [r3] + r1 \\ 
&\to
\bm
\format \r&\quad\r&\quad\r&\quad\c\quad&\r\\
 1 & 1 & 0& | & -1 \\ 0 & 1 & 0 & | & -5/2 \\ 0 & 0 & 1 & | & -1\em
\qquad -(1/2)[r2]\\ 
&\to
\bm
\format \r&\quad\r&\quad\r&\quad\c\quad&\r\\
 1 & 0 & 0& | & -3/2 \\ 0 & 1 & 0 & | & -5/2 \\ 0 & 0 & 1 & | & -1\em
\qquad -[r2] + r1 
\endalign $$
This corresponds to the system
$$
\bm 1 & 0 & 0 \\ 0 & 1 & 0 \\ 0 & 0 & 1 \em X = 
\bm 
\format\r\\-3/2 \\ -5/2 \\ -1 \em
\qquad\text{or}\qquad X = 
\bm \format\r\\-3/2 \\ -5/2 \\ -1 \em
$$
which is the desired solution:  $x_1 = -3/2, x_2 = -5/2, x_3 = -1$.
(Check it by plugging back into the original matrix equation.)
\endexample

The combined method employed in the previous example is called
{\it Gauss-Jordan reduction}.   The strategy is clear.  Use a sequence
\outind{Gauss--Jordan reduction}%
of row operations to reduce the coefficient matrix $A$ to the identity
matrix I.   {\it If this is possible\/}, the same sequence of row operations
will transform the matrix $B$ to a new matrix $B'$, and the corresponding
matrix equation will be
$$
IX = B' \qquad\text{or}\qquad X = B'.
$$

It is natural at this point to conclude that $X = B'$ is the
solution of the original system, but there is a subtlety
involved here.
The method  outlined above shows the following: {\it if there
is a solution\/}, and if it is possible to reduce $A$ to $I$ by a sequence
of row operations, then the solution is $X = B'$.   In essence, this says
that if the solution exists, then it is unique.   It does not demonstrate
that any solution exists.    Why are we justified in concluding
that we do in fact have a solution when the reduction is possible?
	To understand
that, first note that {\it every possible row operation is reversible}.
Thus, to reverse the effect of adding a multiple of one row to another,
just subtract the same multiple of the first row from the (modified)
second row.   To reverse the effect of multiplying a row by a non-zero
scalar, just multiply the (modified) row by the reciprocal of that
scalar.  Finally, to reverse the effect of interchanging two rows,
just interchange them back.  Hence, the effect of any sequence of
row operations on a system of equations is to produce
an {\it equivalent\/} system of equations.   Anything which is
a solution of the initial system is necessarily a solution
of the transformed system and vice-versa.
Thus, the system  $AX = B$ is equivalent to the system
$X = IX = B'$, which is to say $X = B'$ is a solution
of $AX = B$.
\outind{row operations, reversibility of}%

% Advice to instructor.  Assign next subsection as reading assignment.
\subhead Elementary Matrices and the Effect of Row Operations on Products
\endsubhead
Each of the elementary row operations may be accomplished by multiplying
by an appropriate square matrix on the left.   Such matrices of course
should have the proper size for the matrix being multiplied.
\outind{elementary matrices}%

To add $c$ times the $j$th row of a matrix to the $i$th row (with $i\not = j$),
multiply that matrix on the left by the matrix $E_{ij}(c)$ which has diagonal
entries 1, the $i,j$-entry $c$, and all other entries 0.   This
matrix may also be obtained by applying the specified row operation
to the identity matrix.   You should try out a few examples to convince
yourself that it works.  

\example{Example} For $n = 3$,
$$
E_{13}(-4) = \bm \format\r&\quad\r&\quad\r\\
         1 & 0 & -4 \\ 0 & 1 & 0 \\ 0 & 0 & 1 \em.
$$
\endexample

To multiply the $i$th row of a matrix by $c \not= 0$, multiply that matrix 
on the left by the matrix $E_i(c)$ which has diagonal entries 1 except for
the $i,i$-entry which is $c$ and which has all other entries zero.   $E_i(c)$
may also be obtained by multiplying the $i$th row of the identity matrix by
$c$.

\example{Example}  For $n = 3$,
$$
E_2(6) = \bm 1 & 0 & 0 \\ 0 & 6 & 0 \\ 0 & 0 & 1 \em.
$$
\endexample

To interchange the $i$th and $j$th rows of a matrix, with 
$i\not = j$, multiply by the matrix on the left by the matrix $E_{ij}$
which is obtained from the identity matrix by interchanging its
$i$th and $j$th rows.   The diagonal entries of $E_{ij}$ are 1 except
for its $i,i,$ and $j,j$-entries which are zero.   Its $i,j$ and
$j,i$-entries are both 1, and all other entries are zero.

\example{Examples}
For $n = 3$,
$$
E_{12} = \bm 0 & 1 & 0 \\ 1 & 0 & 0 \\ 0 & 0 & 1 \em
\qquad\qquad E_{13} = \bm 0 & 0 & 1 \\ 0 & 1 & 0 \\ 1 & 0 & 0 \em.
$$

Matrices of the above type are called {\it elementary matrices}.

The fact that row operations may be accomplished by matrix multiplication
by elementary matrices
has many important consequences.   Thus, let $E$ be an elementary matrix
corresponding to a certain elementary row operation.   The associative
law tells us
$$
  E(AB) = (EA)B
$$
as long as the shapes match.   However, $E(AB)$ is the result of
applying the row operation to the product $AB$ and $(EA)B$ is the
result of applying the row operation to $A$ and then multiplying
by $B$.   This establishes the important principle enunciated earlier
in this section and upon which Gauss-Jordan reduction is based.
\bigskip
\input chap10.ex4
\bigskip

\nextsec{Singularity, Pivots, and Invertible Matrices}
\head \sn.   Singularity, Pivots, and Invertible Matrices \endhead
Let $A$ be a square coefficient matrix.   Gauss-Jordan reduction 
will work  as indicated in the previous section
 if $A$ can be reduced by
a sequence of elementary row operations to the identity matrix $I$.
A square matrix with this property is called {\it non-singular\/}
or {\it invertible}.   (The reason for the latter terminology will
be clear shortly.)  If it cannot be so reduced, it is called 
{\it singular}.  Clearly, there are singular matrices.   For example,
\outind{singular matrix}%
\outind{matrix, singular}%
\outind{matrix, non-singular}%
\outind{non-singular matrix}%
the matrix equation
$$
\bm 1 & 1 \\ 1 & 1 \em \bm x_1 \\ x_2 \em = \bm 1 \\ 0 \em
$$
is equivalent to the system of 2 equations in 2 unknowns
$$\align
x_1 + x_2 &= 1 \\
x_1 + x_2 & = 0
\endalign$$
which is {\it inconsistent\/} and has no solution.   Thus Gauss-Jordan
reduction certainly can't work on its coefficient matrix.
\outind{system, inconsistent}%
\outind{inconsistent system}%

To understand how to
tell if a square matrix $A$ is non-singular or not, we
look more closely at the Gauss-Jordan reduction process.   The
basic strategy is the following.   Start with the first row, and
 use type (1) row operations
to eliminate all entries in the first column below the $1,1$-position.   A
leading non-zero entry when used in this way is called a
{\it pivot}.   
There is one problem with this course of action: the leading
non-zero entry in the first row may not
be in the $1,1$-position.  In that case, 
 {\it first\/}
 interchange the first row with  a
succeeding row which does have a non-zero entry in the
first column. 
 (If you think about it, you may still see
a problem.  We shall come back to this and related issues later.)

 After the first
reduction, the coefficient matrix will have been transformed to
a matrix of the form
$$
\bm p_1 & * & \hdots & * \\ 0 & * & \hdots & * \\
\vdots & \vdots & \hdots & \vdots \\
0 & * & \hdots & * \em
$$
where $p_1$ is the (first) pivot.  We now do something mathematicians
\outind{pivot}%
(and computer scientists) love: repeat the same process for the
submatrix consisting of the second and subsequent rows.   If we
are fortunate, we will be able to transform $A$ ultimately by
a sequence of elementary row operations into matrix  of the form
$$
\bm p_1 & * & *  &\hdots & * \\ 0 & p_2 & * & \hdots & * \\
 0 & 0 & p_3 &\hdots & * \\
\vdots  &\vdots & \vdots & \hdots & \vdots \\
0 & 0 & 0 & \hdots & p_n \em
$$
with pivots on the diagonal and nonzero-entries in those pivot
positions.  (Such a matrix is also called
an {\it upper triangular matrix\/} because it has zeroes below
\outind{upper triangular matrix}%
the diagonal.) 
  We may now start in the lower right hand
corner and apply the Jordan reduction process.   In this way
each of the entries above the diagonal pivots may be eliminated,
so we obtain a diagonal matrix
$$
\bm p_1 & 0 &   0 &\hdots & 0 \\ 0 & p_2 & 0 & \hdots & 0 \\
 0 & 0 & p_3 &\hdots & 0 \\
\vdots  &\vdots & \vdots & \hdots & \vdots \\
0 & 0 & 0 & \hdots & p_n \em
$$
with non-zero entries on the diagonal.  We may now finish off the
process by applying type (2) operations to the rows as needed
and finally obtain the identity matrix $I$ as required.

The above analysis makes clear that the placement of the pivots
is what is essential to non-singularity.  What can go wrong?
It may happen for a given row that the leading non-zero entry
{\it is not in the diagonal position\/}, and there is no way to
remedy this by interchanging with a subsequent row.  In that
case, we just do the best we can.  {\it We use a pivot  as far to
the left as possible (after suitable row interchange with a 
subsequent row where necessary)}. 
 In the
extreme case, it may turn out that the submatrix we are working
with consists only of zeroes, and there are no possible pivots
to choose, so we stop.  For a square matrix, this
extreme case must occur, since we will run out of
pivot positions before we run out of rows.   Thus, the
Gaussian reduction will still transform $A$ to an upper triangular
matrix $A'$, but some of the diagonal entries will be zero and
some of the last rows (perhaps only the last row) will consist
of zeroes.   That is the singular case.

\nextex
\example{Example \en} 
$$
\align
\bm \format\r&\quad\r&\quad\r\\
1 & 2 & -1 \\ 1 & 2 & 0 \\ 1 & 2 & -2 \em
&\to 
\bm\format\r&\quad\r&\quad\r\\
1  & 2 & -1 \\ 0 & 0  & 1 \\ 0 & 0 & -1\em
\qquad\text{clear 1st column}
\\
&\to 
\bm\format\r&\quad\r&\quad\r\\
1  & 2 & 0 \\ 0 & 0  & 1 \\ 0 & 0 & 0\em
\qquad\text{no pivot in $2,2$ position}
\endalign $$
Note that the last row consists of zeroes.
\endexample

We showed in the previous section that if the $n\times n$ matrix $A$ is
non-singular, then every equation of the form  $AX = B$ (where
both $X$ and $B$ are $n\times p$ matrices) has a
solution and also that the solution $X = B'$ is {\it unique}.
On the other hand, if $A$ is {\it singular\/}, an equation of the form
$AX = B$ {\it may\/} have a solution, but {\it there will certainly
be matrices $B$ for which $AX = B$ has no solutions}.  
This is best illustrated by an example.

\nextex
\example{Example \en}
Consider the system
$$
\bm \format\r&\quad\r&\quad\r\\
1 & 2 & -1 \\ 1 & 2 & 0 \\ 1 & 2 & -2 \em\x = \b
$$
where $\x$ and $\b$ are $3\times 1$ column vectors.  Without specifying
$\b$, the reduction of the augmented matrix for this system
would follow the scheme
$$
\bm 
\format\r&\quad\r&\quad\r\quad $|$ &\quad \r\\
1 & 2 & -1 & b_1 \\ 1 & 2 & 0 & b_2 \\ 1 & 2 & -2 & b_3 \em
\to 
\bm
\format\r&\quad\r&\quad\r\quad $|$ &\quad \r\\
1  & 2 & -1 & * \\ 0 & 0  & 1 & *\\ 0 & 0 & -1 & *\em
\to 
\bm
\format\r&\quad\r&\quad\r\quad $|$ &\quad \r\\
1  & 2 & 0 & b_1' \\ 0 & 0  & 1 & b_2' \\ 0 & 0 & 0 & b_3'\em.
$$
Now simply choose $b_3' =1$ (or any other non-zero value), so
the reduced system is inconsistent.  (Its last equation would be
$0 = b_3' \not= 0$.)  Since, the two row operations may be reversed,
we can now work back to a system with the original coefficient
matrix which is also inconsistent.  (Check in this case that if you
choose $b_1' = 0, b_2' = 1, b_3' = 1$, then reversing the
operations yields $b_1 = -1, b_2 = 0, b_3 = -1$.)

The general case is completely analogous.  Suppose
$$
  A \to \dots \to A'
$$
is a sequence of elementary row operations which transforms $A$ to a matrix
$A'$ for which the last row consists of zeroes.   Choose any
$n\times p$ matrix $B'$ for which the last row {\it does not\/}
consist of zeroes.   Then the equation
$$
A'X = B'
$$
cannot be valid since the last row on the left will necessarily consist
of zeroes.  Now reverse the row operations in
the sequence which transformed $A$ to $A'$.  Let
$B$ be the effect of this reverse sequence on $B'$.
$$\gather
    A \leftarrow \dots \leftarrow A' \\
    B \leftarrow \dots \leftarrow B'
\endgather$$
Then the equation
$$
AX = B
$$
cannot be consistent because the equivalent system $A'X = B'$
is not consistent.

	We shall see later that when $A$ is a singular $n\times n$
matrix, if $AX = B$ has a solution $X$ for a particular $B$,
then it has infinitely many solutions.

There is one unpleasant possibility we never mentioned.  It is
conceivable that the standard sequence of elementary row operations
transforms $A$ to the identity matrix, so we decide it is
non-singular, but some other bizarre sequence of elementary
row operations transforms it to a matrix with some rows consisting
of zeroes, in which case we should decide it is singular.  Fortunately
this can never happen because singular matrices and non-singular
matrices have diametrically opposed properties.  For example,
if $A$ is non-singular then $AX = B$   has a solution for
every $B$, while if $A$ is singular, there are many $B$ for
which $AX = B$ has no solution.  This fact does not depend on
the method we use to find solutions.

\subhead Inverses of Non-singular Matrices \endsubhead
Let $A$ be a non-singular $n\times n$ matrix.   According to the
above analysis, the equation 
$$
AX = I
$$
(where we take $B$ to be the $n\times n$ identity matrix $I$)
has a unique   $n\times n$ solution matrix  $X = B'$.
This $B'$ is called the {\it inverse of $A$}, and it is usually
denoted $A^{-1}$.   That explains
why non-singular matrices are also called {\it invertible}.
\outind{invertible matrix}%
\outind{matrix, invertible}%
\outind{inverse of a matrix}%
\outind{matrix inverse}%

\nextex
\example{Example \en}  Consider
$$
A = \bm
\format\r&\quad\r&\quad\r\\
 1 & 0 & -1 \\
 1 & 1 & 0 \\
 1 & 2 & 0 \em
$$
To solve $AX = I$, we reduce the augmented matrix $[A\,|\,I]$.
$$
\align
\bm\format\r&\quad\r&\quad\r&\quad\c\quad&\r&\quad\r&\quad\r\\
 1 & 0 & -1&|& 1 & 0 & 0 \\
 1 & 1 & 0&|& 0 & 1 & 0 \\
 1 & 2 & 0&|& 0 & 0 & 1 \em
&\to
\bm\format\r&\quad\r&\quad\r&\quad\c\quad&\r&\quad\r&\quad\r\\
 1 & 0 & -1&|& 1 & 0 & 0 \\
 0 & 1 & 1&|& -1 & 1 & 0 \\
 0 & 2 & 0&|& -1 & 0 & 1 \em \\
&\to
\bm\format\r&\quad\r&\quad\r&\quad\c\quad&\r&\quad\r&\quad\r\\
 1 & 0 & -1&|& 1 & 0 & 0 \\
 0 & 1 & 1&|& -1 & 1 & 0 \\
 0 & 0 & -1&|& 1 & -2 & 1 \em
 \\
&\to
\bm\format\r&\quad\r&\quad\r&\quad\c\quad&\r&\quad\r&\quad\r\\
 1 & 0 & -1&|& 1 & 0 & 0 \\
 0 & 1 & 1&|& -1 & 1 & 0 \\
 0 & 0 & 1&|& -1 & 2 & -1 \em\\
&\to
\bm\format\r&\quad\r&\quad\r&\quad\c\quad&\r&\quad\r&\quad\r\\
 1 & 0 & 0&|& 0 & 2 & -1 \\
 0 & 1 & 0&|& 0 & -1 & 1 \\
 0 & 0 & 1&|& -1 & 2 & -1 \em .
\endalign$$
(You should make sure you see which row operations were used in
each step.)   Thus, the solution is
$$
X = A^{-1} =  \bm\format\r&\quad\r&\quad\r\\
  0 & 2 & -1 \\
  0 & -1 & 1 \\
  -1 & 2 & -1 \em.
$$
Check the answer by calculating
$$
A^{-1}A = 
 \bm\format\r&\quad\r&\quad\r\\
  0 & 2 & -1 \\
  0 & -1 & 1 \\
  -1 & 2 & -1 \em
\bm
\format\r&\quad\r&\quad\r\\
 1 & 0 & -1 \\
 1 & 1 & 0 \\
 1 & 2 & 0 \em 
 =
\bm 1 & 0 & 0 \\ 
    0 & 1 & 0 \\
    0 & 0 & 1 \em.
$$
\endexample

There is a subtle point about the above calculations.   The
matrix inverse  $X = A^{-1}$ was derived as the {\it unique\/}
solution of the equation $AX = I$, but we checked it by calculating
$A^{-1}A = I$.   The definition of $A^{-1}$ told us only that
$AA^{-1} = I$.   Since matrix multiplication is not generally
commutative, how could we be sure that the product {\it in the
other order\/} would also be the identity $I$?  The answer is
provided by the following tricky argument.   Let $Y = A^{-1}A$.
Then
$$
   AY = A(A^{-1}A) = (AA^{-1})A = IA = A
$$
so that $Y$ is the {\it unique\/} solution of the equation
$AY = A$.  However, $Y=I$ is {\it also\/} a solution of
that equation, so we may conclude that $A^{-1}A = Y = I$.
The upshot is that for a non-singular square matrix $A$, we have
{\it both\/}  $AA^{-1} = I$ and $A^{-1}A = I$.

The existence of matrix inverses for non-singular square matrices
suggests the following scheme for solving matrix equations of the
form
$$
  AX = B.
$$
First, find the matrix inverse $A^{-1}$, and then take $X = A^{-1}B$.
This is indeed the solution since
$$
  AX = A(A^{-1}B) = (AA^{-1})B = IB = B.
$$
However, as easy as this looks, one should not be misled by the formal
algebra.   Note that the only method we have for finding the
matrix inverse is to apply Gauss-Jordan reduction to the augmented
matrix $[A \,|\, I]$.   If $B$ has fewer than $n$ columns,
then applying Gauss-Jordan reduction directly to $[A\,|\, B]$
would ordinarily involve less computation that finding
$A^{-1}$.  Hence, in the most common cases, applying Gauss-Jordan
reduction  to the original system of equations is the best strategy.   
\medskip
\subhead Numerical Considerations in Computation \endsubhead
The examples we have chosen to illustrate the principles employ
small matrices for which one may do exact arithmetic.  The worst
that will happen is that some of the fractions may get a bit
messy.  In real applications, the matrices are often quite
large, and it is not practical to do exact arithmetic.   The
introduction of rounding and similar numerical approximations
complicates the situation, and computer programs for solving
systems of equations have to deal with problems which arise
from this.  If one is not careful in designing such a program,
one can easily generate answers which are very far off, and even deciding
when an answer is sufficiently accurate sometimes involves rather
subtle considerations.   Typically, one encounters problems
for matrices where the entries differ radically in size.  Also,
because of rounding, few matrices are ever {\it exactly\/}
singular since one can never be sure that a very small numerical
value at a potential pivot would have been  zero if the
calculations had been done exactly.  On the other hand, it
is not surprising that matrices which are close to being singular
can give computer programs indigestion.

If you are interested in such questions, there are many introductory
texts which discuss numerical linear algebra.   
Two such are {\it Introduction to Linear Algebra\/} by Johnson, Riess,
and Arnold  and
{\it Applied Linear Algebra\/} by Noble and Daniel.   One of the
computer assignments in your programming course is concerned
with some of the problems of numerical linear algebra.
\bigskip
\input chap10.ex5
\bigskip
\nextsec{Gauss-Jordan Reduction in the General Case}
\head \sn.  Gauss-Jordan Reduction in the General Case \endhead

Gauss-Jordan reduction works just as well if the coefficient
matrix $A$ is singular or even if it is not a square matrix.
Consider the system
$$
  A\x = \b
$$
where the coefficient matrix $A$ is an $m\times n$ matrix.
We shall concentrate on the case that $\x$ is an $n\times 1$
column vector of unknowns and $\b$ is a given $m\times 1$
column vector.  (This illustrates the principles, and the
case $AX=B$ where $X$ and $B$ are $n\times p$ matrices with
$p > 1$ works in a similar manner.)  The method is to apply
elementary row operations to the augmented matrix
$$
[A\,|\,\b] \to \dots \to [A'\, |\, \b']
$$
making the best of it with the coefficient matrix $A$.
We may not be able to transform $A$ to the identity matrix,
but we can always pick out a set of pivots, one in each non-zero
row, and otherwise mimic what we did in the case of a square
non-singular $A$.   If we are fortunate, the resulting system
$A'\x = \b'$ will have solutions.

\nextex
\example{Example \en} Consider
$$
\bm\format\r&\quad\r&\quad\r\\
1 & 1 & 2 \\
-1 & -1 & 1 \\
1 & 1 & 3 \em
\bm x_1\\ x_2\\ x_3 \em
=
\bm\format\r\\
1\\5\\3\em .
$$
Reduce the augmented matrix as follows
$$
\bm\format\r&\quad\r&\quad\r&\quad\c\quad&\r\\
1 & 1 & 2 &| & 1\\
-1 & -1 & 1& | & 5 \\
1 & 1 & 3 & | & 3 \em
\to
\bm\format\r&\quad\r&\quad\r&\quad\c\quad&\r\\
1 & 1 & 2 &| & 1\\
0 & 0 & 3 & | & 6 \\
0 & 0 & 1 & | & 2 \em
\to
\bm\format\r&\quad\r&\quad\r&\quad\c\quad&\r\\
1 & 1 & 2 &| & 1\\
0 & 0 & 3 & | & 6 \\
0 & 0 & 0 & | & 0 \em
$$
This completes the `Gaussian' part of the reduction with pivots
in the $1,1$ and $2,3$ positions, and the last row of the
transformed coefficient matrix consists of zeroes.
Let's now proceed with the `Jordan' part of the reduction.
Use the last pivot to clear the column above it.
$$
 \bm\format\r&\quad\r&\quad\r&\quad\c\quad&\r\\
1 & 1 & 2 &| & 1\\
0 & 0 & 3 & | & 6 \\
0 & 0 & 0 & | & 0 \em
\to 
 \bm\format\r&\quad\r&\quad\r&\quad\c\quad&\r\\
1 & 1 & 2 &| & 1\\
0 & 0 & 1 & | & 2 \\
0 & 0 & 0 & | & 0 \em
\to 
 \bm\format\r&\quad\r&\quad\r&\quad\c\quad&\r\\
1 & 1 & 0 &| & -3\\
0 & 0 & 1 & | & 2 \\
0 & 0 & 0 & | & 0 \em
$$
and the resulting augmented matrix corresponds to the system
$$\align
x_1 + x_2 \hphantom{+ x_3} &= -3 \\ 
                      x_3  &= 2 \\
0 &= 0
\endalign $$
Note that the last equation could just as well have read
$0 = 6$ (or some other non-zero quantity) in which case the
system would be inconsistent and not have a solution.  Fortunately,
that is not the case in this example.  The second equation
tells us $x_3 = 2$, but the first equation only gives a relation
$x_1 = -3 - x_2$ between $x_1$ and $x_2$.   That means that the
solution has the form
$$
\x = \bm x_1 \\ x_2 \\ x_3 \em
 = \bm -3 - x_2 \\ x_2 \\ 2 \em = \bm\format\r\\ -3\\ 0 \\ 2 \em
+ x_2\bm\format\r\\-1\\ 1\\ 0\em
$$
where $x_2$ can have {\it any value whatsoever}.   We say that
$x_2$ is a free variable, and the fact that it is arbitrary means
that there are {\it infinitely many solutions}.   $x_1$ and
$x_3$ are called {\it bound\/} variables.

It is instructive to reinterpret this geometrically in
$\R^3$.  The original system of equations may be written 
$$\align
x_1 + x_2 + 2x_3  &= 1\\
-x_1  - x_2 + x_3 &= 5 \\
x_1 + x_2 + 3x_3 &= 3 
\endalign
$$
which are equations for 3 planes in $\R^3$.   Solutions 
$$
\x = \bm x_1\\ x_2 \\x_3 \em
$$
correspond to points lying in the common
intersection of those planes.   Normally, we would expect
three planes to intersect in a single point.   That would have
been the case had the coefficient matrix been non-singular.
However, in this case the planes intersect in a line,
and the solution obtained above may be interpreted as the
vector equation of that line. 
If we put $x_2 = s$ and rewrite the equation using vector notation,
we obtain
$$
\x = \lb  -3, 0, 2 \rb + s\lb -1, 1, 0 \rb.
$$
\endexample

Example \en\ illustrates many features of the general procedure.
Gauss-Jordan reduction of the coefficient matrix is always
possible, but the pivots don't always end up on the diagonal.
In any case, the Jordan part of the reduction will yield
a 1 in each pivot position with zeroes above and below the
pivot in that column.   In any given row of the reduced
coefficient matrix, the pivot will be
on the diagonal or to its right, and  all entries {\it to the left
of the pivot\/} will be zero.  (Some of the entries to the right
of the pivot may be non-zero.)   
If the number of pivots is smaller than the number of rows
(which will always be the case for a singular square matrix),
then some rows of the reduced coefficient matrix will
 consist entirely of zeroes.
If there are non-zero entries in those rows 
to the right of the
divider {\it in the augmented matrix\/}, the system is inconsistent and
has no solutions.  Otherwise, the system does have solutions.
Such solutions are obtained by writing out the corresponding
system, and transposing all terms {\it not associated with the
pivot position\/} to the right side of the equation.  Each
unknown in a pivot position is then expressed in terms of
the non-pivot unknowns (if any). The pivot unknowns
are said to be {\it bound}.  The non-pivot unknowns may
be assigned any value and are said to be {\it free}.


\nextex
\xdef\HomEx{\en}
\example{Example \en}  Consider
\nexteqn
$$
\bm\format\r&\quad\r&\quad\r&\quad\r\\
  1& 2& -1& 0\\
  1& 2& 1& 3 \\
  2& 4& 0& 3 \em
\bm x_1\\x_2\\x_3\\x_4\em = \bm 0\\0\\0 \em.\tag\eqn
$$
Reducing the augmented matrix yields
$$
\align
\bm\format\r&\quad\r&\quad\r&\quad\r&\quad\c\quad&\r\\
  1& 2& -1& 0&|&0\\
  1& 2& 1& 3&|&0 \\
  2& 4& 0& 3&|&0 \em
&\to
\bm\format\r&\quad\r&\quad\r&\quad\r&\quad\c\quad&\r\\
  1& 2& -1& 0&|&0\\
  0& 0& 2& 3&|&0 \\
  0& 0& 2& 3&|&0 \em
\to
\bm\format\r&\quad\r&\quad\r&\quad\r&\quad\c\quad&\r\\
  1& 2& -1& 0&|&0\\
  0& 0& 2& 3&|&0 \\
  0& 0& 0& 0&|&0 \em\\
&\to
\bm\format\r&\quad\r&\quad\r&\quad\r&\quad\c\quad&\r\\
  1& 2& -1& 0&|&0\\
  0& 0& 1& 3/2&|&0 \\
  0& 0& 0& 0&|&0 \em
\to
\bm\format\r&\quad\r&\quad\r&\quad\r&\quad\c\quad&\r\\
  1& 2& 0& 3/2&|&0\\
  0& 0& 1& 3/2&|&0 \\
  0& 0& 0& 0&|&0 \em.
\endalign$$
(Note that since there are zeroes to the right of the divider,
we don't have to worry about possible inconsistency
in this case.)
The system corresponding to the reduced augmented matrix is
$$\align
x_1 + 2x_2 \hphantom{+ x_3} + (3/2)x_4 &= 0 \\
               x_3          + (3/2)x_4 &= 0 \\
0 &= 0
\endalign$$
Thus,
$$\align
x_1 &= -2x_2 - (3/2)x_4 \\
x_3 &= \hphantom{-2x_2} -3(/2)x_4
\endalign$$
with $x_1$ and $x_3$ {\it bound\/} and $x_2$ and $x_4$ {\it free}.
   A general solution has the
form
$$\align
\x &= \bm x_1\\x_2\\x_3\\x_4\em
  = \bm -2x_2 - (3/2)x_4 \\
       \hphantom{-2}x_2\hphantom{- (3/2)x_4}\\
       \hphantom{-2x_2} -(3/2)x_4 \\
       \hphantom{-2x_2 -} x_4 \em
  = \bm \format\r\\-2x_2\\ x_2\\ 0\\ 0 \em +
   \bm\format\r\\-(3/2)x_4\\ 0\\-(3/2)x_4\\ 0 \em \\ 
\x &= 
   x_2\bm \format\r\\-2\\ 1\\ 0\\ 0 \em +
   x_4\bm\format\r\\-3/2\\ 0\\-3/2\\ 0 \em \\ 
\endalign$$
where $x_2$ and $x_4$ can assume any value.

This solution may also be interpreted geometrically in $\R^4$.
Introduce two vectors
$$\align
\v_1 &= \lb -2, 1, 0, 0 \rb \\
\v_2 &= \lb -3/2, 0, -3/2, 0 \rb
\endalign
$$
in $\R^4$.   Note that neither of these vectors is a multiple of
the other.  Hence, we may think of them as spanning a (2-dimensional)
plane in $\R^4$.  Putting $s_1 = x_2$ and $s_2 = x_4$, we
may express the general  solution vector as 
$$
 \x = s_1\v_1 + s_2\v_2,
$$
so the solution set of the system (\eqn) may be identified with the
plane spanned by $\{\v_1, \v_2\}$.
\endexample

Make sure you understand
 the procedure used in the above examples to express the
general solution vector $\x$  entirely in terms of the free
variables.  We shall use it quite generally.

Any system of equations with real coefficients may be interpreted
as defining a locus in $\R^n$, and studying the structure---in
particular, the dimensionality---of such a locus is something
 which will concern us later.

\nextex
\example{Example \en}  Consider 
$$
\bm\format\r&\quad\r\\
    1 & 2 \\
    1 & 0 \\
    -1 & 1 \\
    2 & 0 \em
\bm x_1\\ x_2 \em =
\bm\format\r\\ 1 \\ 5 \\ -7 \\ 10 \em.
$$
Reducing the augmented matrix yields
$$\split
\bm\format\r&\quad\r&\quad\c\quad&\r\\
    1 & 2& | & 1 \\
    1 & 0& | & 5 \\
    -1 & 1 & | & -7\\
    2 & 0 & | & 10 \em
\to
\bm\format\r&\quad\r&\quad\c\quad&\r\\
    1 & 2& | & 1 \\
    0 & -2& | & 4 \\
    0 & 3 & | & -6\\
    0 & -4 & | & 8 \em
\to
\bm\format\r&\quad\r&\quad\c\quad&\r\\
    1 & 2& | & 1 \\
    0 & -2& | & 4 \\
    0 & 0 & | & 0\\
    0 & 0 & | & 0 \em \\
\to
\bm\format\r&\quad\r&\quad\c\quad&\r\\
    1 & 2& | & 1 \\
    0 & 1& | & -2 \\
    0 & 0 & | & 0\\
    0 & 0 & | & 0 \em
\to
\bm\format\r&\quad\r&\quad\c\quad&\r\\
    1 & 0& | & 5 \\
    0 & 1& | & -2 \\
    0 & 0 & | & 0\\
    0 & 0 & | & 0 \em
\endsplit
$$
which is equivalent to
$$
\align
x_1 &= 5 \\
x_2 &= -2.
\endalign$$
Thus the unique solution vector is
$$
\x = \bm\format\r\\5\\-2\em.
$$
\endexample

These examples and the preceding discussion lead us to certain
conclusions about a system of the form
$$
A\x = \b
$$
where $A$ is an $m\times n$ matrix, $\x$ is an $n\times 1$
column vector of unknowns, and $\b$ is an $m\times 1$ column
vector that is given.

  The number $r$ of pivots of $A$
is called the {\it rank\/} of $A$, and clearly it plays
\outind{rank of a matrix}%
\outind{matrix, rank of}%
an crucial role.  It is the same as the number of non-zero rows
at the end of the Gauss-Jordan reduction since there is exactly
one pivot in each non-zero row.  The rank is certainly not greater than 
either the number of rows $m$ or the number of columns $n$ of $A$.

If $m = n$, i.e., $A$ is a square matrix, then $A$ is
non-singular when its rank is $n$ and it is singular when its
rank is smaller than $n$. 

  More generally for an $m\times n$ matrix $A$, if the  rank $r$ is smaller
than the number of rows $m$, then there are $m\times 1$
column vectors $\b$ such that the system $A\x = \b$  {\it does not
have any solutions}.  The argument is basically the same
as for the case of a singular square matrix.  Transform $A$ by
a sequence of elementary row operations to a matrix $A'$
with its last row consisting of zeroes, choose $\b'$ so that
$A'\x = \b'$ is inconsistent, and reverse the operations to
find an inconsistent $A\x = \b$.
 
If for a given $\b$, the system $A\x = \b$
does have solutions, then the unknowns
$x_1, x_2, \dots, x_n$ may be partitioned into two sets:
$r$ bound unknowns and  $n - r$
free unknowns.  The bound unknowns are
expressed in terms of the free unknowns.  The number $n - r$
of free unknowns is sometimes called the {\it nullity\/} of
the matrix $A$.  If the nullity $n - r > 0$, i.e., $n > r$,
then (if there are any solutions at all) there are infinitely
many solutions.
    
Systems of the form
$$
A\x = 0
$$
are called {\it homogeneous}.  Example \HomEx\ is a homogeneous
\outind{homogeneous system}%
\outind{system, homogeneous}%
system.    Gauss-Jordan reduction of a homogeneous system always
succeeds since the matrix $\b'$ obtained from $\b = 0$ is
also zero.  If $m = n$, i.e., the matrix is square, and $A$
is non-singular, the {\it only solution} is 0, but if $A$
is singular, i.e., $r < n$, then there are definitely non-zero
solutions since there are some free unknowns which can be assigned
non-zero values.   This rank argument works for any $m$ and $n$:
if $r < n$, then there are definitely non-zero solutions for
the homogeneous system  $A\x = 0$.  One special case of interest
is $m < n$.  Since $r \le m$, we must have $r < n$ in that case.
That leads to the following important principle:  {\it a
homogeneous system of linear algebraic equations for which there
are more unknowns than equations always has some non-trivial
solutions.} 

%Exer  Explain why a system $A\x = \b$ with $A$ an $m\times n$ matrix
%Exer of rank $r$ will always have solutions if $m = r$.   Explain
%Exer why it will not have solutions for some choices of $\b$ if
%Exer $r < m$.

\medskip
\subhead Pseudo-inverses \endsubhead   It some applications, one
needs to try to find `inverses' of non-square matrices.   Thus,
if $A$ is a $m\times n$ matrix, one might need to find an
$n\times m$ matrix $A'$ such that
$$
AA' = I\qquad\text{the $m\times m$ identity}.
$$
Such an $A'$ would be called a {\it right pseudo-inverse}.   Similarly,
an $n\times m$ matrix $A''$ such that
$$
A''A = I\qquad\text{the $n\times n$ identity}
$$
is called a left pseudo-inverse.
\outind{pseudo-inverse of a matrix}%

If $m > n$, i.e., $A$ has  more rows
than columns, then {\it no
 right pseudo-inverse is possible}.  For, suppose we could find an
$n\times m$ matrix $A'$ such that
$AA' = I$ (the $m\times m$ identity matrix).
Then for any $m\times 1$ column vector $\b$, $\x = A'\b$
is a solution of $A\x = \b$ since
$$
A\x = A(A'\b) = (AA')\b = I\b = \b.
$$
On the other hand, we know that since $m > n \ge r$, there is
at least one $\b$ such that $A\x = \b$ does not have a solution.

\medskip
On the other hand,
if $m < n$ and the rank of $A$ is $m$
(which is as large as it can get in any case),
 then it is  always possible to find a right pseudo-inverse.
 To see this, 
let
$$
 X = \bm x_{11} & x_{12} & \hdots & x_{1m} \\
         x_{21} & x_{22} & \hdots & x_{2m} \\
       \vdots & \vdots & \hdots & \vdots \\
         x_{n1} & x_{n2} & \hdots & x_{nm} \em
$$
and consider the matrix equation
$$
AX = I.
$$
It may be viewed as $m$ separate equations of the form
$$
A\x = \bm 1\\0\\\vdots \\0\em,
A\x = \bm  0\\1\\\vdots \\0\em,
\dots,
A\x = \bm 0\\0\\\vdots \\1\em,
$$
one 
for each column of $I$.   Since $r = m$, each of these
equations has a solution.  (In fact it will generally have
infinitely many solutions.)  
\bigskip
\input chap10.ex6
\bigskip

\nextsec{Vector Spaces}
\head \sn.  Vector Spaces \endhead

Many of the notions we encountered when studying linear differential
equations have interesting analogues in the theory of solutions
of systems of algebraic equations.   Both these theories have
remarkable similarities to the theory of vectors in two and three
dimensions.    We give some examples.
\outind{vector space}%

The general solution of a first order linear equation
$$
y' + a(t)y = b(t)
$$
has the form
$$
y = y_p + c h
$$
where  $y_p$ is  one solution, $h$ is a solution of
the corresponding homogeneous equation,  and $c$ is a scalar
which may assume any value.  Similarly, we saw in Example 1 in the
previous section that the general solution of the system of equations
had the form
$$
\x = \x_0 + s \v
$$
where $\x_0 = \lb -3, 0, 2 \rb$ is one solution (for $s = 0$),
$\v = \lb -1, 1, 0 \rb$, and $s$ is a scalar which may assume
any value.   Each of these is reminiscent of
the equation of a line in $\R^3$
$$
\r = \r_0 + s\v
$$
where $\r_0$ is the position vector of a point on the line,
$\v$ is a vector parallel to the line, and $s$ is a scalar which
may assume any value.

Furthermore, we saw that the general solution of the second order
linear homogeneous differential equation
$$
y'' + p(t)y' + q(t)y = 0
$$
has the form
$$
y = c_1y_1 + c_2y_2
$$
where $\{y_1, y_2\}$ is a linearly independent pair of solutions
and $c_1$ and $c_2$ are two scalars which may assume any values.
Similarly, we saw in Example 2 in the previous section that
the general solution of the system had the form
$$
\x = s_1\v_1 + s_2\v_2
$$
where $\v_1 = \lb -2, 1, 0, 0 \rb, \; \v_2 =  \lb -3/2, 0, -3/2, 0 \rb$
and $s_1$ and $s_2$ are scalars which may assume any values.
Note that the pair $\{\v_1, \v_2\}$ is {\it linearly independent\/}
in exactly the same sense that a pair of solutions of a differential
equation is linearly independent, i.e.,
 {\it neither vector is a scalar multiple of the other}.
Both these situations are formally similar to what happens for
a plane in $\R^3$ which passes through the origin.   If $\{\v_1, \v_2\}$
is a pair of vectors in the plane, and neither is a multiple of the
other, then a general vector in that plane can be expressed
as a linear combination
$$
\v = s_1\v_2 + s_2\v_2.
$$

As one studies these different theories, more and more similarities
arise.   For example,  we just remarked that
the general solution of the differential
equation
$$
y'' + p(t)y' + q(t)y = f(t)
$$
has the form
$$
y = y_p + \text{a general solution of the corresponding homogeneous equation}
$$
where $y_p$ is  particular solution of the inhomogeneous equation.
Exactly the same rule applies to the general solution of
the inhomogeneous algebraic system
$$
A\x = \b.
$$
Namely, suppose  $\x_p$ is {\it one particular solution\/}
 of this inhomogeneous
\outind{inhomogeneous system, particular solution}%
system, and $\x$ is any other solution.  Then,
$$\align
A\x &= \b \\
A\x_p &= \b
\endalign$$
and subtraction yields
$$
A\x - A\x_p = A(\x - \x_p) = 0,
$$
i.e., $\z = \x - \x_p$ is a solution of the homogeneous system
$A\z = 0$.   Thus
$$
\x = \x_p + \z = \x_p + \text{a general solution of the homogeneous system}.
$$
The important point to note is that not only is the conclusion the
same, but the argument used to derive it is also essentially the same.

Whenever mathematicians notice that the same phenomena are observed
in different contexts, and that the same arguments are used to
study those phenomena, they
look for a common way to describe what is happening.   One of the
major accomplishments of the late nineteenth and early twentieth
centuries was the realization among mathematicians that there
is a single concept, that of an {\it abstract vector space\/}, which
may be used to study many diverse mathematical phenomena, including
those mentioned above.  They discovered
that the common aspects of all such theories were based on
the fact that they share certain  {\it operations\/},
and that these operations, although defined differently in each
individual case, {\it all obey common rules}.   Any
argument which uses only those rules will be valid in all cases.

%Advice to instructor.  This is all done in words so as to
% discourage spending a lot of time on it.

There are two basic operations which must be present for a collection
of objects to constitute a vector space.  (Additional
operations  may also be present, but for the moment we ignore
them.)  First, there should be an operation of addition which
obeys the usual rules you are familiar with for addition of
vectors in space.  Thus, addition should be an associative operation,
it should obey the commutative law, and there should be an element
called zero ($0$) which added to any other element results in
the same element.
 Finally, every element must have a
{\it negative\/} which when added to the original element yields zero.

For  example, in $\R^n$, addition $\x + \y$
 is  done by
adding corresponding entries of the two $n$-tuples.  The
zero element is the $n$-tuple $\bold 0$ for which all entries are
zero.  For any $\x$ in $\R^n$, the negative of $\x$ is just
the $n$-tuple $-\x$.

Alternatively, let $\bold S$ denote the set of all solutions of
the second order homogeneous linear differential equation
$$
y'' + p(t)y' + q(t)y = 0.
$$
In this case, the addition operation $y_1 + y_2$ is just the
ordinary addition of functions, and the zero element is the
function which is identically zero.   The negative of a function
is defined as expected by the rule $(-y)(t) = -y(t)$. 
Since the equation is homogeneous, the sum of two solutions is
again a solution, the zero function is also a solution, and
the negative of a solution is a solution.
If such were not the case, the set $\bold S$ would not be
{\it closed\/} under the operations, so it would not be a
vector space.

To have a vector space, we also need a second operation:
we must be able to multiply objects by scalars.   This operation must also
obey the usual rules of vector algebra.  Namely, it must satisfy
the associative law, distributive laws, and multiplying an object by the
scalar 1 should not change the object.
 
For example, in $\R^n$, we multiply a vector $\x$ by a scalar
$c$ in the usual way.

For $\S$, the vector space of solutions of a 2nd order homogeneous
linear differential equation, 
 a function (solution) is multiplied by a
scalar by the rule $(cy)(t) = cy(t)$.  Again, since the differential equation
is homogeneous, any scalar multiple of a solution is a solution,
so the set $\S$ is closed under the operation. 

In courses in abstract algebra, one studies in detail the list of
rules (or axioms) which govern the operations in a vector space.
In particular, one derives all the usual rules of vector algebra
from the properties listed above.  In this course, we shall assume
all that has been done, so you may safely manipulate objects
in any vector space just the way you would manipulate ordinary
vectors in space.   Note the different levels of abstraction
you have seen in this course for the concept `vector'.   First
a vector was just a quantity in the plane or in space with a magnitude
and a direction.   Vectors were added or multiplied by scalars using
simple geometric rules.  Later, we introduced the concept of a
`vector' in $\R^n$ which was an $n$-tuple of real numbers.  Such
`vectors' were added or multiplied by scalars by doing the same to
their components.  Finally, we are now considering `vectors' which
are {\it functions}.  Such `vectors' are added or multiplied by
scalars by doing the same to the function values.   As noted above,
what is important is not the nature of the individual object
we call a `vector', but the properties of the operations we define
on the {\it set\/} of all such `vectors'.
\outind{vector space}%

There are many other important examples of vector spaces.   Often
they
are {\it function spaces\/}, that is, their elements are functions
defined on some common domain.  Here are a few examples of
such vector spaces.
\outind{function space}%

Let  $I$ be a real interval.   The set $\Cal F(I)$ of all
real valued functions
defined on $I$ is vector space if we use the operations of adding
functions and multiplying them by scalars.  The set $\Cal C(I)$
of all {\it continuous real valued functions\/} defined on $I$ is also a vector
space because the sum of two continuous functions is continuous
and any scalar multiple of a continuous function is continuous.
Similarly, the set
$\Cal C^2(I)$ of all {\it twice differentiable real valued functions\/}
defined on $I$ {\it with continuous second derivatives\/} is a
vector space.   As mentioned before,
 the set of all such differentiable functions which are solutions of
a specified second order homogeneous linear differential equation
is a vector space.  Finally, the set of all real valued analytic functions
defined on $I$ is also a vector space.

	Not every set of `vector' like objects is a vector space.
We have already seen several such examples.   For example, the
vector equation of a line in $\R^3$ has the form
$$
\r = \r_0 + s\v
$$
where $\r$ is the position vector connecting the origin to a general
point on the line, $\r_0$ is the position vector of one particular
point on the line, and $\v$ is a vector parallel to the line.
If the line does not pass through the origin, the sum of two
position vectors with end points on the line won't be a vector
with endpoint on the line.   Hence, the set is not closed under
addition.  It is also not closed under multiplication by scalars.
\medskip
\centerline{\epsfbox{s10-2.ps}}
\medskip
Similarly, while the set of solutions of the homogeneous
first order equation
$$
y' + a(t) y = 0
$$
is a vector space, the set of solution of the (inhomogeneous)
first order equation
$$
y' + a(t) y = b(t)
$$
is not a vector space if $b(t)$ is not identically 0.   
 This rule holds in general.  The
set of solutions of a homogeneous linear equation (of any kind)
is a vector space, but the set of solutions of an inhomogeneous
linear equation is not a vector space.

%Exer  Show that the set of solutions of the first order equation
%Exer  $y' + a(t)y = b(t)$ is a vector space if and only if
%Exer  $b(t) = 0$. 


\medskip
\subhead The Domain of Acceptable Scalars \endsubhead    In the
previous discussion,  we assumed
implicitly that all scalars were {\it real}.   However, there are
circumstances where it would be more appropriate to allow scalars
which are either real or {\it complex}.  For example, we know that
it is easier to study solutions of a second order linear equation
with constant coefficients if we allow {\it complex valued\/}
solutions and use complex scalar constants when writing out the
general solution.   Thus, one must specify in any given case
whether one is talking about a {\it real\/} vector space, where
the set of possible scalars is restricted to $\R$, or a 
{\it complex\/} vector space, where the set of possible
scalars is the larger domain $\C$. 
\outind{scalar, complex}%
\outind{complex scalar}%

One important complex vector space is the set $\C^n$ of all
$n\times 1$ column vectors with {\it complex\/} entries.
Another is the set of all complex valued solutions of a given
homogeneous linear differential equation.
\outind{$\noexpand\C^n$}%

One confusing point is that every complex vector space may also
be considered to be a real vector space simply by agreeing to
allow only real scalars in any expression in which a `vector'
is multiplied by a scalar.  For example, the set $\C$ of all
complex numbers may itself be viewed as a real vector
space in this way.  If we do so, then, since any complex number
$a + bi$ is determined by the pair $(a,b)$ of its real and
imaginary parts, as a real vector space $\C$ is essentially
the same as $\R^2$.
\medskip
\subhead Subspaces \endsubhead   You may have noticed that in
many of the examples listed previously, some vector spaces
were {\it subsets\/} of others.   If $\V$ is a vector space,
and $\W$ is a non-empty subset, then we call $\W$
a {\it subspace\/} of $\V$ if whenever two elements of $\V$
\outind{subspace of a vector space}%
are in $\W$ so is their sum and whenever an element of $\V$
is in $\W$ so is any scalar multiple of that element.   This  
means that $\W$ becomes a vector space in its own right under
the operations it inherits from $\V$.

For $\V = \R^3$, most subspaces are what you would expect.
Any line passing through the origin yields a subspace, but a
line which does not pass through the origin does not.
Similarly, any plane passing through the origin yields a subspace
but a plane which does not pass through the origin does not.
A less obvious subspace is the {\it zero subspace\/}
which consists of the single vector $\bold 0$.   Also, in
general mathematical usage, any set is considered to be
a subset of itself, so $\R^3$ is also a subspace of $\R^3$.

For any vector space $\V$,
 the zero subspace and the whole vector
space are subspaces of $\V$. 


For a less obvious example, let $\Cal C^2(I)$ be
the vector space of all functions, defined on the real interval
$I$, with continuous second derivatives.    Then the
set of solutions of the homogeneous differential equation
$$
y'' + p(t)y' + q(t)y = 0
$$
is a subspace. 
In essence, we know this from earlier work, but let's derive it again by
a general argument.
  Consider the {\it operator\/} 
$$
L = \frac{d^2}{dt^2} + p(t)\frac{d}{dt} + q(t)
$$
acting on twice differentiable functions $y(t)$, i.e., let
$$
L(y) = y'' + p(t)y' + q(t)y.
$$
With this notation,
the differential equation may be written $L(y) = 0$.

$L$ is what we call a {\it linear operator\/} because it
obeys the following rules:
\outind{linear operator}%
\nexteqn
\xdef\eqnone{\eqn}
\nexteqn
\xdef\eqntwo{\eqn}
$$  
\align
L(y_1 + y_2) &= L(y_1) + L(y_2) \qquad\text{for functions }
y_1, y_2 \tag\eqnone \\
L(cy) &= cL(y) \qquad\text{for a function } y\ \text{
and a scalar } c. \tag\eqntwo
\endalign
$$
The fact that the set of solutions of $L(y) = 0$
 is a subspace is a direct
consequence of these rules.  Namely, if $L(y_1) = 0$ and
$L(y_2) = 0$, then rule (\eqnone) immediately gives
$L(y_1 + y_2) = L(y_1) + L(y_2) = 0$.  Similarly, if
$L(y) = 0$, rule (\eqntwo) immediately gives
$L(cy) = cL(y) = 0$ for any scalar $c$.

This same argument would work {\it for any linear operator\/},
and we shall see that
one of the most common ways to obtain
a subspace is as the solution set or {\it null space\/}
of a linear operator.
\outind{null space of a linear operator}%
For example, 
the solution set of a homogeneous system of algebraic
equations
$$
A\x = \bold 0
$$
is also a subspace because it is the null space
of an appropriate linear operator.  (See the Exercises.)

%Exer A\x = 0  L(\x) = A\x.
\bigskip
\input chap10.ex7
\bigskip

\nextsec{Linear Independence, Bases, and Dimension}
\head \sn.  Linear Independence, Bases, and Dimension \endhead

Let $\V$ be a vector space.   In general, $\V$ will have
infinitely many elements, but it is often possible to specify
$\V$ in terms of an appropriate finite subset.  For example,
we know that the vector space of solutions of a homogeneous
second order linear differential equation consists of all
linear combinations
$$
c_1y_1 + c_2y_2
$$
where $\{y_1, y_2\}$ is a linearly independent pair of solutions,
and $c_1, c_2$ are arbitrary scalars.   We say in this case that
the pair $\{y_1, y_2\}$ is a {\it basis\/} for the
 space of solutions. 
We want to generalize this to `bases' with more than two elements. 

As before, let $\V$ be any vector space and let 
$\{\v_1, \v_2, \dots, \v_k\}$ be a non-empty finite subset
of elements of $\V$.   Such a {\it set\/}
is called {\it linearly independent\/} if 
\outind{linear independence}%
\outind{independence, linear}%
no element of the set can be expressed
as a {\it linear combination\/} of the other elements in the
\outind{linear combination}%
set.  For a set $\{\v_1, \v_2\}$
 with two vectors, this subsumes the
previous definition: neither vector should be a scalar
multiple of the other.   For a set $\{\v_1, \v_2, \v_3 \}$
with three elements it means that {\it no\/} relation of
any of the following forms is possible:
$$\align
\v_1 &= a_2\v_2 + a_3\v_3 \\    
\v_2&= b_1\v_1 + b_3\v_3 \\
\v_3 &= c_1\v_1 + c_2\v_2.
\endalign $$
The opposite of `linearly independent' is `linearly dependent'.


  To get a better hold on the concept,
consider the (infinite) set of all
{\it linear combinations}
\nexteqn
$$
c_1\v_1 + c_2 \v_2 + \dots + c_k\v_k = \sum_{i=1}^k c_i\v_i\tag\eqn
$$
where each coefficient $c_i$ is allowed to range
arbitrarily over the domain of scalars.   It is not very
hard to see that this infinite set is a {\it subspace\/} of $\V$.
It is called the {\it subspace spanned by\/}
$\{\v_1, \v_2, \dots, \v_k\}$.  You can think of the elements
of this subspace as forming a general solution of some
(homogeneous)
problem. 
We would normally want to be sure
that there aren't any {\it redundant elements\/} in the
spanning set $\{\v_1,\v_2, \dots, \v_k\}$.
If one  $\v_i$ could be
expressed linearly in terms of the others, that expression 
for $\v_i$
could be substituted in (\eqn), and the result could be
simplified by combining terms.  We could thereby omit
$\v_i$ and  express
the general element in 
(\eqn) as a linear combination of the other elements in the
spanning set. 

\nextex
\example{Example \en}
Consider the set consisting of the following four vectors in $\R^4$.
$$
\v_1 = \bm 1\\ 0 \\ 0 \\ 0 \em,\;
\v_2 = \bm 1\\ -1 \\ 0 \\ 0 \em,\;
\v_3 = \bm 0\\ 1 \\ 0 \\ 0 \em,\;
\v_4 = \bm 0\\ 1 \\ 1 \\ -1 \em.
$$
This set is not linearly independent since
\nexteqn
$$
\v_2 = \v_1 - \v_3.\tag\eqn
$$
Thus, any element is the subspace spanned by $\{\v_1,\v_2,\v_3,\v_4\}$
can be rewritten
$$\align
c_1\v_1 + c_2\v_2 + c_3\v_3 + c_4\v_4
&= c_1\v_1 + c_2(\v_1 - \v_3) + c_3\v_3 + c_4\v_4\\
&= (c_1 + c_2)\v_1 + (c_3 - c_2)\v_3 + c_4\v_4 \\
 &= c_1'\v_1 + c_3'\v_3
+ c_4\v_4.
\endalign$$ 

On the other hand, if we delete the element $\v_2$,
the set consisting of the vectors
$$
\v_1 = \bm 1\\ 0 \\ 0 \\ 0 \em,\;
\v_3 = \bm 0\\ 1 \\ 0 \\ 0 \em,\;
\v_4 = \bm 0\\ 1 \\ 1 \\ -1 \em.
$$
is linearly independent.
To see this, just look carefully at the pattern of zeroes. For
example, $\v_1$ has first component 1, and the other two have
first component 0, so $\v_1$ could not be a linear combination
of $\v_2$ and $\v_3$.  Similar arguments eliminate the other
two possible relations.  (What are those arguments?)

%Exer What are the arguments?

In the above example, we could just as well have written
$$
\v_1 = \v_2 + \v_3
$$
and eliminated $\v_1$ from the spanning set without loss.
   In general, there are many possible ways to delete
redundant vectors from a spanning set.

A linearly independent subset $\{\v_1,\v_2, \dots, \v_n\}$
of  a vector space $\V$ which also spans $\V$ 
is called a {\it basis} for $\V$.  Many of the algorithms we
\outind{basis of a vector space}%
have for solving homogeneous problems yield bases for the
 solution space.   For example, as noted above, any
linearly independent pair of solutions of a second order homogeneous
linear equation is a basis for its solution space.   Much of
what we do later will be designed to generalize that to higher
order differential equations and to systems of differential
equations.  

	Let $A$ be an $m\times n$ matrix, and
let $\W$ be the solution space of the homogeneous system
$$
A\x = 0.
$$
(To be definite, assume the matrix has real entries and that
$\W$ is the solution subspace of $\R^n$.  However, the corresponding theory
for a complex matrix with solution subspace in $\C^n$ is basically
the same.)    The Gauss-Jordan reduction method always generates
a basis for $\W$.    We illustrate this with an example.
(You should also go back and look at Example 2 in Section 6.)
%Back reference?

\nextex
\example{Example \en}  Consider
$$
\bm 
\format\r&\quad\r&\quad\r&\quad\r&\quad\r\\
1 & 1 & 0 & 3 & -1 \\
1 & 1 & 1 & 2 & 1 \\
2 & 2 & 1 & 5 & 0 \em
\bm\format\r\\x_1\\x_2\\x_3\\x_4\\x_5 \em
= 0.
$$
To solve it, apply Gauss-Jordan reduction
$$
\align
\bm 
\format\r&\quad\r&\quad\r&\quad\r&\quad\r&\quad\c\quad&\r\\
1 & 1 & 0 & 3 & -1 &|&0 \\
1 & 1 & 1 & 2 & 1 &|&0 \\
2 & 2 & 1 & 5 & 0  &|&0\em
&\to
\bm\format\r&\quad\r&\quad\r&\quad\r&\quad\r&\quad\c\quad&\r\\
1 & 1 & 0 & 3 & -1 &|&0 \\
0 & 0 & 1 & -1 & 2 &|&0 \\
0 & 0 & 1 & -1 & 2 &|&0 \em\\
&\to
\bm\format\r&\quad\r&\quad\r&\quad\r&\quad\r&\quad\c\quad&\r\\
1 & 1 & 0 & 3 & 0 &|&0 \\
0 & 0 & 1 & -1 & 2 &|&0 \\
0 & 0 & 0 & 0 & 0  &|&0\em.
\endalign
$$
  The last matrix is fully reduced with pivots in the $1,1$ and
$2,3$ positions.  The corresponding system is
$$
\align
x_1 + x_2 \hphantom{+ x_3} + 3x_4 \hphantom{+ 2x_5} & = 0 \\
                       x_3  -\hphantom{3}x_4 + 2x_5 & = 0 
\endalign$$
with $x_1, x_3$ bound and $x_2, x_4$, and $x_5$ free. Expressing
the bound variables in terms of the free variables yields
$$
\align
x_1 &=  - x_2 - 3x_4 \\
x_3 &=  \hphantom{-x_2} +\hphantom{3}x_4 - 2x_5.
\endalign$$
The general solution vector, when expressed in terms of the
free variables, is
$$
\align
\x = 
\bm\format\r\\x_1\\x_2\\x_3\\x_4\\x_5 \em
= \bm  -x_2 - 3x_4 \\ x_2 \\
x_4 - 2x_5\\ x_4 \\x_5 \em
&= \bm\format\r\\ -x_2\\x_2\\0\\0\\0\em
+ \bm\format\r\\ -3x_4\\0\\x_4\\x_4\\0\em
+\bm\format\r\\ 0\\ 0\\-2x_5\\0\\x_5\em \\
&= x_2\bm\format\r\\ -1\\1\\0\\0\\0\em
+ x_4\bm\format\r\\ -3\\0\\1\\1\\0\em
+x_5\bm\format\r\\ 0\\ 0\\-2\\0\\1\em .
\endalign$$
If we put
$$
\v_1 = 
\bm\format\r\\ -1\\1\\0\\0\\0\em,\quad
\v_2 =\bm\format\r\\ -3\\0\\1\\1\\0\em,\quad
\v_3 =\bm\format\r\\ 0\\ 0\\-2\\0\\1\em,
$$
and $c_1 = x_2,\, c_2 = x_4$, and $c_3 = x_5$, then the general
solution takes the form
$$
\x = c_1\v_1 + c_2\v_2 + c_3\v_3
$$
where the scalars $c_1, c_2, c_3$ (being new names for the free
variables) can assume any values.
Also, the set $\{\v_1, \v_2, \v_3\}$ is linearly independent.
This is clear for the following reason.  Each vector is associated
with one of the free variables and has a 1 in that position where
the other vectors necessarily have zeroes.  Hence, none of the
vectors can be linear combinations of the others.
It follows that $\{\v_1, \v_2, \v_3\}$ is a basis for the
solution space.
\endexample

The above example illustrates all the important aspects of the
solution process for a homogeneous system
$$
A\x = 0.
$$
We state the important facts about the solution without going
through the general proofs since they are just the same as what
we did in the example but with a lot more confusing notation.
The general solution has the form
$$
\x = c_1 \v_1 + c_2\v_2 + \dots + c_k\v_k
$$
where $\v_1, \v_2, \dots, \v_k$ are {\it basic
solutions\/} obtained {\it by successively setting each free
variable equal to 1 and the other free variables equal to
zero}.  $c_1, c_2 \dots, c_k$ are just new names for the
free variables.  The set $\{\v_1  , \v_2, \dots, \v_k\}$
is linearly independent because of the pattern of 1's and
0's at the positions of the free variables, and since it
spans the solution space, it is a basis for the solution
space.

There are some special cases which are a bit confusing.
First, suppose there is only one basic solution $\v_1$.
Then, the set $\{\v_1\}$ with one element is indeed a basis.
In fact, in any vector space, the set $\{\v\}$ consisting
of a single {\it non-zero vector\/} is linearly independent.
Namely, there are no other vectors in the set which it
could be a linear combination of.   In this case, the subspace
spanned by $\{\v\}$ just consists of all multiples $c\v$
where $c$ can be any scalar.   A much more confusing case
is that in which the spanning set is the {\it empty set\/},
i.e., the set with no elements.   (That would arise, for example,
 if the zero solution were the unique solution of the 
homogeneous system, so there would be no free variables and no
basic solutions.)  This is dealt with as follows.   First, the
empty set is taken to be linearly  independent by convention.
Second, again by convention, we take every linear
combination of {\it no vectors\/} to be  zero.  It follows that
the empty set spans the zero subspace $\{0\}$, and is a basis for it.
(Can you see why the above conventions imply that the set
$\{0\}$  is {\it not\/} linearly independent?)

Let $\V$ be a vector space.   If $\V$ has a basis
$\{\v_1, \v_2,\dots, \v_n\}$ with $n$ elements, then we
say that $\V$ is $n$-dimensional.   That is, the dimension
\outind{dimension of a vector space}%
of a vector space is the number of elements in a basis.

  For example,
since the solution space of a 2nd order homogeneous linear
differential equation has a basis with two elements,
that solution space is 2-dimensional.  

Not too surprisingly, the dimension of $\R^n$ is $n$.  To
see this we note that the set consisting of the vectors
$$
\e_1 = \bm 1\\0\\0\\\vdots\\0\em,\quad
\e_2 = \bm 0\\1\\0\\\vdots\\0\em,\quad
\dots,
\e_n = \bm 0\\0\\0\\\vdots\\1\em,\quad
$$
is a basis.   For, the set is certainly linearly independent
because the pattern of 0's and 1's precludes any dependence
relation.  It also spans $\R^n$ because any vector $\x$
in $\R^n$ can be written
$$
\align
\x = \bm x_1\\x_2\\\vdots\\x_n \em
 &= x_1\bm 1\\0\\0\\\vdots\\0\em
+ x_2\bm 0\\1\\0\\\vdots\\0\em
+
\dots
+ x_n\bm 0\\0\\0\\\vdots\\1\em\\
&= x_1 \e_1 + x_2\e_2 + \dots + x_n\e_n.
\endalign
$$
$\{\e_1, \e_2, \dots, \e_n\}$ is called the {\it standard basis\/}
\outind{standard basis}%
\outind{basis, standard}%
for $\R^n$.
Note that in $\R^3$ the vectors $\e_1, \e_2$, and $\e_3$ are
what we previously called $\i, \j$, and $\k$.

In this chapter we have defined the concept 
dimension only for vector spaces, but the notion is considerably
more general.  For example, a plane in $\R^3$ should be considered
two dimensional even if it doesn't pass through the origin.  Also,
a surface in $\R^3$, e.g., a sphere or hyperboloid, should also
be considered two dimensional.  (People are often confused about
curved objects because they seem to extend in extra dimensions.
The point is that if you look at a small part of a surface,
it normally looks like a piece of a plane, so it has the
same dimension. Also, as we have seen, a surface can normally
be represented parametrically with only two parameters.)  
Mathematicians have develped a very general theory of
dimension which applies to almost any type of set.  In cosmology,
one envisions the entire universe as a certain type of
four dimensional object.    Certain
bizarre sets can even have a fractional dimension, and that
concept is useful in what is called `chaos' theory.
\medskip
\subhead Coordinates \endsubhead
Let $\V$ be a vector space and suppose  $\{\v_1, \v_2, \dots, \v_n\}$
is a linearly independent subset of $\V$.
  Suppose $\v$  is in the subspace spanned by $\{\v_1, \v_2, \dots, \v_n\}$,
i.e.,
$$
\v = c_1\v_1 + c_2\v_2 + \dots + c_n\v_n
$$
for appropriate coefficients $c_1, c_2, \dots, c_n$.
{\it The coefficients in such a linear combination are unique}.
For, suppose we had
$$
\v = c_1\v_1 + c_2\v_2 + \dots + c_n\v_n
   = c'_1\v_1 + c'_2\v_2 + \dots + c'_n\v_n.
$$
Subtract one expression from the other to obtain
$$
   (c_1 - c'_1)\v_1 + (c_2 - c'_2)\v_2 + \dots + (c_n - c'_n)\v_n = 0.
$$
We would like to conclude that all these coefficients are
zero, i.e., that
$$
\align
c_1 &= c_1'\\
c_2 &= c_2'\\
&\vdots \\
c_n &= c_n'.
\endalign
$$
If that were not the case, one of the coefficients would be
non-zero, and we could divide by it and transpose, thus
expressing one of the vectors $\v_i$ as a linear combination
of the others.  But since, the set is linearly independent, we
know that is impossible.   Hence, all the coefficients are
zero as required.


Suppose $\{\v_1, \v_2, \dots, \v_n\}$ is a basis for $\V$.
The above argument
shows that {\it any\/} vector $\v$ in $\V$ may be expressed {\it uniquely\/}
$$
\v = c_1\v_1 + c_2\v_2 + \dots + c_n\v_n,
$$
and the coefficients $c_1, c_2, \dots, c_n$ are called the
{\it coordinates\/} of the vector $\v$ with respect to the
\outind{coordinates with respect to a basis}%
basis $\{\v_1, \v_2, \dots, \v_n\}$.   A convenient way to
exhibit the relationship between a vector and its coordinates
is as follows.  Put the coefficients $c_i$
on the other side of the basis vectors, and write
$$
\v  = \v_1c_1 + \v_2c_2 + \dots \v_nc_n 
    = \bm \v_1 & \v_2 & \dots &\v_n\em
       \bm c_1\\c_2\\ \vdots \\ c_n\em.
$$
The column vector on the right is a bona-fide 
element of $\R^n$
(or of $\C^n$ in the case of complex scalars), but the
`row vector' on the left is not really an $n\times 1$
matrix since its entries are vectors, not scalars.

\nextex
\example{Example \en}   Consider the vector space $\S$
of all real solutions
of the differential equation
$$
  y'' + k^2y = 0.
$$
The solutions $y_1 = \cos kt$ and $y_2 = \sin kt$ constitute
a linearly independent pair of solutions, so that gives a basis
for $\S$.   On the other hand
$$
y = \cos(kt + \delta)
$$
is also a solution, so it should be expressible as a linear
combination of the basis elements.  Indeed, by trigonometry,
we have
$$\align
y = \cos(kt + \delta) &= \cos(kt)\cos\delta - \sin(kt)\sin\delta \\
             &= y_1\cos\delta + y_2(-\sin\delta) \\
     & \bm y_1 & y_2 \em \bm \format\r\\ \cos\delta\\-\sin\delta \em.
\endalign
$$
Thus
$$
 \bm \format\r\\\cos\delta\\-\sin\delta \em
$$
is a vector in $\R^2$ giving the coordinates of $\cos(kt + \delta)$
with respect to the basis $\{y_1, y_2 \}$.
\endexample

Given a basis for a vector space $\V$, one may think of the
elements of the basis as unit vectors pointing along coordinate
axes in $\V$.   The coordinates with respect to the basis
then are the coordinates relative to these axes.  If one
starts (as one does normally in $\R^n$) with some specific
set of axes, then the axes associated with a new basis need
not be mutually perpendicular, and also the unit of length
may be altered, and we may even have different units of
length on each axis.


\subhead Invariance of Dimension \endsubhead
There is a subtle point involved in the definition of dimension.
The dimension of $\V$ is the number of elements in a basis for
$\V$, but it is at least conceivable that two different bases
have different numbers of elements.  If that were the case,
$\V$ would have two different dimensions, and that does not
square with our idea of how such words should be used.
\outind{dimension, invariance of}%

In fact it can never happen that two different bases have different
numbers of elements.      To see this, we shall prove something
slightly different.  Suppose $\V$ has a basis with $m$ elements.
We shall show that 

{\narrower\it any linearly independent subset of $\V$
has at most $m$ elements.}

   This would suffice for what we 
want because if we had two bases one with $n$ and the other with
$m$ elements, either could play the role of the basis and the
other the role of the linearly independent set.  (Any basis is
also linearly independent!)  Hence, on the one hand we would have
$n \le m$ and on the other hand $m\le n$, whence it follows that
$m = n$.

Here is the proof
of the above assertion about linearly independent subsets,
(but you might want to skip it your first time through the subject).
%Advice  Skip this proof in class.
Let $\{\u_1, \u_2, \dots, \u_n\}$ be a linearly independent subset. 
  Each
$\u_i$ can be expressed uniquely in terms of the basis
$$
\align
\u_1 = \sum_{j=1}^m \v_j p_{j1} &= 
\bm \v_1 & \v_2 & \dots & \v_m \em 
\bm p_{11}\\ p_{21}\\ \vdots \\ p_{m1} \em \\
\u_2 = \sum_{j=1}^m \v_j p_{j2} &= 
\bm \v_1 & \v_2 & \dots & \v_m \em 
\bm p_{21}\\ p_{22}\\ \vdots \\ p_{m2} \em \\
&\vdots \\
\u_n = \sum_{j=1}^m \v_j p_{jn} &= 
\bm \v_1 & \v_2 & \dots & \v_m \em 
\bm p_{1n}\\ p_{2n}\\ \vdots \\ p_{mn} \em .
\endalign
$$
Each of these equations represents one column of the complete
matrix equation
$$
\bm \u_1 & \u_2 & \hdots & \u_n \em
=
\bm \v_1 & \v_2 & \dots & \v_m \em
\bm p_{11} & p_{12} & \hdots & p_{1n} \\
    p_{21} & p_{22} & \hdots & p_{2n}\\
   \vdots & \vdots & \hdots & \vdots \\
    p_{m1} & p_{m2} & \hdots & p_{mn} \em.
$$
Note that the  matrix on the right is an $m\times n$ matrix.
Consider the homogeneous system
$$
\bm p_{11} & p_{12} & \hdots & p_{1n} \\
    p_{21} & p_{22} & \hdots & p_{2n}\\
   \vdots & \vdots & \hdots & \vdots \\
    p_{m1} & p_{m2} & \hdots & p_{mn} \em
\bm x_1\\x_2\\ \vdots \\ x_n \em = 0.
$$
Assume, contrary to what we hope, that $n > m$. Then,
 we know by the theory of homogeneous
linear systems, that there is a non-trivial solution to this system,
i.e., one with at least one $x_i$ not zero.   Then
$$
\multline
 \bm \u_1 & \u_2 & \hdots & \u_n \em
\bm x_1\\x_2\\ \vdots \\ x_n \em
= \\ 
\bm \v_1 & \v_2 & \dots & \v_m \em
\bm p_{11} & p_{12} & \hdots & p_{1n} \\
    p_{21} & p_{22} & \hdots & p_{2n}\\
   \vdots & \vdots & \hdots & \vdots \\
    p_{m1} & p_{m2} & \hdots & p_{mn}\em
\bm x_1\\x_2\\ \vdots \\ x_n \em = 0.
\endmultline
$$
Thus,  $0$ has a non-trivial representation
$$
 0 = \u_1x_1 + \u_2x_2 + \dots + \u_nx_n
$$
which we know can never happen for a linearly independent set.
Thus, the only way out of this contradiction is to believe that
$n \le m$ as claimed.

One consequence of this argument is the following fact.  {\it The
dimension of a subspace cannot be larger than the dimension
of the whole vector space.}  The reasoning is that a basis
for a subspace is necessarily a linearly independent set
and so it cannot have more elements than the dimension of
the whole vector space.
  
It is important to note that two different bases of the same
vector space might have no elements whatsoever in common.
All we can be sure of is that they have the same size.




\medskip
\subhead Infinite Dimensional Vector Spaces \endsubhead
Not every vector space has a finite basis.   We shall not prove it
rigorously
here, but it is fairly clear that the vector space of
all continuous functions $\Cal C(I)$  cannot have a finite basis
$\{f_1, f_2 \dots, f_n\}$.   For, if it did, then that would mean
{\it any\/} continuous function $f$ on $I$ could be written
as a finite linear combination 
$$
 f(t) = c_1f_1(t) + c_2f_2(t) + \dots + c_n f_n(t),
$$
and it is not plausible that any  finite set of continuous functions
could capture the full range of possibilities of continuous functions
in this way.   

If a vector space has a finite basis, we say that it is {\it finite
dimensional\/}; otherwise we say it is {\it infinite dimensional}.

Most interesting function spaces 
are infinite dimensional.   Fortunately, the subspaces of these
spaces which are solutions of homogeneous linear differential
equations are finite dimensional, and these are what we shall
spend the next chapter studying.

We won't talk much about infinite dimensional vector spaces in
this course, but you will see them again in your course on
Fourier series and partial differential equations, and you will
also encounter such spaces when you study quantum mechanics.



\bigskip
\goodbreak
\input chap10.ex8
\bigskip
\nextsec{Calculations in $\noexpand\R^n$ or $\noexpand\C^n$}
\head \sn. Calculations in $\R^n$ or $\C^n$ \endhead

\outind{linear independence in $\noexpand\R^n$}%
Let  $\{\v_1, \v_2, \dots, \v_k\}$ be a collection of vectors in
$\R^n$  (or $\C^n$ in the case of complex scalars.)  It is
often useful to have a way to pick out a linearly independent
{\it subset\/} which spans the same subspace, i.e., which is
a basis for that subspace.   The basic idea (no pun intended)
is to throw away redundant vectors until that is no longer
possible, but there is a systematic  way to do this all at once.
  Since the vectors $\v_i$ are elements of
$\R^n$, each may be realized as a  $n\times 1$ column vector.
Put these vectors together to form an $n\times k$ matrix
$$
A = \bm \v_1 & \v_2 & \hdots & \v_k \em.
$$
(This is the same notation we used in the previous section,
but now since the $\v_i$ are column vectors, rather than elements
of some abstract vector space, we really do get a
matrix.)  To find a basis, apply Gaussian reduction to the matrix
$A$,  and pick out the columns
of  $A$ which in the transformed reduced matrix end up with pivots.

\nextex
\example{Example \en}
Let 
$$
\v_1 = \bm\format\r\\1\\0\\1\\1\em,\quad
\v_2 = \bm\format\r\\2\\2\\4\\0\em,\quad
\v_3 = \bm\format\r\\-1\\1\\0\\-2\em,\quad
\v_4 = \bm\format\r\\0\\1\\1\\0\em .
$$
Form the matrix $A$ with these columns and apply Gaussian reduction
$$\align
\bm \format\r&\quad\r&\quad\r&\quad\r\\
    1 & 2 & -1 & 0 \\
    0 & 2 & 1 & 1 \\
    1 & 4 & 0 & 1 \\
    1 & 0 & -2 & 0 \em
&\to
\bm \format\r&\quad\r&\quad\r&\quad\r\\
    1 & 2 & -1 & 1 \\
    0 & 2 & 1 & 1 \\
    0 & 2 & 1 & 1 \\
    0 & -2 & -1 & 0 \em \\
&\to
\bm \format\r&\quad\r&\quad\r&\quad\r\\
    1 & 2 & -1 & 1 \\
    0 & 2 & 1 & 1 \\
    0 & 0 & 0 & 0 \\
    0 & 0 & 0 & 1 \em\\
&\to
\bm \format\r&\quad\r&\quad\r&\quad\r\\
    1 & 2 & -1 & 1 \\
    0 & 2 & 1 & 1 \\
    0 & 0 & 0 & 1 \\
    0 & 0 & 0 & 0 \em.
\endalign$$
This completes the Gaussian reduction, and the pivots are in the
first, second, and fourth columns.  Hence, the vectors
$$
\v_1 = \bm\format\r\\1\\0\\1\\1\em,\quad
\v_2 = \bm\format\r\\2\\2\\4\\0\em,\quad
\v_4 = \bm\format\r\\0\\1\\1\\0\em 
$$
   form a basis for the subspace spanned by $\{\v_1, \v_2, \v_3, \v_4 \}$.
\endexample

Let's look more closely at this example to see why the subset 
is linearly independent and also spans the same subspace as the
original set.  The
proof that the algorithm works in the general case is more complicated
to write down but just elaborates the ideas exhibited in the example.
Consider the homogeneous system
$A\x = 0$.    This may also be written
$$
A\x = \bm \v_1 & \v_2 & \v_3 & \v_4 \em \bm x_1\\x_2\\x_3\\x_4 \em
   = \v_1x_1 + \v_2x_2 + \v_3x_3 + \v_4x_4 = 0.
$$
In the general solution, $x_1, x_2$, and $x_4$ will be bound variables
(from the pivot positions) and $x_3$ will be free.   That means we can
set $x_3$ equal to anything, say $x_3 = -1$ and the other variables will
be determined.   For this choice, the relation becomes
$$
  \v_1x_1 + \v_2x_2 - \v_3 + \v_4x_4 = 0
 $$
which may be rewritten
$$
\v_3 = x_1\v_1 + x_2\v_2 + x_4\v_4.
$$
Thus, $\v_3$ is redundant and may be eliminated from the set without
changing the subspace spanned by the set.  On the other hand, the
set  $\{\v_1, \v_2, \v_4 \}$  is linearly independent, since if
we were to apply Gaussian reduction to the matrix
$$
A' = \bm \v_1 & \v_2 & \v_4 \em
$$
the reduced matrix would have a pivot in every column, i.e., it
would have rank 3.  Thus, the system
$$
 \bm \v_1 & \v_2 & \v_4 \em \bm x_1 \\ x_2 \\ x_4 \em =
\v_1x_1 + \v_2x_2 + \v_4x_4 = 0
$$
has only the trivial solution.    That means that no one of the
three vectors can be expressed as a linear combination of the other
two.  For example, if $\v_2 = c_1\v_1 + c_4\v_4$, we have
$$
   \v_1c_1 + \v_2(-1) + \v_4c_4 = 0.
$$
It follows that the set is linearly independent.
\medskip
\subhead Column Space and Row Space \endsubhead
Let $A$ be an $m\times n$ matrix.   Then the columns
$\v_1, \v_2, \dots, \v_n$ of
$A$ are vectors in $\R^m$ (or $\C^m$ in the complex case),
and $\{\v_1, \v_2,\dots, \v_n\}$
 spans a subspace of $\R^m$ called the {\it column
space of\/}  $A$.   The column space plays a role in the
\outind{column space of a matrix}%
the theory of 
inhomogeneous systems  $A\x = \b$ in the following
way.  A vector $\b$ is in the column space if and only if it
is expressible as a linear combination
$$
\b = \v_1x_1 + \v_2x_2 + \dots +\v_nx_n
   = \bm \v_1 & \v_2 & \dots & \v_n\em
\bm x_1\\x_2\\ \vdots \\ x_n \em
   = A\x. 
$$
Thus, {\it the column space of $A$ consists of all vectors $\b$
in $\R^m$ for
which the system $A\x = \b$ has a solution}.

Note that the method outlined in the beginning of this section
gives a basis for the column space, and the number of elements
in this basis is the rank of $A$.  (The rank is the number of
pivots!)  Hence, {\it the rank of an $m\times n$ matrix $A$ is
the dimension of its column space}.

There is a similar concept for rows;  the row space of an
$m\times n$ matrix $A$ is the subspace of $\R^n$ spanned by
the rows of $A$.   It is not hard to see that {\it the dimension
of the row space of $A$ is also the rank of $A$}.  For,
since each row operation is reversible, applying a row operation
does not change the subspace spanned by the rows.  Hence, the
row space of the matrix $A'$ obtained by Gauss-Jordan reduction
from $A$ is the same as the row space of $A$.   However, the
 set of non-zero rows of the reduced matrix is a basis for this subspace.
To see this, note first that it certainly spans (since leaving
out zero rows doesn't cost us anything).   Moreover, it is also
a linearly independent set because each non-zero row has a 1 in
a pivot position where all the other rows are zero.
\outind{row space of a matrix}%

The fact that both the column space and the row space have the
same dimension is sometimes expressed by saying
``the column rank equals the row rank''.

The column space also has an abstract interpretation.
Consider the linear operator $L:\R^n \to \R^m$ defined by
$L(\x) = A\x$.    The {\it image\/} of this operator is defined
to be the set of all vectors $\b$ in $\R^m$ of the form
$\b = L(\x)$ for some $\x$ in $\R^n$.  Thus, the image of $L$
is just the set of $\b = A\x$, which by the above reasoning is
the column space of $A$, and its dimension is the rank $r$.   
On the other hand, you should recall that the dimension of
the null space of $L$ is the number of basic solutions, i.e.,
$n - r$.    Since these add up to $n$, we have
$$
\dim \text{Image of }L + \dim \text{Null Space of }L = \dim
\text{Domain of }L.
$$

\medskip
\subhead A Note on the Definition of Rank \endsubhead
The rank of $A$ is defined as the number of pivots
in the reduced matrix obtained from $A$ by
an appropriate sequence of elementary row operations.   Since we
can specify a standard procedure for performing such row operations,
that means the rank is a well defined number.   On the other hand,
it is natural to wonder what might happen if $A$ were reduced by
an alternative, perhaps less systematic, sequence of row operations.
The above analysis shows that we would still get the same answer
for the rank.  Namely, the rank is the dimension of the column
space  of $A$, and that number depends only on
 the column space itself, not on any particular basis for it.
(Or you could use the same argument using the row space.)

The rank is also the number of non-zero rows in the
reduced  matrix, so it
follows that this number does not depend on the particular sequence
of row operations used to reduce $A$  to Gauss-Jordan reduced form. 
In fact, the entire matrix obtained  at the end (as
long as it is in Gauss-Jordan reduced form)  depends
only on the original matrix $A$ and not on the particular sequence
or row operations used to obtain it.
The proof of this fact is not so easy, and we omit it here.
\bigskip
\goodbreak
\input chap10.ex9
\endchapter
\closeseg{chap10}
\enddocument
