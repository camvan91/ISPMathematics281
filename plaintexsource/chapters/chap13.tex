\documentstyle{book}
\input extra.tex
\input le2.sty
\input epsf.tex
\input chap12.lnk
\maketoctrue
\indextrue
\mardiagtrue
\openseg{chap13}
\Monograph
\topmatter
\nextchap{Nonlinear Systems}
\title
\chapter{\chapno}Nonlinear Systems
\endtitle
\endtopmatter
\document
\nextsec{Introduction}
\head \sn. Introduction \endhead

So far we have concentrated almost entirely on linear differential
equations.   This is appropriate for several reasons.  First, most of
\outind{non-linear system}%
\outind{system, non-linear}%
classical physics is described by linear equations, so knowing how
to solve them is fundamental in applying the laws of physics.
Second, even in situations where nonlinear equations are necessary
to describe phenomena, it is often possible to begin to understand
the solutions by making linear approximations.  Finally, linear
equations are usually much easier to study than nonlinear
equations.  Be that as it may, nonlinear equations and nonlinear
systems have become increasingly important in applications, and
at the same time a lot of progress has been made in understanding
their solutions.  In this chapter we shall give a very brief
introduction to the subject.

We start with a couple of typical examples.

\nextex
\example{Example \en}  In your physics class you studied the
behavior of an undamped pendulum. 
\outind{pendulum, exact equation}%
  To make our analysis easier, we shall assume that the
pendulum consists of a point mass $m$
 connected to a fixed pivot by a massless rigid rod
of length $L$.

\mar{s13-1.ps}
\noindent
  Then, using polar coordinates to describe the
motion, we have for the tangential acceleration
$$
a_\theta = r\frac{d^2\theta}{dt^2} + 2\frac{dr}{dt}\frac{d\theta}{dt}.
$$
(Refer to Chapter I, Section 2 for acceleration in polar coordinates.)
Since $r = L, dr/dt = 0$, we obtain
$$
a_\theta = L\theta''.
$$
On the other hand, the component of acceleration in the tangential
direction is $-g\sin\theta$, so we obtain the second order differential
equation
\nexteqn
$$
\theta'' = -\frac gL\sin\theta.\tag\eqn
$$
This is a second order nonlinear equation.   It is usually solved
as follows.   {\it Assume $\theta$ is small.}   Then
$$
\sin\theta = \theta + O(\theta^3)
$$
so (\eqn) may be approximated by the second order linear equation
$$
\theta'' = -\frac gL \theta.
$$
It is easy to solve this equation:
$$
\theta = A\cos \left(\sqrt{\frac gL}\,t + \delta \right).
$$
However, this approximation is certainly not valid if $\theta$
is large.   For example, if you give the mass a big enough shove,
it will revolve about the pivot through a complete circuit and in
the absence of friction will continue to do that indefinitely.
We clearly need another approach if we want to understand what happens
in general.

Unfortunately equation (\eqn) can't be solved explicitly in terms of
known functions.  However, it is possible to get a very good qualitative
understanding of its solutions.   To explore this, we consider
the equivalent first order system.   Let $x_1 = \theta$ and
$x_2 = \theta'$.   Then $x_2' = \theta'' = -(g/L) \sin \theta = 
-(g/L)\sin x_1$.  Hence, the desired system is
\nexteqn
\xdef\VecEq{\eqn}
$$\align
x_1' &= x_2 \\
x_2' &= -\frac gL \sin x_1\\
\intertext{or, in vector form}
\x' &= \f(\x) = \bm x_2\\ -\frac gL \sin x_1 \em.\tag\eqn
\endalign$$

As noted earlier, we can't find $\x = \x(t)$ explicitly as a function
of $t$, but it is possible to learn quite a lot about the
the solutions by looking at the geometry of the solution curves.  In this
case, we can describe the geometry by eliminating
$t$ from the differential equations.   We have
$$
\frac{dx_2}{dx_1} = \frac {dx_2/dt}{dx_1/dt} = -\frac gL \frac{\sin x_1}{x_2}.
$$
This equation may be solved by separation of variables.   I leave
the details to you, but the general solution is
\nexteqn
$$
Lx_2{}^2 - 2g\cos x_1 = C.\tag\eqn
$$
This equation may also be derived quite easily
from
the law of {\it conservation of energy}.  Namely, multiplying
by
\outind{conservation of energy}%
\outind{energy, conservation of}%
$mL/2$ and expressing everything in terms of $\theta$ yields
$\dfrac 12 m (L\theta')^2 - mgL\cos \theta = C$.
The first term on the
left represents the kinetic energy and the second the
potential energy.

(\eqn) gives a family of curves called the {\it orbits}
\outind{orbit}%
of the system.   (The term `orbit' arises from celestial mechanics
which has provided much of the motivation for the study of nonlinear
systems.)  You can sketch these orbits by hand, but a computer
program will make it quite a bit easier.
Such a diagram is called a  {\it phase portrait\/} of
the system.  (The term `phase' is used because the $x_1, x_2$-plane
is called the {\it phase plane\/} for historical reasons.)
\outind{phase plane}%
\outind{phase portrait}%
Note that (\eqn) exhibits the orbits as the level curves of a function
but it does not tell you the directions that solutions `flow' along
those orbits.   
It is  easy to
determine these   directions by examining the vector $\dfrac{d\x}{dt}$
at typical points.   For example, if $0< x_1 < \pi$ and $x_2 > 0$,
then from (\VecEq), we see that $\dfrac{dx_1}{dt} = x_2 > 0$
and $\dfrac{dx_2}{dt} = -\dfrac gL \sin x_1 < 0$.   It follows that
the solutions move along the orbits downward and to the right in
that region.
\medskip
\centerline{\epsfbox{s13-2.ps}}
\medskip
Examination  of the phase portrait exhibits some interesting
phenomena.  
First look at the vaguely elliptical orbits which circle the origin.
These represent periodic solutions in which the pendulum swings back
and forth.   (To see that, it suffices
to follow what happens to $\theta = x_1$ as the solution moves
around an orbit 
in the phase plane.) 
\medskip
\centerline{\epsfbox{s13-3.ps}}
\medskip
  Next look at
the origin.   This represents the constant solution
$x_1 = 0, x_2 = 0$,  in which the pendulum does not move
at all  ($\x'(t) = 0$ for all $t$). 
This may be obtained from (\eqn) by taking
$C = -2g$.  There are two other constant solutions represented
in the phase portrait: for $C = 2g$ we obtain $x_1 = \pi, x_2 = 0$
or $x_1 = -\pi, x_2 = 0$.   These represent the same physical
situation; the pendulum is precariously balanced on top of the rod
($\theta = \pi$ or $\theta = -\pi$).   This physical situation is an
example of an {\it unstable equilibrium}.   Given a slight change in
its position $\theta$ or velocity $\theta'$, the pendulum
 will move off
the balance point, and the corresponding solution
 will eventually move quite far from the equilibrium point.
On the other hand,
the constant solution $x_1 = x_2 = 0$ represents a
{\it stable\/} equilibrium.
\outind{equilibrium, stable}%
\medskip
\centerline{\epsfbox{s13-4.ps}}
\medskip

Consider the two orbits which appear to connect the two
unstable equilibrium points.  {\it Neither of the equilibrium
points is actually part of either orbit.}   For, once the pendulum
is in equilibrium it will stay there forever, and if it is not
in equilibrium, it won't ever get there.  You should think this
out carefully for yourself and try to understand what actually
happens to the pendulum for each of these solutions.

The remaining orbits represent motions in
which the pendulum swings around the pivot in circuits which
repeat indefinitely.  
 (The orbits don't appear to repeat in the phase plane, but you
should remember that
values of $x_1 = \theta$ differing by $2\pi$
 represent the same physical configuration
of the pendulum.)


   It should be noted in passing that different solutions of
the system of differential equations can produce the same
orbit.  Indeed, we can start a solution off at $t = t_0$ at
any point $\x_0$ on an orbit, and the resulting solution will trace out
that orbit.
Solutions obtained this way
are the same except for a shift in the time scale.   
 
\endexample

\nextex
\example{Example \en} In the study of ecological systems, one
is often interested in the dynamics of populations.   Earlier
\outind{prey--predator problem}%
\outind{population}%
in this course we considered the growth or  decline of a single
population.  Consider now two populations $x$ and
$y$  where the size of each depends on the other.  For example,
$x$ might represent the number of caribou present in a given
geographical region and $y$ might represent the number of wolves which 
prey on the caribou.   This is a so-called
{\it prey--predator\/} problem.   The mathematical model for
such an interaction is often expressed as a system of
differential equations of the form
\nexteqn
$$\align
\frac{dx}{dt} &= px  -qxy \\
\frac{dy}{dt} &= -ry + sxy
\tag\eqn
\endalign $$
where $p, q, r, s$ are {\it positive\/} constants.   The justification
for such a model is as follows.  In absence of predators, the prey
will follow a Malthusian law  $dx/dt = px$ where $p$ is the birthrate
of the prey.  However, in the presence of predators, there will be
an additional term limiting the rate of growth of 
$x$ which depends on the likelihood of an encounter between prey
and predator.   This likelihood is assumed to be proportional to
the product $xy$ of the two population sizes.   Similarly,
without prey, it is assumed that the population of predators
will decline according to the Malthusian law $dy/dt = -ry$, but
then the term $sxy$ is added to account for the rate of population
growth for the predators which can be supported from the existing prey.
Note that this model is derived from rather simple minded considerations.
Even so, the predictions of such a model may correspond quite well
with observations.   However, one should bear in mind that there
may be other models which work just as well. 

As in the previous example, the system (\eqn) can't be solved
explicitly at a function of $t$, but we can get a pretty good
description of its phase portrait.
$$
\frac{dy}{dx} = \frac{dy/dt}{dx/dt} = \frac{-ry + sxy}{px - qxy}
$$
yields
$$\gather
(ry - sxy)dx + (px - qxy)dy = 0\\
(r - sx)\frac{dx}x + (p - qy)\frac{dy}y = 0 \\
r\ln x - sx + p\ln y - qy = C \\
\ln x^r y^p - (sx + qy) = C.
\endgather $$

 These curves may be graphed (after choosing
plausible values of the constants $p, q, r, s$).   

\medskip
\centerline{\epsfbox{s13-6.ps}}
\medskip

We only included orbits in the first quadrant since those are
the only ones that have significance for population problems.

Note that there  are two constant solutions.  First, 
$x = y = 0$ is certainly an equilibrium point.   
  The other constant solution
may be determined from (\eqn) by setting
$$\align
\frac{dx}{dt} &= px  -qxy = 0\\
\frac{dy}{dt} &= -ry + sxy = 0.
\endalign$$
For, anything obtained  this way will certainly be constant
and also a solution of (\eqn).   In this case, we get
$$\align
x(p - qy) &=0 \\
y(r - sx) &= 0.
\endalign$$
From the first equation $x = 0$ or $y = p/q$.  From the second
$y = 0$ or $x = r/s$.   However, $x = 0$ is not consistent with
$x = r/s$ and similarly $y = 0$ is not consistent with $y = p/q$,
so 
we obtain a second constant solution
$x = r/s, y = p/q$.   It represents an equilibrium
in which both populations stay fixed at non-zero values.    

The positive $x$ axis is an orbit, and it corresponds
to the situation where there are no predators ($x = Ce^{pt}, y = 0$). 
Note that this orbit does not contain
the origin, but  $\lim_{t \to -\infty} x(t) = 0$.
Similarly, the positive $y$ axis represents the situation
with no prey ($x = 0, y = Ce^{-rt}$).   This shows that the
origin represents an unstable equilibrium.  On the other hand, the
point $(r/s, p/q)$ represents a stable equilibrium.  (Can you see
why?)

The remaining orbits correspond to solutions in which each population
varies periodically.   However, the exact relation between the
times of maximum population for prey and predator may be quite
subtle.

\medskip
\centerline{\epsfbox{s13-7.ps}}
\medskip

\endexample

The general nonlinear first order system has the form
$$
\frac{d\x}{dt} = \f(\x, t)
$$
where $\f(\x,t)$ is vector valued function taking values in
$\R^n$ and $\x = \x(t)$ is a ($n$-dimensional)
 vector valued solution.  It is often the case, as in
both examples, that $\f$ doesn't depend explicitly on $t$.
Such systems are called {\it time independent\/} or {\it autonomous}.
\outind{autonomous system}%
\outind{system, autonomous}%
 The {\it critical
points\/} of a system are those points $\a$ satisfying
\outind{critical point of a system}%
\outind{system, critical point of}%
$\f(\a, t) = 0$ for all $t$, or, in the autonomous case,
just $\f(\a) = 0$.  Each critical point corresponds to
a constant solution, $\x(t) = \a$ for all $t$.   The
behavior of solutions near a critical point in the phase portrait
of the system tell us something about the stability of the
equilibrium represented by the critical point.   In the
examples we noted at least two different types of behavior
near a critical point.  
\medskip
\centerline{\epsfbox{s13-8.ps}}
\medskip
In the above case, called a {\it saddle point\/},
The above situation occurred in the pendulum example
at the unstable equilibria.   Some solutions approach the critical
point and then depart, some solutions approach the critical point
asymptotically as $t \to \infty$, some solutions do the reverse
for $t \to -\infty$.
Such a critical point is called a {\it saddle point\/},
\outind{saddle point in phase portrait}%
\outind{critical point, saddle}%
\medskip
\centerline{\epsfbox{s13-9.ps}}
\medskip
The above situation occurred in both examples.  Nearby solutions
repeat periodically.  Such a critical point is called a
{\it center}.
\outind{center in phase portrait}%
\outind{critical point, center}%

There are many other possibilities.  

\nextex
\example{Example \en}
Consider a
\outind{pendulum, damped, exact equation}%
\outind{damped pendulum, exact equation}%
damped rigid pendulum with
damping dependent on velocity.   Newton's second law results in
a differential equation of
the form
$$
\theta'' =  -\frac gL \sin\theta  - a\theta'
$$
where $a > 0$.   It is fairly clear how the damping will affect
the behavior of such a pendulum.   There will still be unstable
equilibria in the phase plane which correspond to the pendulum
precariously balanced on end ($\x_1 = \theta$ equal to an odd multiple
of $\pi$ and $x_2 = \theta' = 0$).   There will be stable equilibria in the phase plane
which correspond to the pendulum hanging downward at rest  ($x_1 = \theta$
equal to an even multiple of $\pi$ and $x_2 = \theta' = 0$).   Near the stable
equilibrium, the pendulum will oscillate with decreasing amplitude
(and velocity), so the corresponding orbits in the phase plane will
spiral inward toward the equilibrium point and approach it asymptotically. 
Here is what the phase portrait looks like in general.
\medskip
\centerline{\epsfbox{s13-10.ps}}
\medskip

The above physical reasoning is convincing, but it is also
useful
to have a more mathematical approach.
To this end. we convert the
second order equation for $\theta$ to the system
\nexteqn
$$\align
x_1' &= x_2 \\
x_2' &= -\frac gL \sin x_1 - ax_2.
\tag\eqn
\endalign$$
The critical points are $(n\pi, 0)$ where $n = 0, \pm 1,
\pm 2, \dots$.
 The method we used for sketching the phase portrait of
the undamped pendulum
doesn't work in this case, but we can use what we learned previously
to guide us to an understanding of the
behavior near the critical point $(0,0)$ as follows.
Let
$$
U = \frac 12 x_2{}^2 -\frac gL\cos x_1.
$$
This quantity was {\it constant\/} in the undamped case by
the law of conservation of energy.  Since some energy is lost
because of the damping term, $U$ is no longer constant.
Indeed,   we have
$$\align
\frac {dU}{dt} &= \frac 12 2x_2x_2' + \frac gL\sin x_1\, x_1'\\
     &= x_2(-\frac gL \sin x_1 - ax_2) + \frac gL \sin x_1\, x_2 \\
     &= -ax_2{}^2.
\endalign$$
Thus, $dU/dt < 0$ except where the path crosses the $x_1$-axis
(i.e., $x_2 = 0$).   However, at points on the $x_1$-axis, the
velocity $d\x/dt$ (given by (\eqn)) is directed off the axis. 
Using this information, it is possible to see that $U$
decreases steadily
along any orbit
of the damped system
as the orbit
descends inward crossing the level curves
$$
L x_2{}^2 - 2g\cos x_1 = 2LU = C
$$
of the undamped system.
In the limit, the orbit
approaches the critical point at the origin.
This confirms
the physical interpretation described above
near the critical point  $x_1 = \theta = 0, \, x_2 = \theta' = 0$.
Here is what the phase portrait looks like in general.
\medskip
\centerline{\epsfbox{s13-10a.ps}}
\medskip

As noted above, the orbits around the critical point at the origin
spiral in toward the critical point and approach it asymptotically.
In general, such a critical point is called a {\it focus\/}.  In this case,
\outind{focus in phase portrait}%
\outind{critical point, focus}%
the critical point represents a stable equilibrium, but, in general,
 the reverse situation where  the orbits
spiral out from the origin is possible.  Then the critical point
represents an unstable equilibrium, but is still called
a focus.

\endexample 

Generally, we may summarize the phenomena illustrated
above by calling a critical point {\it stable\/}
if any solution which comes sufficiently close to the critical point
\outind{critical point, stability of}%
\outind{stability of critical point}%
for one time $t$ stays close to it for all subsequent time.
Otherwise, we shall call the critical point {\it unstable}.
If {\it all nearby\/} solutions  approach the critical point asymptotically
 as
$t \to \infty$, it is called {\it asymptotically\/} stable.
\outind{asymptotic stability for a critical point}%
\outind{critical point, asymptotic stability of}%

As we shall see, the behavior of the phase portrait
near a critical point can often be
reduced to a problem in linear algebra.
\bigskip
\input chap13.ex1
\bigskip

\nextsec{Linear Approximation}
\head \sn. Linear Approximation Again \endhead

Suppose we want to study the behavior of a system
$$
\frac{d\x}{dt} = \f(\x)
$$
near a critical point $\x = \a$.  (We assume the system is
time independent for simplicity.)  One way to do this is to
approximate $\f(\x)$ by a {\it linear\/} function in the
vicinity of the point $\a$.   In order to do this, we need
a short digression about multidimensional calculus.

\subhead The Derivative of a Function $f:\R^n \to \R^m$ \endsubhead
Let $\f:\R^n \to \R^m$ denote a smooth function.  Since $\f$
is an $m$-dimensional vector valued function, it may be
specified by 
$m$ component functions
$$
\f = \bm f_1\\ f_2 \\ \vdots \\ f_m \em
$$
where each component is a scalar function
$f_i(x_1,x_2,\dots, x_n)$ of $n$ variables.  Fix a point
$\a = (a_1, a_2, \dots, a_n)$ in the domain of $\f$.
   For each component,
we have the linear approximation
$$
f_i(\x) = f_i(\a) + \nabla f_i(\a)\cdot (\x - \a) + o(|\x - \a|). 
$$
(See Chapter III, Section 4.)  We may put these together in a
single vector equation
$$
\f(\x) = \bm f_1(\x)\\ f_2(\x)\\ \vdots \\ v_m(\x) \em
  = \bm f_1(\a)\\ f_2(\a)\\ \vdots \\ v_m(\a) \em
 + \bm \nabla f_1(\a)\\ \nabla f_2(\a) \\ \vdots \\ \nabla f_m(\a)\em
(\x - \a) + o(|\x - \a|).
$$
Let
$$
D\f = \bm \nabla f_1 \\ \nabla f_2 \\ \vdots \\ \nabla f_m \em
$$
be the $m\times n$ matrix with rows the gradients of the component
functions $f_i$.    The $i,j$ entry of $D\f$ is
$\dfrac{\d f_i}{\d x_j}$.    Then the above equation may be
written more compactly
\nexteqn
$$
\f(\x) = \f(\a) + D\f(\a)(\x - \a) + o(|\x - \a|).\tag\eqn
$$
\outind{linear approximation}%
The matrix $D\f$ is called the {\it derivative\/} of the function
$\f$.  It plays the same role for vector valued functions that the
\outind{derivative of a vector function}%
\outind{$D\noexpand\f$}%
gradient plays for scalar valued functions.

\nextex
\example{Example \en}
Let $m = n = 2$ and suppose
$$
\f(x_1, x_2) = \bm x_2\\ -\sin x_1 \em.
$$
 Let's consider the behavior of this function near  $\a = (0,0)$.

First calculate the derivative.
$$\align
\nabla f_1 &= \bm 0 & 1 \em \\
\nabla f_2 &= \bm -\cos x_1 & 0 \em\\
\intertext{so}
D\f(0,0) &= \bm 0 & 1 \\ -1 & 0 \em.
\endalign $$
Hence, formula (\eqn) yields
$$\align
\f(x_1, x_2) &= \f(0,0) + D\f(0,0)\x + o(|x|)\\
&\approx  \bm 0\\ 0 \em + \bm 0 & 1\\ -1 & 0 \em \bm x_1\\ x_2 \em 
  =\bm x_2 \\ -x_1 \em.
\endalign$$
\endexample

\nextex
\example{Example \en}
Let $m =2, n = 3$ and take
$$
\f(x_1, x_2, x_3) = \bm x_1 + 2x_2 + x_3 \\ x_1{}^2 + x_2{}^2 + x_3{}^2 \em.
$$
Then
$$
D\f(x_1, x_2, x_3) =
\bm 1 & 2 & 1 \\ 2x_1 & 2x_2 & 2x_3 \em.
$$
Suppose we want to study the behavior of $\f$ near the point
$(1,2,-4)$.  We have
$$
f(1, 2, -4) = \bm\format\r\\ 1\\ 21 \em
$$
and
$$
D\f(1, 2, -4) = \bm\format\r&\quad\r&\quad\r\\ 1 & 2 & 1 \\ 2 & 4 & -8 \em 
$$
 so, according to (\eqn),
$$
\f(x_1, x_2, x_3) \approx \bm\format\r\\ 1\\ 21 \em + 
\bm\format\r&\quad\r&\quad\r\\ 1 & 2 & 1 \\ 2 & 4 & -8 \em
\bm x_1 - 1 \\ x_2 - 2 \\ x_3 + 4 \em.
$$
\endexample

The linear approximation is an invaluable tool for the study
of functions $\R^n \to \R^m$.   We have already seen its
use in a variety of circumstances for scalar valued functions.
It also arose implicitly when we were studying change of
variables for multiple integrals.  For example, a change
of variables in $\R^2$ may be described by a function
$\g:\R^2 \to \R^2$ and the Jacobian determinant used in
the correction factor for
double integrals is just   
$$
\det D\g = \det \bm \dfrac{\d g_1}{\d x_1} & \dfrac{\d g_1}{\d x_2} \\
                         \dfrac{\d g_2}{\d x_1} & \dfrac{\d g_2}{\d x_2}\em.
$$
A similar remark applies in $\R^3$, and indeed the change of
variable rule may be generalized to integrals in $\R^n$ by
using the correction factor  $|\det D\g|$ where $\g:\R^n \to \R^n$
\outind{Jacobian}%
is the function giving the change of variables.

\subhead Analysis of Non-Linear Systems near Critical Points
\endsubhead
We want to apply the above ideas to the analysis of an autonomous
nonlinear system
\nexteqn
\xdef\NonLin{\eqn}
$$
\frac{d\x}{dt}  = \f(\x)\tag\eqn
$$
near a critical point $\a$.  Here, $m = n$ and $\f:\R^n \to \R^n$.
  By definition, $\f(\a) = 0$
at a critical point, so the linear approximation gives
$$
\f(\x) = \f(\a) + D\f(\a)(\x - \a) + o(|\x - \a|)
       =  D\f(\a)(\x - \a) + o(|\x - \a|).
$$
If we put this in (\eqn) and drop the error term, we obtain 
$$
\frac{d\x}{dt} = D\f(\a)(\x -\a).
$$
This may be simplified further by the change of variables
$\y = \x - \a$ which in essence moves the critical point to
the origin.  Since $\dfrac{d\y}{dt} = \dfrac{d\x}{dt}$,
this yields
\nexteqn
\xdef\Lin{\eqn}
$$
\frac{d\y}{dt} = A\y\tag\eqn
$$
where $A = D\f(\a)$ is the derivative matrix at the critical point.
In this way, we have replaced the nonlinear system (\NonLin)
by the {\it linear system\/} (\Lin), at least near the critical
point.   If dropping the error term $o(|\x - \a|)$
 doesn't affect things
too severely, we expect the behavior of solutions of the linear system
to give a pretty good idea of what happens to  solutions of the nonlinear
system near the critical point.
At least in principle, we know how to solve
 linear systems.

\nextex
\example{Example \en}
Consider the pendulum system
$$\align
\frac{dx_1}{dt} &= x_2 \\
\frac{dx_2}{dt} &= -\sin x_1
\endalign$$
where the units have been chosen so $g = L$.   Consider the critical point
$(\pi, 0)$ corresponding to the unstable equilibrium discussed earlier.
We have
$$
D\f = \bm 0 & 1 \\ -\cos x_1 & 0 \em = \bm 0 & 1 \\ 1 & 0 \em
\qquad\text{at } (\pi, 0).
$$
Hence, putting $y_1 = x_1 - \pi, y_2 = x_2$, 
near the critical point, the system is approximated by the linear system
$$
\frac{d\y}{dt} = \bm 0 & 1\\1 & 0 \em \y.
$$
(In the `$y$' coordinates, the critical point is at the origin.)
This system is quite easy to solve.  I leave the details to you.   The eigenvalues
are $\lambda = 1, \lambda = -1$.   A basis of eigenvectors is given by
$$\align
\v_1 &= \bm -1\\ 1\em\qquad\text{corresponding to } \lambda = 1\\
\v_2 &= \bm\format\r\\1 \\ 1 \em \qquad\text{corresponding to } \lambda = -1.
\endalign$$
The general solution of the linear system is
\nexteqn
$$
\y = c_1e^t\v_1 + c_2e^{-t}\v_2\tag\eqn
$$
or
$$\align
y_1 &= -c_1e^t + c_2e^{-t}\\
y_2 &= c_1e^t  + c_2e^{-t}
\endalign$$
The phase portrait for the solution near $(0,0)$
can be worked out by trying various $c_1$ and $c_2$ and sketching the
resulting orbits.   However, it is easier to see what it looks like
if we put $P = \bm \v_1 &\v_2 \em$ and make the change of coordinates
$\y = P\z$.    As in Chapter XII, Section  6, this will have
the effect of diagonalizing the coefficient matrix and yielding the
system
$$\align
\frac{dz_1}{dt} &= z_1 \\
\frac{dz_2}{dt} &= -z_2
\endalign$$
with solutions
\nexteqn
$$
z_1 = c_1 e^t\qquad z_2 = c_2e^{-t}.\tag\eqn
$$
From this, it is clear that the orbits are the family of hyperbolas
given by
$$
z_1z_2 = c_1c_2 = c.
$$
This includes the case $c = 0$ which
gives the degenerate `hyperbola' consisting of the  lines $z_1 = 0$
and $z_2 = 0$.   (The degenerate case actually consists of  five orbits:
 the critical
point $(0,0)$ and the positive and negative half lines on each axis.)
\medskip
\centerline{\epsfbox{s13-11.ps}}
\medskip
The $z_1$ and $z_2$ axes in this diagram are directed 
respectively along the vectors $\v_1$ and $\v_2$.
The direction in which each orbit is traversed
may be determined from the 
explicit solutions (\eqn).

You should compare the above picture 
with the phase portrait derived in Section 1 for the pendulum
problem.    
In this example, the local analysis merely confirms
what we already knew from the phase portrait that was derived in
Section 1. 
 However, in general,  sketching the
phase portrait of the nonlinear system may be very difficult
or even impossible.   Hence, deriving
a picture near each critical point by  linear analysis 
is a useful first step in understanding the whole picture.
\endexample

The previous example illustrates generally what happens for a
$2$ dimensional system when the eigenvalues  are {\it real\/}
and {\it of opposite signs}.   In this case the critical
point is called a {\it saddle point\/}, and it is 
clearly unstable.     

\nextex
\example{Example \en}
Consider the prey-predator problem described by the system
$$\align
\frac{dx_1}{dt} &= x_1 - x_1x_2\\
\frac{dx_2}{dt} &= -x_2 + x_1x_2.
\endalign$$
This assumes that the populations are being measured in bizarre
units, and with respect to these units the constants $p, q, r,s$
are all 1.   This is  rather unrealistic,
but it does simplify the algebra quite a lot.     As before,
there are two critical points in the first quadrant:
$(0,0)$ and
$(r/s, p/q) = (1, 1)$.   Let's study the linear approximation
near $(1,1)$.  We have
$$
D\f = \bm 1 -x_2 & -x_1\\ x_2 & -1 + x_1 \em
 = \bm\format\r&\quad\r\\ 0 & -1\\ 1 & 0 \em\qquad\text{at } (1,1).
$$
Hence, the approximating linear system is
$$
\frac{d\y}{dt} = 
\bm\format\r&\quad\r\\ 0 & -1\\ 1 & 0 \em \y
$$
where $y_1 = x_1 - 1, y_2 = x_2 - 1$.
I leave it to you to work out the solutions of this system.
The eigenvalues are $\lambda = \pm i$, so we need to use
complex solutions.   An eigenvector for $\lambda = i$ is
$$
\u = \bm i\\ 1\em
$$
and a corresponding complex solution is
$$
e^{it}\u.
$$
As usual, we can find a linearly independent pair of real solutions
by taking real and imaginary parts.
To help you work similar problems in homework, we shall do this in
slightly greater generality than necessary in the particular case.
$$
\u = \v + i\w\qquad\text{where } \v = \bm 0\\ 1\em\quad\text{and}\quad
\w =\bm 1 \\ 0 \em. 
$$
Then
$$
e^{it}\u = (\cos t + i\sin t)(\v + i\w) =
     \v \cos t - \w \sin t + i(\v \sin t + \w \cos t).
$$
Taking real and imaginary parts yields the two solutions
$$\align
\v\cos t - \w \sin t &= \bm \v & \w \em \bm\format\r\\ \cos t\\ -\sin t\em \\
\v\sin t + \w \cos t &= \bm \v & \w \em \bm \sin t \\ \cos t \em.
\endalign$$
This suggests that we put  $P = \bm \v & \w \em$ (so that its columns
are the real and imaginary parts of the eigenvector $\u$), and
make the change of variables $\y = P\z$.   Then, in the `$z$'
coordinate system, we obtain two linearly independent solutions
$$
 \bm\format\r\\ \cos t \\ -\sin t \em
\qquad\text{and}\qquad  \bm \sin t\\ \cos t \em.
$$
Any solution is then a linear combination
$$
\z = c_1 \bm\format\r\\ \cos t \\ -\sin t \em + 
c_2 \bm \sin t\\ \cos t \em
= \bm c_1\cos t + c_2\sin t \\ -c_1\sin t + c_2 \cos t \em.
$$
Now put $c_1 = A\cos\delta, c_2 = A\sin\delta$.   The above solution
takes the form
$$
\z = \bm\format\r\\ A\cos(t - \delta)\\-A\sin(t - \delta) \em.
$$
This gives a family of {\it circles\/} which are traversed
clockwise with respect to the $z_1$ and $z_2$ axes.  
However, the $z_1$ and $z_2$ axes have orientation opposite to
that of the original axes.  (Look at  
$\v = \e_2$ and $\w = \e_1$ which are `unit' vectors along
the new axes.)  Thus, 
  with respect to the $y_1, y_2$ axes, the motion is
counter-clockwise.
\medskip
\centerline{\epsfbox{s13-12.ps}}
\medskip
Note that the phase portrait we derived in Section 1 for the prey-predator
system gives a similar picture near the critical point  $(\pi,0)$.
\endexample

A similar analysis applies to any
 2-dimensional  real system if the eigenvalues are complex.  If
the eigenvalues are purely imaginary, i.e., of the form $\pm i\omega$,
then  the results are similar to what we got in Example \en.
In the
`$z$' coordinate system, the orbits look like
circles and they are traversed clockwise.  However, the change 
from the `$y$' coordinates to the `$z$' coordinates may
introduce  changes of scale, different for the two axes.
Hence, the orbits are really
ellipses when viewed in the original coordinate system.  Also,
as in Example \en, the change of coordinates may introduce a reversal
of orientation.  A critical point of this type is called
a {\it center}.
\outind{critical point, center}%

If the eigenvalues are not purely imaginary, then they are of
the form $a \pm bi$ with $a\not=0$, and  all solutions
have the
additional factor $e^{at}$. 
   This has the
effect of turning the `ellipses' into spirals.   If
$a < 0$, $e^{at} \to 0$
as $t\to \infty$, so all solutions spiral in towards the origin.  If
$a > 0$, they all spiral out. In either case, the critical point is called
a {\it focus}.  If $a < 0$ the critical point is stable, and if
\outind{critical point, focus}%
$a > 0$ it is not.
\medskip
\centerline{\epsfbox{s13-13.ps}}
\medskip
The above examples by no means exhaust all the possibilities
even in the 2-dimensional case.  In the remaining cases,
the eigenvalues $\lambda_1, \lambda_2$ are real and of the same sign. 
 The exact nature of the phase portrait
will depend on the how the linear algebra works out,
but in all these cases the critical point is called a {\it node}.
\outind{node}%
\outind{critical point, node}%
Here are some pictures of nodes:
\medskip
\centerline{\epsfbox{s13-14.ps}}
\medskip
%Nodes with \hfill $\lambda_1 > \lambda_2 > 0$  or
%$\lambda_1 < \lambda_2 < 0$\hskip .75truein}
\centerline{\epsfbox{s13-15.ps}}
\medskip
%Nodes with $\lambda_1 = \lambda_2$ and 2 linearly
% independent eigenvectors\hfill}
\centerline{\epsfbox{s13-16.ps}}
\medskip
%Nodes with $\lambda_1 = \lambda_2$, generalized eigenvectors needed
\outind{generalized eigenvectors}%

Note that in each case the critical point is stable if the eigenvalues
are negative.

% Advice to instructor.  The next example should be skipped 
% unless there is lots of time.
\nextex
\example{Example \en}
Consider the linear system
$$
\frac{d\x}{dt} = 
\bm\format\r&\quad\r\\
 -3 & -1 \\ 1 & -1 \em \x.
$$
The only critical point is the origin, and of course the linear
approximation there is just the original system.

It turns out that there is no basis of eigenvectors in this
case, so we must use the method of generalized eigenvectors.
In fact, this example was worked out in Chapter XI, Section 8,
Example 2.
We found there that
$$
\align
\x_1 &= e^{-2t}(\e_1 + t\v_2) = \bm \e_1 & \v_2 \em\bm e^{-2t}\\ te^{-2t}\em \\
\x_2 &= e^{-2t}\v_2 = \bm \e_1 & \v_2 \em \bm 0\\ e^{-2t}\em
\endalign
$$
form a basis for the solution space, where
$$
\e_1 = \bm 1\\ 0\em\qquad\text{and}\qquad \v_2 = (A +2I)\e_1 = 
\bm\format\r\\-1\\ 1\em.
$$ 
The general solution is
$$
\x = c_1\x_1 + c_2\x_2 = 
\bm \e_1 & \v_2 \em\bm c_1e^{-2t}\\c_1te^{-2t} + c_2 e^{-2t}\em.
$$
This suggests making the change of coordinates  $\x = P\z$
where 
$$
P = \bm \e_1 & \v_2 \em = \bm\format\r&\quad\r\\ 1 & -1 \\ 0 & 1 \em.
$$
In the new coordinates, the general solution is given by
$$
\z = \bm  c_1e^{-2t}\\ c_1te^{-2t} + c_2 e^{-2t} \em 
$$
which in components is
$$
\align
z_1 &= c_1e^{-2t} \\
z_2 &= c_1te^{-2t} + c_2 e^{-2t}.
\endalign$$
As $t\to \infty$, both $z_1$ and $z_2$ approach zero because of
the factor $e^{-2t}$.
Also,
$$
\frac{z_2}{z_1} = t + \frac{c_2}{c_1}
$$
so for large $t$, both $z_1$ and $z_2$ have the same sign.
 On the other hand, for $t$
sufficiently negative, $z_1$ and $z_2$ have opposite signs, i.e.,
$\z$ starts off either in the fourth quadrant or the second
quadrant.
(Note that this argument required $c_1 \not= 0$.  What does
the orbit look like if $c_1 = 0$?)

We leave it as an challenge for you to sketch the orbits of
this system.  You should get one of the diagrams sketched
above.   You should first do it in the $z_1, z_2$ coordinate
system.  However, to interpret this in the original $x_1, x_2$
coordinates, you should notice that the '$z$' axes are not
perpendicular.  Namely, the $z_1$-axis is the same as the
$x_1$ axis, but the $z_2$-axis points along the vector
$\v_2$, so it makes an angle of $3\pi/4$ with the positive
$x_1$ axis.
\endexample

If the {\it matrix $D\f(\a)$ is singular\/}, the theory utilized
above breaks
down.  The following example indicates how that might occur.

\nextex
\example{Example \en}  Consider the system
$$
\frac{d\x}{dt}  = \bm\format\r&\quad\r\\
                       -1 & 1\\
                       1 & -1 \em \x.
$$
The origin is a critical point, and since the system is linear,
the linear approximation there is the same as the system itself.

The eigenvalues turn out to be $\lambda = -2$ and $\lambda = 0$.
For $\lambda = -2$ a basic eigenvector is
$$
\v_1 = \bm -1\\ 1\em.
$$
For $\lambda = 0$, a basic eigenvector is
$$
\v_2 = \bm\format\r\\1\\1\em.
$$
Put these together to form a change of basis matrix
$$
P = \bm\format\r&\quad\r\\
        -1 & 1\\ 1 & 1 \em.
$$
Then, by our theory
$$
P^{-1}\bm -1 & 1\\ 1 & -1 \em P = \bm -2 & 0 \\ 0 & 0 \em
$$
and the change of variables $\x = P\z$ yields the `uncoupled'
system
$$\align
\frac{dz_1}{dt} &= -2z_1\\
\frac{dz_2}{dt} &= 0.
\endalign$$
The solution is
$$
z_1 = C_1e^{-2t}\qquad z_2 = C_2
$$
This is a family of half lines perpendicular to the $z_2$ axis.
Every point on the $z_2$-axis is a critical point and is approached
asymptotically from either side as $t \to \infty$.
\medskip
\centerline{\epsfbox{s13-17.ps}}
\medskip
\endexample

In the above discussion, we have been assuming that the behavior
of a non-linear system near a critical point may be determined
from the behavior of the
linear approximation. 
  Unfortunately, that is not always the case.
First of all, if the matrix $D\f(\a)$ is singular, the behavior
near the critical point depends strongly on the higher
order terms which were ignored in forming the linear approximation.
Even if $D\f(\a)$ is non-singular, in some cases,
the higher order terms can
exert enough influence to change the nature of the critical
point.   
\nextex
\xdef\ExX{\en}
\example{Example \en}
Consider the nonlinear system
\nexteqn
$$
\align
\frac{dx_1}{dt} &=  x_2 -x_1(x_1{}^2 + x_2{}^2)\\
\frac{dx_2}{dt} &=  -x_1 -x_2(x_1{}^2 + x_2{}^2).
\tag\eqn
\endalign
$$
$(0,0)$ is clearly a critical point.  (Are there any more?)
Also,
$$
D\f = \bm -3x_1{}^2 -x_2{}^2 & 1 - 2x_1x_2 \\
      -1 -2x_1x_2 & -x_1{}^2 - 3x_2{}^2 \em
= \bm\format\r&\quad\r\\ 0 & 1 \\ -1 & 0 \em\qquad\text{at } (0,0).
$$
Hence, the approximating linear system is
$$
\frac{d\x}{dt} =
\bm\format\r&\quad\r\\ 0 & 1 \\ -1 & 0 \em \x.
$$
The eigenvalues are $\lambda = \pm i$, so the critical point is
a center.   The phase portrait of the linear system consists of
a family of closed loops centered at the origin.

On the other hand, we can solve the nonlinear system exactly in
this case if we switch to polar coordinates in the $x_1, x_2$-plane.
Put $x_1 = r\cos\theta,\, x_2 = r\sin\theta$ in (\eqn).  We get
$$
\align
\frac{dr}{dt}\cos\theta -r\sin\theta \frac{d\theta}{dt}
  &= r\sin\theta -r^3\cos\theta\\
\frac{dr}{dt}\sin\theta +r\cos\theta \frac{d\theta}{dt}
  &= -r\cos\theta -r^3\sin\theta.
\endalign
$$
Multiply the first equation by $\cos\theta$, the second by $\sin\theta$
and add to obtain
$$
\gather
\frac{dr}{dt} = -r^3 \\
-\frac{dr}{r^3} = dt \\
\frac 1{2r^2} = t + c \\
r = \frac 1{\sqrt{2t + c_1}}
\endgather
$$
Similarly, multiplying the first equation by $\sin\theta$, the
second by $\cos\theta$ and subtracting the first from the second
yields
$$
r\frac{d\theta}{dt} = -r.
$$
However, $r = 0$ is the critical point which we already know yields
a constant solution, so we may assume $r\not=0$.  Hence, we get
$$
\gather
\frac{d\theta}{dt} = -1 \\
\theta = -t + c_2.
\endgather
$$
This clearly represents a family of solutions which spiral in towards
the origin, approaching it asymptotically at $t \to \infty$.
Thus, {\it the additional nonlinear terms turned a center into a focus}.
In this case, it is a stable focus with the orbits spiraling in
toward the critical point.   In other cases, the non-linear terms
might perturb things in the other direction so that the orbits
\outind{critical point, center to focus}%
would spiral out from the origin
\medskip
\centerline{\epsfbox{s13-18.ps}}
\medskip
\endexample

In general, for a two dimensional system, a center for the linear
approximation can stay a center or can become a focus in the non-linear
system.
Fortunately, for a two dimensional system,
if  $f$ is sufficiently smooth (i.e., $C^2$),
and $Df$ is non-singular
at the critical point, {\it this is the only case\/} in which the nonlinear
 terms
can significantly affect the structure of the phase portrait.

In every other non-singular case, 
i.e., a node, a saddle point,
or a focus, a theorem of Poincar\'e
\outind{Poincare\noexpand\'e}%
assures us that the linear approximation gives a true picture of the
phase portrait near the critical point.

Note that in the previous examples, the signs of the eigenvalues
of the linear systems (or the signs of the real parts of complex
eigenvalues) played an important role.   It is clear why that
would be the case.  A basic solution will have a factor
of the form $e^{at}$ and if $a < 0$, the basic solution will
necessarily converge to zero at $t \to \infty$.  Hence, if 
these signs are negative for both basic solutions,
 every solution will converge to
zero as $t \to \infty$, so the corresponding critical point
will be asymptotically stable.  On the other hand, if one eigenvalue
is negative and one is positive (a saddle point), the situation is
more complicated.  (What if both signs are positive?)

Needless to say the situation is even more complicated for higher dimensional
systems.  It is still true
that if all the eigenvalues of the linear approximation
 are negative, or, if complex,
have negative real parts, then all solutions near the critical point
 converge to it as $t \to \infty$, i.e., the corresponding equilibrium
is  asymptotically stable.  However, the basic linear algebra
is considerably more complicated, so it is not so easy to classify
exactly what happens even in the linear case.   
\medskip
\subhead Stable orbits and Attractors \endsubhead
The behavior of a nonlinear system near its critical points  
helps us to understand the  system, but it is certainly not the only
thing of interest.   For example, the solution of the
prey-predator problem yields many periodic solutions.  The orbits
traced out by such solutions are 
 stable in the following sense:  any  solution
which at some time $t$ comes close to the orbit remains close to it
\outind{orbit, stability of}%
\outind{stability of an orbit}%
for all subsequent $t$.   That means that if we stray a little from
such an orbit, we won't ever get very far from it.   From this
perspective, a critical point is just a stable orbit which happens
to be a point.  We should be interested in finding all
stable orbits, but of course it is much harder to find the
non-constant ones.

Much of the theory of nonlinear systems was motivated by questions
in celestial
mechanics.   The general mathematical problem in that subject is
the so-called $n$-body problem where we attempt to describe the
motion of an arbitrary number of point masses subject to 
the gravitational forces between them.   A complete solution
of that problem still eludes us despite intense study over several
centuries.   Even fairly simple questions remain unanswered.
For example, one would assume that the Solar System as a whole
will continue to behave more or less as it does now as long as it is
not disturbed by a significant perturbation such as a
star passing nearby.   However, no one has been able to prove, for
example, that the entire system is `stable' in the
sense that it remains bounded
for all time.
Thus, it is conceivable that at some point in time a
  planet might cease to follow its normal orbit and leave the solar
system altogether.  (At least, it is conceivable to mathematicians,
who generally believe that something may happen until they 
prove that it can't happen!)

Modern developments in the study of nonlinear systems are often
concerned with  `stability' questions such  as
those mentioned above.   One such result is a famous theorem of
Poincar\'e and Bendixson.   It asserts the following:  if
\outind{Poincar\noexpand\'e--Bendixson Theorem}%
an orbit of a 2 dimensional nonlinear system enters a bounded region in
the phase
plane
and remains there forever, and if that bounded region contains no
critical points, then either the orbit
is periodic itself, or it approaches a periodic orbit asymptotically.
(If critical points were not excluded from the bounded region,
the constant solutions represented by those points
would violate the conclusion of the theorem.
Also, the presence of critical points could `disrupt' the behavior
 of other paths
in rather subtle ways.)
The following example illustrates  this phenomenon.

\nextex
\example{Example \en}  Consider the system
$$\align
\frac{dx}{dt} &= -y + x(1 - x^2 - y^2)\\
\frac{dy}{dt} &= x + y(1 - x^2 - y^2).
\endalign$$
This system can be solved explicitly if we switch to polar
coordinates.    
Putting $x = r\cos\theta, \, y = r\sin\theta$ in the above system
yields
$$\align
\frac{dr}{dt}\cos\theta - r\sin\theta \frac{d\theta}{dt}
 &= -r\sin\theta + r\cos\theta(1 - r^2) \\
\frac{dr}{dt}\sin\theta + r\cos\theta \frac{d\theta}{dt}
 &= r\cos\theta + r\sin\theta(1 - r^2).
\endalign$$
Multiply the first equation by $\sin\theta$ and subtract it
from $\cos\theta$ times the second equation to obtain
$$
r\frac{d\theta}{dt} = r.
$$
Thus, either $r = 0$ or 
$$\gather
\frac{d\theta}{dt} = 1\\
\theta = t + D,
\endgather$$
where $D$ is an arbitrary constant.
Similarly, multiplying the first equation by $\cos\theta$
and adding it to $\sin\theta$ times the second equation yields
$$
\frac{dr}{dt} = r(1 - r^2).
$$
We see from this that $r = 0$ and $r = 1$ are solutions in which
$dr/dt = 0$.  ($r = -1$ is not a solution because by definition
$r \ge 0$.)  $r = 0$ yields a critical point at the origin. 
$r = 1$ (together with $\theta = t + D$) yields a periodic
solution for which the orbit is a circle of radius 1 centered
at the origin.   If we exclude these solutions, we may
separate variables to obtain
$$
\int \frac{dr}{r(1 - r^2)} = t + c_1.
$$
The left hand side may be computed by the method of partial fractions
which yields
$$
\ln r - \frac 12 \ln|1 - r| - \frac 12 \ln ( 1 + r) = t + c_1.
$$
I did it instead using Mathematica which neglected to include
the absolute values in the second term, but fortunately
I remembered them.   (The absolute values
are not necessary for the other terms since $r, r+1 > 0$.)
 This may be further simplified as follows.
$$\gather
\ln \frac r{\sqrt{|1 - r^2|}} = t + c_1 \\
 \frac r{\sqrt{|1 - r^2|}} = c_2e^t \\
  \frac {r^2}{|1 - r^2|} = c_3e^{2t}.
\endgather$$
Note that the constant necessarily satisfies $c_3 > 0$.
We now consider two cases.   If $0 < r < 1$, we have
$$\gather
 \frac {r^2}{1 - r^2} = c_3e^{2t}\\
r^2 =  c_3e^{2t}(1 - r^2)\\
 r^2(1 + c_3e^{2t}) = c_3e^{2t}\\
r^2 = \frac{c_3e^{2t}}{1 + c_3e^{2t}}
\endgather$$
Divide both numerator and denominator by $c_3e^{2t}$
 and take the square root
to obtain
\nexteqn
\xdef\RLT{\eqn}
$$
r = \sqrt{\frac 1{Ce^{-2t}+1}}.\tag\eqn
$$
If you follow what happened to the constant at each stage, you will
see that the constant we end up with satisfies $C > 0$.

If $r > 1$, we may continue instead
as follows.
$$\gather
 \frac {r^2}{r^2-1} = c_3e^{2t}\\
r^2 =  c_3e^{2t}(r^2-1)\\
 r^2(c_3e^{2t} - 1) = c_3e^{2t}\\
r^2 = \frac{c_3e^{2t}}{c_3e^{2t} - 1}.
\endgather$$
Divide numerator and denominator by $c_3e^{2t}$ to obtain
\nexteqn
\xdef\RGT{\eqn}
$$
r = \sqrt{\frac 1{1 - Ce^{-2t}}}.\tag\eqn
$$
As above $C > 0$.

We may summarize all the above cases except the critical point
$r = 0$ by writing
\nexteqn
$$
\align
r &= \frac 1{\sqrt{1 + Ce^{-2t}}}\\
\theta &= t + D
\tag\eqn
\endalign
$$
where $C > 0$ for $r < 1$, $C = 0$ for $r = 1$,
 and $C < 0$ for $r > 1$.
Note that for $C > 0$, the solution spirals outward from the
origin and approaches the periodic orbit $r = 1$ asymptotically
as $t \to \infty$.   Similarly, for $C < 0$, the solution spirals
inward and approaches the periodic orbit $r =1$ asymptotically
as $t \to \infty$.    All these paths behave as the Poincar\'e--Bendixson
Theorem predicts. 
\medskip
\centerline{\epsfbox{s13-19.ps}}
\medskip
 You can choose for the bounded region any
annular (ring shaped) region containing the circle $r = 1$.  
You can't allow the critical point at the origin in the bounded
region because then the constant solution $\x(t) = 0$ would
violate the conclusion of the theorem.   

\endexample

A periodic orbit that is asymptotically stable (i.e., all
solutions which get sufficiently near it approach it asymptotically)
is called
an {\it attractor}.   In the above example the orbit $r = 1$ is
\outind{attractor}%
an attractor.  On the other hand, the critical point $r = 0$
exhibits the opposite kind of behavior, so it might aptly be
called a {\it repeller}.
\bigskip
\input chap13.ex2

\endchapter
\closeseg{chap13}
\enddocument
