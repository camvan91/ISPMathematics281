\documentstyle{book}
\input epsf.tex
\input extra.tex
\input le2.sty
\input chap1.lnk
\maketoctrue
\mardiagtrue
\indextrue
\openseg{chap2}
\Monograph
\topmatter
\nextchap{Differential Equations, Preview}
\title\chapter{\chapno} Differential Equations, Preview\endtitle
\endtopmatter
\document
\nextsec{Some Elementary Differential Equations}
\head \sn. Some Elementary Differential Equations \endhead

Later in this course, you will study differential equations in
a systematic way, but before that you are likely to encounter
some simple differential equations in Physics and other courses.
Hence, we shall give a brief introduction to the subject here.

A differential equation is an equation involving an independent
variable, a dependent variable, and derivatives of the latter.
The {\it order\/} of a differential equation is the highest order
\outind{differential equation, order of}%
\outind{order of a differential equation}%
of any derivative which appears in it.
\nexteqn
\xdef\FirstEx{\eqn}
For example,
$$
\frac{dx}{dt} = ax\tag\eqn
$$
is an example of a {\it first order\/} differential equation,
while
\nexteqn
\xdef\SecondEx{\eqn}
$$
\frac{d^2x}{dt^2} = -Kx\tag\eqn
$$
is an example of a {\it second order\/} differential equation.

To solve a differential equation, you must find a {\it function\/}
$x = x(t)$, expressing the dependent variable in terms of the
independent variable, which when substituted in the equation
yields a true identity.
For example, substituting $x = Ce^{at}$ (where $C$ is any constant)
and $dx/dt = Cae^{at}$ in equation (\FirstEx) yields
$$
      Cae^{at} = a(Ce^{at})
$$
which is true.   Hence, the function given by $x(t) = Ce^{at}$ is
a solution.   Note the following very important point.  If someone
is kind enough to suggest a solution to you, it is not necessary
to go through any procedure to ``find'' that solution.  All you
need to do is to check that it works.

	If you don't have any idea of what a solution to a given
differential equation might be, then you need some method to try
to find a solution.  How one goes about this is a vast subject,
and we shall go into it later, as mentioned above.  For the moment,
we describe only the method of {\it separation of variables\/} which
\outind{separation of variables for ordinary differential equation}%
works in many simple examples.  We illustrate it by solving
equation (\FirstEx).
$$\align
    \frac{dx}{dt} & = ax \\
   \frac{dx}x & = adt\qquad\text{Variables are separated formally}\\
   \int \frac{dx}x &= \int a dt \qquad\text{Take antiderivatives}\\
    \ln |x| &= at + c \\
    |x| &= e^{at + c} = e^{at}e^c\qquad\text{Exponentiate} \\
     x  &= \pm e^ce^{at}.
\endalign $$
However, $\pm e^c$ is just some constant which we may call $C$.  
Thus we find that $x = Ce^{at}$ is a solution.   

Note that when integrating  there should be an
undetermined constant on each side of the equation.  However, these
constants can be combined in one by transposition, and that is what
we did by putting the $+ c$ only on the right side of the equation
in the fourth line.

Whenever the variables can be so separated, the above procedure
produces a {\it general solution\/} of the equation.
There are some technical difficulties with the method.
(For example, a denominator might vanish after the variables
have been separated.
 Also, the
formal separation procedure needs some rigorous justification.)
However, these difficulties can usually be resolved and one can
make a convincing argument
that {\it any solution\/} will be of the form obtained
by the method.

Note that in the example, the solution produced one {\it arbitrary\/}
constant  $C$, i.e., one constant which is not otherwise determined
by the equation.  ($a$ is also a constant, but it was assumed known
in the original equation.)   You can think of that constant arising
from integration.   To complete the solution, it is necessary to
determine the constant.  There are many ways to do this, depending
on the formulation of the problem giving rise to the
differential equation.  One of the easiest is to
give the value $x_0$ of the dependent variable  for one specified value
$t_0$ of the independent variable and then solve the resulting equation
for the constant $C$.  (This is called satisfying an
{\it initial condition}.)  For example, suppose we are given that
\outind{initial conditions for a differential equation}%
$x = 10$ when $t = 0$.  Substituting these values yields
$$
    10 = Ce^{a(0)} = Ce^0 = C
$$
or $C = 10$.  With that initial condition, the solution is
$x = 10e^{at}$.   

Equation (\FirstEx) describes processes of growth or decay.  $x$
might represent the size of a population, the number of radioactive
atoms, or something similar.   The equation asserts that the
growth rate (decay rate if $a < 0$) is proportional to the amount
present.  In the case of population growth, $a$ is often called the
``birth rate''.  In the case of radioactive decay, $a <0$, so
we put $a = -\gamma$,
and $\gamma$ tells us the
 {\it proportion\/} of atoms present at a given
time which will decay at that time.   In that case,  $\gamma$ is often expressed in
terms of the {\it half life\/} $T$ which is the time for the
$x$ to decrease to half its initial value.  You might check
your proficiency in the use of exponentials and logarithms by
checking the following formula.
$$
     \gamma = \frac{\ln 2}T.
$$

Equation (\SecondEx),
$\dfrac{d^2x}{dt^2} = -Kx$,
  arises in studying {\it simple harmonic motion}.
\outind{harmonic oscillator, simple}%
\outind{oscillator}%
For example, the motion of a particle of mass $m$
 oscillating at the end of 
a spring with spring constant $k$ is governed by
such an equation with $K = k/m$, if we ignore friction.
(\SecondEx) is much harder to solve than (\FirstEx).  
One must distinguish the cases
$K > 0$ and $K < 0$, but the former has more interesting applications
(as in simple harmonic motion), so we shall assume $K > 0$.
The solution proceeds by
 separation of variables as follows.
First, put $v = dx/dt$.  Then the equation becomes
$$\align
   \frac{dv}{dt} &= -Kx \\
    \frac{dv}{dx}\frac{dx}{dt} &= -Kx\qquad\text{using the chain rule}\\
 \text{or}\qquad   \frac{dv}{dx}v &= -Kx \\
     v\, dv &= -Kx\,dx \qquad\text{separating the variables} \\
    \frac{v^2}2 &= -K\frac{x^2}2 + c \\
    v^2 &= -Kx^2 + 2c.
\endalign $$ 

\mar{s2-1.ps}
However, we might just as well rename the constant $2c$ and call it
$C_1$. Note that since $K > 0$, we have $C_1 = v^2 + Kx^2 \ge 0$.  If
$C_1 = 0$, it follows that $v = x = 0$ for all $t$.   That is certainly
possible, but it isn't very interesting, so we assume $C_1 > 0$.
  Now, recalling that $v = dx/dt$, we obtain
$$\align
    v &= \frac{dx}{dt} = \pm\sqrt{C_1 - K x^2} \\
     \frac{dx}{\sqrt{C_1 - K x^2}} &= \pm dt\qquad\text{separating variables}\\
\int\frac{dx}{\sqrt{C_1 - K x^2}} &= \pm t + C_2\qquad\text{integrating}\\
\frac 1{\sqrt K}\cos^{-1}(\sqrt{\frac K{C_1}} x) &= \pm t + C_2.
\endalign$$
Note that we need $K > 0$ and $C_1 > 0$ for the last integration to
be valid.  Continuing, we have
$$
\align
\cos^{-1}(\sqrt{\frac K{C_1}} x) &= \pm \sqrt K t + \sqrt K C_2\qquad\text{or}\\
x &= \sqrt{\frac{C_1}K}\cos(\pm \sqrt K t + \sqrt K C_2) \\
 &= \sqrt{\frac{C_1}K}\cos(\sqrt K t \pm \sqrt K C_2) .
\endalign $$
However, we may define 
$A = \sqrt{\dfrac{C_1}K}$ (so $A > 0$)
 and $\delta = \pm \sqrt K C_2$  to obtain
the general solution
\nexteqn
\xdef\SHSol{\eqn}
$$
  x = A\cos(\sqrt K t + \delta).\tag\eqn
$$
Note that this solution has {\it two\/} arbitrary constants arising
from the two integrations which had to be performed.  Generally,
the solution of an 
$n$th order equation involves $n$ arbitrary constants.

The above derivation depended strongly on the assumption $K > 0$.
For $K < 0$, we would not be able to conclude that
$C_1 > 0$, and the integration step which resulted in 
$\cos^{-1}(\sqrt{\frac K{C_1}} x)$ on the left would not be valid. 
If you are ambitious, you might try doing the
integration under the assumpition $K < 0$ to see what you get.
Later in
this course, we shall derive other methods to solve this equation whatever
the sign of $K$.  

To determine the constants, one may again specify initial conditions.
However, specifying the value $x_0$ at $t_0$ will yield only one
equation for the two constants.  Hence, one needs an additional
condition to obtain another equation.  To get that, one commonly
specifies the derivative $v_0$ at $t_0$.  For example, suppose
$K = 3$, and suppose  $x = 1, dx/dt = -1$ at $t = 0$.  Then
(\SHSol) yields the two equations
$$\align
  1 &= A\cos(\delta) \\
  -1 &= -A\sqrt 3\sin(\delta).
\endalign $$  
Dividing the second equation by the first, yields
$$
  \tan\delta = \frac 1{\sqrt 3}
$$
from which we conclude  $\delta = \pi/6$ or $-5\pi/6$.  Since both
$\sin \delta$ and $\cos\delta$ are positive, this yields $\delta =
\pi/6$.   Putting this in the
first equation yields
$$
   1 = A\cos(\frac\pi 6) = A\frac {\sqrt 3}2
$$
so $A = 2/\sqrt 3$.   We conclude that the solution satisfying these
initial conditions is
$$
   x = \frac 2{\sqrt 3}\cos(\sqrt 3 t + \frac\pi 6).
$$

The quantity $A$ in equation (\SHSol) is called the {\it amplitude\/}
\outind{amplitude of a harmonic oscillator}%
of the solution, and $\delta$ is called the {\it phase}.   One way
to visualize the solution  is as follows.   Consider a
point moving on a circle of radius $A$ centered at the origin.
\medskip
\centerline{\epsfbox{s2-2.ps}}
\medskip
\noindent
The projection of that point on the $x$-axis oscillates back according
to  (\SHSol).   $\delta$ gives the angle the position vector of
the point makes
with the $x$-axis at $t = 0$.  The quantity $\omega = \sqrt K$
gives the angular velocity of the point on the circle.  $\omega$
is also called the {\it angular frequency\/} of the oscillation.
\outind{angular frequency}%
\outind{frequency, angular}%
\outind{frequency}%
    The {\it period\/} $T = 2\pi/\omega$ 
\outind{period of an oscillator}%
is the time required for one circuit (one complete oscillation).
The frequency $f = 1/T = \omega/2\pi$ is the number of circuits
(oscillations) per unit time.   The frequency is usually measured
in Hertz (Hz), i.e., oscillations per second.


 One
often sees the solution written differently in the form obtained
by expanding  (\SHSol) as follows.
$$\align
  x &= A\cos(\omega  t + \delta) \\
    &= A\cos(\omega t)\cos\delta - A\sin(\omega t)\sin\delta \\
  x &= C\cos(\omega t) + D\sin(\omega t)
\endalign$$
where $C = A\cos\delta$ and $D = -A\sin\delta$.   

You should note that had you suspected
 that $x = A\cos(\omega t + \delta)$
is a solution, you could have checked it by substituting
into (\SecondEx), and avoided most of the
work in the derivation above.  You might argue that one would have
little reason to suspect such a solution might work.  That is
true if one is thinking in purely mathematical terms, but given that
one knows the differential equation arises from a physical problem
in which one observes oscillatory behavior, it is not entirely
unreasonable to guess at a solution like (\SHSol).  Of course, without
going through a more refined analysis, you
could not be absolutely sure that (\SHSol) encompasses all possible
solutions (i.e., that it is sufficiently general), but the fact
that it involves {\it two\/} arbitrary constants would strongly
suggest that such is the case.  

Remember the moral of the above discussion.  If somehow or other
a solution to a problem is suggested to you, you don't have to
bother ``deriving'' that solution by a ``mathematical'' procedure.
If the solution works, and if in addition, you have reason to
believe there is only one solution which can work, then the one
you have must be it.

\bigskip
% Exercises for Section 1
\input chap2.ex1

\endchapter
\closeseg{chap2}
\enddocument
