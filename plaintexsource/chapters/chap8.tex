\documentstyle{book}
\input extra.tex
\input epsf.tex
\input le2.sty
\input chap7.lnk
\maketoctrue
\indextrue
\mardiagtrue
\openseg{chap8}
\Monograph
\topmatter
\nextchap{Series}
\title
\chapter{\chapno}Series
\endtitle
\endtopmatter
\document
\nextsec{Series Solutions of a Differential Equation}
\head \sn.  Series Solutions of a Differential Equation \endhead

There are several second order
linear differential equations which arise in mathematical
\outind{series solution of a differential equation}%
\outind{Legendre's equation}%
\outind{Bessel's equation}%
\outind{Euler's equation}%
physics which cannot be solved by the methods introduced in the
previous chapter.  Here are some examples
$$
\gather
y'' -\frac{2t}{1 - t^2}y' + \frac{\alpha(\alpha + 1)}{1 - t^2} y = 0
\qquad\text{Legendre's equation}\\
t^2 y'' + ty' + (t^2 - \nu^2)y = 0\qquad\text{Bessel's equation}\\
ty'' + (1 - t)y' + \lambda y = 0 \qquad\text{Laguerre's equation}
 \endgather $$
A related equation we studied (in the Exercises to Chapter VII, Sections
4 and 8) is
$$
y'' +\frac{\alpha}t y' + \frac{\beta}{t^2} y = 0 \qquad\text{Euler's
equation}.
$$
These equations and others like them arise as steps in solving 
important partial differential equations such as Laplace's
equation, the wave equation, the heat equation, or Schroedinger's
equation.  Their solutions are given appropriate names:
{\it Legendre functions, Bessel Functions\/}, etc.  Sometimes
these {\it special functions\/} can be expressed in terms of
other known functions, but usually they are entirely new
functions.  One way to express these solutions is as {\it infinite
series\/}, and we investigate that method here.

\example{Example}  Consider
\nexteqn
$$
(1 - t^2)y'' - 2t y' + 6 y = 0\tag\eqn
$$
which is Legendre's equation for $\alpha = 2$.   The simplest possible
functions are the {\it polynomial functions\/},
 those which can be expressed as combinations of powers of $t$
$$
y = a_0 + a_1t + a_2t^2 +  a_3t^3 + a_4t^4 + \dots + a_dt^d.
$$
The highest power of $t$ occurring in in such an expression is called
its {\it degree}.  We shall try to find a solution of
(\eqn) which is a polynomial function
{\it without\/} committing ourselves about its degree in advance.  We have
$$
\align
y &= a_0 + a_1t + a_2t^2 +  a_3t^3 + a_4t^4 + \dots + \\
y'&={\hskip 20pt} a_1{\hskip 5pt}+ 2a_2t +  3a_3t^2 + 4a_4t^3 + \dots  \\
y''&={\hskip 45pt}2a_2{\hskip 5pt} + 6a_3t + 12a_4t^2 + \dots 
\endalign $$
Note that these equations tell us that $y(0) = a_0$ and
$y'(0) = a_1$, so the first two coefficients just give
us the initial conditions at $t = 0$.

We have
$$\align
1\,y'' &= 2a_2 + 6a_3t + 12a_4t^2 + 20a_5t^3 + \dots \\
-t^2y''&={\hskip 50pt}-2a_2t^2  -6a_3 t^3  -\dots  \\
-2ty'&={\hskip 15pt}-2a_1t  -4a_2t^2 - 6a_3 t^3 - \dots  \\
6y &=  6a_0 + 6a_1t + 6a_2t^2 + 6a_3t^3 + \dots 
\endalign $$
The sum on the left is $(1 - t^2)y'' -2ty' + 6y$ which is
assumed to be zero.
 If we add up the coefficients
of corresponding powers of $t$ on the right,
we obtain
$$
 0 = 
2a_2 + 6a_0  + (6a_3 + 4a_1)t +
12a_4t^2  +  (20a_5 - 6a_3)t^3 + \dots
$$
Setting the coefficients of each power of $t$ equal
to zero, we obtain
$$
2a_2 + 6a_0 =0\quad 6a_3 + 4a_1= 0\quad
12a_4 = 0\quad 20a_5 - 6a_3 = 0 \quad\dots
$$
Thus,
$$\align
a_2 &= -3a_0 \\
a_3 &= -\frac 23 a_1 \\
a_4 &= 0 \\
a_5 &= \frac 3{10} a_3 = -\frac 15 a_1 \\
&\vdots
\endalign $$
The continuing pattern is more or less clear.  Each succeeding
coefficient $a_n$ is expressible as a multiple of the coefficient 
$a_{n-2}$ two steps lower.   Since $a_4 = 0$, it follows
that all coefficients $a_n$ with $n$ even must vanish for $n \ge 4$.
On the other hand, the best that can be said about the coefficients
$a_n$ with $n$ odd is that ultimately each of them may be
expressed in terms of $a_1$.
It follows that if we separate out even and odd powers
of $t$, the solution looks something like
$$
\align
y &= a_0 -3a_0t^2 + a_1t -\frac 23 a_1 t^3 -\frac 15 a_1 t^5 + \dots \\
 &= a_0(1 - 3t^2) + a_1(t -\frac 23 t^3 -\frac 15 t^5 + \dots).
\endalign
$$
Consider first the solution satisfying the initial conditions
$a_0 = y(0) = 1, a_1 = y'(0) = 0$.  The corresponding solution is
$$
y_1 = 1 - 3t^2
$$
which is a polynomial of degree 2, just as expected.
(However, notice that we had no way to predict in advance that
it would be of degree 2.)  The meaning of the `solution'
obtained by putting $a_0 = 0$ and $a_1 = 1$ is not so clear.
It appears to be a `polynomial'
$$
y_1 = t - \frac 23 t^3 - \frac 15 t^5 + \dots
$$
which goes on forever.  In fact, it is what is called an
{\it power series}.    This is just one example of many
\outind{power series}%
where one must represent a function in terms of an infinite
series.   We shall concern ourselves in the rest of this
chapter with the theory of such series with the ultimate aim
of understanding how they may be used to solve differential
equations.   For many of you this may be review, but you
should pay close attention since some points you are not
familiar with may come up.
\endexample

%Exer  Second solution later identifiable with result of
%Exer of reduction of order.
\bigskip
\input chap8.ex1
\bigskip
\nextsec{Definitions and Examples}
\head \sn. Definitions and Examples \endhead

A series is a sequence of terms $u_n$ connected by `$+$' signs
$$
 u_1 + u_2 + u_3 + \dots + u_n + \dots
$$
The idea is that if there are infinitely
many terms, the summation process is supposed to go on
forever.   Of course, in reality that is impossible, but perhaps
after we have added up sufficiently  many terms, the contributions
of additional terms will be so negligible that they really
won't matter.   We shall make this idea a bit more precise
below.

The terms of the series can be positive, negative or zero.  They
can depend on variables or they can be constant.   The summation
can start at any index.  In the general theory we assume the
first term is $u_1$, but the summation could start just as
well with $u_0$, $u_6$, $u_{-3}$, or any convenient index.

Here are some examples of important series
$$
\align
1 + t + t^2 + \dots + t^n + \dots\qquad&\text{geometric series}\\
1 + \frac 12 + \frac 13 + \dots + \frac 1n + \dots\qquad&\text{harmonic
series}\\
t - \frac{t^3}{3!} + \frac{t^5}{5!} - \dots +(-1)^n\frac{t^{2n+1}}{(2n+1)!}
+\dots \qquad&\text{Taylor series for } \sin t 
\endalign
$$
In the first series, the general term is $t^n$ and the numbering starts
with $n = 0$.   In the second series, the general term is $1/n$ and
the numbering starts with $n = 1$.

Series are often represented more compactly using `$\Sigma$'-notation
$$
\sum_{n = 1}^\infty u_n.
$$
For example,
$$\align
\sum_{n=0}^\infty t^n\qquad&\text{is the geometric series}\\
\sum_{n=1}^\infty \frac 1n\qquad&\text{is the harmonic series}
\endalign $$
\outind{geometric series}%
\outind{harmonic series}%

The notion of the sum of an infinite series is made precise
as follows.  Let
$$
\align
s_1 &= u_1 \\
s_2 &= u_1 + u_2 \\
s_3 &= u_1 + u_2 + u_3 \\
&\vdots \\
s_n &= u_1 + u_2 + \dots + u_n = \sum_{j=1}^n u_j\\
&\vdots
\endalign $$
$s_n$ is called the {\it $n$th partial sum}.  It is obtained by
adding the $n$th term $u_n$ to the previous partial sum,
i.e.,
$$
   s_n = s_{n-1} + u_n.
$$
Consider the behavior of the {\it sequence\/} of
partial sums  $s_1, s_2, \dots, s_n, \dots$ as $n \to \infty$.
If this sequence approaches a finite limit
$$
\lim_{n \to \infty} s_n = s
$$
then we say the series {\it converges\/} and that $s$ is its
{\it sum}.   Otherwise, we say that the series {\it diverges}.

\example{Example}  Consider the geometric series
$$
1 + t + t^2 + \dots + t^n + \dots = \sum_{n=0}^\infty t^n.
$$
The partial sums are
$$
\align
s_0 &= 1 \\
s_1 &= 1 + t \\
s_2 &= 1 + t + t^2 \\
&\vdots\\
s_n &= 1 + t + t^2 + \dots + t^n = \frac{1 - t^{n+1}}{1 -t}\\
&\vdots
\endalign $$
The formula for $s_n$ should be familiar to you from high school
algebra, it is the sum of the first $n + 1$ terms of a geometric
progression with starting term 1 and ratio $t$.  It applies
as long as $t \not= 1$.   Let $n \to \infty$.
There are several cases to consider.

Case (a).  $|t| < 1$.   Then, $\lim_{n\to \infty} t^n = 0$
so $\lim_{n\to\infty} s_n = \dfrac 1{1 -t}$.  Hence, in this
case the series converges and the sum is $1/(1 - t)$.

Case (b)  $t = 1$.  Then $s_n = n \to \infty$, so the
series diverges.  We might also say in a case like this
that the sum is $\infty$.

Case (c)  $t = -1$.   Then the series is
$$
1 - 1 + 1 - 1 + 1 - \dots
$$
and the partial sums $s_n$ alternate between 1 ($n$ odd)
 and 0  ($n$ even).  As a result
they don't approach any definite limit as $n \to \infty$.

Note that in cases (b) and (c) it is {\it not\/} true that
$1/(1- t)$ is the sum of the series, since that sum 
is not well defined.  However, in case (c) ($t = -1$),
we have $1/(1-t) = 1/2$
which is the average of the two possible partial sums
1 and 0.

Case (d).  Assume $t > 1$.  In this case, it might be more
appropriate to write
\nexteqn
$$
  s_n = \frac{t^{n+1} - 1}{t - 1}.\tag\eqn
$$
As $n \to \infty$, $t^{n+1} \to \infty$, so the series
diverges.  However, since $t^n$ increases without limit,
it would also be appropriate in this case to say the sum
is $\infty$.

Case (e).   Assume $t < -1$.   Then in (\eqn), the term
$t^{n+1}$ oscillates wildly between positive and negative
values, and the same is true of $s_n$.  Hence, $s_n$
approaches no definite limit as $n\to \infty$, and the 
series diverges.

To summarize, {\it the geometric series $\sum_{n=0}^\infty t^n$
converges to the sum $\dfrac 1{1-t}$
 for $-1 < t < 1$ and diverges otherwise.}
\endexample

\subhead Properties of Convergence \endsubhead
We list some important properties of convergent and divergent
series.
\medskip
1. {\it Importance of the tail}.  Whether or not a series converges
does not depend on any {\it finite\/} number of terms.  Thus,
\outind{tail of a series, importance of}%
one should not be misled by what happens to the first few
partial sums.  (In this context `few' might mean the first
10,000,000 terms!)  Convergence of a series depends on
the limiting behavior of the sequence of partial sums, and
no finite number of terms in that sequence will tell you
for sure what is happening in the limit.

Of course, the actual {\it sum\/} of a convergent series
depends on all the terms.
\medskip

2. {\it The general term $u_n$ of a convergent series always
approaches zero}.   For, as noted earlier
$$
   s_n = s_{n-1} + u_n.
$$
If $s_n$ approaches a finite limit $s$, then it is not hard
to see that $s_{n-1}$ approaches this same limit $s$.  Hence,
$$
\lim_{n\to \infty} u_n = \lim_{n\to\infty} s_n - \lim_{n\to \infty}
s_{n-1} = s - s = 0.
$$
This rule provides a quick check for divergence in special cases.
For example, the series
$$
\frac 12 + \frac 23 + \dots \frac n{n+1} + \dots
$$
cannot possibly converge because
$$
\lim_{n\to \infty} \frac n{n+1} = 1 \not=0.
$$
(That limit can be calculated using L'H\^opital's Rule or
rewriting $\dfrac n{n+1} = \dfrac 1{1 + 1/n}$.)
\medskip
3. {\it The harmonic series $\sum_{n=1}^\infty 1/n$ diverges.}
This is important because it shows us that the converse of
the previous assertion is not true.  That is, it is possible
that $u_n \to 0$ as $n\to \infty$ and the series still diverges.
Many people find this counter-intuitive.  Apparently, it
is hard to believe that the sum of infinitely many things
can be a finite number, and having convinced oneself that
it can happen, one seeks an explanation.  The explanation
that comes to mind is that it can happen because the
terms are getting smaller and smaller.  However, as the
example of the harmonic series shows, {\it that is not
enough}.  There are of course many other examples.
\smallskip
Here is a {\it proof\/}
 that the harmonic series diverges.  We use the principle
that if $s_n$ approaches a finite limit $s$, then any subsequence
obtained by considering infinitely many selected values of $n$
would also approach this same finite limit.   Consider, in particular,
$$\align
s_2 &= 1 + \frac 12 = \frac 32\\
s_4 &= s_2 + \frac 13 + \frac 14 > \frac 32 + 2\left(\frac 14\right) = \frac 42 \\
s_8 &= s_4 + \frac 15 + \frac 16 + \frac 17 + \frac 18 >
\frac 42 + 4\left(\frac 18\right) = \frac 52\\
s_{16} &= s_8 + \undersetbrace{8\ \text{terms}}
\to{\frac 19 + \dots +\frac 1{16}} > \frac 52 + 8\left(\frac 1{16}\right) = \frac 62\\
&\vdots
\endalign $$
Continuing in this way, we can show in general that
$$
s_{2^k} > \frac {k+2}2.
$$
Since the right hand side is unbounded as $k \to \infty$, the
sequence $s_{2^k}$ cannot approach a finite limit.    It follows that
$s_n$ cannot approach a finite limit.
\medskip
The harmonic series also provides a warning to those who would
compute without thinking.   If you naively add up the terms
of the harmonic series using a computer, the series will appear to
approach a finite sum.  The reason is that after the partial
sums attain a large enough value, each succeeding term will
get lost in round-off error and not contribute to the accumulated
sum.   The exact `sum' you will get will depend on the word
size in the computer, so it will be different for different
computers.  We shall see other examples of this phenomenon below.
The moral is to be very careful about how you add up a large
number of small terms in a computer if you want an accurate
answer.
\medskip
4. {\it Algebraic manipulation of series}.     One may perform
many of the operations with series that one is familiar
with for finite sums, but some things don't work.

For example, the sum or difference of two convergent series
is convergent.  Similarly, any  multiple of  a convergent
series is convergent.

\nextex
\example{Example \en}  Each of the series $\sum_{n=0}^\infty
\dfrac 1{2^n}$ and $\sum_{n=0}^\infty\dfrac 1{3^n}$ is a
geometric series with a $|t| < 1$.  ($t = 1/2$ for the first
and $t = 1/3$ for the second.)  Hence, both series converge
and so does the series
$$
\sum_{n=0}^\infty \left(\frac 1{2^n} + \frac 1{3^n}\right).
$$
In fact, its sum is obtained by adding the sums of the two
constituent series
$$
  \frac 1{1 - 1/2} + \frac 1{1 - 1/3} = 2 + \frac 32 = \frac 72.
$$
\endexample

\nextex
\example{Example \en}
$$
\sum_{n=0}^\infty t^{n+1} = t\sum_{n=0}^\infty t^n = t\frac 1{1-t} = 
\frac t{1-t}
$$
is valid for $-1 < t < 1$.
\endexample

On occasion, you can combine divergent series to get a
convergent series.

\nextex
\example{Example \en}  Consider the two series
$$
\sum_{n=1}^\infty \frac 1n\qquad\text{and}\qquad
\sum_{n=1}^\infty \frac 1{n+1}.
$$
The first is the harmonic series, which we saw is not convergent.
The second series is
$$
\frac 12 + \frac 13 + \frac 14 + \dots + \frac 1{n+1} + \dots
$$
which is the harmonic series with the first term omitted.  Hence,
it is also divergent.  However, the difference of these two
divergent series is
$$
\sum_{n=1}^\infty \left(\frac 1n - \frac 1{n+1}\right)
=
\sum_{n=1}^\infty \frac 1{n(n + 1)},
$$
and this series is convergent.  (This follows by one of the
tests for convergence we shall investigate in the next section,
but it may also be checked directly.  See the
Exercises.)
\medskip
However, {\it not all manipulations are valid}.
\nextex
\example{Example \en}  We saw that the series
$$
1 - 1 + 1 - 1 + 1 - 1 + \dots
$$
is not convergent.  However, we may produce a convergent series
of regrouping terms
$$
(1 - 1) + (1 - 1) + (1 - 1) + \dots = 0 + 0 + 0 + \dots
$$
or
$$
1 + (-1 + 1) + (-1 + 1) + \dots = 1 + 0 + 0 + 0 + \dots
$$
\endexample

{\it Generally speaking, regrouping of terms so as to produce a
new series won't produce correct results\/}, although in certain 
important cases it does work.  We shall return to this point in
Sections 3 and 5.    %Forward Reference
\bigskip
\input chap8.ex2
\bigskip

\nextsec{Series of Non-negative Terms}
\head \sn. Series of Non-negative Terms \endhead

Some of the strange things that happen with series result from
trying to balance one `infinity' against another.  Thus, if
infinitely many terms are positive, and infinitely many terms are
negative, the sum of the former might be `$+\infty$' while
the sum of the latter might be `$-\infty$', and the result
of combining the two is anyone's guess.   For
series $\sum_n u_n$ in which all the terms have the same sign,
 that is not a problem, so
such series are much easier to deal with.  In this section
we shall consider series in which all terms $u_n \ge 0$, but
analogous conclusions are valid for series in which all $u_n \le 0$.

Let $\sum_{n=1}^\infty u_n$ be such a series with non-negative
terms $u_n$.   In the sequence of partial sums
we have
$$
 s_n = s_{n-1} + u_n \ge s_{n-1},
$$
i.e., the next term in the sequence is never less than the
previous term.  There are only two possible ways such a sequence
can behave.
\roster
\item "(a)" The sequence $s_n$ can grow without bound.  In that case
we say $s_n \to \infty$ as $n \to \infty$.
\item "(b)"  The sequence $s_n$ can remain bounded so no $s_n$
ever exceeds some preassigned upper limit.  In that case, the
sequence must approach a finite limit $s$, i.e.,
$s_n \to s$ as $n \to \infty$.
\endroster
The fact that non-decreasing sequences behave this way is a fundamental
property of the real number system called {\it completeness}.

\medskip
\centerline{\epsfbox{s8-1.ps}}
\medskip
\example{Example}  Consider the decimal expansion of $\pi$.  It
may be considered the limit of the sequence
$$
\align
s_1 & = 3 \\
s_2 &= 3.1 \\
s_3 &= 3.14 \\
s_4 &= 3.141 \\
s_5 &= 3.1415 \\
s_6 &= 3.14159 \\
&\vdots
\endalign $$
This is a sequence of numbers which is bounded above (for
example, by 4), so (b) tells us that it approaches a finite
limit, and that limit is the real number $\pi$.
\endexample

Note that for series, with terms alternately positive and
negative, the above principle does not apply.  

\example{Example}  For the series $1 - 1 + 1 - 1 + 1 - \dots$,
the partial sums are 
$$
\align
s_1 &= 1 \\
s_2 &= 0 \\
s_3 &= 1 \\
s_4 &= 0 \\
&\vdots
\endalign
$$
so although the sequence is bounded above (and also below),
it never approaches a finite limit.

{\it For series consisting of terms all of the same sign (or zero),
rearranging the terms does not affect convergence or divergence
or the sum when convergent}.   We won't try to prove this in this
course because we would first have to define just precisely what
we mean by `rearranging' the terms of a series.   You can find a
proof in any good calculus book.  (See for example {\it Calculus\/}
by Tom M.  Apostol).

\subhead The Integral Test \endsubhead
Suppose we have a series $\sum_{n=1}^\infty u_n$ where
each term is positive ($u_n > 0$) and moreover the terms
\outind{integral test for convergence}%
decrease
$$
  u_1 > u_2 > \dots > u_n > \dots
$$
Suppose in addition that $\lim_{n \to \infty} u_n = 0$.
Often it is possible to come up with a function $f(x)$
of a real variable such that $u_n = f(n)$ for each $n$.
This function should be continuous and it should also
be decreasing and approach 0 as $x \to \infty$.
Then, there is a very simple test to check whether
or not the series converges.
\medskip
{\it $\sum_n u_n$ converges if and only if the improper integral
$\int_1^\infty f(x)\,dx$ is finite.}
\medskip
\nextex
\example{Example \en}
The harmonic series $\sum_n 1/n$ satisfies the above conditions
and we may take $f(x) = 1/x$.  Then
$$
\int_1^\infty \frac 1x dx = \lim_{X \to \infty} \int_1^X\frac {dx}x
  = \lim_{X \to \infty}\left. \ln x\right|_1^X =
  \lim_{X \to \infty} \ln X = \infty.
$$
It follows that the series diverges, a fact we already knew.
\endexample

The reason why the integral test works is fairly clear
from some pictures.  
From the diagram below
\medskip
\centerline{\epsfbox{s8-2.ps}}
\medskip
$$
s_n = u_1 + u_2 + \dots + u_n \ge \int_1^{n+1} f(x) dx 
$$
Hence, if $\int_1^\infty f(x)dx = \lim_{n \to \infty}\int_1^{n+1} f(x)dx
 = \infty$, it follows that the sequence $s_n$ is not bounded
and (a) applies, i.e., $s_n \to \infty$ and the series diverges.
On the other hand from the diagram below,
\medskip
\centerline{\epsfbox{s8-3.ps}}
\medskip

$$ s_n - u_1 = u_2 + u_3 + \dots + u_n \le \int_1^n f(x)dx.
$$
Hence, by similar reasoning if the integral is finite, the
sequence $s_n$ is bounded, so by (b) it approaches a limit,
and the series converges.

\nextex
\example{Example \en}  The series $\sum_1^\infty \dfrac 1{n^p}$
with $p > 0$
is called the `$p$-series'.  The conditions of the integral
test apply, and we may take $f(x) = 1/x^p$.  Then, with the
exception of the case $p = 1$, which we have already dealt with,
we have
$$
\int_1^\infty \frac{dx}{x^p} =
\lim_{X\to \infty}\int_1^X\frac{dx}{x^p} =
\lim_{X\to \infty}\left.\frac{x^{-p+1}}{-p + 1}\right|_1^X
 = \lim_{X\to \infty}\frac 1{1-p}[ X^{1 - p} -1].
$$
If $p > 1$, then $X^{1-p} = \dfrac 1{X^{p-1}} \to 0$
so the above limit is $\dfrac 1{p-1}$ (finite), and the
series converges.
On the other hand, if $0 < p < 1$, then $1 - p > 0$ and
$X^{1-p} \to \infty$, so the above limit is not finite and the
series diverges.
For $p = 1$, the series is the harmonic series which we already
discussed.
Hence, we conclude that the `$p$-series' diverges for
$0 < p \le 1$ and converges for $p > 1$.  
\endexample

{\bf Error estimates with the integral test.}
The integral test may be refined to estimate the speed at
which a series converges.   Suppose $\sum_n u_n$ is
\outind{integral test, error estimate}%
\outind{error estimate for sum of a series by integral test}%
convergent and its sum is $s$.   
Decompose $s$ as follows
$$
s = \undersetbrace{s_n}\to{u_1 + u_2 + \dots + u_n}
 + \undersetbrace{R_n}\to{u_{n+1} + u_{n+2} + \dots }
$$
Here, $R_n = s - s_n = \sum_{j = n+1}^\infty u_j$ is the
error you make if you stop adding terms after the $n$th term.
It is sometimes called the {\it remainder}.  From the diagram
below, you can see that
\nexteqn
$$
\int_{n+1}^\infty f(x)dx < R_n = \sum_{j=n+1}^\infty u_n < \int_n^\infty
f(x)dx.\tag\eqn
$$ 
\medskip
\centerline{\epsfbox{s8-4.ps}}
\medskip

\nextex
\example{Example \en}  Consider the series $\sum_{n=1}^\infty 1/n^2$.
We have
$$
\int_{n+1}^\infty \frac{dx}{x^2} < R_n < \int_n^\infty \frac{dx}{x^2}.
$$
Evaluating the two integrals, we obtain
$$
\frac 1{n+1} < R_n < \frac 1n.
$$
This gives us a pretty good estimate of the size of the error.  For
$n$ large, the upper and lower estimates are about the same, so
is makes sense to say $R_n \approx 1/n$.  For example, I used
Mathematica to add up 100 terms of this series and got
(to 5  decimal places) $1.63498$, but it is known that the true
sum of the series is $\pi^2/6$ which (to 5 decimal places) is $1.64493$.  As
you see, they differ in the second decimal place, and the error
is about $1/100 = .01$ as predicted.  

{\bf A common blunder}.   It is often suggested that when adding
up terms of a series on a computer, it suffices to stop when
the next term you would add  is less than the error you
are willing to tolerate.  The above example shows that this
suggestion is nonsense.  For, if you were willing to accept
an error of $.0001 = 10^{-4}$, you would then stop when $1/n^2 < .10^{-4}$
 or
$n^2 > 10^4$ or $n > 100$.  However, the above analysis shows
that the actual error at that stage would be about $1/100 = .01$.

This rule makes even less sense for a divergent series like the
harmonic series because it tells you that its sum is finite.  

\subhead The Comparison Test \endsubhead
By a variation of the reasoning used to derive the
integral test, one may derive the following criteria to
\outind{comparison test for convergence of a series}%
determine if a series $\sum_n u_n$
of non-negative terms converges or diverges.
\medskip
(a)  If you can find a convergent series $\sum_{n=1}^\infty a_n$
such that $u_n \le a_n$ for every $n$, then $\sum_{n=1}^\infty u_n$
also converges.
  \smallskip
(b)  If you can find a divergent series $\sum_{n=1}^\infty b_n$
of non-negative terms such that $b_n \le u_n$, then
$\sum_{n=1}^\infty u_n$ also diverges.
\medskip


\nextex
\example{Example \en}
The series
$\sum_{n=1}^\infty \dfrac {\ln n}{n}$ diverges because
it can be compared to the harmonic series, i.e.,
$$
   \frac{\ln n}n \ge \frac 1n\qquad\text{for } n > 1,
$$
so $\sum \ln n/n$ diverges since $\sum 1/n$ diverges.
(Why doesn't it matter that the comparison fails for $n = 1$?)
\endexample

\nextex
\example{Example \en}  Consider the series
$$
\sum_{n=1}^\infty \frac n{n^3 -n + 2}.
$$
We have
$$
\frac n{n^3 - n + 2} = \frac 1{n^2 - 1 + 2/n} < \frac 1{n^2 -1} 
 < \frac 1{n^2 - n^2/2} = \frac 1{n^2/2} = \frac 2{n^2}.
$$
The first equality is obtained by dividing numerator and denominator
by $n$.  Each inequality is obtained by making the
denominator smaller.   Since $\sum_n 2/n^2 = 2\sum_n 1/n^2$
and $\sum_n 1/n^2$ converges, so does $\sum_n 2/n^2$.  Hence, the
comparison test tells us that $\sum_n n/(n^3 - n + 2)$ converges.
\endexample

The last example illustrates a common strategy when
applying the comparison test.
We try to estimate the approximate behavior of $u_n$ for large
$n$ and thereby come up with an appropriate comparison series.
We then try (usually by a tricky argument) to show that the given
series is less than the comparison series (if we expect convergence)
or greater than the comparison series (if we expect divergence.)
Unfortunately, it is not always easy to dream up the relevant
comparisons.  Hence, it is often easier to use the following
version of the comparison test.

\proclaim{The Limit Comparison Test} Suppose $\sum_n u_n$ and
$\sum_n c_n$ are series of non-negative terms such that
$\lim_{n\to \infty} \dfrac {u_n}{c_n}$ exists and is not
0 or $\infty$.  Then $\sum_n u_n$ converges if and only if
\outind{limit comparison test for convergence of a series}%
$\sum_n c_n$ converges.  If the limiting ratio is 0, and
$\sum_n c_n$ converges, then $\sum_n u_n$ converges.  If the
limiting ratio is $\infty$, and $\sum_n c_n$ diverges, then
$\sum_n u_n$ diverges.
\endproclaim

\example{Example \en, again}  $u_n = n/(n^3 - n + 2)$ looks
about like $c_n = 1/n^2$.  Consider the ratio
$$
\frac{u_n}{c_n} = \frac n{n^3 - n + 2}\frac {n^2} 1
   = \frac{n^3}{n^3 - n + 1} = \frac 1{1 - 1/n^2 + 2/n^3} \to 1
$$
as $n \to \infty$.  Since 1 is finite and non-zero, and since
the comparison series $\sum_n 1/n^2$ converges, the series
$\sum_n n/(n^3 - n + 2)$ also converges.
\endexample

It is important to emphasize that this test does not work
for series for which some terms are positive and some are
negative.

%Advice to instructor:  omit this proof in class.
\subhead The Proof \endsubhead
The proof of the limit comparison test is fairly straightforward,
but if you are not planning to be a Math major and you are
willing to accept such things on faith, you should skip it.
Suppose
$$
\lim_{n\to \infty} \frac{u_n}{c_n} = r\qquad\text{with }
0 < r < \infty.
$$
Then, 
for all $n$ sufficiently large
$$
\frac r2  < \frac{u_n}{c_n} < 2r.
$$
(All the values of $u_n/c_n$ have to be very close to $r$ when
$n$ is large enough.)   Hence, for all $n$ sufficiently large,
$$
\frac r2 c_n < u_n < 2rc_n
$$
If $\sum_n c_n$ converges, then so does $\sum_n 2rc_n$, so by
the comparison test, so does $\sum_n u_n$.   By similar reasoning,
if $\sum_n c_n$ diverges, so does $\sum_n u_n$.   Note that we
have used the remark about the importance of
the tail---see Section 2---which asserts that when deciding
on matters of convergence or divergence, it is enough to look only
at all sufficiently large terms.
%Exer Proofs for the cases of zero or infinite limit.

\bigskip
\input chap8.ex3
\bigskip
\nextsec{Alternating Series}
\head \sn. Alternating Series \endhead

A series with alternating positive and negative terms is called
an {\it alternating series}.  Such a series is usually
represented
$$
  v_1 - v_2 + v_3 - v_4 + \dots
$$
where $v_1, v_2, v_3, \dots$ are all positive.  Of course, variations
of this scheme are possible.  For example, the first term might have
an index other than 1, and also there is no reason not to start
\outind{alternating series}%
with a negative term.   However, to simplify the discussion of
the general theory, we shall assume the series is as presented
above.  For alternating series, we start to encounter the problem
of `balancing infinities', but they are still relatively well
behaved.  Also, many series important in applications are
alternating series.

\example{Examples {\rm of Alternating Series}}
\nexteqn\xdef\Axx{\eqn}
\nexteqn\xdef\B{\eqn}
\nexteqn\xdef\Ce{\eqn}
$$
\gather
\sum_{n=0}^\infty t^n \qquad\text{if } t\ \text{is negative}\tag{\Axx}\\
\sum_{n=1}^{\infty}\frac{(-1)^{n+1}}n = 1 - \frac 12 + \frac 13 - \dots\tag\B \\
\sum_{n=0}^\infty (-1)^n\frac{x^{2n}}{(2n)!} = 1 - \frac{x^2}{2!}
    + \frac{x^4}{4!} - \frac{x^6}{6!} + \dots\tag{\Ce}
\endgather
$$
(\Axx) is a special case of the geometric series, and we shall
see later that (\Ce) is a series
for $\sin x$.
\nextthm
\proclaim{Theorem \cn.\tn}
 Let $v_1 - v_2 + v_3 - \dots$ be an alternating series
with
$$
  v_1 > v_2 > \dots > v_n > \dots \qquad\text{(all $ > 0$)}.
$$
If $\lim_{n\to \infty} v_n = 0$, then the series converges.
If $s$ is the sum of the series, then the error $R_n = s - s_n$
after $n$ terms satisfies
$$
|R_n| < v_{n+1}.
$$
That is, the absolute value of the error after $n$ terms is
bounded by the next term of the series.
\outind{error estimate for sum of an alternating series}%
\endproclaim

\example{Example}  The series
$$
1 - \frac 12 + \frac 13 - \dots + (-1)^{n+1}\frac 1n + \dots
$$
converges since $\dfrac 1n \to 0$.  Also,
$$
s = \undersetbrace{s_n}\to{1 - \frac 12 + \frac 13 - \dots
   + (-1)^{n+1}\frac 1n} + R_n
$$
where $|R_n| < \dfrac 1{n+1}$.  According to Mathematica,
for $n = 100$, $s_n =  0.688172$ (to 6 decimal places), and
the theorem tells us the error is less than $1/100 = .01$.
Hence, the correct value for the sum is somewhere between
$.67$ and $.69$.  

Suppose on the other hand, we want to use enough terms to
get an answer 
such that the error will be less than $.0005 = 5\times 10^{-4}$.
For that, we need  
$$
\gather
\dfrac 1{n+1} \le 5\times 10^{-4}\qquad\text{or}\\
n + 1 \ge \dfrac 1{5\times 10^{-4}} = \frac 15 10^4 = 2\times 10^3.
\endgather $$
Thus, $n = 2000$ would work.
\endexample

\noindent {\it An aside on rounding.}

In the above example, we asked for enough terms to be sure that
\outind{rounding}%
the error is less that $5\times 10^{-4}$.  This is {\it different
from\/}
asking that it be accurate to three decimal places, although the
two questions are closely related.   For example, suppose the
true answer to a problem is $1.32436...$  An error of $+.0004$
would change this to $1.32476..$ which would be rounded {\it up\/}
to the incorrect answer $1.325$ if we used only three decimal places.
An alternative to rounding is {\it truncating\/}, i.e. cutting off
\outind{truncation}%
all digits past the desired one.  That would solve our problem in
the case of a positive error, but an error of $-.0004$ would in
this case result in the incorrect answer  $1.323$ if we truncated.

By using extra decimal places, it is possible to reduce the likelihood
of a false rounding error, but it is not possible to eliminate
it entirely.  For example, suppose the true answer 
is $2.426499967$ and we insist on
enough terms so that the error is less that $5\times 10^{-8}$.
Then, an error of
 $+.00000004$ would cause us to round up incorrectly to
$2.427$.

Since the issue of rounding can be so confusing, we shall instead
generally ask that the error be smaller that five in the next
decimal position.  You should generally read a request for a
certain number of decimal places in this way but you should also
remember that rounding may change the answer a bit in such cases.


{\bf The Proof.}
The proof of the theorem is a little tricky, but it is
illuminating enough that you should try to follow it.  
You will get a better idea of how the error behaves,
and that is important.   The diagram below
makes clear what happens.
%D
\medskip
\centerline{\epsfbox{s8-5.ps}}
\medskip
Here is the same thing in words.
\nexteqn
If $n$ is odd,
$$
  s_{n-1} < s_{n-1} + \undersetbrace{> 0}\to{u_n - u_{n-1}} = s_{n+1}.\tag\eqn
$$
Hence, the 
even numbered partial sums $s_n$
form an increasing sequence. Similarly,
\nexteqn
$$
 s_{n+1} = s_{n-1} -\undersetbrace{> 0}\to{(u_n - u_{n-1})} < s_{n-1},\tag\eqn
$$
so  the odd numbered partial sums $s_n$ form a decreasing sequence.
Also, if $n$ is odd,
$$
s_{n+1} = s_n - v_{n+1} < s_n,
$$
so it follows that every odd partial sum is greater than every
even partial sum:
$$
s_2 < s_4 < s_6 < \dots < s_7 < s_5 < s_3 < s_1.
$$
The even partial sums form an increasing sequence, bounded above,
so by
the completeness property (b), they must approach a finite limit
$s'$.   By similar reasoning,  since the odd partial sums form a
{\it decreasing\/} sequence, bounded below, they also approach a finite
limit $s''$.  Moreover, it is not too hard to see that for $n$ even
$$
     s_n \le s' \le s'' \le s_{n+1}.
$$
However, $s_{n+1} - s_n = v_{n+1} \to 0$ as $n \to \infty$.
It follows that $s''-s' \to 0$ as $n \to \infty$, but since
$s'' - s'$ does not depend on $n$, it must be equal to zero.
In other words the even partial sums {\it and\/} the odd partial
sums must approach the {\it same\/} limit $s = s' = s''$.
It follows that $\lim_{n \to \infty} s_n = s$, and the series
converges.   

\mar{s8-6.ps}
Note also that we get the following refined error estimate from the
above analysis.
$$\gather
   s_n < s < s_n + v_{n+1}\qquad\text{if $n$ is even} \\
   s_n - v_{n+1} < s < s_n\qquad\text{if $n$ is odd}. 
\endgather
$$
%Exer  Use the refined error analysis to estimate $\sum_{n=1}^{1000}$
%Exer  by the sharpest possible inequalities.

\bigskip
\input chap8.ex4
\bigskip
\nextsec{Absolute and Conditional Convergence}
\head \sn.  Absolute and Conditional Convergence \endhead

A series $\sum_n u_n$ is called {\it absolutely convergent\/}
\outind{absolute convergence of a series}%
\outind{conditional convergence of a series}%
if the series $\sum_n |u_n|$ converges.  If $\sum_n u_n$ converges,
but $\sum_n |u_n|$ diverges, the original series is called
{\it conditionally convergent}.  Absolutely convergent series
are quite well behaved, and one can manipulate them almost
as though they were finite sums.   Conditionally convergent
series, on the other hand, are often hard to deal with.

\example{Examples}  The series 
$$
 1 - 1/4 + 1/9 - \dots + (-1)^{n+1}(1/n^2) + \dots = \sum_{n=1}^\infty
\frac{(-1)^{n+1}}{n^2}
$$
is absolutely convergent because $\sum_n 1/n^2$ converges.
The series
 $$1 - 1/2 + 1/3 - \dots + (-1)^{n+1}(1/n) + \dots = \sum_{n=1}^\infty
\frac{(-1)^{n+1}}n
 $$
is conditionally convergent.  It does converge by the alternating
series test, but the series of absolute values $\sum_n 1/n$ diverges.
\endexample

Of course, for series with non-negative terms, absolute convergence
means the same  thing as convergence.  Also, for series with all terms
$u_n \le 0$, absolute convergence means the same thing as convergence.
(Just consider the series $-\sum_n u_n = \sum_n (-u_n)$ for which all
the terms are non-negative.)

\nextthm
\proclaim{Theorem \cn.\tn}  If $\sum_n |u_n|$ converges,
then $\sum_n u_n$ converges.  That is, an absolutely convergent
series is convergent.
\endproclaim

This theorem may look `obvious', but that is the result
of the choice of terminology.  The relation between the
series $\sum_n u_n$ and the corresponding series of absolute
values  $\sum_n |u_n|$ is rather subtle.   For example, even
if they both converge, they certainly won't have the same
sum.  

{\bf The proof.}
Assume $\sum_n |u_n|$ converges.  Define two new series as
follows.   Let
$$
\align
p_n &= \left\{\aligned u_n\quad&\text{if $u_n > 0$}\\
                       0\qquad&\text{otherwise}\endaligned \right. \\
q_n &= \left\{\aligned |u_n| =  -u_n\quad&\text{if $u_n < 0$}\\
                       0\qquad&\text{otherwise}\endaligned \right. .
\endalign $$
Then $\sum_n p_n$ is the series obtained when all negative terms of
the original series are replaced by 0's, and $\sum_n q_n$ 
is the {\it negative\/} of the series obtained if all positive
terms of the original series are replace by 0's.  Both are
series of non-negative terms, and
$$
u_n = p_n + (-q_n) = p_n - q_n\qquad\text{for each } n.
$$
(Either $p_n$ or $-q_n$ is zero, and the other is $u_n$.)
Hence, to show that $\sum_n u_n$ converges, it would suffice
to show that both $\sum_n p_n$ and $\sum_n q_n$ converge.
Consider the former.  For each $n$, we have
$$
0 \le p_n \le |u_n|.
$$
Indeed, $p_n$ equals one or the other of the bounds depending on
the sign of $u_n$.   Hence, by the comparison test, since
$\sum_n |u_n|$ converges, so does $\sum_n p_n$.   A similar
argument shows that $\sum_n q_n$ converges.
 Hence, we conclude that 
$\sum_n u_n$ converges.  That completes the proof.
\medskip

One consequence of the above argument is that for absolutely
convergent series, rearranging the terms of the series does
not affect convergence or divergence or the sum when convergent.
The reason is that  for an absolutely convergent series,
\nexteqn
$$
\sum_n u_n = \sum_n p_n - \sum_n q_n,\tag\eqn
$$
and each of the series on the right is
a convergent series of non-negative
terms.   For such series, rearranging terms
is innocuous.   Suppose on the other hand that
the two series $\sum_n p_n$ and $\sum_n q_n$ are
both divergent.   Then (\eqn) would assert that the $\sum_n u_n$
is the difference of two `infinities', and any attempt to
make sense of that is fraught with
peril.  
  However, in that case
$$
\sum_n |u_n| = \sum_n p_n + \sum_n q_n
$$
is a sum of two divergent series of non-negative terms and certainly
doesn't converge, so the series $\sum_n u_n$ is not absolutely
convergent.

\subhead  The Ratio Test \endsubhead
There is a fairly simple test which will establish absolute
convergence for series to which it applies.   Consider the
ratio
$$
    \frac {|u_{n+1}|}{|u_n|}
$$
of succeeding terms of the series of absolute values.  Suppose this
approaches a finite limit $r$.  (Note that we must have
$r \ge 0$ since it is a limit of non-negative terms.) 
That means that, for large $n$,
the series $\sum_n |u_n|$ looks very much like a geometric series
of the form
$$
   \sum_n ar^n = a\sum_n r^n.
$$
However, the geometric series converges exactly when $0 \le r < 1$.
We should be able to conclude from this that the original
series also converges exactly in those circumstances.  Unfortunately,
because the comparison involves only a limiting ratio, the analysis
is not precise enough to tell us what happens when $r = 1$.  The
precise statement of the test is the following.

\proclaim{The Ratio Test}  Suppose
$$
r = \lim_{n\to\infty} \frac {|u_{n+1}|}{|u_n|}
$$
exists.  If $r < 1$, the series $\sum_a u_n$ is absolutely
convergent.  If $r > 1$, the series $\sum_n u_n$ diverges.
If $r = 1$, the test provides no information.
\endproclaim
\outind{ratio test for convergence of a series}%

\nextex
\example{Example \en}
Consider the series
$$
\sum_{n=0}^\infty \frac{x^n}{n!}.
$$
(As you may already know,
this is the series for $e^x$.)  To apply the ratio test, we
need to consider
$$
\frac{ |x^{n+1}/(n+1)!|}{|x^n/n!|} =
\frac{|x|^{n+1}}{(n+1)!}\,\frac{n!}{|x|^n} = \frac{|x|}{n+1}.
$$
The limit of this as $n \to \infty$ is $r = 0 < 1$.  Hence, by the
ratio test, the series $\sum_n x^n/n!$ is absolutely convergent
for every possible $x$.
\endexample

\nextex
\example{Example \en}  Consider the series
$$
\sum_{n=1}^\infty (-1)^n \frac {x^n}n.
$$
(As we shall see this is the series for $\ln(1 + x)$ at least when
it converges.)  The ratio test tells us to consider
$$
\frac{|(-1)^{n+1}x^{n+1}/n+1|}{|(-1)^nx^n/n|}
= \frac{|x|^{n+1}}{n+1}\,\frac n{|x|^n} = \frac n{n+1} |x|.
$$
Since, $\lim_{n\to\infty}\dfrac n{n+1} = 1$, the limiting ratio
in this case is $|x|$.  Thus, the series $\sum_{n=1}^\infty
(-1)^nx^n/n$ converges absolutely if $|x| < 1$ and diverges
if $|x| > 1$.   Unfortunately, the ratio test does not tell
us what happens when $|x| = 1$.    However, {\it we can settle
those case by using other criteria}.  Thus, for $x = 1$.
the series is
$$
\sum_{n=1}^\infty \frac{(-1)^n}n
$$
and that converges by the alternating series test.  On the other
hand, for $x = -1$, the series is
$$\sum_{n=1}^\infty (-1)^n \frac{(-1)^n}n = \sum_n\frac{(-1)^{2n}}n
= \sum_n \frac 1n
$$
which is the harmonic series, so it diverges.
\endexample

The detailed proof that the ratio test works is a bit subtle.  We
include it here for completeness, but you will be excused if you
skip it.
%Advice to instructor:  Skip in class.

{\bf The proof.}
Suppose $\lim_{n\to\infty}\dfrac{|u_{n+1}|}{|u_n|} = r$.
That means that if $r_1$ is slightly less than $r$,
 and $r_2$ is slightly greater than $r$, then 
for all  sufficiently large $n$
$$
     r_1 < \frac{|u_{n+1}|}{|u_n|} < r_2.
$$
Moreover, by making $n$ large enough, we can arrange for the
numbers $r_1$ and $r_2$ to be as close to $r$ as we might like.

By the general principle that the tail of a series dominates
in discussions of convergence, we can ignore the finite number
of terms for
which the above inequality does not hold. 
  Renumbering, we may assume
$$
r_1|u_n| < |u_{n+1}|< r_2|u_n|
$$
holds for all $n$ starting with $n = 1$.  Then,
$$
  |u_2| < r_2|u_1|, |u_3|< r_2|u_2| < r_2{}^2|u_1|, \dots,
 |u_n| < |u_1|r_2{}^{n-1}
$$
and similarly for the lower bound, so putting $a = |u_1|$,
 we may assume
\nexteqn
$$
   ar_1{}^{n-1} < |u_n| < ar_2{}^{n-1}\tag\eqn
$$
for all $n$.  Suppose $r < 1$.  Then, by taking $r_2$
sufficiently close to $r$, we may assume that $r_2 < 1$.
 So we may use (\eqn) to compare $\sum_n |u_n|$ to
$\sum_n ar_2{}^{n-1}$ which is a convergent geometric series.
Hence, $\sum_n |u_n|$ converges.  Suppose on the other hand
that $r > 1$.  Then by choosing $r_2$ sufficiently close to
$r$, we may assume $r_2 > 1$.   Putting this in (\eqn)
yields
$$
   1 < |u_n|
$$
for all $n$.  Hence, $u_n \to 0$ as $n \to \infty$ is not
possible, and $\sum_n u_n$ must diverge.

Note that the above argument breaks down if $r = 1$.  In that
case, we cannot assume that $r_2 < 1$ or $r_1 > 1$, so
neither part of the argument works.

\subhead The Root Test \endsubhead
There is a test which is related to the ratio test which is
a bit simpler to use.  Instead of considering the ratio of
successive terms, one considers the $n$th root $|u_n|^{1/n}$.
If this approaches a finite limit $r$, then roughly speaking,
for large $n$, $|u_n|$ behaves like $r^n$.   Thus, we are led to compare
$\sum_n |u_n|$ to the geometric series $\sum_n r^n$.  We
get convergence or divergence, as in the ratio test,
which depends on the value of the limit $r$.  As in the ratio
test, the analysis is not precise enough to tell us what
happens if that limit is 1.

\proclaim{The Root Test}  Suppose $\lim_{n\to\infty}|u_n|^{1/n} = r$.
If $0\le r < 1$, then $\sum_n u_n$ converges absolutely.  If
$1 < r$, then $\sum u_n$ diverges.  If $r = 1$, it may converge or
diverge.
\endproclaim
\outind{root test for convergence of a series}%

\nextex
\example{Example \en}  Consider the series
$$
x + \frac {x^2}4 + \frac {x^3}9 + \dots = 
\sum_{n=1}^\infty \frac{x^n}{n^2}.
$$
We have
$$
\left|\frac {x^n}{n^2}\right|^{1/n} = \frac{|x|}{n^{2/n}}
$$
so in this case
$$
\lim_{n\to\infty} |u_n|^{1/n} = |x|\lim_{n\to\infty} n^{-2/n}.
$$
The limit on the right is evaluated by a tricky application
of L'H\^opital's rule.   The idea is that if $\lim_{n\to\infty} \ln a_n
= L$, then $\lim_{n\to\infty} a_n = e^L$.  In this case,
$$
\lim_{n\to\infty} \ln(n^{-2/n}) = \lim_{n\to\infty}(-\frac 2n \ln n)
= \lim_{n\to\infty}\frac{-2\ln n}n = \lim_{n\to\infty}\frac{-2/n}1 = 0.
$$
(The next to last step was done using L'H\^opital's rule.)  Hence,
$$
\lim_{n\to\infty} n^{-2/n} = e^0 = 1.
$$
It follows that the limiting ratio for the series is $r = |x|$.
Thus, if $|x| < 1$ the series $\sum_n \dfrac{x^n}{n^2}$ converges
absolutely, and if $|x| > 1$, it diverges.  For $|x| = 1$,
{\it we must settle the question by other means}.  For
$x = 1$, the series is $\sum_n 1/n^2$ which we know converges.
(It is a `$p$-series' with $p = 2 > 1$.)  For $x = -1$, 
the series is $\sum_n (-1)^n/n^2$, and we just decided its
series of absolute values converges.  Hence, $\sum_n (-1)^n/n^2$
converges absolutely.  (You could also see it converges by the
alternating series test, but absolute convergence is stronger.)
\endexample

The proof of the root test is similar to that of the ratio test.
Again you will be excused if you skip it, but here it is.

{\bf The proof.}
Suppose $\lim_{n\to\infty} |u_n|^{1/n} = r$.   If
$r_1$ is slightly less than $r$ and $r_2$ is slightly larger,
then, for $n$ sufficiently large,
$$
   r_1 < |u_n|^{1/n} < r_2.
$$
Again, by the principle that we may ignore finitely many terms of
a series when investigating convergence,
we may assume the inequalities holds for all $n$. Raising to the $n$ power,
we get
$$
   r_1{}^n < |u_n| < r_2{}^n.
$$
If $r < 1$, we may assume $r_2 < 1$, and compare $\sum_n |u_n|$
with a convergent geometric series.   On the other hand, if
$1 < r$, we may assume $1 < r_2$, and then $|u_n| \to \infty$
which can't happen for a convergent series.
\bigskip
\input chap8.ex5
\bigskip

\nextsec{Power Series}
\head \sn. Power Series \endhead

At the beginning of this chapter, we saw that we might want to
consider series of the form $\sum_{n=0}^\infty a_n t^n$ as solutions
of differential equations.  A bit more generally, if we were
given initial conditions  at $t = t_0$, we might
consider series of the form
$$
\sum_{n = 0}^\infty a_n (t - t_0)^n.
$$
Such a series is called a {\it power series\/} centered at $t_0$.
\outind{power series}%

A power series will generally converge for some, but perhaps not all,
values of the variable.  Usually, you can determine those
values for which it converges by an application of the ratio or
root test.

\nextex
\example{Example \en}  Consider the power series
$$
\sum_{n=0}^\infty \frac n{2^n} (t - 3)^n.
$$
To apply the ratio test, consider
$$
\frac{\dfrac{n+1}{2^{n+1}}|t - 3|^{n+1}}{\dfrac n{2^n} |t - 3|^n}
= \frac{|t - 3|}2 \frac {n + 1}n.
$$
However,
$$
\lim_{n\to \infty} \frac {n + 1}n = 1
$$
(by L'H\^opital's rule, or by dividing numerator and denominator by $n$).
Hence, the limiting ratio is $|t - 3|/2$.  Hence, the series converges
absolutely if
$$
 \gather
\frac {|t - 3|}2 < 1 \\
\text{i.e.,}\qquad |t - 3| < 2 \\
\text{i.e.,}\qquad  -2 < t - 3 < 2 \\
\text{i.e.,}\qquad 1 < t < 5.
\endgather
$$
Similarly, if $|t - 3|/2 > 1$, the series diverges.  That inequality
amounts to $t < 1$ or $5 < t$.   The case $|t - 3|/2 = 1$,
i.e., $t = 1$ or $t = 5$, must be handled separately.  I leave the
details to you.  It turns out that the series 
diverges both for $t = 1$ and $t = 5$.
Hence, the exact range of convergence is
$$
   1 < t < 5
$$
and the series converges absolutely on this interval.
\medskip
\centerline{\epsfbox{s8-7.ps}}
\medskip
%D diagram
The above analysis could have been done equally well using the
root test.
\endexample

The behavior exhibited in Example \en, or in similar  examples
in the previous section, is quite general.  For any power series
$\sum_n a_n (t - t_0)^n$
there is a number $R$ such that  the series converges absolutely
in the interval
$$
   t_0 - R < t < t_0 + R,
$$
diverges outside that interval, and may converge or diverge at
either endpoint.  $R$ is called the {\it radius of convergence\/}
\outind{radius of convergence of a power series}%
of the series.  (The terminology comes from the corresponding
concept in complex analysis where the condition $|z - z_0| < R$
characterizes all points in the complex plane
 inside a circular disk of radius $R$
centered at $z_0$.)
  $R$ may be zero, in
which case the series converges only for $t = t_0$ (where 
all terms but the first are 0).   $R$ may be infinite, in which case
the series converges for all $t$.  In many cases, it is finite,
and it may be important to know its value.
\medskip
\centerline{\epsfbox{s8-8.ps}}
\medskip
%D diagram
Understanding how the ratio or root test is used to determine
the radius of convergence gives you a good idea of why a power
series converges in a connected interval rather than at a random
collection of points,
so you should work enough examples to be sure
you can do such calculations.
Unfortunately, in the general case, neither the ratio nor the root
test may apply, so the argument
justifying the existence of an interval of convergence is a bit
tricky.
%Advice to instructor.  Skip this.
We shall outline the argument here, but you need not study it at
this time if the examples satisfy you that it all works.

{\bf The argument why the range of convergence is an interval.}
If the series converges only for $t = t_0$, then we take $R = 0$,
and we are done.  Suppose instead that there is a $t_1 \not=t_0$
at which the series $\sum_n a_n (t_1 - t_0)^n$ converges.  
Then the general term
$$
    a_n (t_1 - t_0)^n \to 0
$$
as $n \to \infty$.  One consequence is that the absolute value
of the general term must be bounded, i.e., there is a bound $M$ such
that
$$
      |a_n||t_1 - t_0|^n < M
$$
for all $n$.  Put $R_1 = |t_1 - t_0|$.
 Then, if $|t - t_0| < R_1$,
$$
   |a_n| |t - t_0|^n = 
|a_n||t_1 - t_0|^n \left(\frac{|t - t_0|}{R_1}\right)^n
< Mr^n
$$
where $r = |t - t_0|/R_1 < 1$.  Comparing with a geometric
series, we see that $\sum_n |a_n||t - t_n|^n$ converges, so
$\sum_n a_n (t - t_0)^n$ is absolutely convergent
 as long as $|t - t_0| < R_1$.      

Consider next all possible $R_1 > 0$ such that the series
converges absolutely in the range $t_0 - R_1 < t < t_0 + R_1$.
If there is no upper bound to this set of numbers, then the
series converges absolutely for all $t$.  In that case, we set
the radius of convergence $R = \infty$.  Suppose instead that
there is some upper bound to the set of all such $R_1$.  By
a variation of the completeness arguments we have used before
(concerning bounded increasing sequences), it is possible to
show in this case that there is a single value $R$ such that
$\sum_n a_n (t - t_0)^n$ converges absolutely for $|t - t_0| < R$,
but that the series diverges for $|t - t_0| > R$.  This $R$
is the desired radius of convergence.

\subhead Differentiation and Integration of power Series \endsubhead
The rationale behind the use of series to solve differential equations
is that they are generalizations of polynomial functions.  By the
usual rules of calculus, polynomial functions are easy to differentiate
or integrate.  For example, if
$$
\align
f(t) &= a_0 + a_1(t - t_0) + a_2(t - t_0)^2 + \dots + a_n(t - t_0)^n + \dots \\
\intertext{then}
f'(t) &= 0 + a_1\,1 + 2a_2(t - t_0) + \dots + na_n(t - t_0)^{n-1} + \dots
\endalign $$
That is, to differentiate a polynomial (or any finite sum), you differentiate
each term, and then add up the results.  A similar rule works for integration.
Unfortunately, these rules do not always work for infinite sums.  It
is possible for the derivatives of the terms of a series to add up to
something other than the derivative of the sum of the series.  

%Advice to instructor:  Example below may be skipped.
\example{Example}
Consider the series
$$
\sum_{n=1}^\infty \left(\frac{\sin nt}n - \frac{\sin (n+1)t}{n+1}\right).
$$
The partial sums are
$$\align
s_1(t) &= \sin t - \frac 12 \sin 2t \\
s_2(t) &= \sin t - \frac 12 \sin 2t + \frac 12 \sin 2t - \frac 13 \sin 3t
     = \sin t - \frac 13 \sin 3t \\
&\vdots \\
s_n(t) &= \dots = \sin t - \frac 1{n+1}\sin (n+1)t\\
&\vdots
\endalign $$
However,  $\dfrac{|\sin (n+1)t|}n \le \dfrac 1n$ for any $t$,
so its limit is 0 as $n \to \infty$.  Hence,
$$
\lim_{n\to\infty} s_n(t) = \sin t
$$
so the series converges for every $t$ and its sum is $\sin t$.
On the other hand, the series of derivatives is
$$
\sum_{n=1}^\infty (\cos nt - \cos (n+1)t).
$$
The partial sums of this series are calculated essentially
the same way, and
$$
s_n'(t) = \cos t - \cos (n+1)t.
$$
For most values of $t$, $\cos (n+1)t$ does not approach a
definite limit as $n\to \infty$.  (For example, try $t = \pi/2$.
You alternately get $0, -1$, or 1 for different values of $n$.)
Hence, the series of derivatives is generally not even a convergent
series.

Even more bizarre examples exist in which the series of derivatives
converges but to something other than the derivative of the sum
of the original series.
\endexample

Similar problems arise when you try to integrate a series term
by term.  Some of this will be discussed when you study Fourier
Series and Boundary Value Problems, next year.
\medskip
   Fortunately, for
a power series, {\it within its interval of absolute convergence\/}
the derivative of the sum is the sum of the derivatives and similarly for
integrals.   This fact is fundamental for any attempt to use power
series to solve a differential equation.  It may also be used to
make a variety of other calculations with series.

\nextthm
\proclaim{Theorem \cn.\tn}  Suppose $\sum_{n=0}^\infty a_n(t - t_0)^n$
converges absolutely to
 $f(t)$ for $|t - t_0| < R$.  Then

(a)  $f'(t) = \sum_{n=1}^\infty na_n(t - t_0)^{n-1}$ for
$|t - t_0| < R$.

(b)  $\int_{t_0}^t f(s) ds = \sum_{n=0}^\infty \dfrac {a_n}{n+1}(t - t_0)^{n+1}$
for $|t - t_0| < R$.
\outind{radius of convergence of a power series}%
\outind{differentiation of a power series}%
\outind{integration of a power series}%
Moreover, the series $\sum_{n=1}^\infty na_n(t - t_0)^{n-1}$
and $\sum_{n=0}^\infty \dfrac {a_n}{n+1}(t - t_0)^{n+1}$ have the
same radius of convergence as $\sum_{n=0}^\infty a_n(t - t_0)^n$.
\endproclaim

The proof of this theorem is a bit involved.  An outline is given in
the appendix to this section, which you may want to skip.

\nextex
\example{Example \en}
We know
$$
\frac 1{1 - t} = \sum_{n=0}^\infty t^n \qquad\text{for } |t| < 1.
$$
(1 is the radius of convergence.)  Hence,
$$
\frac 1{(1 - t)^2} = \sum_{n=1}^\infty nt^{n-1} = \sum_{n=0}^\infty
   (n+1)t^n.
$$
The last expression was obtained by substituting $n + 1$ for
$n$.  This has the effect of changing the lower limit `$n = 1$' 
to `$n + 1 = 1$' or `$n = 0$'.  Such
substitutions are often useful when manipulating power series,
particularly in the solution of differential equations.

The differentiated series also converges absolutely for $|t| < 1$.
\endexample
\nextex
\example{Example \en}  If replace $t$ by $-t$ in the formula
for the sum of a geometric series, we get
$$
\frac 1{1 + t} = \sum_{n=0}^\infty (-t)^n = \sum_{n=0}^\infty (-1)^n t^n
$$
and this converges absolutely for $|-t| = |t| < 1$.  Hence, by (b),
we get 
$$
\int_0^t \frac{ds}{1 +s} = \sum_{n=0}\frac{(-1)^n}{n+1}t^{n+1}
    =  \sum_{n=1}^\infty \frac{(-1)^{n+1}}n t^n
$$
and this is valid for $|t| < 1$.  Note the shift obtained by
replacing $n$ by $n - 1$.   Doing the integral on the left
yields
$$
\ln (1 + t) = \sum_{n=1}^\infty \frac{(-1)^{n+1}}n t^n =
 t -\frac{t^2}2 + \frac{t^3}3 + \dots\qquad\text{for } |t| < 1.
$$

Series of this kind may be used to make numerical calculations.
For example, suppose we want to calculate $\ln 1.01$ to
within $5\times 10^{-6}$, i.e., {\it loosely speaking\/}, we want
the answer to be accurate to five decimal places.   We can use the series
$$
\ln (1 + .01) =  .01 -\frac{(.01)^2}2 + \frac{(.01)^3}3 + \dots + 
(-1)^{n+1}\frac {(.01)^n}n + \dots
$$
and include enough terms so the error $R_n$ satisfies $|R_n| < 5\times 10^{-6}$.
Since the series is an alternating series with decreasing terms,
we know that $|R_n|$ is bounded by the absolute value of
the next term in the series,
i.e., by  $\dfrac{(.01)^{n+1}}{n+1}$.  Hence, to get the desired accuracy,
it would suffice to choose $n$ large enough so that
$$
      \frac{(.01)^{n+1}}{n+1} = \frac{10^{-2(n+1)}}{n+1}   < 5 \times 10^{-6} .
$$
There is no good way to determine such an $n$ by a deductive
process, but trial and error works reasonably well.  Certainly,
$n = 2$ would be good enough.   Let's see if $n = 1$ would
also work.
$$
\frac{10^{-4}}2 = .5\times 10^{-4} = 5\times 10^{-5},
$$
and that is not small enough.  Hence, $n = 2$ is the best
that this particular estimate of the error will give us.
Hence,
$$
\ln(1.01) = .01 - \frac{(.01)^2}2 
 \approx 0.00995.
$$
You should check this with your calculator to see if it is
accurate.  (Remember however that the calculator is also using
some approximation, and it doesn't know the true value any
better than you do.)
\endexample

The estimate $R_n$ of the error after using $n$ terms of a
series is tricky to determine.  For alternating series,
we know that the next term rule applies, but we shall see
examples later of where much more sophisticated methods
need to be used.

\subhead Appendix.  Proof of the Theorem \endsubhead

We first note that it is true that the sum of a power series
within its radius of convergence is always a {\it continuous
function}.   The proof is not extremely hard, but we shall
omit it here.   You may see some further discussion of
this point in your later mathematics courses.
We shall use this fact implicitly several places in what
follows.  In particular, knowing the sum of a series is
continuous allows us to conclude it is integrable and also
to apply the fundamental theorem of calculus.

Statement(a) in the
theorem may be proved once statement (b) has been established.  
To see this, argue
as follows.  Let $\sum_{n=0}^\infty a_n(t - t_0)^n = f(t)$
have radius of convergence $R > 0$.
  Consider the differentiated series $\sum_{n=1}^\infty
 na_n(t - t_0)^{n-1}$.   If the ratio (or root) test applies,
it is not hard to see that this series has the same radius
of convergence $R$ as the original series.  (See the Exercises.)
If the ratio test does not apply, there is a tricky argument
to establish this fact in any case.  Suppose that point is
settled.   Define
$$
g(t) = \sum_{n=1}^\infty  na_n(t - t_0)^{n-1}\quad\text{for } 
|t - t_0| < R.
$$
(We hope that $g(t) = f'(t)$, but we don't know that yet.)
Assume (b) is true and apply it to the series for $g(t)$.  We
get
$$
\int_{t_0}^t g(s)ds = \sum_{n=1}^\infty \frac nn a_n(t - t_0)^n
= \sum_{n=1}a_n(t - t_0)^n.
$$
However,
$$
\align
f(t) &= \sum_{n=0}^\infty a_n(t - t_0)^n = a_0 + \sum_{n=1}^\infty
a_n(t - t_0)^n \\
&= a_0 + \int_{t_0}^t g(s)ds.
\endalign
$$
This formula, together with the fundamental theorem of
calculus assures us that $f(t)$ is a differentiable function
and 
$$
 f'(t) = 0 + \frac{d}{dt}\int_{t_0}^t g(s)ds =  g(t)
$$
as needed.
\medskip
The proof of (b) is harder.  Let
$$
f(t) = \sum_{n=0}^\infty a_n(t - t_0)^n\quad\text{for } |t - t_0| < R.
$$
Fix $t > t_0$.  (The argument for $t < t_0$ is similar.)
We want to integrate $f(s)$ over the range $t_0 \le s \le t$. 
For any given $N$, we may decompose the series for $f(s)$ into two parts
$$
f(s) = \sum_{n=0}^N a_n(s - t_0)^n + 
\undersetbrace{R_N(s)}\to{\sum_{n = N+1}^\infty a_n(s - t_0)^n}.
$$
Integrate both sides to obtain
$$\align
\int_{t_0}^t f(s)ds &= \sum_{n=0}^N\int_{t_0}^t a_n(s - t_0)^n ds
  + \int_{t_0}^t R_N(s)ds \\
&= \sum_{n=0}^N \frac{a_n}{n+1} \left.(s - t_0)^{n+1}\right|_{t_0}^t
   + \int_{t_0}^t R_N(s)ds \\
&= \sum_{n=0}^N \frac{a_n}{n+1} (t - t_0)^{n+1} + \int_{t_0}^t R_N(s)ds.
\endalign $$

To complete the proof, we need to see what happens in the last
equality as $N \to \infty$.  The first term is the $N$th partial
sum of the desired series, so it suffices to show that the second
error term approaches zero.  However.
$$
\left|\int_{t_0}^t R_N(s)ds \right | \le \int_{t_0}^t |R_N(s)|ds.
$$
On the other hand, we have
$$
\align
|R_N(s)| &= \left|\sum_{n=N+1}^\infty a_n(s - t_0)^n\right|
\le \sum_{n = N+1}^\infty |a_n||s - t_0|^n \\
&\le \sum_{n = N+1}^\infty |a_n||t - t_0|^n.
\endalign
$$
(This follows by the same reasoning as in the comparison test.)
Note that since
$|t - t_0| < R$, the series $\sum_n a_n(t - t_0)^n$ converges
{\it absolutely\/}, so the series on the right of the above
inequality is the tail of a convergent series; call it
$T_N$.  Then, we have
$$
|R_N(s)| \le T_N
$$
for  $|s- t_0| < |t - t_0|$, where $T_N$ is independent of $s$.  Hence,
$$
\left|\int_{t_0}^t R_N(s)ds\right| \le
\int_{t_0}^t |R_N(s)|ds \le \int_{t_0}^t T_N ds = T_N (t - t_0).
$$
Since $T_N$ is the tail of a convergent series, it goes to zero
as $N \to \infty$.   That completes the proof.
\bigskip
\input chap8.ex6
\bigskip
\nextsec{Analytic Functions and Taylor Series}
\head \sn. Analytic Functions. Taylor Series \endhead

Our aim, as enunciated at the beginning of this chapter, is to use
power series to solve differential equations.   So suppose that
$$
f(t) = \sum_{n= 0}^\infty a_n(t - t_0)^n
\qquad\text{for } |t - t_0| < R,
$$
where $R$ is the radius of convergence of the series on the right.
(Note that this assumes that $R > 0$.  There are power series
with $R = 0$.  Such a series converges only at $t = t_0$ and
won't be of much use.  See the Exercises for the
previous section for an example.)
A function $f:\R \to \R$ is said to be {\it analytic\/} if
\outind{analytic function}%
at each point $t_0$ in its domain, it may be represented by
such a power series in an interval about $t_0$.  Analytic
functions are the best possible functions to use in applications.
For example, we know by the previous section, that such a function
is differentiable.  Even better, its derivative is also analytic
because it has a power series expansion with the same radius of
convergence.  Hence, the function also has a second derivative,
and by extension of this argument, {\it must have derivatives
of every order}.   Moreover, it is easy to relate the
coefficients of the power series to these derivatives.  We have
$$
\align
f(t) &= \sum_{n= 0}^\infty a_n(t - t_0)^n = a_0 + a_1(t - t_0) + a_2(t -t_0)^2
+ \dots \\
f'(t) &= \sum_{n= 1}^\infty na_n(t - t_0)^{n-1} =  a_1 + 2a_2(t -t_0)
+ \dots \\
f''(t) &= \sum_{n= 2}^\infty n(n-1)a_n(t - t_0)^n =  2a_2
+ \dots \\
&\vdots\\
f^{(k)}(t) &= \sum_{n=k}^\infty n(n-1)(n-2)\dots(n-k+1)a_n(t - t_0)^n = 
k(k-1)\dots 2\cdot 1 a_k +  \dots \\
&\vdots
\endalign
$$
Here $f^{(k)}$ denotes the $k$th derivative of $f$.  Note  that
the series for $f^{(k)}(t)$ starts off with $k!a_k$.  Make sure
you understand why!  Now put $t = t_0$ in the above formulas.  All
terms involving $t - t_0$ to a positive power vanish, and we get
$$
\align
f(t_0) &= a_0 \\
f'(t_0) &= a_1 \\
f''(t_0) &= 2a_2 \\
 &\vdots \\
f^{(k)}(t_0) &= k!a_k \\
&\vdots
\endalign $$
We can thus solve for the coefficients
$$
\align
a_0 &= f(t_0) \\
a_1 &= f'(t_0) \\
a_2 &= \frac{f''(t_0)}2 \\
&\vdots \\
a_k &= \frac{f^{(k)}(t_0)}{k!} \\
&\vdots
\endalign $$
Hence, within the radius of convergence, the power series expansion
of $f(t)$ is
$$
f(t) = \sum_{n=0}^\infty \frac{f^{(n)}(t_0)}{n!}(t - t_0)^n.
$$
This series is called the {\it Taylor series\/} of the function $f$.
\outind{Taylor series}%

The above series was derived under the assumption that $f$ is
analytic, but we can form that series
for any function whatsoever, provided
derivatives of all orders exist at $t_0$. If the function is analytic,
then it will equal its Taylor series within its radius of convergence.

\nextex
\example{Example \en}  Let $f(t) = e^t$ and let $t_0 = 0$.
Then,  
$$f'(t) = e^t, f''(t) = e^t, \dots, f^{(n)}(t) = e^t, \dots,$$
so
$$
a_n = \frac {e^0}{n!} = \frac 1{n!}.
$$
Hence, the Taylor series for $e^t$ at $0$ is
$$
\sum_{n=0}^\infty \frac 1{n!} t^n.
$$
It is easy to determine the radius of convergence of this series
by means of the ratio test.
$$
\frac{|t|^{n+1}/(n+1)!}{|t|^n/n!} = \frac{|t|}{n+1} \to 0 < 1.
$$
Hence, the series converges for all $t$ and the radius of convergence
is infinite.  We shall see that $e^t$ equals the series for every
$t$, so f(t) = $e^t$ defines an analytic function  on $\R$.  However,
this need not always be the case.  For example,  we might have
$f(t)$ equal to the Taylor series for some but not all values in the interval
of convergence of that series. 
\endexample

\nextex
\example{Example \en}  Let $f(t) = \cos t$ and let $t_0 = 0$.
Then, $f'(t) = -\sin t, f''(t) = -\cos t, f'''(t) = \sin t$,
and $f^{(4)}(t) = \cos t$.  This pattern then repeats indefinitely
with a period of 4.  In particular,
$$
f^{(n)}(0) = \left\{\aligned 1\quad&\text{if $n$ is even and a multiple of 4}\\
-1\quad&\text{if $n$ is even and not a multiple of 4}\\
 0\quad&\text{otherwise}
\endaligned \right..
$$
Hence, the Taylor series is
$$
1 - \frac{t^2}{2} + \frac{t^4}{4!} - \frac{t^6}{6!} + \dots
= \sum_{n=0}^\infty (-1)^n\frac{t^{2n}}{(2n)!}.
$$
As above, a simple application of the ratio test shows that this
series has infinite radius of convergence.  We shall see below
that $\cos t$ is in fact analytic and equals its Taylor series for
all $t$.
\endexample
\nextex
\example{Example \en}  We showed in the previous section that the
expansion
$$
\ln (1 + t) = \sum_{n=1}^\infty (-1)^{n+1} \frac{t^n}n
$$
is valid for $t$ in the interval of convergence of the series,
($|t| < 1$).    This tells us that the function $\ln (1 + t)$
is certainly analytic for $-1 < t < 1$, and the series on the
right is its Taylor series.

If we substitute $t - 1$ for $t$ in that expansion we get
$$
\ln t = \sum_{n=1}^\infty (-1)^{n+1} \frac{(t -1)^n}n
$$
and this gives us the Taylor series of $\ln t$ for $t_0 = 1$.
\endexample

\subhead Taylor's Formula with Remainder \endsubhead
Because analytic functions are so important, it is useful
to have a mechanism for determining when a function is equal
to its Taylor series.  To this end, let
$$
R_N(t) = f(t) - \sum_{n=0}^N\frac{f^{(n)}(t_0)}{n!}(t - t_0)^n.
$$
The {\it remainder\/}, $R_N(t)$, is the difference 
\outind{remainder, Taylor series}%
\outind{Taylor series, remainder}%
between the value of the function
and the value of the appropriate partial sum of its Taylor series.
To say the series converges to the function is to say that the
limit of the sequence of partial sums
is $f(t)$, i.e., that  $R_N(t) \to 0$.   To determine if this
is the case, it is worthwhile having a method for estimating
$R_N(t)$.   This is provided by the following formula which is
called the {\it Cauchy form of the remainder}.
\nexteqn
\xdef\AE{\eqn}
$$
R_N(t) = \frac 1{N!}\int_{t_0}^t (t - s)^N f^{(N+1)}(s)\,ds.\tag\eqn
$$
This formula is in fact valid as long as the $f^{(N+1)}(s)$
exits and is continuous in the interval from $t_0$ to $t$.
We shall derive this formula later in this section, but it
is instructive to look at the first case $N = 0$.  Here
$$
\gather
f(t) = f(t_0) + R_0(t) \\
\text{where}\qquad R_0(t) = \int_{t_0}^t f'(s)ds
\endgather $$
so the formula tells us that the change $f(t) - f(t_0)$ in the
function is the integral of its rate of change $f'(s)$.  The general
case may be considered an extension of this for higher derivatives,
but just why the factor $(t - s)^n$ comes up in the integral won't
be clear until you see the proof.

In most cases, formula (\eqn) isn't much use as written.  The point
is that we want to use the partial sum
$$
  \sum_{n=0}^N\frac{f^{(n)}(t_0)}{n!}(t - t_0)^n
$$
as an approximation to the function value $f(t)$.  If we could
calculate the remainder term  $R_N(t)$ exactly, we could also
calculate $f(t)$ exactly, and there would be no need to use
an approximation.   Hence, in most cases, we want to get an
{\it upper bound\/} on the size of $R_N(t)$ rather than an
exact value.  {\it Suppose it is known that} 
$$
    |f^{(N+1)}(s)| \le M
$$
{\it for $s$ in the interval between $t_0$ and $t$}.   Then
replacing $|f^{(N+1)}(s)|$ by $M$ in (\eqn) yields
(for $t > t_0$)
$$
|R_N(t)| \le \frac 1{N!}\int_{t_0}^t (t - s)^N M ds 
 = -\frac M{N!}\left. \frac{(t - s)^{N+1}}{N+1}\right|_{t_0}^t
 = \frac M{(N + 1)!} (t - t_0)^{N+1}.
$$
A similar argument works for $t < t_0$ except for some fiddling with
signs.  The result which holds in both cases is
\nexteqn\xdef\AI{\eqn}
$$
|R_N(t)| \le \frac M{(N+1)!}|t - t_0|^{N+1}.\tag\eqn
$$
The expression on the right is what we get
get from the {\it next term\/} of the Taylor series if we
{\it replace $f^{(N+1)}(t_0)$ by the maximum value of that
derivative in the interval from $t_0$ and $t$}.

\nextex
\example{Example \en}  Consider the Taylor series for
$f(t) = \cos t$ centered at $t_0 = 0$   
$$
\sum_{n=0}^\infty \frac{(-1)^n}{(2n)!}t^{2n}.
$$
Only even numbered terms appear in this series.   We want to estimate
the error $R_N(t)$ using the inequality (\AI).   We might as well
take $N = 2K+1$ odd.  (Why?)  We know the even derivatives are
all $\pm\cos t$, so we can take $M= 1$ as an upper bound for
$|f^{(N+1)}(s)| = |f^{(2K + 2)}(s)|$. Hence, the estimate for the
remainder is
$$
|R_{2K+1}(t)| \le \frac 1{(2K+2)!} |t|^{2K+2}.
$$
(Note that in this case the estimate on the right is just the
absolute value of the
next term in the series.)
Thus for $K = 5, t = 1$ radian, we get
$$
|R_{11}| \le \frac 1{12!} 1^{12} = 2.08768\times 10^{-9}.
$$
Because the odd numbered terms of the series are zero,
the number of non-zero terms in the approximating sum is 6.
According to Mathematica
$$
\sum_{n=0}^5 (-1)^n\frac 1{(2n)!} =  0.540302304
$$
to 9 decimal places.  Use your calculator to see if it agrees that
this result is as accurate as the above error estimate suggests.
\endexample

\nextex
\example{Example \en}  Take $f(t) = \sin t$ and $t_0 = 0$. 
Calculations as above lead to the Taylor series
$$
\sum_{n=0}^\infty (-1)^n\frac{t^{2n+1}}{(2n + 1)!}.
$$
All the even numbered terms are zero.  We get the estimate
for the remainder for $N = 2K + 2$---(why don't we use $N = 2K + 1$?)---
$$
|R_{2K+2}(t)| \le \frac{|t|^{2K + 3}}{(2K + 3)!}.
$$
\endexample

\nextex
\emar{s8-9.ps}{-125}
\example{Example \en}  Take $f(t) = e^t$ and $t_0 = 0$.   The Taylor
series is
$$
\sum_{n=0}^\infty \frac{t^n}{n!}.
$$
The estimate of the remainder, however, is a bit more involved.
The crucial parameter $M$ is the upper bound
of $|f^{(N+1)}(s)| = e^s$ for $s$ in the interval from $0$ to $t$,
but this
depends on the sign of $t$.  
Suppose first that $t < 0$.   Then, since $e^s$ is an increasing
function of $s$, its maximum value is attained at the right endpoint,
i.e., at $s = 0$.  Hence, we should take $M = e^0 = 1$.
Thus,
$$
|R_N(t)| \le \frac 1{(N+1)!} |t|^{N+1}
$$
is a plausible estimate for the remainder.
On the other hand, if $t > 0$, the same reasoning shows that
we should take $M = e^t$.  This seems a bit circular, since
it is $e^t$ that we are trying to calculate.  However, there is
nothing preventing us from using an $M$ {\it larger than\/}
$e^t$ if it is easier to calculate.
For example, suppose we want to compute $e = e^1$ using the
series
$$
  \sum_{n=0}^\infty \frac 1{n!} 1^n =
\sum_{n=0}^\infty \frac 1{n!}.
$$
The maximum value of $e^s$ in this case would be $e^1 = e$, but
that is certainly less than 3.  Hence, we take $M = 3$ and use the
error estimate
$$
|R_N(1)| \le \frac 3{(N+1)!} |1|^{N+1}.
$$
Thus, for $N = 15$ (actually 16 terms), we could conclude
$$
|R_{15}(1)| \le \frac 3{16!} = 1.43384\times 10^{-13}.
$$
According to Mathematica, the sum for $N = 15$ is
$2.71828182846$.  You should try this on your calculator to
see if it agrees.
\endexample

We may use the estimate of the error to determine values of
$t$ for which the Taylor series converges to the function.
The following lemma is useful in that context.

\nextthm
\proclaim{Lemma \cn.\tn} For any number $t$,
$$
 \lim_{n\to\infty} \frac{t^n}{n!} = 0.
$$
\endproclaim
\demo{Proof}
We saw earlier (by the ratio test) that the Taylor series
for $e^t$ 
$$
\sum_{n=0}^\infty \frac{t^n}{n!}
$$
has infinite radius of convergence.  Hence, it converges 
 for all $t$, and its general term  $t^n/n!$ must
approach 0.
\enddemo

For each of the functions we considered above, the estimate
of the remainder involved $|t|^n/n!$ or something related to it.  Hence,
the Lemma tells us $R_N(t) \to 0$ as $N \to \infty$ in each
of these cases, for any $t$.  Thus, we have
the following Taylor series expansions, valid for all $t$,
$$
\align
\cos t &= \sum_{n=0}^\infty (-1)^n\frac {t^{2n}}{(2n)!} \\
\sin t &= \sum_{n=0}^\infty (-1)^n\frac {t^{2n+1}}{(2n+1)!} \\
e^t &=\sum_{n=0}^\infty \frac{t^n}{n!}.
\endalign $$

In principle, each of these series may be used to calculate
the function to any desired degree of accuracy by using sufficiently
many terms.  However, in practice this is only useful for relatively
small values of $t$.  For example,  we saw that 
$$
 |R_{2K+1}(t)| \le \frac {|t|^{2K+2}}{(2K+2)!}
$$
is a plausible estimate of the error when using the Taylor series
for $\cos t$.   However, here are some values for 
 $t = 10$
$$
\align
  n\qquad & t^n/(2n)! \\
1\qquad & 50 \\
2\qquad & 416.667 \\
3\qquad &1388.89\\
4\qquad &  2480.16 \\
5\qquad & 2755.73 \\
6\qquad & 2087.68 \\
&\vdots \\
10\qquad& 41.1032 \\
&\vdots \\
20\qquad & 1.22562 10^{-8} \\
&\vdots
\endalign $$
As you see, the terms can get quite large before the factorial
term in the denominator begins to dominate.  Using the series
to calculate $\cos t$ for large $t$ might lead to some nasty surprises.
(Can you think of a better way to do it?)

\subhead Derivation of the Cauchy Remainder \endsubhead
We assume that all the needed derivatives exist and are
continuous functions.   
Consider the expression
$$
R_N(t,s) = f(t) - f(s) - f'(s)(t - s) - \frac{f''(s)}2(t -s)^2
            - \dots - \frac{f^{(N)}(s)}{N!}(t - s)^N
$$
as a function of both $t$ and $s$.  Take the derivative of both
sides with respect to $s$.  Note that by the product rule,
the derivative of a typical term is
$$
\frac{\d}{\d s}\left(-\frac{f^{(n)}(s)}{n!}(t -s)^n\right)
= -\frac{f^{(n+1)}(s)}{n!}(t - s)^n + \frac{f^{(n)}(s)}{(n-1)!}(t - s)^{n-1}
$$
If we write these terms in the reverse order, we get
$$
\align
\frac{\d R_N}{\d s}
&= 0 - f'(s)  + f'(s) - f''(s)(t -s) + f''(s)(t - s) - \dots
      - \frac{f^{(N+1)}(s)}{N!}(t - s)^N \\
&=   - \frac{f^{(N+1)}(s)}{N!}(t - s)^N.
 \endalign
$$
since all terms except the last cancel.
Integrating with respect to $s$, we obtain
$$
\int_{t_0}^t \frac{\d R_N}{\d s} ds
= - \frac 1{N!}\int_{t_0}^t f^{(N+1)}(s)(t - s)^N ds.
$$
However, the integral on the left is
$$
\left. R_N(t,s)\right|_{s = t_0}^{s = t} = R_N(t,t) - R_N(t,t_0).
$$
Since $R_N(t,t) = f(t) - f(t) - 0  - \dots -0 = 0$, it follows that
$$
-R_N(t,t_0) = 
 - \frac 1{N!}\int_{t_0}^t f^{(N+1)}(s)(t - s)^N ds,
$$
which is the desired formula except for the minus signs.

\subhead Determining the Radius of Convergence of a Taylor Series
\endsubhead
We have seen that generally speaking the Taylor series expansion
$$
f(t) = \sum_{n=0}^\infty \frac{f^{(n)}(t_0)}{n!} (t - t_0)^n
$$
is valid in some interval $t_0 - R < t < t_0 + R$.   There is
a simple rule for determining $R$ from $f$, which applies in
many cases.

Consider  the example
$$
f(t) = \frac 1{1 - t} = 1 + t + t^2 + \dots + t^n + \dots
$$
We know the series on the right converges absolutely for
$-1 < t < 1$  and that the expansion of the function is valid
in this range.  ($t_0 = 0, R = 1$.)  You will note that the
function in this case has a singularity at $t = 1$.  That suggests
the following rule:

\proclaim{Radius of Convergence of a Taylor Series}
  For each point $t_0$, let $R$ be the
distance to the nearest singularity of the function $f$.  Then
the radius of convergence of
the Taylor series for $f$ centered at $t_0$ is $R$, and the series 
converges absolutely
to the function in the interval $t_0 - R < t < t_0 + R$.
\endproclaim
\outind{radius of convergence of a power series}%

Unfortunately, it is easy to come up with an example for which
this rule appears to fail.  Put  $u = -t^2$  in
the formula 
$$
\frac 1{1 - u} = 1 + u + u^2 + \dots + u^n + \dots
$$
to get
$$
\frac 1{1 - (-t^2)} = 1 + (-t^2) + (-t^2)^2 + (-t^2)^3
+ \dots + (-t^2)^n + \dots
$$
This is valid for $|u| = |-t^2| < 1$, i.e., $|t| < 1$.
Cleaning up the minus signs yields
$$
\frac 1{1 + t^2} = 1 - t^2 + t^4 - t^6 + \dots + (-1)^nt^{2n} + \dots
$$
for $-1 < t < 1$.    The function $f(t) = 1/(1 + t^2)$ does
not appear to have a singularity either at $t = 1$ or $t= -1$,
which contradicts the rule.  However, if you consider the function
of a {\it complex\/} variable $z$ defined by
$$
f(z) = \frac 1{1 +z^2}
$$
this function does have singularities at $z = \pm i$ where the
denominator vanishes.   Also, very neatly, the distance
in the complex plane from $z_0 = 0$ to each of these singularities
is $R = 1$, and that is the proper radius of convergence of the
series.  {\it Thus the rule does apply if we extend the function
$f(t)$ to a function $f(z)$ of a complex variable and look for
the singularities of that complex function}.   In many simple
cases, it is obvious how to do this, but in general, the
notion of singularity for  functions $f(z)$ of a complex
variable and how to go about extending  functions $f(t)$ of a real
variable requires extensive theoretical discussion.
This will be done in your complex analysis course.  We felt
it necessary, however, to enunciate the rule, even though
it could not be fully explained, because it is so important
in studying the behavior of series solutions of differential
equations.  It is one of many circumstances in  which 
behavior in the complex plane, which would be invisible just
by looking at real points, can affect the behavior of a mathematical
system and of physical processes modeled by such a system.

%% Advice to instructor:  Skip next sub-section in class.
\subhead A Subtle Point \endsubhead
We have been using implicitly the fact that the sum
 $f(t) = \sum_{n=0}^\infty a_n (t - t_0)^n$ of a power
series is an analytic function within the interval
of convergence $t_0 - R < t < t_0 + R$ of that series.
If so, then the Taylor series of $f(t)$
at any {\it other\/}
point $t_1$ in $(t_0 - R, t_0 + R)$ should also converge to $f(t)$.
(Look carefully at the definition of `analytic' at the beginning
of this section!)
In fact, such is the case, and the interval of convergence 
of the Taylor series at $t_1$
 extends at least to
the closer of the two endpoints  $t_0 - R, \, t_0 + R$ of the
interval of convergence of the original power series.  
One must prove these assertions to justify
the conclusion that the sum of a power series
is analytic, but we won't do that in this course.
It would be rather difficult to do without getting
into complex variable theory.
Hence, you will have to wait for a course in that subject for
a proof.   

\smallskip
\mar{s8-10.ps}
\bigskip
\input chap8.ex7
\bigskip

\nextsec{More Calculations with Power Series}
\head \sn.  More Calculations with Power Series \endhead

There are two approaches to finding the Taylor series expansion
of an analytic function.  You can start with the function and
its derivatives and calculate the coefficients $a_n = f^{(n)}(t_0)/n!$.
This is often quite messy.  (Also, it may be quite difficult
to show
that the series converges to the function in an appropriate
range  by  
estimating $R_N(t)$.)  Another approach is to start with the
series for a related function and then derive the desired series 
by differentiating, integrating or other manipulations.

\nextex
\example{Example \en}  By {\it substituting\/}  $u = -t^2$ in
the series expansion
$$
\frac 1{1 - u}  = 1 + u + u^2 + \dots
$$ we may obtain
the expansion
$$
\frac 1{1 + t^2} = 1  - t^2 + t^4 - t^6 + \dots + (-1)^nt^{2n} + \dots.
$$
Since the first expansion is valid for $-1 < u < 1$, the second
expansion is valid for $0\le t^2 < 1$, i.e., $-1< t < 1$.
In this range,
we may integrate the series term by term to obtain
$$
\int_0^t\frac 1{1 + s^2}ds = t -\frac{t^3}3 + \frac{t^5}5 -
\dots + (-1)^n\frac {t^{2n+1}}{2n+1} + \dots
$$
or after evaluating the left hand side
$$
\tan^{-1}t = t -\frac{t^3}3 + \frac{t^5}5 -\dots + (-1)^n\frac {t^{2n+1}}{2n+1} + \dots
$$
for $-1 < t < 1$.  You might try to derive this Taylor series by taking
derivatives of the $\tan^{-1} t$.  You will find it is a lot harder.

The above series also converges  for $t = 1$ by the alternating series
test, and its sum is $\tan^{-1} (1)$.  (This does not follow simply
by the above analysis, since that only applied in the range $-1 < t <1$,
but it can be demonstrated by a more refined analysis.)  Thus
we get the formula
$$
\frac{\pi}4 =
1 -\frac 13 + \frac 15 - \frac 17 + \dots +  (-1)^n\frac 1{2n+1} + \dots
$$
In principle, this series may be used to compute $\pi$ to any
desired degree of accuracy.   Unfortunately, it converges rather
slowly, but similar expansions for other rational multiples of
$\pi$ converge rapidly.  Recently, such series have been used
to test the capabilities of modern computers by computing $\pi$
to exceptionally large numbers of decimal places.

The Taylor series does not converge for
 $t = -1$,  since in that case you get
$$
-1 -\frac 13 - \frac 15 - \frac 17 - \dots - \frac 1{2n+1} - \dots
$$
(Apply the integral test to the negative of the series.)  That fact
should make you cautious about trying to extend results to the
endpoints of the interval of convergence.
\endexample

\nextex
\example{Example \en}  The quantities $\sin t/t$ and $(1 - \cos t)/t$
occur in a variety of applications.  
%Exer  Tunnel through circular earth.
The limits $\lim_{t \to 0} \sin t/t = 1$ and $\lim_{t \to 0}(1 - \cos t)/t
= 0$ must be determined in order to calculate the derivatives of the
$\sin$ and $\cos$ functions.  We can use the current theory to get
more precise information about how the above ratios behave near
$t = 0$.   We discuss the case of $\sin t/t$.  $(1 - \cos t)/t$
is similar.  Start with
$$
\sin t = t - \frac{t^3}{3!} + \frac{t^5}{5!} - \dots +(-1)^n\frac{t^{2n+1}}
{(2n +1)!} + \dots
$$
which is valid for {\it all\/} $t$.  Multiply by $1/t$ to obtain
$$
\frac{\sin t}t = 1 - \frac{t^2}{3!} + \frac{t^4}{5!} - \dots +
(-1)^n\frac{t^{2n}}{(2n+1)!} + \dots
$$
which is also valid for all $t \not= 0$.  (The analysis breaks down if $t = 0$,
but the formula may be considered valid in the sense that
the right hand side is
1 and the left hand side has limit 1.) 
Note that the series is an alternating series so we may estimate the
error if we stop after $N$ terms by looking at the {\it next\/}
term.  By taking $n=1$, we conclude that
$$
\frac{\sin t}t \approx 1
$$
for $t$ small with the difference behaving like $O(t^2)$.
\endexample

Sometimes one can use Taylor series to evaluate an integral
quickly and to high accuracy in a case where it is not possible
to determine an elementary anti-derivative.

\nextex
\example{Example \en}  We shall calculate $\int_0^{\pi/2}\dfrac{\sin x}x dx$
so that the error is less than
$5\times 10^{-4}$,
which {\it loosely speaking\/} says is it accurate
to three decimal places.   Note that the function $\sin x/x$ does not have
an elementary anti-derivative.   As in the previous example,
$$
\frac{\sin x}x =
 1 - \frac{x^2}{3!} + \frac{x^4}{5!} - \dots +
(-1)^n\frac{x^{2n}}{(2n+1)!} + \dots
$$
and we may consider this expansion valid for all $x$ by using the
limit 1 for the value of $\sin x/x$ for $x = 0$.   Integrating term
by term yields
$$
\int_0^{\pi/2} \frac{\sin x}x dx
=
\frac\pi 2 - \frac 1{3\cdot 3!}\left(\frac \pi 2\right)^3
+ \frac 1{5\cdot 5!}\left(\frac \pi 2\right)^5 
- \frac 1{7\cdot 7!}\left(\frac\pi 2\right)^7 + \dots
$$
Also, it is possible to show that the right hand side is an
alternating series {\it in which the terms decrease}.  Indeed,
checking the first few values, we have (to 4 decimal places)
$$\align
\frac\pi 2 &= 1.5708 \\
\frac 1{18}\left(\frac\pi 2\right)^3 &= 0.2153 \\
\frac 1{600}\left(\frac\pi 2\right)^5 &= 0.0159 \\
\frac 1{35280}\left(\frac\pi 2\right)^7 &= 0.0007 \\
&\vdots
\endalign
$$
The next term is $(1/9\cdot 9!)(\pi/2)^9\approx 2\times 10^{-5}$.
Hence, the error in just using the terms listed is certainly small
enough that it won't affect the 3rd decimal place.  Adding up
those terms (including signs) yields $1.371$ to 4 decimal places.
\endexample

\subhead The Binomial Theorem \endsubhead
An important Taylor series is that for the function
$f(t) = (1 + t)^a$.   That series is called the {\it
binomial series}.
Perhaps you recall from high school the formulas
$$
\align
(1 + t)^2 &= 1 + 2t + t^2 \\
(1+t)^3 &= 1 + 3t + 3t^2 + t^3 \\
(1 + t)^4 &= 1 + 4t + 6t^2 + 4t^3 + t^4 \\
&\vdots
\endalign $$
The general case for $a$ a positive integer is the
{\it binomial theorem\/} formula
\outind{binomial theorem}%
\nexteqn
$$
(1 + t)^a = 1 + at + \frac{a(a - 1)}2 t^2 + \dots + at^{a-1}t^a
 = \sum_{n=0}^a \binom an t^n,
$$
where
$$
\binom an = \frac{a(a - 1)(a - 2)\dots (a - n +1)}{n!}
       = \frac{a!}{n!(a - n)!}.
$$
(By convention $\binom an = 0$ if $n < 0$ or $n > a$.)
If we take $a$ to be a negative integer, a fraction
 or indeed {\it
any non-zero real number\/}, then we may still define
binomial coefficients
$$
\binom an = \frac{a(a - 1)(a - 2)\dots (a - n +1)}{n!}
$$
with the convention that $\binom a0 = 1$.
\outind{binomial coefficients}%
Unlike the ordinary binomial coefficients, 
these quantities aren't necessarily positive,
and moreover they don't vanish for $n > a$.  Thus, instead of
a polynomial, we get a {\it series expansion}
\nexteqn
$$
(1 + t)^a = \sum_{n=0}^\infty \binom an t^n\tag\eqn
$$
which is valid for $-1 < t < 1$.  
\nextex
\example{Example \en}  For $a = -2$, the coefficients are
$$
\align
\binom {-2}0 &= 1 \\
\binom {-2}1 &= -2 \\
\binom {-2}2 &\frac{(-2)(-3)}2 = 3 \\
\binom {-2}3 &= \frac{(-2)(-3)(-4)}{3!} = -4 \\
&\vdots \\
\binom {-2}n &= \frac{(-2)(-3)\dots(-n)(-n-1)}{n!} = (-1)^n(n+1)
\endalign
$$
Hence,
$$
(1 + t)^{-2} = \sum_{n=0}^\infty (-1)^n(n+1)t^n.
$$
(Compare this with the series for $(1 + t)^{-2}$ you  obtained
in the Exercises for Section 6
by differentiating the series for $(1 + t)^{-1}$.)
%  Back reference to Exercises for Section 6
\endexample

The general binomial theorem (\eqn)
 was discovered and first proved by Isaac Newton.
Since then there have been several different proofs.  We
shall give a rather tricky proof based on some of the
ideas we developed above.

First note that the series $\sum_n\binom an t^n$
 has radius of convergence
1.  For, the ratio
$$
\multline
\frac{|\binom a{n+1}||t|^{n+1}}{|\binom an||t|^n} =\\
  |t|\frac{|a(a-1)\dots(a - n + 1)(a - (n+1)+1)|}
{(n+1)!}\,
\frac{n!}{|a(a-1)\dots(a - n + 1)|} \\
= |t|\frac{|a - n|}{n+1}
\endmultline
$$
approaches $|t|$ as $n \to \infty$.  Thus, the series converges
absolutely for $|t| < 1$.  Let
$$
f(t) = \sum_{n=0}^\infty \binom an t^n\qquad\text{for } -1 < t < 1.
$$
Then
\nexteqn
$$
f'(t) = \sum_{n=1}^\infty \binom an n t^{n-1}.\tag\eqn
$$
Multiply this by $t$
to obtain
$$
tf'(t) = \sum_{n=1}^\infty \binom an n t^n =
\sum_{n=0}^\infty \binom an n t^n,
$$
where we put back the $n=0$ term which is zero in any case.
Similarly, putting $n+1$ for $n$ in (\eqn) yields
$$
f'(t) = \sum_{n=0}^\infty \binom a{n+1}(n+1)t^{n-1},
$$
so
$$
f'(t) + tf'(t) = \sum_{n=0}^\infty \left(\binom a{n+1}(n+1)
           + \binom an n\right)t^n.
$$
However,
$$
\align
\binom a{n+1}(n+1) &+ \binom an n \\
&=\frac{a(a-1)\dots(a - (n+1) + 1)}{(n+1)!}(n+1) 
+ \frac{a(a-1)\dots(a - n +1)}{n!}n \\
&= \frac{a(a -1)\dots(a - n +1)}{n!}(a -n) + 
\frac{a(a-1)\dots(a - n +1)}{n!}n \\
&= \frac{a(a-1)\dots(a -n +1)}{n!}a = \binom an a.
\endalign
$$
Hence,
$$
(1 + t)f'(t) = \sum_{n=0}^\infty a\binom an t^n = a
\sum_{n=0}^\infty \binom an t^n = af(t).
$$
In other words, $f(t)$ satisfies the differential equation
$$
f'(t) - \frac a{1+t}f(t) = 0.
$$
This is a linear equation, and we know the solution
$$
f(t) = Ce^{\int \frac a{1 +t}dt} = Ce^{a\ln(1 + t)} = C(1 + t)^a.
$$
To determine $C$, note that it follows from the definition
of $f$ that $f(0) = 1$.  Hence, $1 = C1^a = C$, and
$$
f(t) = (1 +t)^a\qquad\text{for } -1 < t < 1
$$
as claimed.

%Exer  \sqrt binomial series.  Expansion in relativity.

\subhead Other Manipulations \endsubhead
Other manipulations with series are possible.  For example, series
may be added, subtracted,  multiplied or even divided, 
but determining the radius
of convergence of the resulting series may be somewhat tricky.

\nextex
\example{Example \en}  We find the Taylor series expansion
for $e^{2t}\cos t$ for $t_0 = 0$.  We have
$$\align
e^{2t} &= 1 + 2t + 2t^2 + \frac 43 t^3 + \dots \\
\cos t &= 1 - \frac 12 t^2 + \frac 1{24} t^4 - \dots
\endalign $$
so
$$
\align
e^{2t}\cos t
&= 1 + 2t + 2t^2 + \frac 43 t^3 + \dots \\
       &-\frac 12 t^2 - t^3 - \dots 
\endalign
$$
where we have listed only the terms of degree $\le 3$.  Combining
terms, we obtain
$$
e^{2t}\cos t = 1 + 2t + \frac 32 t^2  + \frac 13 t^3 + \dots
$$
\endexample
\nextex  
\example{Example \en}   We know that
$$
\frac 1{1 + u} = 1 - u + u^2 - u^3 + \dots\qquad\text{for } |u| < 1.
$$
Since $(1 + t)^2 = 1 + 2t + t^2$, we may substitute $u = 2t + t^2$
in the above equation to get
$$
\align
\frac 1{(1 + t)^2} = \frac 1{1 + 2t + t^2}
  &= 1 - (2t + t^2) + (2t + t^2)^2 - (2t + t^2)^3 + \dots \\
  &= 1 -2t -t^2 + 4t^2 + 4t^3 + t^4 - 8t^3 -12t^4 - 6t^5 - t^6 + \dots \\
  & = 1 - 2t + 3t^2 - 4t^3 + \dots
\endalign $$
Note that I stopped including terms where I could be sure that the higher
degree terms would not contribute further to the given power of t.
(The term $(2t + t^2)^4$ would contribute a term involving $t^4$.)
Note also that it is not clear from the above computations
what radius of convergence
to specify for the last expansion to be valid.

You should compare the above expansion with what you would obtain
by using the binomial theorem for $(1 + t)^{-2}$.
\endexample

One consequence of the fact that we may manipulate series in this
way is
that the sum, difference, or product of analytic functions is
again analytic.  Similarly, the quotient of two analytic
functions is also analytic, at least if we exclude from the
domain points where the denominator vanishes.  Finally,
it is even possible
to substitute one series in another, so the composition of
two analytic functions is generally analytic.

\bigskip
\input chap8.ex8
\bigskip
\nextsec{Multidimensional Taylor Series}
\head \sn. Multidimensional Taylor Series \endhead

We want to extend the notions introduced in the previous sections
to functions of more than one variable, i.e.,   $f(\r)$ with
$\r$ in $\R^n$.   In this section, we shall concentrate entirely
on the case $n = 2$ because the notational difficulties get
progressively worse as the number of variables increases.  However,
with a good understanding of that case, it is not hard to see how
to proceed in higher dimensional cases.

Suppose then that $n = 2$, and a scalar valued function is given
by $f(\r) = f(x,y)$.  We want to expand this function in a
multidimensional power series.   From our previous discussion
of linear approximation, we know it should start
$$
f(x,y) = f(0,0) + f_x(0,0)x + f_y(0,0)y + \dots,
$$
and we need to figure out what the higher degree terms should be.

To this end, 
consider series of the
 form
$$
a_{00} + a_{10}x + a_{01}y + a_{20}x^2 + a_{11}xy + a_{02}y^2
 + \dots = \sum_{n=0}^\infty(\sum_{i+j = n} a_{ij}x^iy^j).
$$
Notice the way the terms are arranged.  A typical term will
be some multiple of a {\it monomial\/} $x^iy^j$.   The coefficient
of that monomial is denoted $a_{ij}$ where the subscripts specify the
powers of $x$ and $y$ in the monomial.   Moreover, we call 
$n = i+j$ the {\it total degree\/} of the monomial $x^iy^j$, and 
we group all monomials of the same degree together as
$\sum_{i+j = n} a_{ij}x^iy^j$.

The above series is centered at $(0,0)$.  We can similarly
discuss a multidimensional power series of the form
$$
    \sum_{n=0}^\infty (\sum_{i+j = n} a_{ij}(x - x_0)^n(y - y_0)^n)
$$
centered at the point $(x_0,y_0)$.   For the moment we will
concentrate on series centered at $(0,0)$ in order to save writing.
Once you understand that, you can get the general case by
substituting $x - x_0$ and $y - y_0$ for $x$ and $y$ and making
other appropriate changes.

A function $f(x,y)$ with domain some open set in $\R^2$
is said to be {\it analytic\/} if at each point in its domain
it may be expanded in a power series centered at that point
in some neighborhood of the point.   Suppose then that
$$
f(x,y) =  
a_{00} + a_{10}x + a_{01}y + a_{20}x^2 + a_{11}xy + a_{02}y^2
 + \dots = \sum_{n=0}^\infty(\sum_{i+j = n} a_{ij}x^iy^j)
$$
is valid in some neighborhood of $(0,0)$.  By putting $x = 0, y = 0$
we see that 
$$f(0,0) = a_{00}.$$
  Just as in the case
of power series in one variable, term by term differentiation
(or integration) is valid.   Hence, we have
\nexteqn
\xdef\EqOne{\eqn}
$$
\frac{\d f}{\d x} = a_{10} + 2a_{20}x + a_{11}y + \dots
   = \sum_{n=1}^\infty (\sum_{i + j = n} ia_{ij}x^{i-1}y^j).\tag\eqn
$$
Putting $x = 0, y = 0$ yields
$$
\frac{\d f}{\d x}(0,0) = a_{10},
$$
just as we expected.  Similarly,  $\frac{\d f}{\d y}(0,0) = a_{01}$.

Take yet another derivative  to get
$$
\frac{\d^2 f}{\d x^2} = 2a_{20} + \dots =
\sum_{n=2}^\infty \sum_{i + j = n}i(i-1)a_{ij}x^{i-2}y^j.
$$  
Hence,
$$
\frac{\d^2 f}{\d x^2}(0,0) = 2a_{20}.
$$
Similar reasoning applies to partials with respect to $y$, and
continuing in this way, we discover that
$$
\align
a_{n0} &= \frac 1{n!}\frac{\d^nf}{\d x^n}(0,0)\\
a_{0n} &= \frac 1{n!}\frac{\d^nf}{\d y^n}(0,0).
\endalign
$$
However, this still leaves the bulk of the coefficients undetermined.
To find these, we need to find some {\it mixed partials}.  Thus,
from equation (\EqOne), we get
$$
\frac{\d^2 f}{\d x\d y} = 
a_{11} + \dots = \sum_{n=2}^\infty (\sum_{i+j = n} ij a_{ij}x^{i-1}y^{j-1}).
$$
Thus,
$$
\frac{\d^2 f}{\d x\d y}(0,0) = a_{11}.
$$
Here the total degree is $1 + 1 = 2$.
Suppose we consider the case of total degree $r + s = k$ where
$r$ and $s$ are the subdegrees for $x$ and $y$ respectively.
Then, it is possible to show that
$$
\frac{\d^kf}{\d x^r\d y^s} = \sum_{n = k}^\infty \sum_{i + j = n}
i(i -1)(i -1)\dots (i - r + 1)\,j(j -1)(j - 1)\dots (j - s + 1)
a_{ij}x^{i - r}y^{j - s}.
$$
The leading term in the expansion on the right for $n = k$
has only one non-zero term, the one with 
$i = r$ and $j = s$, i.e.,
 $$
r!s!a_{rs}.
$$
Hence, since $k = r + s$,
$$
a_{rs} = \frac 1{r!s!}\frac{\d^{r + s}f}{\d x^r\d y^s}(0,0).
$$
(You should do all the calculations for $a_{21}$ and $a_{12}$
to convince yourself that it really works this way.  If you
don't quite see why the calculations work as claimed in the
general case, you should probably just accept them.)

Thus, the power series expansion centered at $(0,0)$ is
$$
f(x,y) = \sum_{n= 0}^\infty (\sum_{i + j = n}
\frac 1{i!j!}\frac{\d^nf}{\d x^i\d y^j}(0,0)\,x^iy^j).
$$
Similarly, the power series expansion centered at $(x_0,y_0)$ is
$$
f(x,y) = \sum_{n= 0}^\infty \sum_{i + j = n}
\frac 1{i!j!}\frac{\d^nf}{\d x^i\d y^j}(x_0,y_0)\,(x - x_0)^i(y - y_0)^j.
$$
The right hand side is called the (multidimensional) {\it Taylor
series\/} for the function centered at $(x_0,y_0)$.
\outind{multidimensional Taylor series}%
\outind{Taylor series, multidimensional}%

There is another way to express the series.  Consider the terms
of total degree $n$
$$
\sum_{i + j = n}\frac 1{i!j!}\frac{\d^nf}{\d x^i\d y^j}\,(x - x_0)^i(y - y_0)^j,
$$
where to save writing we omit explicit mention of the fact that the
partial derivatives are to be evaluated at $(x_0, y_0)$.
To make this look a bit more like the case of one variable, we divide
by a common factor of $n!$.  Of course, we must then also multiply
each term by $n!$ to compensate, and the extra $n!$ may be incorporated
with the other factorials to obtain
$$
\frac{n!}{i!j!} = \binom ni.
$$
Hence, the terms of degree $n$ may be expressed
\nexteqn
\xdef\EqTwo{\eqn}
$$
\frac 1{n!}\sum_{i + j = n} \binom ni \frac{\d^nf}{\d x^i\d y^j}\,
(x - x_0)^i(y - y_0)^j.\tag\eqn
$$

\nextex
\example{Example \en}
Let $f(x,y) = e^{x +y}$ and $x_0 = 0, y_0 = 0$.   Then it is not
hard to check that
$$
\frac{\d^n f}{\d x^i \d y^j} = e^{x+y} = e^0 = 1\quad\text{at }
(0,0).
$$
Hence, the series expansion is
$$
e^{x+y} = \sum_{n=0}^\infty \frac 1{n!}\sum_{i + j = n} \binom n i
x^i y^j.
$$

An easier way to derive this same series is to start with
$$
e^u = \sum_{n=0}^\infty \frac 1{n!}u^n
$$
and put $u = x+y$.  Since
$$
(x + y)^n = \sum_{i+j = n} \binom ni x^i y^j,
$$
we get the same answer.
\endexample

\medskip
There is yet another way to express formula (\EqTwo).  To save
writing, put $\Delta x = x - x_0, \Delta y = y - y_0$.
Consider the operator  $D = \Delta x \dfrac \d{\d x}
+ \Delta y \dfrac \d{\d y}$.  Then, symbolically,
$$
\align
  D^n =  (\Delta x \frac \d{\d x}+ \Delta y \frac \d{\d y})^n
     &= \sum_{i + j = n}\binom ni (\Delta x \frac \d{\d x})^i
(\Delta y\frac\d{\d y})^j\\
 &=\sum_{i+j = n}\binom ni \Delta x^i \Delta y^j \frac {\d^n}
{\d x^i \d y^j}.
\endalign
$$
Hence, 
$$
\sum_{n=0}^\infty \frac 1{n!} (D^n f)(x_0,y_0)
 = \sum_{n=0}^\infty \frac 1{n!}\sum_{i + j = n}\binom ni
\Delta x^i \Delta y^j \frac{\d^n f}{\d x^i \d y^j}(x_0,y_0),
$$
which is the expression arrived at earlier in formula (\EqTwo).

\subhead The Error \endsubhead
One can analyze the error $R_N$ made if you stop with terms of
degree $N$ for multidimensional Taylor series in a manner similar
to the case of ordinary Taylor series.
The  exact form of this remainder  $R_N$ is not as important as the
\outind{remainder, multidimensional Taylor series}%
fact
 that {\it as a function of $|\Delta\r| = \sqrt{\Delta x^2 + \Delta y^2}$\/}
it is $O(|\Delta\r|^{N+1})$.
Thus, we may write in general
$$
f(x + \Delta x, y + \Delta y)
 = \sum_{n=0}^N \frac 1{n!}\sum_{i + j = n}\binom ni
\frac{\d^n f}{\d x^i \d y^j}
\Delta x^i \Delta y^j + O(|\Delta r|^{N+1}).
$$
  One important case is
$N = 1$ which gives
$$
f(x + \Delta x, y + \Delta y) = f(x,y) + f_x\Delta x + f_y\Delta y +
 O(|\Delta\r|^2).
$$
(You should compare this with our previous discussion of the linear
approximation in Chapter III.)  

The case $N= 2$ is also worth writing
out explicitly.  It may be put in the following form
\nexteqn
$$
\multline
f(x + \Delta x, y + \Delta y) \\
 = f(x,y) + f_x\Delta x + f_y\Delta y
 + \frac 12 (f_{xx}\Delta x^2 + 2f_{xy}\Delta x \Delta y + f_{yy} \Delta y^2)
+ O(|\Delta\r|^3).
\endmultline
$$
We shall use this relation in the next section.

\subhead Derivation of 
the Remainder Term \endsubhead
You may want to skip the following discussion.

There is a multidimensional analogue of Taylor's formula
{\it with remainder}.  
The easiest way to get it is to derive it from the
one dimensional case as follows.   We concentrate, as above,
on the case $n = 2$.   Consider the function $f(\r) = f(x,y)$
near the point $\r_0 = (x_0,y_0)$ and put
$\Delta\r = \lb \Delta x, \Delta y \rb = \lb x - x_0, y - y_0 \rb$.
Consider the line segment from $\r_0$ to $\r = \r_0 + \Delta\r$
parameterized by
$$
   \r_0 + t\Delta r,\qquad\text{where } 0\le t \le 1.
$$
Define a function of one variable
$$
g(t) = f(\r_0 + t\Delta\r), \qquad 0\le t \le 1.
$$
By the 1-dimensional Taylor's formula, we have
$$
g(t) = g(0) + g'(0)t + \frac 12 g''(0)t^2 + \dots +
       \frac 1{N!}g^{(N)}(0)t^N + R_N(t)
$$
where the remainder $R_N(t)$ is given by a certain integral.
Putting $t = 1$, we obtain
$$
g(1) = g(0) + g'(0) + \frac 12 g''(0) + \dots +
       \frac 1{N!}g^{(N)}(0) + R_N(1)
$$
where
$$
R_N(1) = \frac 1{N!}\int_0^1 (1 - s)^N g^{(N+1)}(s)ds.
$$
We need to express the above quantities in terms of the
function $f$ and its partial derivatives.  This is
not hard if we make use of the chain rule
$$
\frac{dg}{dt} = \nabla f\cdot \frac{d}{dt}(\r_0 + t\Delta\r)
    = \nabla f\cdot\Delta\r.
$$
We can write this as a symbolic relation between operators
$$
\frac{d}{dt}g  = \Delta\r\cdot\nabla f = (\Delta x\frac{\d}{\d x}
     + \Delta y\frac{\d}{\d y})f = D f.
$$
Hence,
$$
\frac{d^n}{dt^n}g = D^n f
$$
so evaluating at $t = 0, \r = \r_0$ yields
$$
f(x,y) = \sum_{n=0}^N \frac 1{n!} D^n f(x_0,y_0) + R_N(1).
$$
To estimate the remainder, consider
$$
\frac{d^{N+1}}{ds^n}g(s) = \sum_{i+j = N+1}\binom {N+1}i
\Delta x^i\Delta y^j \frac{\d^{N+1}f}{\d x^i \d y^j}(\r_0 + s\Delta\r). 
$$
Assume that within the indicated range
$$
\left |\frac{\d^{N+1}f}{\d x^i \d y^j}\right | \le M. 
$$
Then
$$
|\frac{d^{N+1}}{ds^n}g(s)| \le \sum_{i+j = N+1}\binom {N+1}i
|\Delta x|^i|\Delta y|^j M = M(|\Delta x| + |\Delta y|)^{N+1}. 
$$
The quantity $|\Delta x| + |\Delta y|$ represents the sum of
the lengths of the legs of a right triangle with hypotenuse
$|\Delta\r|$.  A little plane geometry will convince you that
$|\Delta x| + |\Delta y| \le \sqrt 2 |\Delta\r|$.  Hence,
$$
\left |\frac{d^{N+1}}{ds^n}g(s)\right | \le 
M(\sqrt 2 |\Delta\r|)^{N+1} = 2^{(N+1)/2}M |\Delta\r|^{N+1}.
$$
Hence,
\nexteqn
$$
\align
|R_N(1)| &\le \frac 1{N!}2^{(N+1)/2}M |\Delta\r|^{N+1}\int_0^1(1 -s)^N ds
\intertext{or}
|R_N(1)| &\le \frac {2^{(N+1)/2}M}{(N+1)!}|\Delta\r|^{N+1}.\tag\eqn
\endalign $$

\mar{s8-11.ps}
A similar analysis works for functions $f:\R^n \to \R$, but the
factor in the numerator will be more complicated for more than
two variables.
\bigskip
\input chap8.ex9
\bigskip
\nextsec{Local Behavior of Functions and Extremal Points}
\head \sn. Local Behavior of Functions and Extremal Points \endhead

In your one variable calculus course, you learned how to find
maximum and minimum points for graphs of functions.   We want
to generalize the methods you learned there to functions of
several variables.   Among such problems, one usually distinguishes
{\it global\/} problems from {\it local\/} problems.  A
{\it global\/} maximum (minimum) point is one at which the
function takes on the largest (smallest) possible value for
all points in its domain.  A {\it local\/} maximum (minimum)
point is one at which the function is larger (smaller)
than its values at all nearby points. 
Thus, the bottom of the crater in the volcano Mauna Loa is a local
minimum but certainly not a global minimum,   Generally, one may use
the methods of differential calculus to determine local
maxima or minima, but usually other considerations must be
brought into play to determine the global maximum or minimum.
In this section, we shall concern ourselves only with 
local extremal points.

First, let's review the single variable case.   Suppose
$f:\R \to \R$ is a function, and we want to determine
where it has local minima.  If $f$ is sufficiently smooth,
we may begin to expand it near a point $x$ in a Taylor series 
$$
f(x + \Delta x) = f(x) + f'(x)\Delta x +  O(\Delta x^2)
$$
or
$$
f(x + \Delta x) - f(x) = f'(x)\Delta x + O(\Delta x^2).
$$
The terms on the right represented by $O(\Delta x^2)$ are generally
small compared to the linear term $f'(x)\Delta x$, so provided
$f'(x) \not = 0$ and $\Delta x$ is small, we may write
\nexteqn
$$
f(x + \Delta x) - f(x) \approx f'(x)\Delta x. \tag\eqn
$$
 On the other hand, at a local minimum,
we must have  $f(x + \Delta x) - f(x) \ge 0$, and that contradicts
(\eqn) since the quantity on the right changes sign when $\Delta x$
changes sign.  The only way out of this dilemma is to conclude
that $f'(x) = 0$ at a local minimum. 
  Similar reasoning applies at
a local maximum.

Suppose $f'(x) = 0$.   Then, the approximation (\eqn) is
no longer valid, and we must consider higher order terms.
Continuing the 
Taylor expansion yields
$$
\align
f(x + \Delta x) - f(x) &= f'(x)\Delta x + \frac 12 f''(x)\Delta x^2
   + O(\Delta x^3) \\
&= \frac 12 f''(x)\Delta x^2 + O(\Delta x^3).
\endalign $$
Reasoning as above, we conclude that if $f''(x) \not= 0$
and $\Delta x$ is
 small, the quadratic term on the right dominates.
That means that $f(x + \Delta x) \ge f(x)$ if
$f''(x) > 0$, in which case we conclude that $x$ is
a local minimum point.  Similarly, $f(x + \Delta x) \le f(x)$
if $f''(x) < 0$, in which case we conclude that $x$ is a local
maximum point.   If $f''(x) = 0$, then of course no conclusion is possible.
As you know, in that case, $x$ might be  a local minimum point,
a local maximum point, or a point at which the graph has a point
of inflection.  
Taking the Taylor series one step further might yield
additional information, but we will leave that
for you to investigate on your own.

We now want to generalize the above analysis to functions
$f:\R^2 \to \R$.   If we assume that $f$ is sufficiently smooth,
then it may be expanded near a point $(x,y)$
$$
f(x + \Delta x, y + \Delta y) = f(x,y) + \nabla f\cdot\Delta\r
 + O(|\Delta\r|^2)
$$
which may be rewritten
$$
f(x + \Delta x, y + \Delta y) - f(x,y) = \nabla f\cdot\Delta\r
 + O(|\Delta\r|^2).
$$
As above, if $|\Delta\r|$ is small enough, and $\nabla f
\not= \bold 0$, then the linear term on
the right dominates.  Since in that case the linear term can
be either positive or negative depending on $\Delta\r$,
it follows that the quantity on the left cannot be of a single
sign.  Hence, if the point $(x,y)$ is either a local maximum
or a local minimum point, we must necessarily have
$
\nabla f = 0.
$ 
In words, {\it at a local extremal point of the function $f$,
the gradient $\nabla f = 0$}.

A point $(x,y)$ at which $\nabla f = \bold 0$ is called
a {\it critical point\/} of the function.   For a smooth
\outind{critical point of a function}%
function, every local maximum or minimum is a critical
point, but the converse is not generally true.  (This
parallels the single variable case.)  All the assertion
$\nabla f = 0$ tells us is that the tangent plane
to the graph of the function is horizontal at the critical
point.

\example{Examples}
Let $f(x,y) = x^2 + 2x + y^2 -6y$.   Then
$$
\nabla f = \lb f_x, f_y \rb = \lb 2x + 2, 2y - 6 \rb.
$$
Setting this to zero yields $2x + 2 = 0, 2y - 6 = 0$ or
$x = -1, y = 3$.  Hence, $(-1,3)$ is a critical point.
We can see that this point is actually a local minimum point
by completing the square.
$$
 f(x,y) = x^2 + 2x + y^2 - 6y = x^2 + 2x + 1 + y^2 - 6y + 9 - 10
   = (x + 1)^2 + (y - 3)^2 -10. 
$$
The graph of this function is a circular paraboloid (bowl) with vertex
 $(-1,3,-10)$, and it certainly has a minimum at $(-1,3)$.
(It is even a global minimum!)

Consider on the other hand, the function $f(x,y) = x^2 - y^2$.
Setting $\nabla f = \bold 0$ yields
$$
\lb 2x, -2y \rb = \lb 0, 0\rb
$$
or $x = y = 0$.   Hence, the origin is a critical point.  However,
we know the graph of the function is a hyperbolic paraboloid
with a {\it saddle point\/} at $(0,0)$.   Near	
 a saddle point,
the function increases in some directions and decreases in others,
so such a point is neither a local maximum nor a local minimum.
\endexample.

\smallskip
\centerline{\epsfbox{s8-12.ps}}
\medskip
As in the single variable case, the analysis may be carried
further by considering second derivatives and quadratic terms.  Suppose
$\nabla f = \bold 0$, and extend the Taylor expansion to
include the quadratic terms.  We have
$$
f(x+ \Delta x, y + \Delta y) - f(x,y)
= \frac 12(f_{xx}\Delta x^2 + 2 f_{xy}\Delta x\Delta y
 + f_{yy}\Delta y^2) + O(|\Delta\r|^3).
$$
In most circumstances, the quadratic terms will dominate,
so we will be able to determine the local behavior by
examining them.  However, because the expression is so
much more complicated than in the single variable case,
the theory is more complicated.  To save writing,
put $A = f_{xx}, B = f_{xy}, C = f_{yy}, u = \Delta x$,
and $v =\Delta y$.   Then we want to look at
the graph of
$$
 z = Q(u,v) =  Au^2 + 2Buv + Cv^2
$$
in the neighborhood of the origin.
Such an expression is called a {\it quadratic form}.

Before considering the general case we consider some examples.

(a) $z = 2u^2 + v^2$.  The graph is a bowl shaped surface which
opens upward, so the origin is a local minimum.  Similarly,
the graph of $z = -u^2 - 5v^2$ is a bowl shaped surface
opening downward, and the origin is a local maximum.
\medskip
\centerline{\epsfbox{s8-13.ps}}
\medskip

(b) $z = uv$.  The graph is a saddle shaped surface, and the
origin is neither a local minimum nor a local maximum.
\medskip
\centerline{\epsfbox{s8-14.ps}}
\medskip
(c) $z = u^2$.  The graph is a `trough' centered on the $u$-axis
and opening upward.  The entire $v$-axis is a line of local minimum
points.  Similarly, the graph of $z = -v^2$ is a trough centered
on the $u$-axis and opening downward.
\medskip
\centerline{\epsfbox{s8-15.ps}}
\medskip
We now consider the general case.
  Suppose
first that $A\not=0$.  Multiply through by $A$ and then
complete the square 
$$
\align
AQ(u,v) &= A^2u^2 + 2ABuv + B^2v^2 - B^2v^2 + ACv^2 \\
&= (Au + Bv)^2 + (AC - B^2)v^2.
\endalign
$$
The first term is a square, so it is always non-negative, so
the signed behavior of the expression depends on the quantity
$\Delta = AC - B^2$.   $\Delta$ is called the
{\it discriminant of the quadratic form}. 
\outind{discriminant of a quadratic form}%

If $\Delta > 0$, the expression is
a sum of squares so it is always non-negative.  Even better,
it can't be zero unless $u = v = 0$.  For, if the sum is
zero, it follows $Au + Bv = 0$ and also $v = 0$.  Since
we assumed $A \not=0$, that means that $u = 0$.  Hence,
if $A\not= 0$ and $\Delta > 0$, then $AQ(u,v)$
is always positive (except for $u=v=0$), and $Q(u,v)$
 has the {\it same sign as\/} $A$.  Hence, the origin
is a local minimum if $A > 0$, and it is a local maximum
if $A < 0$.  In this case the form is called {\it definite}.
\outind{definite quadratic form}%
\outind{quadratic form}%

If $\Delta < 0$, the expression is a difference of squares,
so it is sometimes positive and sometimes negative.   The
graph of $z = AQ(u,v) = (Au + Bv)^2 - |\Delta|v^2$ is a saddle shaped
surface which intersects  the $u,v$-plane in the locus
$$
Au + Bv = \pm \sqrt{|\Delta|}v,
$$
which is a pair of lines intersecting at the origin.
These lines divide the $u,v$-plane into four regions with
the surface above it in two of the regions and below it
in the other two.  In this case, the form is called
{\it indefinite}.
\outind{indefinite quadratic form}%
\medskip
\centerline{\epsfbox{s8-15a.ps}}
\medskip

If $\Delta = 0$, 
$$
 z = AQ(u,v) =  (Au + Bv)^2
$$
defines a trough shaped surface which intersects the $u,v$-plane
along the line $Au + Bv = 0$.   The graph of $z = Q(u,v)$
either lies above or  below the $u,v$-plane,
depending on the sign of $A$.   If $\Delta = AC - B^2 = 0$,
we say that the quadratic form is {\it degenerate}.
\outind{degenerate quadratic form}%

If $A = 0$ and $B\not= 0$, then $\Delta = AC - B^2 = -B^2 < 0$, and
$$
Q(u,v) = 2Buv + Cv^2 = (2Bu + Cv)v
$$
can be positive or negative.  This just amounts to a special
case of the analysis above for an indefinite form.  

If $A = B = 0$ and $C\not = 0$, then $\Delta = 0$, and
$$
Q(u,v) = Cv^2
$$
and this is a special case of a degenerate form.   

If $A = B = C = 0$, there isn't really anything to consider
since we don't really have a quadratic form.


The above analysis tells us how the quadratic terms in
$$
f(x+\Delta x, y+\Delta y) - f(x,y)
 = \frac 12(f_{xx}\Delta x^2 + 2f_{xy}\Delta x \Delta y
+ f_{yy}\Delta y^2) + O(|\Delta\r|^3)
$$
behave.   
If $|\Delta\r|$ is small, the behavior on the
right is determined primarily by
 $Q(\Delta x, \Delta y)$, with the cubic and higher order
providing a slight distortion.   In particular we have

\proclaim{Sufficiency Conditions for Local Extrema}
Assume $f:\R^2 \to \R$ has continuous second partials, and that
$(x,y)$ is a critical point.  Let
$$
\Delta = f_{xx}f_{yy} - f_{xy}{}^2
$$
where everything is evaluated at the critical point.

(a) If $\Delta > 0$,
then the critical point is a local minimum
if $f_{xx} > 0$,  or a local maximum if $f_{xx} < 0$. 
(The quadratic approximation is an elliptic paraboloid.)

(b) If $\Delta < 0$, then the critical point is neither
a local maximum nor a local minimum.
(The quadratic approximation is a saddle.)

(c) If $\Delta = 0$, any of the above possibilities might
occur.
\endproclaim

The proofs of these assertions require a careful analysis in
which the different order terms are 
compared to one another.  We shall not go into these matters
here in detail, but a few remarks might be enlightening.
The reason why case (c) is ambiguous is not hard to see.
If $\Delta = 0$, the graph of the degenerate quadratic
form is a trough which contains a line of minimum (or
maximum) points.  {\it Along that line\/}, the cubic order terms
will control what happens.  We could have either a local
maximum or a local minimum or neither depending on the
shape of the trough and the contribution from the cubic
terms.   In case $\Delta < 0$,  it 
 is harder to see how the cubic terms perturb the saddle
shaped graph for the quadratic form, but it may still
be thought of as a saddle. 

There is one slightly confusing point in the use of the
above formulas.  It appears that $f_{xx}$ is playing a
special role in (a).  However, it is in fact true
in case (a) that $f_{xx}$ and $f_{yy}$ have the same
sign.  Otherwise,
$\Delta = f_{xx}f_{yy} - f_{xy}{}^2 < 0$. 

\nextex
\example{Example \en} We shall classify the critical points of
the function defined by  $f(x,y) = x\sin y$.  
First, to find the critical points, solve
$$
\align
\frac{\d f}{\d x} &= \sin y = 0 \\
\frac{\d f}{\d y} &= x\cos y = 0.
\endalign
$$ 
From the first equation, we get $y = k\pi$ where $k$ is any
integer.  Since $\cos (k\pi) \not=0$, the second equation
yields $x = 0$.   Hence, there are infinitely many critical
points $(0,k\pi)$ where $k$ ranges over all possible integers.

To apply the criteria above, we need to calculate the discriminant.
We
have
$$\align
f_{xx} &= 0 \\
f_{yy} &= -x\sin y \\
f_{xy} &= \cos y \\
\Delta &= 0 - \cos^2y = -\cos^2 y.
\endalign
$$
At $(0,k\pi)$, we have $\Delta = -(\pm 1)^2 = -1 < 0$.
Hence, (b) applies and every critical point is a saddle point.
\endexample

\nextex
\example{Example \en}  Let  
$$
f(x,y) = x^3 + y^3 - 3x^2 + 3y^2 +2.
$$
To find the critical points, solve
$$
\align
f_x &= 3x^2 -6x = 0 \\
f_y &= 3y^2 + 6y = 0.
\endalign
$$
The solutions are $x = 0, 2$ and $y = 0, -2$, so the critical
points are $$(0,0), (0,-2), (2,0), (2, -2).$$

To classify these, calculate
$$
\align
f_{xx} &= 6x - 6 \\ 
f_{yy} &= 6y + 6 \\
f_{xy} &= 0 \\
\Delta &= 36(x -1)(y + 1).
\endalign
$$

At $(0,0)$, $\Delta = - 36 < 0$, so $(0,0)$ is a saddle
point.

At $(0,-2)$, $\Delta = 36 > 0$.  Since, $f_{xx} = -6 < 0$,
$(0,-2)$ is a local maximum.

At $(2,0)$, $\Delta = 36 > 0$.  Since, $f_{xx} = 6 > 0$,
$(2,0)$ is a local minimum. 

At $(2,-2)$, $\Delta = -36 < 0$, so $(2, -2)$ is a saddle
point. 
\endexample
\smallskip
\centerline{\epsfbox{s8-17.ps}}
\medskip
\nextex
\example{Example \en}
Let 
$$ f(x,y) = x^3 + y^2.$$
We have
$$
f_x = 3x^2\qquad f_y = 2y,
$$
so $(0,0)$ is the only critical point.
Moreover
$$
f_{xx} = 6x,\qquad f_{yy} = 2,\qquad\text{and } f_{xy} = 0.
$$
Hence, $\Delta = 12x = 0$ at $(0,0)$.  Hence, the
criteria yield no information in this case.  In fact,
the point is not a maximum, a minimum, or a saddle point.
\medskip
\centerline{\epsfbox{s8-18.ps}}
\medskip
The example $f(x,y) = x^4 + y^2$ is very similar.  It also
has a degenerate critical point at $(0,0)$ but in this
case it is a minimum point.
\endexample

If the quadratic terms vanish altogether, then the cubic
terms dominate and various interesting possibilities arise.
One of these is called a `monkey saddle', and you can imagine
its shape from its name.  (See the Exercises.) 

\subhead Higher Dimensional Cases \endsubhead
For a {\it smooth\/}
function $f:\R^n \to \R$ with $n > 2$, many of the above
considerations still apply.    If $\nabla f = 0$ at a point
$\r$, then the point is called a {\it critical point}.  Local maxima
\outind{critical point of a function}%
and minima occur at critical points, but there are many
other possibilities.  Examination of the quadratic terms
may allow one to determine precisely what happens, but even in
the case $n = 3$ this can be much more complicated than the
case of functions of two variables.

\bigskip
\input chap8.ex10
\endchapter
\closeseg{chap8}
\enddocument
