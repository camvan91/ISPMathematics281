\documentstyle{book}
%\syntax
\loadbold
\input extra.tex
\input epsf.tex
\input le2.sty
\maketoctrue
\mardiagtrue
\indextrue
\openseg{chap1}
\Monograph
\startup
\topmatter
\nextchap{Vectors}
\title\chapter{\chapno} Vectors\endtitle
\endtopmatter
\document
\nextsec{Introduction}
\head \sn. Introduction \endhead
%\def\i{\bold i}\def\j{\bold j}\def\k{\bold k}
%\def\u{\bold u}\def\v{\bold v}

A {\it vector\/} is a quantity which is characterized by
a {\it magnitude\/} and a {\it direction}.   Many quantities are best
described by vectors rather than
numbers.  For example, when driving a car,
\outind{vector}%
 it may be sufficient to
know your speed, which can be described by a single number,
 but the motion of an airplane must be described
by a vector quantity called velocity which takes into account its
direction as well as its speed.
%D1 vector
Ordinary numerical quantities are called {\it scalars\/}
when we want to emphasize that they are not vectors.

\mar{s1-1.ps}
In print, vectors are almost always denoted by bold face symbols, e.g.,
$\bold F$, but when written, one uses a variety of mechanisms for
distinguishing vectors from scalars.  One common such notation is
$\overrarrow F$.  The magnitude of a vector $\bold F$ is denoted
$|\bold F|$.   Indicating its direction symbolically is a bit more
difficult, and we shall discuss that later.      

Vectors are  represented {\it geometrically\/} by
{\it directed line segments.} The length of the line segment is
the magnitude of the vector and its direction is the direction of
\outind{directed line segment}%
\outind{vector, representation by directed line segment}%
the vector.
%D2  Two dir line segs for same vector.  $\overrarrow{AB}=\overrarrow{BC}$
Note that
parallel lines segments of the {\it same length\/} and {\it same direction\/} 
represent the {\it same vector}.  In drawing pictures, there
is a tendency to identify a directed line segment with the vector
it represents, but this can lead to confusion about
how vectors are used.   In general, there are infinitely many
directed line segments which could be used to represent the
same vector.   You should always remember that {\it you
are free to put the tail end of a vector at any point which might
be convenient for your purposes.}  We shall often use the notation
$\overrarrow{AB}$ for the vector represented by the directed line
segment from $A$ to $B$.

\mar{s1-2.ps}
Many of the laws of physics relate vector quantities.  For example,
Newton's Second Law
$$
        \bold F = m\bold a
$$
relates two vector quantities:  force $\bold F$ and acceleration
$\bold a$.  The constant of proportionality $m$ is a scalar quantity
called the mass.


\subhead Operations with vectors \endsubhead
It is a familiar experience that when velocities are combined, their
magnitudes do not add.  For example, if an airplane flies North
at 600 mph against  an east  wind of 100 mph,  the resultant  velocity
will be somewhat west of north and its magnitude certainly won't
be 700 mph.  To calculate the correct velocity in this case, we need
{\it vector addition\/}, which is defined geometrically as follows.
Suppose  $\bold a, \bold b$ are vectors.  To add them, choose directed line
\outind{vector addition}%
\outind{parallelogram law}%
segments representing them, with tails at the same point, and consider
the vector represented by the diagonal of the resulting parallelogram.
%D3   Parallelogram  law
We call that vector $\bold a + \bold b$.  This is called the parallelogram
law of addition.   There is also another way to do the same thing
called the triangle law of addition.  See the diagram.
\medskip
\centerline{\epsfbox{s1-3.ps}}
\medskip
Note that for the triangle law, we place the tail of one directed
line segment on the head of the other, taking advantage of of freedom
to place the tail where needed.  If the vectors $\bold a$ and $\bold b$
have the same or opposite directions, then the diagrams become degenerate
(collinear) figures.   (You should draw several cases for yourself to
make sure you understand.)

As we saw in Newton's Law, we sometimes want to {\it multiply\/}
a vector by a scalar.  This is defined as follows.  If $s$ is a scalar
and $\bold v$ is a vector, then $s\bold v$ is the vector with magnitude
$|s||\bold v|$ and its direction is either the same as that of $\bold v$,
if $s > 0$, or opposite to $\bold v$, if $s < 0$.

\mar{s1-4.ps}
Note that the above definition omits the case that $s = 0$.  In that
case, we run into a problem, at least if we represent vectors by directed
line
segments.  It does not make sense to talk of the direction of a
directed line segment with length zero.  For this reason, we introduce
a special quantity we call the {\it zero vector\/} which has
magnitude zero and no well defined direction.  It is not 
a vector as previously defined, but we allow it as
a degenerate case which is needed so operations with vectors will make
sense in all cases.  
 The zero vector is denoted by a bold face zero
\outind{zero vector}%
$\bold 0$ or in writing by $\overrarrow 0$.  With that notation,
$s\v = \bold 0$ if $s = 0$.  Note also that a degenerate
case of the parallelogram law yields $\v + \bold 0 = \v$ for
any vector $\v$.

You can {\it subtract\/} a vector $\bold a$ from a vector
\outind{vectors, subtraction of}%
$\bold b$  by adding the opposite vector $-\bold a = (-1)\bold a$.
Study the accompanying diagram for some idea of how
$\bold b - \bold a$ might be represented geometrically.
\medskip
\centerline{\epsfbox{s1-5.ps}}
\medskip
Note that if $\bold a$ and $\bold b$ are represented by directed
line segments with the same tail, then the line segment from the end
of $\bold a$ to the end of $\bold b$ represents $\bold b - \bold a$.

	In the study of motion, which is of interest both in physics
and in mathematics, it is common to use the so called {\it position
vector\/} (also called {\it radius vector}).  Thus, if a particle
\outind{position vector}%
moves on a path, its position can be specified as the endpoint 
$P$ of
a directed line segment from a common {\it origin} $O$.  The position vector
is $\bold r = \overrarrow{OP}$.   In this context, we are often interested
in comparing the position vectors $\bold r_1$ and $\bold r_2$ of the
same particle at different times (or of two different particles).  As
the diagram indicates, the difference $\bold r_2 - \bold r_1$ is
associated with the directed line segment from the first position to the
second position.  This is called the {\it displacement vector\/}, and
it would be the same for any choice of common origin $O$.
\medskip
\centerline{\epsfbox{s1-6.ps}}
\medskip

The operations defined above satisfy the usual laws of algebra.
Here are some of them
\def\b{\bold b}\def\a{\bold a}\def\c{\bold c}
$$
\align
(\a + \b) + \c &= \a + (\b + \c)\qquad \text{Associative Law}\\
\a + \b &= \b + \a\qquad \text{Commutative Law}\\
s(\a + \b) &= s\a + s\b \\
(s + t)\a &= s\a + t\a \\
(st)\a &= s(t\a)
\endalign$$
There are several others which you can probably invent for yourself.
These rules are not too difficult to check from the definitions we
gave, but we shall skip that here.  (You might try doing it yourself
by drawing the appropriate diagrams.)

\subhead Components \endsubhead
The geometric definitions above result in nice pictures and nourish
our intuition, but they are quite difficult to calculate with.  To
make that easier,
\outind{components}%
it is necessary to introduce coordinate systems and thereby
reduce geometry to algebra and arithmetic.   We start with the
case of vectors in the plane since it is easier to visualize, but of course
to deal with the the real world, we shall have to also extend our
notions to space.

\emar{s1-6a.ps}{-75}
Recall that a coordinate system in the plane is specified by choosing
an origin $O$ and then choosing two perpendicular axes meeting at
the origin.  These axes are chosen in some order so that we know which
axis (usually the $x$-axis) comes first and which (usually the
$y$-axis) second.   Note that there are many different coordinate systems
which could be used although we often draw pictures
\outind{orientation of axes}%
 as if there were only one.
%D8  Two orientations 
In physics, one often has to think carefully about the
coordinate system because
choosing it appropriately  may greatly simplify the
resulting analysis.  Note that the axes are usually drawn with {\it
right hand orientation\/} where the right angle from the positive
$x$-axis to the positive $y$-axis is in the counter-clockwise
direction.  However, it would be equally valid to use the
{\it left hand orientation\/} in which that angle is in the
clockwise direction.  One can easily switch the orientation of
\outind{left hand orientation}%
a coordinate system by reversing one of the axes.   (The concept of
orientation is quite fascinating and it arises in mathematics,
physics, chemistry, and even biology in many interesting ways.
Note that almost all of us base our intuitive concept of orientation
on our inborn notion of ``right'' versus ``left''.)
\medskip
\centerline{\epsfbox{s1-7.ps}}
\medskip
	Given a vector $\bold v$, the projections---of any directed line
segment representing it---onto the coordinate axes are called the 
{\it components\/} of the vector.   The components of $\bold v$ are
often displayed
as an ordered pair $\lb v_1, v_2 \rb$ where each component is numbered
according to the axis it is associated with.  Another common notation
is $\lb v_x, v_y\rb$.   We shall use both. 
%D9 Components
Notice that the Pythagorean Theorem tells us that
$$
        |\v| = \sqrt{v_1{}^2 + v_2{}^2}.
$$
  Our notation
 distinguishes
between the coordinates $(x,y)$ of a point $P$ in the plane, and 
the components $\lb v_x, v_y\rb$ of a vector.   This is to emphasize
the difference between a vector and a point.  That distinction
is a bit esoteric, particularly in the analytic context,
since a
pair of numbers is just that  no matter what notation we use.  Hence,
you will find that many mathematics and physics books make no such
distinction.
(Review in your mind the distinction between a point $P$ and the
position vector $\overrarrow{OP}$.  What relation exists between
the coordinates of $P$ and the components of $\overrarrow{OP}$?)

\mar{s1-8.ps}
There is another common way to indicate components.  Let $\bold i$
denote a vector of length 1 pointing in the positive direction along
the $x$-axis, and let $\bold j$ denote the corresponding 
unit vector for the $y$-axis.  (These are also commonly written
$\hat\bold x$ and $\hat\bold y$.)   Then the diagram
makes clear that
$$
    \bold v = v_1\bold i + v_2 \bold j .
$$

Vector operations are mirrored by corresponding operations for
components.  Thus, if $\bold u$ has components $\lb u_1,u_2\rb$ and
$\bold v$ has components $\lb v_1, v_2\rb$, then
$$\align
  \bold u + \bold v &= u_1 \i + u_2\j + v_1\i + v_2\j \\
&= (u_1 + v_1)\i + (u_2 + v_2)\j
\endalign
$$
from which we conclude that the components of $\u + \v$ are
$\lb u_1 + v_1, u_2 + v_2\rb$.  (The diagram below exhibits
 a more
geometric argument.
\medskip
\centerline{\epsfbox{s1-9.ps}}
\medskip
   (Note that I only drew one of the many diagrams
needed to handle all possible cases.)  In words, we may restate the rule
as follows
\block
{\it the components of the sum of two vectors are the sums of the
components of the vectors.}
\endblock
A similar argument shows that the components of $s\v$ are
$\lb sv_1, sv_2\rb$, i.e.,
\block
{\it the components of a scalar multiple of a vector are that
multiple of the components of the vector.}
\endblock

The same sort of considerations  apply in space.  To set up a coordinate
system, we first choose an origin $O$, and then choose, in a some
order, three mutually perpendicular axes through $O$,
 each with a specified
positive direction.  These are usually called the $x$-axis, the
$y$-axis, and the $z$-axis.  
Since any two {\it intersecting\/} lines in space determine a plane,
we can think of the first two axes generating an $x,y$-plane.  Since
the $z$-axis must be perpendicular to this plane, the line along which
it lies is completely determined, but we have two possible choices
for its positive direction.   
(See the diagram.)
%D10  Orientation of axes
A set of axes in space has the {\it right hand orientation\/}
\outind{orientation of axes in space}%
if when you point the fingers of your right hand from the positive
$x$-axis to the positive $y$-axis, your upright thumb points in the
direction of the positive $z$-axis.  Otherwise, it has
the {\it left hand orientation}.   As in the plane case, reversing the
direction of one axis reverses the orientation.   Almost all authors
today use the right
hand orientation for coordinate axes.

\mar{s1-10.ps}
	Given a set of coordinate axes,  a point $P$ in space is
assigned coordinates $(x,y,z)$  as follows.  Let the origin
$O$ and the point $P$ be opposite vertices of a rectangular box.
%D11
The coordinates $x, y$, and $z$ are the (signed) magnitudes of the
sides of this box.   Points with $x > 0$ are in front of the
$y,z$-plane, and points  with $x < 0$ are in back of that plane.  You
should think out the possibilities for the signs of $y$ and $z$.
  A point has coordinate $x = 0$ if and only if it lies in the
$y,z$-plane (in which case the ``box'' is degenerate).  Similarly,
$y = 0$ characterizes the $x,z$-plane and $z = 0$ characterizes the
$x,y$-plane.
\medskip
\centerline{\epsfbox{s1-11.ps}}
\medskip
Our previous discussion of vectors generalizes in a more or less obvious
way to space.  The components $\lb v_1, v_2, v_3\rb$ (sometimes
$\lb v_x, v_y, v_z \rb$) of a vector $\v$ are obtained by projecting
a directed line segment representing the vector onto each of the
coordinate axes.
%D12   Components
As before, we have
$$
  |\v| = \sqrt{v_1{}^2 + v_2{}^2 + v_3{}^2}.
$$
(Look at the diagram for an indication of why this is so.  It requires
two applications of the Pythagorean Theorem.)   In addition, the
same rules for components of sums and scalar multiples apply as before
except, of course, there is one more component to worry about.
%D13  Length

In space, in addition to $\i$ and $\j$, we have a third unit vector
$\k$ pointing along the positive $z$-axis, and any vector can be
resolved
$$
      \v = v_1\i + v_2\j + v_3\k
$$
in terms of its components.

\mar{s1-12.ps}
Often, to save writing, we shall not distinguish between a vector
and its components, and we will write
$$
\align
\v &= \lb v_1, v_2 \rb\qquad \text{in the plane,}\\
\v &= \lb v_1, v_2, v_3\rb\qquad \text{in space.}
\endalign
$$
This is an `abuse of notation' since a
 vector is not the same as its set
of components, which generally 
{\it depend on the choice of coordinate axes\/}.  But, it is
a convenient notation, and you usually won't get in trouble   
if you use it with discretion.

\subhead Higher dimensions and $\bold R^n$ \endsubhead
One can't progress very far in the study of science and mathematics
without encountering a need for higher dimensional ``vectors''.  For
example, physicists have known since Einstein that the physical
universe is best thought of as a 4-dimensional entity called
spacetime in which time plays a role close to that of the 
3 spatial coordinates.  Since, we don't have any way to deal %
\outind{$\noexpand\R^n$}%
intuitively with any higher dimensional geometries, we must
proceed by analogy with two and three dimensions, and the easiest
way to proceed is to generalize the analytic approach by adding 
additional coordinates.  Thus, in general, we consider
$n$-tuples
$$
     (x_1,x_2,\dots, x_n)
$$
where $n$ can be any positive integer.  The collection of all such
$n$-tuples is denoted $\bold R^n$, where the $\bold R$ refers to
the fact that the entries (coordinates) are supposed to be
real numbers.  From this perspective, it doesn't make a whole
\outind{$n$-tuples}%
lot of sense to distinguish points from vectors, so the two terms
are often used interchangeably.   
Vector operations in $\bold R^n$ may be {\it defined\/}
in terms of components.  
$$\align
     |(x_1,x_2,\dots, x_n)|&=
     \sqrt{x_1{}^2 + x_2{}^2 + \dots + x_n{}^2}, \\
     (x_1,x_2,\dots, x_n) +
     (y_1,y_2,\dots, y_n) &=
     (x_1+y_1,x_2+y_2,\dots, x_n+y_n),\\
     s(x_1,x_2,\dots, x_n) &=
     (sx_1,sx_2,\dots, sx_n).
\endalign $$
The case $n=1$ yields  ``geometry'' on a line, 
the cases $n = 2$ and $n = 3$ geometry in the plane and in space, and
the case $n = 4$ yields the geometry of ``4-vectors'' which
are  used in the special theory of relativity.
Larger values of $n$ are used in a
variety of contexts, some of which we shall encounter later in this
course.
\bigskip

% exercises together with answers
\input chap1.ex1

\nextsec{Kinematics and Vector Functions}
\head \sn. Kinematics and Vector Functions. \endhead

In physics, you will study the motion of particles, so we need to
investigate here the associated mathematics.

\nextex
\example{Example \en}
You probably learned in high school that the path of a projectile
moving under the force of gravity near the earth is pretty close
to a parabola.
%D1  Parabola with $v_0$
   If we  choose coordinates $x$ (for horizontal displacement)
and $y$ (for vertical displacement), the  path 
may be described by the equations
$$\align
  x &= v_{x 0}t \\
  y &= v_{y 0}t - \frac 12 g t^2,
\endalign $$
where $t$ denotes time,
 $v_{x0}$ and $v_{y0}$ are the components of a vector
$\v_0$ called the {\it initial velocity\/}, and $g$ is a constant
giving the acceleration of gravity at the surface of the Earth.
This can be simplified if we combine the two coordinate functions
of $t$ in a single vector quantity
$$ \r =  x\i + y\j = (v_{x0}t)\i + (v_{y0}t - \frac 12 g t^2)\j.$$
\endexample
\medskip
\centerline{\epsfbox{s1-13.ps}}
\medskip
We can think of this as expressing $\r$ as {\it a single vector
function\/} of time $t$:  $\r = \r(t)$.
(Note that $\r$ is just the position vector 
connecting the origin to the position of projectile
at time $t$.)   

	In general, we can think of any vector function
$\r = \r(t)$ as describing the {\it path\/}
 of a particle as it moves through space.   For such a function
\outind{vector function}%
we will have
\nexteqn
$$
\r = x(t)\i + y(t)\j + z(t)\k \tag\eqn
$$
so giving a vector function is equivalent to giving three
scalar functions  $x(t)$, $y(t)$, and $z(t)$.
The single vector equation (\eqn) can also be written as
three scalar equations
$$\align
   x &= x(t)\\
   y &= y(t)\\
   z &= z(t)
\endalign$$
 If, as in Example \en,  the motion is
restricted to a plane, then, with suitably a chosen coordinate
system,  we may  omit one coordinate, e.g., we may write 
$ \r = x(t)\i + y(t)\j$, or  $x = x(t), y = y(t)$.

\nextex
\xdef\tmpref{\en}
\example{Example \en.  Uniform Circular Motion}
Let
$$\align
x &= R \cos \omega t \\
y &= R \sin \omega t.
\endalign$$
The (end of) the position vector $\r = x\i + y\j$ traces out a
circle of radius $R$ centered at the origin.
%D3  Picture of circle.
To see this note that $x^2 + y^2 = R^2(\sin^2 \omega t + \cos^2 \omega t)
= R^2$ so the path is certainly a subset of that circle.  The exact part
of the circle traced out depends on which values of $t$ are prescribed
(i.e., on the {\it domain\/} of the function.)  If $t$ extends over
an interval of size $2\pi/\omega$ (i.e., $\omega t$ extends over
an interval of size $2\pi$), the circle will be traced exactly once,
but for other domains, only part of the circle may be traced or the circle
might be traced several times.  The constant $\omega$ determines the
\outind{uniform circular motion}%
rate at which the particle moves around the circle.  You should try some
representative values of $t$, say with $\omega = 1$ to convince yourself
that the circle is traced counterclockwise if $\omega > 0$ and
clockwise if $\omega < 0$.  (What about $\omega = 0$?)  

The vector equation for the circle would be
$\r = (R\cos \omega t)\i + (R\sin \omega t)\j$
or, in component notation,
$\r = \lb R\cos \omega t, R\sin \omega t\rb$.
\endexample

\nextex
\mar{s1-14.ps}
\example{Example  \en}
Let
$$\align
x &= R \cos \omega t \\
y &= R \sin \omega t \\
z &= b t.
\endalign$$
Then the position vector $\r = x\i + y\j + z\k$ traces out a path
in space.  The projection of this path in the $x,y$-plane
(obtained by setting $z = 0$) is the circle described in
Example \tmpref.  At the same time, the particle is rising
(assuming $b  > 0$) in the $z$-direction at a constant rate.
The resulting path is called a {\it helix}.
\outind{helix}%
(What if $b < 0$ or $b = 0$?)

The vector equation for the helix would be
$\r = (R\cos \omega t)\i + (R\sin \omega t)\j + bt\,\k$
or, in component notation,
$\r = \lb R\cos \omega t, R\sin \omega t, bt\rb$.
\endexample
\smallskip
\mar{s1-15.ps}
\smallskip
\subhead Velocity and Acceleration \endsubhead
\exno=0
Suppose the path of a particle is described by a vector function
$\r = \r(t)$.   The derivative of such a function may be defined
exactly as in the case of a scalar function of a real variable.
Let $t$ change by a small amount $\Delta t$, and put
$\Delta \r = \r(t + \Delta t) - \r(t)$.  Then define
$$
  \r'(t) = \frac{d\r}{dt} = \lim_{\Delta t \to 0}\frac{\Delta\r}{\Delta t}.
$$
Although this looks formally exactly like the scalar case, the geometry is
\outind{velocity vector}%
quite different.  Study the accompanying diagram.
\medskip
\centerline{\epsfbox{s1-16.ps}}
\medskip
$\Delta\r$ is the displacement vector from the position of the
particle at time $t$ to its position at time $t + \Delta t$.  It
is represented by the directed {\it chord\/} in the diagram.
As $\Delta t \to 0$, the direction of the chord approaches a limiting
direction, which we call the {\it tangent direction\/} to the curve.
The derivative $\r'(t)$ is also called {\it the instantaneous velocity},
and it is usually denoted $\v$ (or $\v(t)$ if we want to emphasize its
functional dependence on $t$.)
It is, of course, a vector, and we usually picture it with its tail
at the point with position vector $\r(t)$ and pointing (appropriately)
along the tangent line to the curve.




Calculating the derivative (or velocity) is much easier if we use
components.  If $\r = x\i + y\j + z\k$, then we may write
$\Delta r = \Delta x \i + \Delta y \j + \Delta z \k$, and
$$\align
\r'(t) &= 
\lim_{\Delta t \to 0}\frac{\Delta\r}{\Delta t}\\
& = \lim_{\Delta t \to 0}\frac{\Delta x}{\Delta t}\i +
\lim_{\Delta t \to 0}\frac{\Delta y}{\Delta t}\j
\lim_{\Delta t \to 0}\frac{\Delta z}{\Delta t}\k\\
&= \frac{dx}{dt}\i +
 \frac{dy}{dt}\j +
 \frac{dz}{dt}\k.
\endalign$$
In other words, {\it the components of the derivative of a vector function
are just the derivatives of the component functions}.

Of course, in calculus, one need not stop with the first derivative.
The second derivative 
$$\frac{d^2\r}{dt^2} = \r''(t) = 
\frac{d^2x}{dt^2}\i +
\frac{d^2y}{dt^2}\j +
\frac{d^2z}{dt^2}\k 
$$
is called the {\it acceleration\/}, and it is often denoted $\a$.
It may also be described as $\a = \dfrac{d\v}{dt}$.
\outind{acceleration vector}%

\nextex
\example{Example \en}
Let $x(t) = v_{x0}t, y(t) = v_{y0}t - (1/2)gt^2$ as in the previous
section.  Then
$$
     \v = \r'(t) = v_{x0}\i + (v_{y0} - gt)\j
$$
and, in particular, at $t = 0$, we have  $\r'(0) = v_{x0}\i + v_{y0}\j$.
In other words, the velocity vector at $t = 0$ is the vector
$\v_0$ with components $\lb v_{x0}, v_{y0} \rb$, as expected.
(Where on the path does the velocity vector point horizontally?)
Similarly, the acceleration
$$
 \a = \frac{d\v}{dt} =  0\i + (0 -g)\j = -g\j.
$$
is directed vertically downward and has magnitude $g$.   You are probably
familiar with this from your previous work in physics.
\medskip
\centerline{\epsfbox{s1-17.ps}}
\medskip
\endexample

\nextex
\example{Example \en}
Let $\r = R \cos \omega t \i + R \sin \omega t \j$, (i.e.,
$x = R \cos \omega t, y = R \sin \omega t$.) Then
%D6   circle with velocity
$$
\v = \frac{d\r}{dt} = (-R \omega \sin \omega t)\i + (R \omega \cos \omega t)\j
$$ 
and a little trigonometry will convince you that $\v$ is perpendicular
to $\r$.  For,  $\r$ makes angle $\theeta = \omega t$ with the positive
$x$-axis while $\v$ makes angle $\theeta + \pi/2$.   Hence, $\v$ is tangent
to the circle as expected.  Also,
\nexteqn
$$
 |\v| = \sqrt{R^2\omega^2\sin^2\omega t + R^2\omega^2\cos^2\omega t}
      = R\omega.\tag\eqn
$$
Hence, the {\it speed\/} $|\v|$ is constant. 

\mar{s1-18.ps}
	The acceleration is given by
$$
\a = \frac{d\v}{dt} = (-R\omega^2\cos\omega t)\i + (-R\omega^2\sin\omega t)\j
        = -\omega^2 \r.
$$
Hence, the acceleration is directed opposite to the position vector
and points from the position of the particle toward the origin.  This
is usually called {\it centripetal acceleration}.  Also,
$|\a| = \omega^2|\r| = \omega^2 R$.   By equation (\eqn),  $\omega = 
\dfrac{|\v|}R$, so
$$
 |\a| = \frac{|\v|^2}{R^2} R = \frac{|\v|^2}R.
$$ 
Note that in this example, acceleration results entirely from changes
in the direction of the velocity vector since its magnitude is constant.
In general, both the direction and magnitude of the velocity vector will
be changing.
\endexample

\nextex
\example{Example \en}
Let $\r = \lb R\cos\omega t, R\sin \omega t, bt \rb$ represent
a   
 helix as  above. 

\mar{s1-19.ps}
The velocity  and acceleration are  given by
$$\align
 \v &= \lb -R\omega\sin\omega t, R\omega\cos\omega t, b\rb \\
 \a &= \lb  -R\omega^2\cos\omega t, R\omega^2\sin\omega t, 0\rb.
\endalign$$
Since its third component is zero, $\a$ points parallel to the
\outind{helix}%
$x,y$-plane.  Also, if you compare the first two components with
what we obtained in the example of uniform circular motion, you will
see that $\a$ points from the position of the particle directly at
the $z$-axis.
\endexample

\nextex
\example{Example \en.  Uniformly Accelerated Motion}
Suppose the acceleration vector $\a$ is {\it constant}.   Then, just
as would be the case for scalar functions, we can {\it integrate\/}
the equation
$$
  \frac{d\v}{dt} = \a
$$
to obtain
$$
   \v = \a t + \c
$$
where $\c$ is a {\it vector constant}.  In fact, putting $t = 0$
shows that $\c = \v(0)$, the value of $\v(t)$ at $0$, and this
is usually denoted $\v_0$.   Thus, we may write  
$\dfrac{d\r}{dt} = \v = t\a + \v_0$.
(It is customary to put the scalar $t$ in front of the vector,
but it is not absolutely necessary.)  If we integrate this, we obtain
$$
  \r = \frac 12 t^2 \a + t \v_0 + \bold C,
$$
but putting $t = 0$, as above, yields $\bold C =\r(0) = \r_0$, 
the value of the position vector at $t = 0$.  Thus, we obtain
$$
  \r = \frac 12 t^2 \a + t \v_0 + \r_0.
$$
The path of such a particle is a {\it parabola}.  (Can you see
why?)

\mar{s1-20.ps}
The special case $\a = \bold 0$ is of interest.  This is called 
{\it uniform linear motion\/} and is described by
$$
   \r = t \v_0 + \r_0.
$$
The path is a straight line.  (See the diagram.)
Moreover, $\v = d\r/dt = \v_0$, so the velocity
vector is constant.  Of course, it points along the line of motion.
The particle moves along this line at constant speed.
\endexample
\medskip
\centerline{\epsfbox{s1-21.ps}}
\medskip
All our discussions have assumed implicitly that $\v \ne \bold 0$.
  If $\v$
vanishes for one particular value of $t$, there may not be a well
defined tangent vector at the corresponding point on
the path.  A simple example of this
would be
a particle which rises vertically in a straight line
with decreasing velocity until it reaches its maximum height where
it reverses direction and falls back along the same line.
At the point of maximum height, the velocity would be zero, and
it does not really make sense to talk about a tangent {\it vector\/} there
since we can't assign it a well defined direction.
If $\v(t)$ vanishes for all $t$ in an interval, then the particle
stays at one point without moving.
It would make even
less sense to talk about a tangent vector in that case.

% We have also assumed that all the vector functions we encountered
%  are continuous,
% but there are clearly cases where we may want to relax that
% assumption.  For
% example, a ball bouncing off the side of a billiard table
% may be modelled as a particle with velocity switching
% instaneously from one vector to another at the point of impact.
% In that case, it wouldn't make sense to talk about the velocity
% vector at the point of impact.  (That would be an idealized model,
% and a figuring out what a real physical billiard ball actually
% does at the `point' of impact is something I leave for your
% physics professor.)  

\mar{s1-22.ps}
\remark{Remark}
In our discussion of derivatives (and integrals) of vector
functions, we shall ignore for the present the issue of how {\it limits\/} are handled
for such functions.  This would be a matter of some importance
for a completely rigorous treatment because those concepts are 
defined by limits.  These issues are best postponed to a course in
{\it real analysis\/} where there is time for such matters.  It suffices
for now to say that such limits behave exactly as you would expect.
Also, one way to avoid worrying about the matter is to reduce all
limits for vector functions of a single variable $t$ to statements about
limits for the scalar component functions.
\endremark

\subhead Polar Coordinates \endsubhead
For motion in the plane where there is some sort of circular symmetry,
it is often more convenient to use polar coordinates.  This would
certainly be the case for circular motion, but it is also useful,
for example, in celestial mechanics where we think of a planet as
a particle moving in the gravitational field of a star.  In that
case, the gravitational force may be assumed to point directly toward
the origin, and Newton, confirming what Kepler had demonstrated, showed
that the motion would lie in a plane and follow a conic section.

Recall how polar coordinates are defined in the plane.  We choose
an origin $O$, and a (right handed) cartesian coordinate system with
that origin.  The polar coordinates $(r, \theeta)$
 of a point $P$ are the distance $r =|\r| = |\overrarrow{OP}|$ to
the origin and the angle $\theeta$ which the position vector $\r$
makes with the positive $x$-axis.
\outind{polar coordinates}%
\medskip
\centerline{\epsfbox{s1-23.ps}}
\medskip
By definition $r \ge 0$ since it is a distance.  Generally, $\theeta$
is allowed to have any value, but so that the point $P$ will
uniquely determine its polar angle $\theeta$, it is common to restrict
$\theeta$ to some interval of length $2\pi$.  Common choices are
$0\le \theeta < 2\pi$ or $-\pi \le \theeta < \pi$, but many other
choices are possible.  Note that in any case, there is no way to
define $\theeta$ unambiguously at the origin where $r = 0$.  

The set of points with a fixed value $r = a > 0$ constitute a circle
with radius $a$.  The set of points with a fixed value $\theeta = \alpha$
constitute a {\it ray\/}, i.e., a half line, emanating from the origin,
making angle $\alpha$ with the positive $x$-axis.
%D8 circle and ray.

\mar{s1-24.ps}
It is common in elementary mathematics books to interpret a point
with polar coordinates $(r,\theeta)$ where $r < 0$ as lying on
the ray opposite to the ray for the given value of $\theeta$.
 This
is convenient in some formulas, but it adds another degree of ambiguity
since we can get to the opposite ray just as well by adding $\pi$ to
$\theeta$.  Such practices are best avoided.  In this course, you may
generally assume $r \ge 0$, but each time you encounter polar coordinates
in other contexts, you will have to check what the author intends.
%D9 negative r

$r$ and $\theeta$ are the most commonly used symbols for
polar coordinates in mathematics
and physics books, but there are others you may encounter.  For example,
you may sometimes see $\rho$ in place of $r$ or $\phii$ in place
of $\theeta$.  Later in this course, we shall introduce coordinates
in space which are analogous to polar coordinates in the plane.  For
these, there is even more variation of symbols in common use.   It
is specially important when you see any of these coordinate systems
used that you concentrate on the geometric and physical meaning of
the quantities involved rather than on the particular letters
used to represent them.


You may remember the following formulas relating rectangular and
polar coordinates.  In any case, they are clear by elementary trigonometry
from the diagram.
$$
\align
x &= r\cos\theeta \\
y &= r\sin\theeta \\
\intertext{and}
r &= \sqrt{x^2 + y^2} \\
\tan\theeta &= \frac yx,\qquad\text{if } x \ne 0.
\endalign$$
%D10 Formulas

\mar{s1-25.ps}
The description of motion in polar coordinates is both simpler and
more complicated than in rectangular coordinates.  Instead of
expressing vectors in terms of $\i$ and $\j$, it is useful instead
to use unit vectors associated with the polar directions.  These
depend on the value of $\theeta$ at the point $P$ under consideration
and should be viewed as placed with their tails at that point.
$\u_r$ is chosen to point directly away from the origin, so
it is parallel to the position vector $\r = \overrarrow{OP}$.
$\u_\theeta$ is chosen perpendicular to $\u_r$ and pointing in
the counter-clockwise direction (positive $\theeta$),
so it is tangent to a circle passing through $P$ and centered
at $O$.  
\medskip
\centerline{\epsfbox{s1-26.ps}}
\medskip
\outind{$\noexpand\u_r$}%
\outind{$\noexpand\u_\noexpand\theeta$}%
(These are also commonly denoted $\hat\r$ and $\hat\bold\theeta$).
Because of the definition of $\u_r$, we have
$$
    \r = r\u_r
$$
(This equation is a little confusing since the position vector
$\r$ is commonly thought of with its tail at $O$ while $\u_r$
was supposed to be thought of with its tail at $P$.  However,
if you remember that the tail of a vector can be placed wherever
convenient without changing the vector, you won't have a problem.)

It follows that the velocity vector is given by
$$
    \v = \frac{d\r}{dt} = \frac{d(r\u_r)}{dt} =
           \frac{dr}{dt}\u_r + r\frac{d\u_r}{dt},
$$
where we have calculated the product using the {\it product rule\/}
for derivatives.  Since,  $\u_r$ changes its direction as we
move along the path, it is necessary to keep the second term.
(How does the case of rectangular coordinates differ?)   To
calculate further, we first express $\u_r$ and $\u_\theeta$ in 
rectangular coordinates.  (See the diagram.)
$$\align
    \u_r &= \cos\theeta\,\i + \sin\theeta\,\j, \\
\u_\theeta &= -\sin\theeta\,\i + \cos\theeta\,\j.
\endalign$$
\medskip
\centerline{\epsfbox{s1-27.ps}}
\medskip
Hence,

\nexteqn
\edef\eqnone{\eqn}
$$
    \frac{d\u_r}{dt} = -\sin\theeta \frac{d\theeta}{dt} \i +
                        \cos\theeta \frac{d\theeta}{dt} \j
                     = \frac{d\theeta}{dt}\u_{\theeta}. \tag\eqn
$$
A similar calculation shows that
\nexteqn
$$
     \frac{d\u_\theeta}{dt} = -\frac{d\theeta}{dt}\u_r.\tag{\eqn}
$$ 
Putting (\eqnone) in the expression for $\v$ yields
$$
   \v = \frac{dr}{dt}\u_r + r\frac{d\theeta}{dt}\u_\theeta.
$$
This same process can be repeated to obtain the acceleration
$\a = \dfrac{d\v}{dt}$.
Using both (\eqnone) and (\eqn) yields after considerable calculation
\nexteqn
$$
\a = \left(\frac{d^2r}{dt^2} -r\left(\frac{d\theeta}{dt}\right)^2\right)\u_r + 
\left(r\frac{d^2\theeta}{dt^2} + 2\frac{dr}{dt}\frac{d\theeta}{dt}\right)\u_\theeta. \tag\eqn
$$
I leave it as a challenge for your algebraic prowess to try to verify
this formula.

\example{Example. Uniform Circular Motion}
\outind{uniform circular motion}%
Suppose the particle moves in a circle of radius $R$ centered at
the origin, so that $r = R$, and $dr/dt = 0$.  Suppose in addition
that $d\theeta/dt = \omega$ is constant.  Then $d^2r/dt^2 = d^2\theeta/dt^2
= 0$, and putting all this in the above expressions yield
$$\align
\v &= R\omega \u_\theeta \\
\a &= -R\omega^2 \u_r = -\frac{|\v|^2}R \u_r
\endalign
$$
as we discovered previously.
\bigskip
\endexample

% Exercises for section 2 with answers
\input chap1.ex2

\bigskip
\nextsec{The Dot Product}
\head \sn. The Dot Product \endhead
Let $\a$ and $\b$ be vectors.  We assume they are placed so their
tails coincide.  Let $\theeta$ denote the {\it smaller\/} of the
two angles between them, so $0\le \theeta \le \pi$.
Their {\it dot product\/} is defined to be
$$
     \a\cdot\b = |\a|\,|\b|\,\cos \theeta.
$$
%%%  This really should be theeta
\outind{dot product}%
\outind{scalar product}%
This is also sometimes called the {\it scalar product\/} because
the result is a scalar.
Note that $\a\cdot\b = 0$ when either $\a$ or $\b$ is zero or,
more interestingly, if their directions are perpendicular.
If the two vectors have parallel directions, $\a\cdot\b$ is
the product of their magnitudes if they point the same way
or the negative of that product if they point in opposite
directions.  (Make sure you understand why.)

\mar{s1-28.ps}
The dot product is a useful concept when one needs to find the
component of one vector in the direction of another.  For example,
in a typical inclined plane problem in elementary mechanics, one
needs to resolve the vertical gravitational force into components,
one parallel to the inclined plane, and one perpendicular to it.
To see how the dot product enters into that, note that
%D2    projection diagram
the quantity $|\b|\cos\theeta$ is just the perpendicular projection
of $\b$ onto any line parallel to $\a$.  (See the diagram.)  Hence,
we have
\nexteqn
\def\eqone{\eqn}
$$
     \a\cdot\b = |\a|\, (\text{projection of $\b$ on $\a$}).\tag\eqn
$$
(Note that this description is symmetric; either vector could be
projected onto a line parallel to the other.)  In particular, if
$\a$ is a unit vector ($|\a| = 1$), $\a\cdot\b$ is just the value of
the projection.  (What does it mean if $\a\cdot\b < 0$?)
\medskip
\centerline{\epsfbox{s1-29.ps}}
\medskip
The above projection is a scalar quantity, but one is often
interested in the {\it vector\/} obtained by by multiplying the scalar
 projection
of $\b$ on  $\a$ with a unit vector in the
direction of $\a$.  (See the accompanying diagram.)
$\u = \dfrac 1{|\a|}\a$ is such a unit vector, so this {\it vector
projection\/} of $\b$ on $\a$ is given by
\nexteqn
\def\eqtwo{\eqn}
$$
   (\u\cdot\b)\u = (\frac 1{|\a|}\a\cdot\b)\frac 1{|\a|}\a
         = \frac{\a\cdot\b}{|\a|^2}\a. \tag\eqn  
$$

The dot product satisfies certain simple algebraic rules.
$$\align
  \a\cdot\b &= \b\cdot\a\qquad\text{commutative law},\\
  \a\cdot(\b + \c) &= \a\cdot\b + \a\cdot\c\qquad \text{distributive law},\\
  (s\a)\cdot\b &= \a\cdot(s\b) = s(\a\cdot\b),\\
\intertext{and}
   \a\cdot\a &= |\a|^2.
\endalign$$
      These can be proved without too much difficulty from the
geometric definition.  See, for example, the accompanying diagram
which illustrates the proof of the distributive law.
\medskip
\centerline{\epsfbox{s1-30.ps}}
\medskip
\nextthm
\proclaim{Theorem \cn.\tn}  Let the components of
 $\a$ be $\lb a_1, a_2, a_3 \rb$
and let those of $\b$ be  $\lb b_1, b_2, b_3 \rb$.  Then
$$
    \a\cdot\b = a_1b_1 + a_2b_2 + a_3b_3.
$$
\endproclaim

For plane vectors, the same principle applies without the third components.

\example{Examples}

For $\lb 1, 2, -1 \rb$ and $\lb 2, -1, 3 \rb$, the dot product
is $ 2 -2 -3 = -3$.   In particular, that means the angle between
the two vectors is obtuse.

For $\lb 1, 1, 2 \rb$ and $\lb -1, -1, 1 \rb$, the dot product
is $ -1 - 1 + 2 = 0$.   That means the 
two vectors are perpendicular.
\endexample

\demo{Proof of the theorem}
We have
$$\align
\a &= a_1\i + a_2\j + a_3\k \\
\b &= b_1\i + b_2\j + b_3\k,
\endalign$$
so the aforementioned rules of algebra yield
$$\align
   \a\cdot\b 
&= a_1b_1\i\cdot\,\i + a_1b_2\,\i\cdot\j + a_1b_3\,\i\cdot\k \\
&+ a_2b_1\,\j\cdot\i + a_2b_2\,\j\cdot\j + a_2b_3\,\j\cdot\k \\
&+ a_3b_1\,\k\cdot\i + a_3b_2\,\k\cdot\j + a_3b_3\,\k\cdot\k .
 \endalign$$
However, $\i$, $\j$, and $\k$ are mutually perpendicular
unit vectors, so the off-diagonal dot products (e.g., $\i\cdot\j,
\ \i\cdot\k$, etc.) are all zero while the diagonal dot products
(e.g., $\i\cdot\i$) are all one.  Hence, only the three diagonal
terms survive and we obtain
$a_1b_1 + a_2b_2 + a_3b_3$ as claimed.
\qed\enddemo

The theorem gives us a way to calculate the angle between two
vectors {\it in case we know their components}.  Indeed, the formula
may be rewritten
$$
   \cos \theeta = \frac{\a\cdot\b}{|\a||\b|}.
$$
The right hand side may be computed using
$\a\cdot\b = a_1b_1 + a_2b_2 + a_3b_3$, $|\a| =
\sqrt{a_1{}^2 + a_2{}^2 + a_3{}^2}$,  and similarly for $|\b|$, and
from this we can determine $\theeta$.

\example{Example}
Suppose $\a = \i + 2\j -\k, \b = 2\i -\j + 3\k$.  Then
$$\align
    \a\cdot\b &= (1)(2) + (2)(-1) + (-1)(3) = -3 \\
     |\a| &= \sqrt{ 1 + 4 + 1} = \sqrt 6 \\
     |\b| &= \sqrt{ 4 + 1 + 9} = \sqrt{14} \\
\intertext{so}
     \cos \theeta &= \frac{-3}{\sqrt{84}} \\
    \theeta &= 1.90427\text{\ radians}
\endalign $$

The use of components simplifies other calculations also.  For example,
suppose we want to know the {\it projection\/} of the vector
$\b$ on a line parallel to $\a$.  We know from (\eqone)
that  this is
$$
    |\b|\cos \theeta = \frac{\a\cdot\b}{|\a|} = \frac{-3}{\sqrt{6}}.
$$
Similarly, from (\eqtwo), we see that the vector projection of $\b$
on $\a$ is given by
$$
         \frac{\a\cdot\b}{|\a|^2}\a = \frac{-3}6(\i +2\j -\k)
            = -\frac 12 \i - \j + \frac 12 \k.
$$ 
    
Note that in all these calculations we must rely on the use of
components to make the formulas useful.  
\endexample

\subhead The Law of Cosines \endsubhead
  Let $A, B$, and $C$ be the vertices of a triangle,
and define vectors
$$\align
  \a &= \overrarrow{CB} \\
  \b &= \overrarrow{CA} \\
   \c &= \overarrow{BA} = \b - \a.
\endalign $$

\mar{s1-31.ps}
\noindent
Then we have
$$\align
|\c|^2 = \c\cdot\c &= (\b - \a)\cdot(\b - \a) \\
        &= \b\cdot\b -\b\cdot\a - \a\cdot\b + \a\cdot\a \\
        &= |\b|^2 - 2\a\cdot\b + |\a|^2 \\
\intertext{which may be rewritten}
|\c|^2 &= |\a|^2 + |\b|^2 - 2|\a||\b|\cos \theeta . 
\endalign $$
You should recognize that as the {\it Law of Cosines\/}
from Trigonometry.

\subhead Generalizations \endsubhead
We have already mentioned higher dimensional vectors being
characterized as $n$-tuples for values of $n \ge 4$.   The
dot product of $n$-tuples can be defined by analogy with
the formula derived above.  Thus for $4$-vectors,
$\a = \lb a_1, a_2, a_3, a_4 \rb, \b = \lb b_1, b_2, b_3, b_4 \rb$,
we would define
$$
 \a\cdot\b = a_1b_1 + a_2b_2 + a_3b_3 + a_4b_4.
$$
\outind{dot product in $n$ dimensions}%
In the theory of special relativity, it turns out to be useful to
consider instead a ``dot product'' of $4$-vectors of the form
$$
  a_1b_1 + a_2b_2 + a_3b_3 - c^2 a_4b_4
$$
where $c$ is the speed of light.  (Some authors prefer the negative of
this quantity.)  This is a bit bizarre, because the ``dot product''
of a vector with itself can be zero without the vector being zero.

We shall consider some of these strange and wonderful ideas later in
this course.      

\subhead Differentiation of Dot Products \endsubhead
The usual differentiation rules, e.g., the product rule, the chain
rule, etc., apply to vector functions as well as to scalar functions.
Indeed, we have already used some of these rules without making a
point of it.   The proofs are virtually the same as those in the
scalar case, so we need not go through them again in this course.
However, since there are so many different vector operations, it
is worth exploring the consequences of these rules in interesting
cases.  They are sometimes not what you would expect.
For example, the {\it product rule\/} for the dot product of two
vector functions $\bold f(t)$ and $\bold g(t)$ takes the form
\nexteqn
$$
    \frac d{dt}\bold f(t)\cdot \bold g(t) =
    \frac{d\bold f(t)}{dt} \cdot \bold g(t) +
          \bold f(t)\cdot\frac{d\bold g(t)}{dt}.
$$
\outind{dot product, differentiation of}%
Let us apply this to the case $\bold f(t) = \bold g(t) = \r(t)$,
the position vector of a particle moving in the plane or in space.  
{\it Assume the particle moves so that its distance to the origin
is a constant  $R$}.    Symbolically, this can be written
$ \r\cdot\r = |\r|^2 = R^2$.   Then the product rule gives
$$
    0 = \frac{d(\r\cdot\r)}{dt} = \frac{d\r}{dt}\cdot\r +
  \r\cdot\frac{d\r}{dt} = 2\r\cdot \v.
$$
It follows from this that either $\v = 0$ or $\v\perp\r$.   In the
plane case, this is yet another confirmation that for motion in
a circle, the velocity vector is perpendicular to the radius vector.

Similar reasoning shows that if $|\v|$ is constant, then the
acceleration vector $\a$ is perpendicular to $\v$.  (You should write
it out to convince yourself you understand the argument.)
%Ex  Show that if the velocity vector $\v$ is perpendicular to the
%Ex acceleration vector $\a$ at every point of a path, then the
%Ex speed $|\v|$ is constant.
\bigskip

% Exercises for section 3 and answers
\input chap1.ex3

\nextsec{The Vector Product}
\head \sn. The Vector Product \endhead

Let $\a$ and $\b$ be two vectors in space placed so their tails
coincide, and let $\theeta$ be the smaller of the two angles between
them (i.e., $0\le \theeta \le \pi$).  
 Previously, we defined the dot or scalar product $\a\cdot\b$.
There is a second product  called
 the {\it vector product\/} or {\it cross product\/}. It is denoted
\outind{vector product}%
\outind{cross product}%
$\a\times\b$, and as its name suggests, it is a 
{\it vector}.  It is used extensively in mechanics for such notions
as torque and angular momentum, and we shall use it shortly in
studying solid analytic geometry.

$\a\times \b$ is defined as follows.
    Its magnitude is 
\nexteqn
$$
|\a\times\b| = |\a||\b|\sin\theeta.\tag\eqn
$$
The quantity on the right has a very simple interpretation.
$|\b|\sin\theeta$  is the height of the parallelogram spanned by
$\a$ and $\b$, so $|\a||\b|\sin\theeta$ is
 the {\it area of the parallelogram}. 
 Note it is zero if the vectors point 
 in the same ($\theeta = 0$) or  opposite  ($\theta =
\pi$)  directions, but otherwise it is non-zero.

\mar{s1-32b.ps}
The direction of $\a\times \b$ (when it isn't zero)
 is a bit harder to describe.  First,
$\a\times \b$ is perpendicular to both $\a$ and $\b$, and given that we know
its magnitude that leaves precisely two possibilities.   We specify that
it has the {\it right hand\/} orientation.  That is, 
\outind{right hand rule for vector product}%
if the fingers of your right hand
point from $\a$ to $\b$ through the angle $\theeta$,   $\a\times\b$
should point in the direction of your thumb.  

The vector product has some surprising algebraic properties.  First,
as already noted,
$$
   \a\times \b = \bold 0
$$
when $\a$ and $\b$ point in the same or opposite directions.
In particular, $\a\times\a = \bold 0$ 
for any vector $\a$.  
Secondly,  
the commutative law fails, and we have instead
$$
    \a\times \b = - \b\times \a.
$$
(Point from $\b$ to $\a$, and your thumb reverses direction.)
The vector product does satisfy other rules of
algebra such as  
$$\align
    \a\times (\b + \c) &= \a\times\b + \a\times\c \\
     (s\a)\times \b &= \a\times (s\b) = s(\a\times \b),
\endalign$$
 but they are a bit tricky to verify from the geometric
definition.  (See below for another approach.)


The vector products of the basis vectors are easy to
calculate from the definition.  We have
$$\align
  \i\times\j &= -\j\times\i  = \k, \\
\j\times\k &= -\k\times\j = \i\\
 \k\times\i &=  -\k\times\i = \j.
\endalign
$$

\mar{s1-33.ps}
To calculate vector products in general, we expand in terms of components.
$$\align
\a\times\b &= (a_1 \i + a_2\j + a_3\k) \times (b_1\i + b_2\j + b_3\k) \\
       &= a_1b_1\i\times\i + a_1b_2\i\times\j + a_1b_3\i\times\k \\
       &+ a_2b_1\j\times\i + a_2b_2\j\times\j + a_2b_3\j\times\k \\
       &+ a_3b_1\k\times\i + a_3b_2\k\times\j + a_3b_3\k\times\k \\
       &= \bold 0  + a_1b_2\k - a_1b_3\j \\
       &- a_2b_1\k + \bold 0 + a_2b_3 \i \\
       &+ a_3b_1\j - a_3b_2\i + \bold 0 \\
\intertext{so}
\a\times \b &= (a_2b_3 - a_3b_2)\i - (a_1b_3 - a_3b_1)\j + (a_1b_2 - a_2b_1)\k.
\endalign$$
(Note that this makes extensive use of the rules of algebra for vector
products, so one should really prove those rules first.)

There is a simple way to remember the formula.  Recall that
a $2\times 2$ determinant is defined by
$$
 \det\bm a & b \\ c & d \em = ad - bc.
$$
\outind{determinant, $2\noexpand\times 2$}%
Now consider the {\it matrix\/} or array
$$
    \bm a_1 & a_2 & a_3 \\ b_1 & b_2 & b_3 \em
$$
formed from the components of $\a$ and $\b$, and calculate the
$2\times 2$ determinants obtained by omitting successively each of
the columns.  For the first and third, use a ($+$) sign and for the
second use a ($-$) sign.   


\nextex
\example{Example \en}
Let $\a = \lb 1, 2, -1\rb$ and $\b = \lb 1, -2, 3 \rb$.  Then
$$
 \a \times \b = \lb 6-2, -(3 + 1), -2 -2\rb = \lb 4, -4, -4 \rb.
$$
We can check that this vector is indeed perpendicular to both
$\a$ and $\b$ by calculating dot products.
$$\align
   \a\cdot (\a \times \b) &= 4 - 8 + 4 = 0 \\
   \b\cdot (\a \times \b) &= 4 + 8 - 12 = 0 .
\endalign
$$
\endexample

(Many texts suggest defining
$\a\times\b$ by a  $3\times 3$
determinant
\nexteqn
\xdef\ForOne{\eqn}
$$
  \a\times\b = \det 
    \bm \i & \j & \k \\ a_1 & a_2 & a_3 \\ b_1 & b_2 & b_3 \em.\tag\eqn
$$
\outind{determinant, $3\noexpand\times 3$ and vector product}%
If you are not familiar with $3\times 3$ determinants, see the
Exercises.)

\subhead The triple product \endsubhead
Let $\a$, $\b$, and $\c$ be three vectors in space and suppose they
are placed so their tails coincide.  Then the product
$$
     (\a\times\b)\cdot \c
$$
has a simple geometric interpretation.  The vectors are three
of the twelve sides of a {\it parallelepiped}.  $|\a\times\b|$
is the area of the base of that parallelepiped.  Moreover the
projection  of $\c$ on $\a\times\b$ is either the {\it altitude\/}
of the parallelepiped or the negative of that altitude depending
on the relative orientation of the vectors.   Hence, the dot product
is, except for sign, the {\it volume\/} of the parallelepiped.
It is positive if $\c$ is on the same side of the plane determined
by $\a$ and $\b$ as $\a\times\b$, and it is negative if they are
on opposite sides.   
\medskip
\centerline{\epsfbox{s1-34.ps}}
\medskip
\nextex
\example{Example \en}
Let $\a = \lb 0, 1, 1 \rb, \b = \lb 1, 1, 0 \rb,$ and $\c = \lb 1, 0, 1 \rb$.
Then
$\a\times \b = \lb -1, 1, -1\rb$, so
$(\a\times\b)\cdot\c = -1 + 0 + (-1) = -2$.  Hence, the volume is
2.  You should study the diagram to make sure you understand why the
answer is negative. 
\endexample

An immediate consequence of the geometric interpretation is the
formula
\nexteqn
\xdef\TDP{\eqn}
$$
   (\a\times\b)\cdot\c = \a\cdot(\b\times\c). \tag\eqn
$$
For, except for sign, both sides may be interpreted as the volume
of the same parallelepiped, and careful inspection of all
possible relative orientations shows
that the signs will be the same.

Using $3\times 3$ determinants---see the Exercises---you can
check the formula
\nexteqn
\xdef\ForTwo{\eqn}
$$
    \a\cdot(\b\times \c) = \det \bm a_1 & a_2 & a_3 \\
 b_1 & b_2 & b_3 \\
 c_1 & c_2 & c_3 \em. \tag\eqn
$$
\outind{determinant, $3\noexpand\times 3$ and volume}%

It is worth noting that we can use (\TDP) to verify the formula
$$
    \a\times(\b + \c) = \a\times\b + \a\times\c.
$$
For, it would suffice to show that both sides have the same components.
Consider the $x$-component.  In general, the $x$-component of a
vector $\v$ is its projection on the $x$-axis, that is, it is $\i\cdot\v$.
However,
$$\align
  \i\cdot(\a\times(\b+\c)) &= (\i\times\a)\cdot(\b + \c) \\
&= (\i\times\a)\cdot\b + (\i\times\a)\cdot\c \\
&= \i\cdot(\a\times\b) + \i\cdot(\a\times\c) \\
&= \i\cdot(\a\times\b + \a\times\c).
\endalign$$
(Here we used the fact that the distributive law does hold for the
dot product.)  It follows that the two sides have the same $x$-component.
Similar arguments work for the $y$ and $z$-components, so the two
sides are the same.

\subhead Product Rule \endsubhead
Like the dot product, the cross product also satisfies the
product rule.  (Also, the proof is virtually identical to the
proof for scalar functions.)   Thus, if $\f(t)$ and $\g(t)$ are
vector valued functions, then
\nexteqn
$$
\frac{d}{dt}(\f\times\g) = \frac{d\f}{dt}\times \g + \f\times \frac{d\g}{dt}.
\tag\eqn
$$ 
This formula has many useful consequences.  We shall illustrate one
such by showing that if a particle moves in space so that the
acceleration $\a$ is parallel to the position vector $\r$, then
the motion must be constrained to a plane.  To see this, consider
the quantity $\bold L = \r\times\v$.  We have
$$\align
    \frac{d\bold L}{dt} &= \frac{d\r}{dt}\times \v + \r\times \frac{d\v}{dt} \\
		 &= \v\times \v + \r\times\a = \r\times\a.
\endalign
$$
However, since $\r$ and $\a$ are parallel, $\r\times\a = \bold 0$, and
it follows that $d\bold L/dt = 0$, i.e., $\bold L$ is constant.  In particular,
the direction of $\bold L = \r \times\v$ does not change, so since
$\r\perp \bold L$, it follows that the position vector is constrained
to the plane perpendicular to $\bold L$ and containing the origin.

\mar{s1-35.ps}
	The quantity $\r\times \v$ seems to have been pulled from
a hat, but you will learn in physics how the related quantity
$\r\times(m\v)$, which is called {\it angular momentum\/}, is used
\outind{angular momentum}%
to make sense of certain aspects of motion. 

\bigskip

% Exercises with answers for section 4
\input chap1.ex4

\bigskip
\nextsec{Geometry of Lines and Planes}
\head \sn. Geometry of Lines and Planes \endhead

We depart for the moment from the study of concepts of immediate use
in dynamics to discuss some analytic geometry in space.   The notions
we shall introduce have a variety of applications, but, more important,
they will help you develop your spatial intuition and teach you something
about expressing such intuition analytically.

\subhead Lines \endsubhead
We saw before that the motion of a particle moving with constant speed
%%%  Reference?
on a line is described by a vector equation of the form
$$
\r = t\v + \r_0
$$
\outind{line, vector equation of}%
\outind{line, parametric representation of}%
where $\v$ is the constant velocity vector pointing along the line
in the direction of motion, and $\r_0$ is its position at $t = 0$.
(Previously $\v$ was denoted $\v_0$, but since the velocity is constant,
we can use either, and we drop the subscript to save writing.)
%D1   Line pictures
It is often more convenient to use a slightly different form
of this equation
$$
  \r = (t - t_0)\v + \r_0.
$$
Here, `$t$' is replaced by `$t - t_0$' which denotes the time
`elapsed' since some initial time $t_0$, and $\r_0$ denotes the
position at $t = t_0$.   If you multiply this out, you will see
it is really the same as the previous equation with $\r_0$
being replaced by the constant vector $-t_0\v + \r_0$.  (Why is
`elapsed' in quotes?  Think about the case $t < t_0$.) 

\mar{s1-36.ps}
There are a couple of points that should be stressed at this point.
First, the above equations apply in the plane as well as in space.
You should think about how the ordinary analytic geometry of lines
in the plane can be related to this approach.
Secondly, while we have been thinking of $t$ as representing ``time'',
it is not absolutely necessary to do so.  It is better in some circumstances
to think of it as just another variable which ranges over the {\it domain\/}
of a vector function described by the equation
$$
   \r = \r(t) = t\v + \r_0,
$$
and the line is the {\it image\/} of this function, i.e., the set of
all $\r(t)$ so obtained.   Such a variable is called a {\it parameter\/},
\outind{parameter}%
and the associated description of the line is called a {\it parametric
representation}.   This is a more static point of view since we need not
think of the line as traced out by a moving particle.  Also, there is no
need to use the symbol `$t$' for the parameter.  Other letters, such as
`$s$' or `$x$' are perfectly acceptable, and in some circumstances may
be preferable because of geometric connotations in the problem. 

\nextex
\xdef\exone{\en}
\example{Example \en}
Consider the line through the points $P_0$ and
$P_1$ with respective coordinates  $(1,1,0)$ and
$(0,2,2)$.
%D2  Line diagram
Choose $\v = \overrarrow{P_0P_1}$.   This is certainly a vector
pointing along the line, and it does not matter what its magnitude
is.  (Thinking in kinematic imagery, we don't care how fast the
line is traced out.)  Its components are
$\lb 0 - 1 , 2 - 1, 2 - 0 \rb = \lb -1, 1, 2\rb$.
Since the line passes through $P_0$, we may choose $\r_0 = \overrarrow{OP_0}$
which has components $\lb 1, 1, 0 \rb$.   With some abuse of notation,
we may write the parametric equation of the line
$$\align
   \r &= t\lb -1, 1, 2 \rb + \lb 1, 1, 0 \rb =
          \lb -t + 1, t + 1, 2t \rb  \\
\intertext{or}
    \r &= (1-t)\i + (1+t)\j + (2t)\k.
\endalign$$
This can also be written as 3 component equations
$$\align
   x &= 1 - t, \\
   y &= 1 + t,   \\
   z &= 2t.
\endalign $$
Note that there are other ways we could have approached this problem.
We could have reversed the roles of $P_0$ and $P_1$, we could have
used $\frac 12 \v$ instead of $\v$, etc.   Each of these would have
produced a valid parametric equation for the line, but they would all
have looked different.
\endexample

\mar{s1-37.ps}
\smallskip
\nextex
\example{Example \en}
Consider the line in the plane with parametric equation
$$\align
   \r &= t\lb -1, 1 \rb + \lb 1, 3 \rb \\
\intertext{or}
    x &= -t + 1, \\
    y &= t + 3.
\endalign$$
%D3  Line diagram  (1,3)
We may eliminate $t$ algebraically by writing
$$
    t = 1 - x  = y - 3
$$
whence we obtain $y = -x + 4$.  This is the usual way of
representing a line in the plane by an equation of the form
$y = mx + b$, where $m$ is the slope, and $b$ is the $y$-intercept.
\endexample

\mar{s1-38.ps}
Eliminating the parameter makes sense for lines in space, but unfortunately
it does not lead to a single equation.
The vector equation $\r = t\v + \r_0$ can be written as 3 scalar equations
$$\align
    x &= ta + x_0, \\
    y &= tb + y_0, \\
    z &= tc + z_0
\endalign$$
where $\v = \lb a, b, c \rb$.  If none of the components of $\v$
are zero, we can solve these equations for $t$ to obtain
$t = \dfrac{x-x_0}a$,
   $t = \dfrac{y-y_0}b$, and
   $t = \dfrac{z - z_0}c$.   Eliminating $t$ yields so called
{\it symmetric equations\/} of the line
\outind{line, symmetric equations of}%
$$
   \frac{x-x_0}a
   = \frac{y-y_0}b
   = \frac{z - z_0}c.
$$
These have the advantage that the components of $\v$ and the coordinates
of $P_0$ are clearly displayed.  However, symmetric equations are not
directly applicable if, as in Example \exone, one or more of the components
of $\v$ are zero.

\example{Example \exone\ {\rm  revisited}}
We found $\v = \lb -1, 1, 2\rb$ and $\r_0 = \lb 1, 1, 0 \rb$ so
the symmetric equations are
$$
  \frac{x -1}{-1} = \frac{y - 1}{1} = \frac z2.
$$
\endexample

%Ex  What does it say about the line described parametrically
%Ex by $\r = t\lb a,b,c \rb + \lb x_0, y_0, z_0 \rb$ if $a = 0$?  If $b = 0$?
%Ex If $a = b = 0$?   Can we have $a = b = c = 0$?
The geometry of lines in space is a bit more complicated than that of lines
in the plane.  Lines in the plane either intersect or are parallel.
In space,  we have to be a bit more careful about what we mean by
`parallel lines', since lines with entirely different directions can
still fail to intersect.
\medskip
\centerline{\epsfbox{s1-39.ps}}
\medskip
\nextex
\example{Example \en}  
Consider the lines described by
$$\align
\r &= t\lb 1, 3, -2 \rb + \lb 1, 2, 1 \rb \\
\r &= t\lb -2, -6, 4\rb + \lb 3, 1, 0 \rb.
\endalign
$$
They have parallel directions since $\lb -2, -6, 4 \rb = -2\lb 1, 3,-2 \rb$.
Hence, in this case we say the lines are {\it parallel}.  (How can
we be sure the lines are not the same?)
\endexample

\nextex
\example{Example \en}
Consider the lines
$$\align
\r &= t\lb 1, 3, -2 \rb + \lb 1, 2, 1 \rb \\
\r &= t\lb 0, 2, 3\rb + \lb 0, 3, 9 \rb.
\endalign $$
They are not parallel because neither of the vectors
 $\v$ is a multiple
of the other.  They may or may not intersect.  (If they don't,
we say the lines are {\it skew}.)  How can we find out?  One
method, is to set them equal and see if we can solve for the point
of intersection.  There is one tricky point here.  If we think of
the parameter $t$ as time, even if the lines do intersect, there
is no guarantee that particles moving on these lines would arrive
at the point of intersection {\it at the same instant}.  Hence,
the way to proceed is to introduce a second parameter, call it $s$,  
for one of the lines, and then try to solve for the point of
intersection.  Thus, we want
$$
\r = t\lb 1, 3, -2 \rb + \lb 1, 2, 1 \rb =
s\lb 0, 2, 3\rb + \lb 0, 3, 9 \rb,
$$
which after collecting terms yields
$$
    \lb t + 1, 3t + 2, -2t + 1 \rb = \lb 0, 2s + 3, 3s + 9\rb.
$$
Picking out the components yields three equations
$$\align
    t + 1 &= 0 \\
    3t +2 &= 2s + 3 \\
    -2t + 1 &=  3s + 9
\endalign
$$
in 2 unknowns  $s$ and $t$.  This is an {\it overdetermined\/}
system, and it may or may not have a consistent solution.  In this
case, the first two equations yield $t = -1$  and $s = -2$.  Putting
these values in the last equation yields $(-2)(-1) + 1 = 3(-2) + 9$
which checks.  Hence, the equations are consistent, and the lines
do intersect.   To find the point of intersection, put $t = -1$
in the equation for the first line (or
$s = -2$ in that for the second) to obtain  $\lb 0, -1, 3 \rb$.  
%D  Intersecting lines.
\endexample

\nextex
\example{Example \en}
Consider the lines
$$\align
\r &= t\lb 1, 3, -2 \rb + \lb 1, 2, 1 \rb \\
\r &= s\lb 0, 2, 3\rb + \lb 0, 3, 8 \rb.
\endalign $$
We argue exactly as above, except in this case, we obtain component
 equations
$$\align
    t + 1 &= 0 \\
    3t +2 &= 2s + 3 \\
    -2t + 1 &=  3s + 8
\endalign
$$
Again, the first two equations yield $t= -1, s = -2$, but these
values are not consistent with the third equation.  Hence, the lines
are skew, they are not parallel and they don't intersect.
\endexample.

\subhead Planes \endsubhead
A plane in space may be characterized geometrically in several different
ways.  Here are some of the most common characterizations.
\roster
\item   Any three non-collinear points $P_1, P_2$, and
$P_3$ (points not on a common line)
determine a plane. 
\item  There is a unique plane passing through a given point $P_0$ and
perpendicular to a given line $l$.  (A plane is perpendicular to a line
$l$ if each line in the plane through the point of intersection with
$l$ is perpendicular to $l$.)
\item  Any two distinct lines $l_1$ and $l_2$
 which intersect determine a plane.
\item  A line $l$ and a point $P_0$ not on $l$ determine a plane.
\endroster

We want to describe planes analytically.  For this, it is best to
start with the second characterization.   Let $P_0$  with
coordinates $(x_0,y_0,z_0)$ be the given
point.  Clearly, we will get the same plane if we replace the
perpendicular line
$l$ with any line parallel to $l$, so we may as well assume that
$l$ passes through $P_0$.  Choose a vector $\N$ pointing in the
direction of $l$.  If $P$ is any point in the plane, then the
displacement vector  $\overrarrow{P_0P}$ is perpendicular to
$\N$.  However,  $\overrarrow{P_0P} = \r - \r_0$, so the
perpendicularity may be described algebraically by
\nexteqn
$$
    \N\cdot(\r - \r_0) = 0.  \tag\eqn
$$
\outind{plane, equation of}%
\outind{plane, normal to}%
\medskip
\centerline{\epsfbox{s1-40.ps}}
\medskip

Suppose $\N$ has components $\lb a, b, c \rb$.  Since $\r - \r_0$
has components $\lb x - x_0, y - y_0, z - z_0 \rb$, this equation
may be rewritten
\nexteqn
$$
     a(x - x_0) + b(y - y_0) + c(z - z_0) = 0. \tag\eqn
$$
This is called a {\it normal form\/} of an equation of a plane.

\nextex
\example{Example \en}
We shall find an equation for the plane through the points
with coordinates $(1,0,0)$, $(0,1,0)$, and $(0,0,1)$. 
 By symmetry, the angles
which a normal vector makes with the coordinate axes (called its
{\it direction angles\/}), should
be equal, so the vector $\N$ with components $\lb 1, 1, 1 \rb$
should be perpendicular to this plane.  For $P_0$, we could use
any of the three points; we choose it to be the point with coordinates
$(1,0,0)$.   With these choices, we get the normal form of the equation
of the plane
$$
   1(x - 1) + 1(y - 0) + 1(z - 0) = 0
$$
which can be rewritten more simply
$$\align
   x + y + z - 1 &= 0\\
\text{or}\quad x + y + z &= 1.
\endalign
$$
Note that we chose the normal direction by being clever, but there
is a quite straightforward way to do it.  If the three points are
denoted $P_0, P_1$ and $P_2$, then the directed line
segments $P_0P_1$
and $P_0P_2$ lie in the plane, so the vector product
$\overrarrow{P_0P_1}\times \overrarrow{P_0P_2}$ is perpendicular
to the plane.  In this case, $\overrarrow{P_0P_1}$ has components
$\lb 0 - 1, 1 - 0, 0 - 0 \rb = \lb -1, 1, 0 \rb$, and
 $\overrarrow{P_0P_2}$ has components
$\lb 0 - 1, 0 - 0, 1 - 0 \rb = \lb -1, 0, -1 \rb$.  Hence, the
cross product has components $\lb -1, -1, -1 \rb$.  This yields
the normal form
$$
\align
     -1(x - 1) + (-1)(y - 0) + (-1)(z - 0) &= 0 \\
\text{or}\quad -x - y - z = -1.
\endalign $$
\endexample
\mar{s1-41.ps}
As the above example illustrates, there is nothing unique about
the equation of a plane.   For example, if $\N$ with components
$\lb a, b, c \rb$ is a normal vector, so is $s\N$ with components
$\lb sa, sb, sc \rb$ for any non-zero scalar $s$.  Replacing
$\N$ by $s\N$ just multiplies the normal form by the factor $s$,
and clearly this does not change the locus of the equation.  (The
locus of an equation is the set of all points whose coordinates
satisfy the equation.)

In general, the normal form may be rewritten
$$\align
    ax + by + cz - ax_0 -by_0 - cz_0 &= 0 \\
\text{or}\qquad ax + by + cz &= d
\endalign $$
where $d = ax_0 + by_0 + cz_0$.  Conversely, the locus of any such
linear equation, $ax + by +cz = d$, in which not all of the coefficients
$a,b$, and $c$ are zero, is a plane.  This is not hard to prove in
general, but we illustrate it instead in an example.
\nextex
\example{Example \en}
Consider the locus of the equation
$$
2x - 3y + z = 6.
$$
The choice of a normal vector is clear,  $\N = \lb 2, -3, 1 \rb$.
Hence, to express the above equation in normal form, it suffices
to find one point $P_0$ in the plane.  This could be done in many
different ways, but one method that works in this case would be
to set $x = y = 0$ and to solve for $z$.  In this case, we get
$z = 6$, so the point with coordinates $(0,0, 6)$ lies in the
plane.
Write
$$\align
  2x - 3y + z &= 6\qquad\text{for a general point}\\
  2(0) - 3(0) + 6 &= 6\qquad\text{for the specific point}
\endalign$$
and then subtract to obtain
$$
  2(x - 0) -3(y - 0) + 1(z -  6) = 0.
$$
That is an equation for the same plane in normal form. It may also
be written
$$
   \lb 2, -3, 1 \rb \cdot \lb x - 0,y -  0,z -  6 \rb = 0
$$
to express the perpendicularity of $\overrarrow{P_0P}$ to $\N$.

You should convince yourself that this method works in complete
generality.  You should also ask what you might do for an equation
of the form $2x + 3y = 12$ where you can't set $x = y = 0$ and solve
for $z$. 
\endexample

Various special cases where one or more of the coefficients in
the linear equation $ax + by + cz = d$ vanish represent special
orientations of the plane with respect to the coordinate axes.
For example, $x = d$ has as locus a plane parallel to the
$y,z$-plane (i.e., perpendicular to the $x$-axis), and passing
through the point $(d, 0, 0)$.   $ax + by = d$ would have as
locus a plane {\it perpendicular\/} to the $x,y$-plane and intersecting
it in the line in that plane with equation $ax + by = c$.
%D  Diagram of ax + by = d
{\it It is important to note that the equation $ax + by = d$ does not
by itself specify a locus.  You must say whether you are considering
a locus in the plane or in space.}    You should consider other
variations on these themes and draw representative diagrams to make
sure you understand their geometric significance.

\mar{s1-42.ps}
We saw that the coefficients $a, b$, and $c$ in the linear equation
may be chosen to be the components of a normal vector $\N$. 
 The significance
of the constant $d$ is a bit trickier.  We have
$$
   d = ax_0 + by_0 + cz_0 = \N\cdot\r_0
$$
  It is easiest to understand this 
if $\N$ is a unit vector, i.e., $a^2 + b^2 + c^2 = 1$, 
so we suppose that to be the case.  Then $\N\cdot\r_0$
is the {\it projection\/} of the position vector $\r_0$ of the
reference point $P_0$ on the normal direction.   Since the 
vector $\N$ can be placed wherever convenient, we move it so its
tail is at the origin.  Then it is clear that this projection
is just the {\it signed\/} distance of the plane to the origin.
The sign depends on whether the origin is on the side of the
plane pointed at by the normal or the other side.  (How?) 

This reasoning can be extended further as follows.  Let $P_1$
be any point, not necessarily the origin, and let
$(x_1,y_1, z_1)$ be its coordinates.  Then the
projection of $\overrarrow{P_0P_1}
= \r_1 - \r_0$ on the vector $\N$ gives the signed perpendicular distance
from the point $P_1$ to the plane.  Thus, if $\N$ is a unit vector,
the {\it distance\/} is given by
$$
   D = |\N\cdot (\r_1 - \r_0)|.
$$
If $\N$ is not a unit vector, we may replace it by $\n = \dfrac 1{|\N|}\N$.
This yields the formula for that distance
\nexteqn
$$\align
   D &= \frac 1{|\N|} |\N\cdot\r_1 - \N\cdot\r_0| \\ 
     &= \frac 1{\sqrt{a^2 + b^2 + c^2}}|ax_1 + by_1 + cz_1 - d|. \tag\eqn
\endalign
$$

\mar{s1-43.ps}
\nextex
\example{Example \en}
The distance of the point  $(-1, 1, 0)$ to the plane with equation
$x + y + z = 1$ is 
$$
   \frac 1{\sqrt 3}| 1(-1) + 1(1) + 1(0) - 1| = \frac 1{\sqrt 3}.
$$
Note that the sign inside the absolute values is negative which
reflects the fact that $(-1, 1, 0)$ is on the side of the plane
opposite to that pointed at by  $\lb 1, 1, 1 \rb$.
\endexample

\subhead Intersection of Planes \endsubhead
In general, two distinct planes in space are either parallel or they
intersect in a line.  In the first case, the normal vectors are
parallel.  
\outind{line as intersection of planes}%

\nextex
\example{Example \en}
The loci of
$$\align
     2x - 3y + 2z &= 12 \\
\text{and}\qquad -6x + 9y - 6z &= 12
\endalign $$
are parallel.  Indeed the indicated normal vectors are
$\lb 2, -3, 2\rb$ and $\lb -6, 9, -6\rb$ which are multiples of
one another, so they are parallel.
(How do you know the planes are not identical?)
\endexample

\mar{s1-44.ps}
In the second case, there is a simple way to find an equation for
the line of intersection of the two planes.

\nextex
\example{Example \en}   
We find the line of intersection of the planes which are loci of
the linear equations
\nexteqn
\xdef\eqtwo{\eqn}
$$
\align
   6x + 2y -z &= 2 \\
   x - 2y + 3z &= 5. \tag\eqn
\endalign 
$$
\medskip
\centerline{\epsfbox{s1-45.ps}}
\medskip
The line of intersection of the two planes is perpendicular to
normals to both planes.  Thus it is perpendicular to
$\lb 6, 2, -1 \rb$ and also to $\lb 1 , -2, 3 \rb$.   Thus, the
cross product 
$$
\v = \lb 6, 2, -1 \rb\times \lb 1 , -2, 3 \rb = \lb 4, -19, -14 \rb
$$
 is parallel to the desired line.  To find a parametric representation
of the line, it suffices to find one point $P_0$ in the line.
We do this as follows.  In the system of equations (\eqn), choose
some particular value of $z$, say $z = 0$.  this yields the system
$$
\align
   6x + 2y  &= 2 \\
   x - 2y &= 5 
\endalign 
$$
which may be solved by the usual methods of high school algebra to
obtain  $x = 1, y = -2$.  That tells us that the point $P_0$ with
coordinates $(1, -2, 0)$ is a point on the line.  Hence,
$$
  \r = t\v + \r_0 = t\lb 4, -19, -14 \rb + \lb 1, -2, 0\rb
        = \lb 4t + 1, -19t -2, -14t\rb
$$
is a vector parametric representation of the line.  (What would you
have done if $x$  did not appear in either equation, i.e.,
appeared with $0$ coefficients?  In that case, you would not be able
to set $z = 0$ and solve for $x$.)
\endexample

Note that the symmetric equations of a line
$$
   \frac{x- x_0}a = \frac{y - y_0}b = \frac{z - z_0}c
$$
may be interpreted in this light as asserting the intersection
of two planes.  The first equality could be rewritten
$$
    bx - ay = bx_0 - ay_0
$$
which defines a plane perpendicular to the $x,y$-plane. Similarly, the
second equality could be rewritten
$$
     cy - bz = cy_0 - bz_0
$$
which defines a plane perpendicular to the $y,z$-plane.  The line
is the intersection of these two planes.

In general, the intersection of three planes in space is a point,
but there are many special cases where it is not.  For example, the
planes could be parallel, or
the line of intersection of two of the planes might be parallel to
the third, in which case they do  not intersect at all.  Similarly, all
three planes could intersect in a common line.  This geometry is
reflected in the algebra.  If the planes intersect in a single point,
their defining equations provide a system of 3 equations in 3
unknowns which have a unique solution.  In the second case, the
equations are inconsistent, and there is no solution.  In the third
case there are infinitely many solutions.   We shall return to
a study of such issues when we study linear algebra and the solution
of systems of linear equations.

\subhead Elaborations \endsubhead
The analytic geometry of lines and planes in space is quite interesting
and presents us with many challenging problems.  We shall not go
into this in detail in this course, but you might profit from trying
some of the exercises.  One interesting problem, is to determine
the perpendicular distance between skew lines in space.  If the
lines are given parametrically by equations
$$\align
   \r &= t\v_1 + \r_1  \\
    \r &= t\v_2 + \r_2,
\endalign $$
then the distance between the lines is given by
\nexteqn
$$
 \frac{|(\r_2 - \r_1)\cdot(\v_2\times\v_1)|}{|\v_2\times\v_1|}.\tag\eqn
$$
See if you can derive this formula!   (There are some hints in the
exercises.)
\bigskip

% Exercises (with answers) for section 5
\input chap1.ex5


\bigskip
\nextsec{Integrating on Curves}
\head \sn. Integrating on Curves \endhead
\subhead Arc Length \endsubhead
Suppose a particle follows a path in space (or in the plane) described
by $\r = \r(t)$.  Suppose moreover it starts for $t = a$ at
$\r(a)$  and ends
for $t = b$ at $\r(b)$.
We want to calculate the total distance it travels on the curve. 
The correct formula for this distance is
\nexteqn
\xdef\eqnone{\eqn}
$$
   L = \int_a^b |\r'(t)| dt \tag\eqn
$$
\outind{arc length}%
where $\r'(t) = \dfrac{d\r}{dt}$ is the velocity vector at time $t$
(when the particle has position vector $\r(t)$).
\medskip
\centerline{\epsfbox{s1-46.ps}}
\medskip
This makes sense, because $|\r'(t)|$ should be the {\it speed\/}
\outind{speed}%
of the particle at time $t$, so $|\r'(t)|dt$ should
be a good approximation to the distance traveled in a small time
interval $dt$. (Integration amounts to adding up the incremental
distances.)  

\nextex
\xdef\exone{\en}
\example{Example \en}
Let $\r = R\cos t\,\i + R\sin t\, \j$ with $0\le t \le 2\pi$.  As we
saw earlier, this describes one circuit of a circle of radius $R$
centered at the origin.  We have
$$
   \frac{d\r}{dt} = -R\sin t \, \i + R\cos t \, \j,
$$
so $|\r'(t)| = \sqrt{R^2\cos^2 t + R^2\sin^2 t} = R$.  Hence,
$$
    L = \int_0^{2\pi} R dt =\left . R t \right\vert_0^{2\pi} = 2\pi R
$$
just as we would expect.
\endexample


\mar{s1-47.ps}
Formula (\eqnone) may be taken as the {\it definition\/} of the
arc length of the curve, but a little more discussion will clarify
 its ramifications.
Choose $n + 1$ points $\r_0, \r_1, \r_2, \dots, \r_n$ on the curve (as 
indicated in the diagram) by choosing a partition of the
time interval
$$  
a = t_0 < t_1 < t_2 < \dots < t_{n-1} < t_n = b,
$$
and letting the $i$th position vector $\r_i = \r(t_i)$.  
\medskip
\centerline{\epsfbox{s1-48.ps}}
\medskip
For two
neighboring points $\r_{i-1}$ and $\r_i$ on the curve, the
displacement vector $\Delta \r_i = \r_i - \r_{i-1}$ is represented in
the diagram
by the {\it chord\/} connecting the points.  Hence, for any plausible
definition of length, the length of arc $\Delta s_i$ on the
curve from $\r_{i-1}$ to $\r_i$ is approximated quite well by
the length of the chord, at least if the points are closely spaced.
Thus
$$
      \Delta s_i \approx |\Delta \r_i|.
$$
(`$\approx$' means `is approximately equal to'.)  Hence, if we
add everything up, we get
$$
    L = \sum_{i=1}^n \Delta s_i \approx \sum_{i=1}^n |\Delta \r_i|,
$$
that is, {\it the length of the curve is approximated by the length of
the polygonal path\/} made up of the chords.
To relate the latter length to the integral,
note that if we put 
$\Delta t_i = t_i - t_{i-1}$,  then for closely spaced points
$$
     \Delta \r_i \approx \r'(t_{i-1})\Delta t_i.
$$
Putting this in the prior formula for L yields
$$
    L \approx \sum_{i=1}^n |\r'(t_{i-1})| \Delta t_i,
$$
and the sum on the right is one of the approximating (Riemann) sums
which approach the integral $\displaystyle {\int_a^b|\r'(t)| dt}$
as $n \to \infty$.

In our discussions, we have denoted the independent variable
by `$t$' and thought of it as
time.  Although this is helpful in kinematics, from a mathematical
point of view this is not necessary.  Any vector valued function
$\r = \r(u)$ of a variable $u$ can be thought of as yielding a
curve as $u$ ranges through the set of values in the domain of
the function.  As mentioned elsewhere, the variable $u$ is usually
called a {\it parameter\/}, and often it will have some geometric
or other significance.
For example, in Example \exone, it might have made more sense to call
the parameter $\theeta$ (instead of $t$) and to think of it as the
angle the position vector makes with the positive real axis.

\nextex
\example{Example \en}
We shall find the length of the parabola with equation $y = x^2$
on the interval $-1\le x \le 1$.   One parametric representation
of the parabola  would be $\r = x \i + y \j = t \i + t^2 \j$
where $-1 \le t \le 1$.  However, there is no real need to introduce
a new variable; we might just as well use $x$, and write instead 
  $$
     \r = x\,\i + x^2\,\j,\qquad -1 \le x \le 1.
$$
The formula (\eqnone) still applies with $t$ replaced by $x$.
Then,
$$
     \frac{d\r}{dx} = \i + (2x)\j,
$$
so $|\r'(x)| = \sqrt{ 1 + 4x^2}$.
Hence,
$$
  L = \int_{-1}^1 \sqrt{1 + 4x^2} dx = \sqrt 5 + 
\frac 14\ln\frac{\sqrt 5 + 2}{\sqrt 5 - 2}.
$$

\mar{s1-49.ps}
You should ponder what we just did, and convince yourself there is
nothing peculiar about the use of $x$ as parameter.
\endexample

As the previous example shows, calculation of lengths often results
in difficult integrals because of the square root.  Recourse to
integral tables or appropriate computer software is advised.

There is one very tricky point in using formula (\eqnone) to
{\it define\/} length of arc on a curve.  The same curve might
be given by two different parametric representations
$\r = \r_1(u), a\le u \le b$ and $\r = \r_2(v), c \le v \le d$.
How can we be sure we will get the {\it same\/} answer for
the length if we compute
$$
   \int_a^b |\r'_1(u)| du\qquad\text{and}\qquad
\int_c^d |\r_2'(v)| dv ?
$$
The superficial answer is that in general we won't always
get the same answer.  For example,
a circle might be represented parametrically in two different
ways, one which traverses the circle  once, and one
which traverses it twice, and the answers for the length
would be different.  The curve {\it as a point set\/}
does not by itself determine the answer if part or all of it
is covered more than once in a given parametric representation.
This issue, while something to
worry about in specific examples, is really a red herring.  If we
are careful to choose parametric representations which trace each
part of the curve {\it exactly once}, then we expect always
to get the same answer.  (Such representations are called
{\it one-to-one}.)  Our reason for believing this is that
we think of length as a physical quantity which can be measured
by tapes (or more sophisticated means), so we expect the mathematical
theory to fall in line with reality.  Unfortunately, it is not logically
or philosophically legitimate to 
identify the world of  mathematical discourse with
physical reality, so it is incumbent on mathematicians to {\it prove\/}
that all one-to-one parametric representations yield the same answer.
We won't actually do this in this course, but some of the relevant
issues are discussed in the Exercises.


\subhead Line Integrals \endsubhead
In elementary physics, the work $W$ done by a constant force
$F$ exerted over a distance $\Delta$ is defined to be
the product $F\Delta$.
%D   Work.
\outind{work}%
In more complicated situations, physicists need to deal with
forces which may vary or which may not point along the direction
of motion.  In addition, they have to worry about motion which
is not constrained to a line.  We shall investigate the mathematical
concepts needed to generalize the concept of work in those
situations.  First suppose that the work is done along a line,
but that the force varies with position on the line.  If $x$
denotes a coordinate describing that position, and the force
is given by $F = F(x)$, then the work done going from
$x = a$ to $x = b$ is
$$
   W = \int_a^b F(x)dx.
$$
\medskip
\centerline{\epsfbox{s1-50.ps}}
\medskip
The justification for that formula is that if we think of the
interval $[a,b]$ being partitioned into small segments by
division points
$$
 a= x_0 < x_1 < x_2 < \dots < x_{n-1} < x_n = b,
$$
then the work done going from $x_{i-1}$ to $x_i$ should be
given approximately by $\Delta W_i \approx F(x_{i-1})\Delta x_i$
where $\Delta x_i = x_i - x_{i-1}$.  Adding up and taking a limit
yields the integral.

\nextex
\example{Example \en}
Suppose $F(x) = -kx, 0\le x \le D$.  (This is the restoring force
of a spring, where the constant $k$ is the {\it spring constant}.)
Then
$$
     W = \int_0^D (-kx)dx =\left . -k\frac{x^2}2\right\vert_0^D = - \frac {kD^2}2.
$$
\endexample

More generally, the force may not point in the direction of
the displacement.  If that is the case, we simply use the
component of the force in the direction of the displacement.
Thus, if the force $\F$ is a constant vector, and the displacement
is given by a vector $\Delta \r$, then the component in the
direction of of $\Delta \r$ is $|\F|\cos \theeta$ and the work
is $(|\F|\cos \theeta)|\Delta \r|$, which you should recognize
as the dot product
$$
    \F\cdot \Delta\r.
$$
If the force is not constant, this formula will still be approximately
valid if the magnitude of the displacement $\Delta \r$ is sufficiently
small.

\mar{s1-51.ps}
We are now ready to put all this together for the most general
situation.  Suppose we have a force $\F$ which may vary from point
to point, and it is exerted on a particle moving on a path
given parametrically by $\r = \r(u), a \le u \le b$.  (We assume
also that the representation is one-to-one, although it turns out
that this is not strictly necessary here.)  The correct mathematical
quantity to use for the work done is the integral
\nexteqn
\xdef\LineInt{\eqn}
$$
   \int_a^b \F(\r(u))\cdot\r'(u) du. \tag\eqn
$$
\outind{line integral}%
\medskip
\centerline{\epsfbox{s1-52.ps}}
\medskip
The idea behind this formula is that $d\r = \r'(u)du$ is the
displacement along the curve produced by a small increment
$du$ in the parameter, so $\F\cdot d\r = \F\cdot\r'(u)du$
is a good approximation to the work done in that displacement.
As usual, the integral sign suggests a summing up of the
incremental contributions to the total work.

\nextex
\xdef\CircEx{\en}
\example{Example \en}
Let $\Cal C$ be a circle of radius $R$ centered at the origin and
traversed in the counter-clockwise direction.  $\Cal C$
may be represented parametrically by
$$
\r = (R\cos\theeta)\i + (R\sin\theeta)\j,\qquad 0\le \theeta \le 2\pi.
$$
Suppose the the force
is given by $\F = -y\,\i + x\,\j$.   To calculate the work
by formula (\LineInt), we calculate
$$
     \r'(\theeta) - (-R\sin\theeta)\i + (R\cos\theeta)\j,
$$
and note that since $x = R\cos\theeta, y = R\sin\theeta$ {\it on the
circle\/}, we have
$$
   \F = -y\,\i + x\,\j = (-R\sin\theeta)\i + (R\cos\theeta)\j.
$$
Hence,
$$
W = \int_0^{2\pi} (R^2\cos^2\theeta + R^2\sin^2\theeta)d\theeta =
           \int_0^{2\pi} R^2 d\theeta = 2\pi R^2.
$$

\mar{s1-53.ps}
Note that had we not substituted for $x$ and $y$ in terms of $\theeta$
in the expression for $\F$, we would have gotten a meaningless
answer with $x$'s and $y$'s in it.   Generally, when working
such problems, one must be sure that  the relevant quantities in the integrand
have all been expressed in terms of the parameter.
\endexample

Some additional discussion will clarify the meaning of formula
(\LineInt).   Suppose a curve $\Cal C$ is represented
parametrically by a vector function
$\r = \r(u), a\le u \le b$, and suppose a  force $\F$
is defined on $\Cal C$ but may vary from point to point.  Choose
a sequence of points on the curve with position vectors
$\r_0, \r_1, \dots, \r_n$ by subdividing the parameter domain
$$
    a = u_0 < u_1 < u_2 < \dots < u_{n-1} < u_n = b
$$
and letting $\r_i = \r(u_i)$.  Let $\Delta \r_i = \r_i - \r_{i-1}$.
\medskip
\centerline{\epsfbox{s1-54.ps}}
\medskip
Then it is plausible that the work done moving along the curve
from $\r_{i-1}$ to $\r_i$
 is approximated by $\F(\r_{i-1})\cdot\Delta\r_i$,
at least if $|\Delta r_i|$ is small.  Moreover, adding it all up,
we have
$$
      W \approx \sum_{i=1}^n \F(\r_{i-1})\cdot\Delta \r_i.
$$
Just as in the case of arc length, the right hand side approaches
the integral in formula (\LineInt) as a limit as $n \to \infty$
and as the points get closer together.

We introduce the following notation for the limit of the sum
\nexteqn
\def\LineIntII{\eqn}
$$
\int_{\Cal C} \F\cdot d\r = \lim \sum_{i=1}^n \F(\r_{i-1})\cdot\Delta \r_i
            = \int_a^b\F(\r(u))\cdot\r'(u)\,du,
$$
and call it  a {\it line integral}.  It first appears in
physics in discussions of work and energy where $\F$ represents a
force, but it may be thought of as a mathematical entity
defined for any vector function of position $\F$.  

The curve $\Cal C$ and ultimately the line integral have been
discussed in terms of a specific parametric representation
$\r = \r(u)$.  As in the case of arc length, it is possible to
show that different parametric representations produce the same
answer {\it except for one additional difficulty}.   In the
formula for arc length, we dealt with $|\Delta \r|$, so the
{\it direction\/} in which the curve was traced was not relevant.
In the case of line integrals, we use $\F\cdot\Delta\r$, so
reversing the direction changes the sign of $\Delta\r$ and
hence it changes the sign of the line integral.  Line integrals,
in fact, must be defined for {\it oriented\/} curves for which
one of the two  possible directions has been specified.   If
$\Cal C'$ and $\Cal C$ are the same curve but traversed in opposite
directions then
$$
    \int_{\Cal C'}\F\cdot d\r = - \int_{\Cal C} \F\cdot d\r.
$$ 
\medskip
\centerline{\epsfbox{s1-55.ps}}
\medskip
There are several different notations for line integrals. 
For example, you may see
$$
  \int_{\Cal C}\F\cdot d\bold l\qquad\text{or}\qquad
  \int_{\Cal C} \F\cdot d\bold s.
$$
The `$d\bold s$' suggests a vector version of the element of
arc length $ds = |\r'(t)| dt$.   We can also write formally,
$\F = F_x\i + F_y\j + F_z\k, d\r = dz\,\i + dy\,\j + dz\,\k$,
so $\F\cdot d\r = F_xdx + F_ydy + F_zdz$, and
$$
  \int_{\Cal C} \F\cdot d\r = \int_{\Cal C} F_xdx + F_ydy + F_zdz.
$$
\mar{s1-56.ps}
The notation
$$
   \oint \F\cdot d\r
$$
is sometimes used to denote a line integral for an unspecified
closed path (i.e., one which starts and ends at the same point.)
Finally, in some cases the line integral depends only on the
endpoints of $A$ and $B$ of the path, so you may see the notation
$$
   \int_A^B\F\cdot d\r
$$
or something equivalent.  We shall discuss some of these concepts
and special notations later in this course.
\smallskip
\mar{s1-57.ps}
\smallskip


\goodbreak
\subhead Geometric Reasoning \endsubhead
\nobreak
\example{Example \CircEx, {\rm revisited}}
In applying formula (\LineInt) to Example \CircEx, you may have
noticed that, on the circle, the force
$$\align
  \F &= -R\sin\theeta\, \i + R\cos\theeta \,\j \\
\intertext{and the displacement}
  d\r &= \r'(\theeta)d\theeta = (-R\sin\theeta\, \i + R\cos\theeta\,\j)d\theeta
\endalign$$
have the same direction.  
  If we had been able to visualize this
geometric relationship, we could have calculated the line integral
more directly.   
Thus,
$$
     \F\cdot d\r = |\F||d\r|\cos 0 = |\F||d\r|.
$$
On the other hand, $|\F| = \sqrt{(-y)^2 + x^2} = \sqrt{R^2} = R$,
and $|d\r| = ds = Rd\theeta$.  (The length of arc on a circle is always
the radius of the circle times the subtended angle).   Hence,
we could have written directly
$$
  \int_{\Cal C} \F\cdot d\r = \int_0^{2\pi} R\, R d\theeta = 2\pi R^2.
$$
\endexample

\mar{s1-58.ps}
You will often see this approach used by physicists or engineers. 
 Note that
it emphasizes the geometric or physical significance of the
quantities, and it often makes clear underlying simplicity
 which might otherwise be hidden by complicated
formulas.   The approach used previously seems more straightforward,
but that is because the the geometric (or physical) part of the problem
had already been solved for you by giving you the
parametric representation.  For a real problem, you would have to
do that yourself.

To approach line integral problems geometrically, it is useful
to introduce one final bit of notation.  We can think
of the displacement $d\r$ as connecting two very close points on
the curve, or to a high degree of approximation as being tangent
to the curve.  We have already introduced the notation $ds$
for the magnitude of $d\r$, and we can specify its direction by
giving the unit vector $\T$ which is tangent to the curve at the
point and which points in the preferred direction along the curve.
Then $d\r = \T\,ds$, and we may write
$$
   \int_{\Cal C} \F\cdot d\r = \int_{\Cal C}\F\cdot \T\, ds.
$$  
It is important to note that this last formula is only useful
if you argue geometrically.

\nextex
\example{Example \en}
Suppose $\F = 2x\j$ and let $\Cal C$ be the path which starts at
$(0,0)$, moves to $(1,0)$ and then moves to $(1,1)$.   (See
the diagram.)  In this case, the path consists of two segments
$\Cal C_1$ and $\Cal C_2$ joined at a corner where the direction
changes discontinuously.  Clearly, the proper definition of
the line integral in this situation should be
$$
   \int_{\Cal C}\F\cdot d\r = 
   \int_{\Cal C_1}\F\cdot d\r + 
   \int_{\Cal C_2}\F\cdot d\r. 
$$
For the first segment $\Cal C_1$, $\F = 2x\j$ is perpendicular to
$\T$ (hence, to $d\r$), so $\F\cdot d\r = 0$.   Thus the first
integral vanishes.   For the second segment, $\Cal C_2$, 
$\F = 2x\j = 2\j$, and $d\r = \T\, ds = \j\,dy$.  $\F$ and $\T$
point the same way, so $\F\cdot d\r = 2 ds = 2 dy$.   Hence,
$$
   \int_{\C_2}\F\cdot d\r = \int_0^1 2\, dy = \left. 2y\right\vert_0^1
                = 2.
$$
To find the total answer, add up the answers for the two segments to
get $0 + 2 = 2$.
\endexample
\centerline{\epsfbox{s1-59.ps}}
\medskip
\nextex
\example{Example \en}
Let $\F = 2x\j$ as in the previous example, but let $\Cal C$ be the
straight line segment from $(0,0)$ to $(1,1)$.   Then, $\F = 2x\j$
makes angle $\pi/4$ with $\T$ (hence, with $d\r$).  Thus,
$$\F\cdot d\r = |\F|ds\, \cos(\frac\pi 4) = 2x ds\frac 1{\sqrt 2}.$$
Choosing $x$ as the parameter, we may write $ds = \sqrt 2 \, dx$,
so
$$
 \int_{\Cal C}\F\cdot d\r = \int_0^1 \frac{2x\sqrt 2 dx}{\sqrt 2}
                   = \int_0^1 2x\,dx = \left. x^2\right\vert_0^1 = 1.
$$

Here is another slightly different way to do the calculation.
We have $\F = 2x \j, \, d\r = dx\,\i + dy\, \j$.  Hence,
$\F\cdot d\r = 2x\,dy$.   It would seem appropriate to choose
$y$ as parameter, so since $x = y$ on the given line, we have
$$
 \int_{\Cal C}\F\cdot d\r = \int_0^1 2y\,dy = 1.
$$

Finally, the calculation could be done using formula (\LineInt)
as follows.  The given line
segment can be described by the parametric representation
$\r = t\,\i + t \,\j, \, 0\le t \le 1$.   Then
$\r'(t) = \i + \j$, and on the line $\F = 2x\,\j = 2t\,\j$.
Hence, by formula (\LineInt),
$$
  \int_{\Cal C} \F\cdot d\r = \int_0^1(2t\,\j)\cdot(\i + \j)dt
 = \int_0^1 2t\,dt = 1.
$$ 
\endexample

\subhead Polar Coordinates \endsubhead
In polar coordinates, the velocity vector is given by
$$
   \frac{d\r}{dt} = \frac{dr}{dt} \u_r + r\frac{d\theeta}{dt}\u_\theeta
$$
so we may write symbolically
$$
   d\r = \frac{d\r}{dt} dt = dr\,\u_r + rd\theeta\,\u_\theeta.
$$
Hence, if $\F = F_r\u_r + F_\theeta\u_\theeta$ is resolved into
polar components, we may write
$$
  \int_{\Cal C} \F\cdot d\r = \int_{\Cal C} F_rdr + F_\theeta r\,d\theeta.
$$

\example{Example}
Let $\F = -\dfrac 1{r^2} \u_r$.    Such a force has magnitude
 inversely proportional to the square of the distance to the origin
and is directed toward the origin.   (Does it remind you of anything?)
Let  $\Cal C$ be any path whatsoever, which starts at a point
with $r = a$ and ends up at a point with $r = b$.   We have
$$
    \F\cdot d\r = (-\frac 1{r^2} \u_r) \cdot (dr\,\u_r + rd\theeta\,\u_\theeta)
      = -\frac 1{r^2} dr.
$$
(In other words, only the radial component of the displacement counts
since the force is entirely radial in direction.)
If we choose $r$ as the parameter, we have
$$
   \int_{\Cal C} \F\cdot d\r = \int_a^b -\frac 1{r^2} dr
            = \left.\frac 1r\right\vert_a^b = \frac 1b - \frac 1a.
$$
In particular, the answer does not depend on the path, only on the
values of $r$ at the start and finish.
\medskip
\centerline{\epsfbox{s1-60.ps}}
\medskip
Note that the argument (and diagram) presumes that the particle moves
in such a way that the radius always increases.  Otherwise, it would
not make sense to use $r$ as the parameter, since there should be
exactly one point on the curve for each value of the parameter.
  Can you see how to make
the argument work if the path is allowed to meander in such a way
that $r$ sometimes increases and sometimes decreases?
\endexample
\bigskip

% Exercises (with answers) for section 6
\goodbreak
\input chap1.ex6

\goodbreak
\endchapter
\closeseg{chap1}
\enddocument
